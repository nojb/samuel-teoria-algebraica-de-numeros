\documentclass[bibtotoc,leqno,spanish]{amsbook}
%\usepackage[linktocpage=false,pagebackref=true]{hyperref}

\usepackage[paperwidth=14.81cm, paperheight=21.01cm, right=1.5cm, left=1.5cm, top=1.5cm, bottom=1.5cm, includehead, includefoot]{geometry}

%,es-noindentfirst
%,es-nosectiondot
\usepackage{babel}
\usepackage{amsmath}
\usepackage{amsthm}
\usepackage{amssymb}
\usepackage{xspace}
\usepackage{frenchmath}
%\usepackage[color]{showkeys}

\let\emph\relax % there's no \RedeclareTextFontCommand
\DeclareTextFontCommand{\emph}{\bfseries\upshape}

\unaccentedoperators

\newcommand{\RR}{\mathbf{R}}
\newcommand{\QQ}{\mathbf{Q}}
\newcommand{\ZZ}{\mathbf{Z}}
\newcommand{\NN}{\mathbf{N}}
\newcommand{\FF}{\mathbf{F}}
\newcommand{\CC}{\mathbf{C}}
\newcommand{\HH}{\mathbf{H}}

\newcommand{\idl}[1]{\mathfrak{#1}}
\newcommand{\QED}{LQQD.}
\newcommand{\oline}[1]{\overline{#1}}
\newcommand{\then}{\ensuremath{\Rightarrow}\xspace}
\newcommand{\disc}{\mathfrak{D}}
\newcommand{\diff}{\mathfrak{d}}
\newcommand{\abs}[1]{\left\lvert#1\right\rvert}
\renewcommand{\to}[1][]{\xrightarrow{#1}}
\newcommand{\leg}[2]{\left(\frac{#1}{#2}\right)}

%\DeclareMathOperator{\card}{card}
%\DeclareMathOperator{\Ker}{Ker}
\DeclareMathOperator{\mcm}{mcm}
\DeclareMathOperator{\mcd}{mcd}
\DeclareMathOperator{\pr}{pr}
\DeclareMathOperator{\Tr}{Tr}
%\DeclareMathOperator{\Hom}{Hom}

\numberwithin{equation}{section}
\renewcommand{\theequation}{\arabic{equation}}

\addto\captionsspanish{\renewcommand{\chaptername}{Cap\'itulo}}
\addto\captionsspanish{\renewcommand{\contentsname}{\'Indice general}}
\addto\captionsspanish{\renewcommand{\translname}{Traducido por}}

\newenvironment{comm}%
	{\begin{list}{}{\setlength{\leftmargin}{2\parindent}\setlength{\topsep}{\baselineskip}}\item\itshape}
	{\end{list}}

\newtheoremstyle{defi}{}{}{}{}{\bfseries}{.}{ }{}

\newtheoremstyle{note}% name
  {}%      Space above
  {}%      Space below
  {\itshape}%         Body font
  {}%         Indent amount (empty = no indent, \parindent = para indent)
  {\bfseries}% Thm head font
  {.}%        Punctuation after thm head
  { }%     Space after thm head: " " = normal interword space;
     %       \newline = linebreak
  {}%Thm head spec (can be left empty, meaning `normal')

\theoremstyle{note}
\newtheorem*{definition*}{Definici\'on}
\newtheorem{definition}{Definici\'on}
\newcommand{\exername}{}
\newtheorem*{namedexer}{\exername}
\newenvironment{exer*}[1]{\renewcommand{\exername}{#1}\begin{namedexer}}{\end{namedexer}}

\theoremstyle{note}
\newtheorem{theorem}{Teorema}
\newtheorem*{theorem*}{Teorema}
\newtheorem{proposition}{Proposici\'on}
\newtheorem*{proposition*}{Proposici\'on}
\newtheorem{lemma}{Lema}
\newtheorem*{lemma*}{Lema}
\newtheorem{corollary}{Corolario}
\newtheorem*{corollary*}{Corolario}
\newcommand{\namedname}{???}
\newtheorem*{namedthm}{\namedname}
\newenvironment{named}[1]%
	{\renewcommand{\namedname}{#1}%
	\begin{namedthm}}%
	{\end{namedthm}}

\newtheoremstyle{rem}
  {}{}{}{}{\itshape}{.}{ }{}

\theoremstyle{rem}
\newtheorem{example}{Ejemplo}
\newtheorem*{example*}{Ejemplo}
\newtheorem{remark}{Observaci\'on}
\newtheorem*{remark*}{Observaci\'on}

\renewcommand{\thechapter}{\Roman{chapter}}
\renewcommand{\thesection}{\arabic{section}}

\numberwithin{theorem}{section}
\numberwithin{proposition}{section}
\numberwithin{definition}{section}
\numberwithin{lemma}{section}
\numberwithin{corollary}{section}
\numberwithin{example}{section}
\renewcommand{\thetheorem}{\arabic{theorem}}
\renewcommand{\theproposition}{\arabic{proposition}}
\renewcommand{\thedefinition}{\arabic{definition}}
\renewcommand{\thelemma}{\arabic{lemma}}
\renewcommand{\thecorollary}{\arabic{corollary}}
\renewcommand{\theexample}{\arabic{example}}

\numberwithin{footnote}{section}%restart in each section
\renewcommand{\thefootnote}{\arabic{footnote}}

\title{Teor\'ia de n\'umeros algebraica}
\author{Pierre Samuel}
\translator{Nicol\'as Ojeda B\"ar}
\date{}

\setcounter{tocdepth}{1}

\begin{document}

\frontmatter

\maketitle

\newpage

Pierre Samuel, doctor en ciencias, profesor de la universidad Paris XI, naci\'o en 1921.
Sus trabajos conciernen principalmente el \'algebra conmutativa y la geometr\'ia algebraica.

\tableofcontents

\chapter*{Introducci\'on}

La Teor\'ia de N\'umeros, o Aritm\'etica, a veces es llamada
la ``Reina de la Matem\'atica''. La simplicidad del objeto
de estudio (los n\'umeros enteros y sus generalizaciones), la elegancia,
la diversidad de los m\'etodos y los numeros problemas irresueltos
ejercen una poderosa atracci\'on sobre los matem\'aticos,
ya sean principiantes, te\'oricos de n\'umeros profesionales o
especialistas en otras ramas de la matem\'atica. Es por eso que no
debe sorprender al lector que el autor de este libro es un ge\'ometra
algebraico, que no tiene ninguna publicaci\'on original dentro
del dominio de la Aritm\'etica propiamente dicho.

De hecho, este libro no se distingue ni por su profundidad ni
por su alcance. M\'as a\'un, s\'olo presenta uno de los posibles
puntos de vista con los cuales se puede abordar la teor\'ia de n\'umeros,
a saber, el punto de vista algebraico. Salvo un resultado elemental
de Minkowski sobre reticulados en $\RR^{n}$, no se tocan en ning\'un
momento los bellos, y f\'ertiles, m\'etodos anal\'iticos.

La preponderancia otorgada al punto de vista algebraico me parece
justificada por diversas razones. En primer lugar, permite ponerse
r\'apidamente en el marco donde los problemas de la aritm\'etica
se enuncian en su forma
m\'as natural, incluso cuando s\'olo involucran los n\'umeros enteros
usuales. Veremos, por ejemplo, que la b\'usqueda de soluciones
en enteros de la ecuaci\'on de {Pell-Fermat} $x^{2}-dy^{2}=\pm 1$
($d$: un entero dado libre de cuadrados) es un problema que esencialmente
concierne al cuerpo cuadr\'atico $\QQ(\sqrt{d})$. Para la
``gran'' ecuaci\'on de {Fermat} $x^{n}+y^{n}=z^{n}$, es el cuerpo de
ra\'ices  $n$-\'esimas de la unidad que juega el papel decisivo.
Para escribir un entero como suma de dos (respectivamente, cuatro)
cuadrados, veremos que es muy ventajoso trabajar con el anillo
$\ZZ[i]$ de los enteros de Gauss (respectivamente, con un
anillo de cuaterniones conveniente). La ley de reciprocidad
cuadr\'atica hace intervenir al mismo tiempo los cuerpos cuadr\'aticos y las ra\'ices de
la unidad. A lo largo de todo esto, aparecen cuerpos m\'as generales que
$\QQ$, anillos m\'as generales que $\ZZ$, como tambi\'en sus cuerpos y
anillos cocientes, es decir, cuerpos finitos y \'algebras
sobre \'estos.

As\'i, si bien no agota la Teor\'ia de N\'umeros, el m\'etodo algebraico
permite obtener rapidamente resultados substanciales. De hecho, de continuar en
esta misma direcci\'on, arrivar\'iamos a teoremas m\'as profundos,
como aquellos de la teor\'ia de cuerpos de clases.

Por otra parte, a\'un aquellos que prefieren los m\'etodos anal\'iticos
no ignoran que \'estos s\'olo adquieren su m\'axima expresi\'on cuando se los aplica
a cuerpos de n\'umeros algebraicos y no s\'olamente a $\QQ$.
Por ejemplo, no vale la pena estudiar s\'olamente la funci\'on
$\zeta(s)$ sin tratar al mismo tiempo la funci\'on $\zeta_{K}(s)$
de un cuerpo de n\'umeros $K$, ni otras numerosas ``series $L$.''

Finalmente, el desarollo del m\'etodo algebraico tiene la ventaja de presentar
al estudiante con numerosos ejemplos ilustrativos de las nociones introducidas
en el curso de \'algebra: grupos, anillos, cuerpos, ideales, anillos y cuerpos cociente,
homomorfismos e isomorfismos, m\'odulos y espacios vectoriales. Al mismo tiempo
se introducen numerosas nociones algebraicas que son fundamentales
en otras ramas de la matem\'atica, como la Geometr\'ia Algebraica. Por ejemplo,
los elementos enteros sobre un anillo, las extensiones de cuerpos, la teor\'ia de Galois,
los m\'odulos sobre dominios de ideales principales, los anillos y m\'odulos noetherianos,
los anillos de Dedekind y los anillos de fracciones.

Lo anterior describe implicitamente lo que el lector encontrar\'a
en este libro y aquello que le ser\'a imposible encontrar aqu\'i.
He asumido que \'el conoce el \'algebra de un primer curso de licenciatura:
nociones b\'asicas sobre los grupos, anillos, cuerpos, polinomios,
espacios vectoriales---el manejo de los subobjetos, objetos cociente
y objetos producto---el mecanismo del paso al cociente por un ideal o
un submodulo---las diversas nociones de homomorfismo e isomorfismo.
Todo lo necesario sobre estas cuestiones puede encontrarse en un
libro b\'asico de \'algebra ``moderna,'' por ejemplo el
excelente ``Cours d'Alg\`ebre'' de {R. Godement} o el ``Algebra''
de {S. Lang\footnote{Desde ya, estas dos obras tienen un alcance mucho mayor.}.}
Por eso, utilisar\'e este lenguaje
y resultados sin mencionarlo y espero poder mostrarle
al lector que son muy eficaces a la hora de obtener rapidamente
teoremas substanciales de la aritm\'etica.
Por otra parte, pens\'e que, si bien estos temas se cubren en
la orientaci\'on de ``algebra'' de la Licenciatura, ser\'ia
m\'as c\'omodo para el lector poder encontrar aqu\'i todo lo
necesario sobre elementos enteros de un anillo, extensiones
algebraicas de cuerpos, teor\'ia de Galois, modulos y
anillos noetherianos y anillos de fracciones. He intenatado
presentar estos temas sin ninguna laguna (??), pero al mismo
tiempo evitando cualquiera sofisticacion inutil.

Este libro debe su existencia a un curso de ``Matem\'atica avanzada''
dictado en la Universidad de Paris en 1965 y de nuevo en 1966.
Notas fotocopiadas de {Alfred Vidal-Madjar,} alumno del \'Ecole
Normale Sup\'erieure, al cual le agradezco vivamente, sirvieron
como una primera versi\'on. Algunos pasajes provienen de cursos dicatdos
en el \'Ecole Normale Sup\'erieure de Jeunes Filles y en la Universidad
de Clermont-sur-Tiretaine. Por \'ultimo, me fueron muy importantes el consejo
y la influencia de numerosos matem\'aticos. Entre estos, quiero agradecer
especialmente al maestro de mi generaci\'on, {N. Bourbaki,} que tuvo la
amabilidad de mostrarme algunos de sus manuscritos que todav\'ia no han sido
publicados. Tambi\'en les agradezo a mis amigos {Emil Grosswald,} {Georges
Poitou,} {Jean-Pierre Serre} y {John Tate.}

\section*{Prefacio a la segunda edici\'on}

Mis sinceros agradecimientos a numerosos lectores de la primera
edici\'on, entre ellos Germaine Revuz, Alain Bouvier y Pierre Cartier,
que env\'iaron listas de correcciones sumamente \'utiles. Las observaciones
del traductor de la edici\'on inglesa, Alan Silberger, tambi\'en han sido
muy valiosas.

\begin{flushright}
julio 1971
\end{flushright}

{
  \newpage
  \thispagestyle{empty}
  \itshape
  \hspace{0pt}\vfill
  \begin{flushright}
    \`A NICOLE

    qui a su cr\'eer autour de moi

    une atmosph\`ere favorable \`a ce livre.
  \end{flushright}
  \vfill
}

\chapter*{Repaso de notaciones, definiciones y resultados}

Utilizamos las notaciones usuales de la teor\'ia de conjuntos:
$\in$, $\subset$, $\cup$, $\cap$.
El complemento de un subconjunto $B$ de un conjunto $A$ se nota $A - B$.
El cardinal (o potencia o n\'umero de elementos) de un conjunto $A$ lo notaremos
$\card(A)$; si $A$ es un grupo tambi\'en hablamos del orden de $A$.

Suponemos que el lector est\'a familiarizado
con las nociones de grupo, anillo, cuerpo y espacio vectorial,
como as\'i tambi\'en la teor\'ia b\'asica de los espacios vectoriales
(llamada tambi\'en ``algebra lineal''). En este libro, salvo en el cap\'itulo~\ref{cap5},
\S\ref{sec5.7}, ``anillo'' (resp. ``cuerpo'') quiere decir anillo (resp. cuerpo)
\emph{conmutativo y con unidad.}

Dado un grupo finito $G$ y un subgrupo $H$ de $G$, recordamos que $\card(H)$ divide
a $\card(G)$; el cociente $\card(G)/\card(H)$ se llama el \'indice de $H$ en $G$ y
se nota $(G:H)$.

Dados dos subconjuntos $A$ y $B$ de un grupo $G$ (escrito aditivamente), $A+B$
denota el conjunto de sumas $a+b$ con $a\in A$ y $b\in B$.

Dado un anillo $A$, denotamos por $A[X]$ o $A[Y]$ (letras may\'usculas)
el anillo de polinomios (formales) en una variable sobre $A$; escribimos
$A[X_{1},\dots,X_{n}]$ para el anillo de polinomios en $n$ variables y $A[[X]]$ para las
series formales.

Por convenci\'on, un subanillo $A$ de un anillo $B$ contiene el elemento unidad de $B$.
Dado un anillo $B$, un subanillo $A$ de $B$ y un elemento $x\in B$, denotamos por
$A[x]$ el subanillo de $B$ generado por $A$ y $x$, es decir, la intersecci\'on de todos
los subanillos de $B$ que contienen $A$ y $x$; es el conjunto de sumas de la forma
$a_{0}+a_{1}x+\dots+a_{n}x^{n}$ ($a_{i}\in A$); Escribimos $A[x_{1},\dots,x_{n}]$
para el subanillo de $B$ generado por $A$ y una colecci\'on finita $(x_{1},\dots,x_{n})$
de elementos de $B$.

Un anillo se dice \emph{dominio \'integro} (o sin divisores de cero) si el producto de dos
elementos no nulos es no nulo y si $A$ tiene m\'as de un elemento.

Un ideal $\idl{b}$ de un anillo $A$ es un subgrupo aditivo tal que $x\in\idl{b}$
y $a\in A$ implican $ax\in\idl{b}$. El anillo completo y el conjunto que consiste del
\'unico elemento $0$ (escrito $\{0\}$) son ideales, algunas veces llamados ``triviales''.
Un cuerpo no posee ning\'un otro ideal, y esta propiedad caracteriza a los cuerpos
entre todos los anillos. Dada una familia $(b_{i})$ de elementos de un anillo $A$, la intersecci\'on
de los ideales de $A$ que contienen a los $b_{i}$ es un ideal de $A$, llamado el ideal
generado por los $b_{i}$; es el conjunto de sumas finitas $\sum_{i}a_{i}b_{i}$ con $a_{i}\in A$.
Un ideal generado por un elemento $b$ se dice principal y lo escribimos $Ab$ \'o $(b)$.

Dado un anillo $A$ y un ideal $\idl{b}$ de $A$, las clases de equivalencia $a+\idl{b}$
($a\in A$) forman un anillo, llamado anillo cociente de $A$ por $\idl{b}$ y notado
$A/\idl{b}$. Los ideales de $A/\idl{b}$ son de la forma $\idl{b}'/\idl{b}$, donde
$\idl{b}'$ recorre el conjunto de ideales de $A$ que contienen a $\idl{b}$. para que
$A/\idl{b}$ sea un cuerpo es necesario y suficiente que $\idl{b}$ sea maximal entre los ideales
de $A$ distintos a $A$; en este caso decimos que $\idl{b}$ es maximal. Un ideal $\idl{p}$
se dice primo si $A/\idl{p}$ es un dominio \'integro.

Dados dos anillos $A$, $A'$ con elementos neutros $e$ y $e'$, un homomorfismo $f:A\to A'$ es una
funci\'on $f$ de $A$ en $A'$ tal que:
\begin{gather*}
f(a+b)=f(a)+f(b),\quad f(ab) = f(a)f(b),\quad f(e) = e'.
\end{gather*}
Dado un anillo $A$, una \emph{$A$-\'algebra} es un anillo $B$ equipado con un homomorfismo
$\varphi:A\to B$. Si $A$ es un cuerpo $\varphi$ es injectivo, y en este caso generalmente
identificamos $A$ con su imagen $\varphi(A)$ (que es un subanillo de $B$).

Dado un cuerpo $L$ y un subcuerpo $K$ de $L$, decimos que $L$ es una extension de $K$.

El elemento neutro de un anillo $A$ sera escrito casi siempre como $1$.

La noci\'on de \emph{m\'odulo} sobre un anillo $A$ (o de $A$-m\'odulo) es una generalizaci\'on
directa del concepto de espacio vectorial sobre un cuerpo. Un $A$-modulo $M$ es un grupo abeliano
(escrito aditivamente) junto con una aplicaci\'on $A\times M\to M$ (escrita multiplicativamente)
tal que $a(x+y)=ax+ay$, $(a+b)x = ax+bx$, $a(bx) = (ab)x$ y $1x = x$ ($a,b\in A$, $x,y\in M$).
Se tienen las nociones de subm\'odulo y m\'odulo cociente. Dados dos $A$-m\'odulos $M$ y $M'$,
un homomorfismo (o aplicaci\'on $A$-lineal) de $M$ en $M'$ es una
funci\'on $f:M\to M'$ tal que
\begin{gather*}
f(x+y) = f(x)+f(y),\quad f(ax) = af(x)\quad(a\in A, x,y\in M).
\end{gather*}
Dado un homomorfismo $f:X\to X'$ (de grupos, anillos o m\'odulos), llamamos \emph{n\'ucleo} de
$f$, y lo escribimos $\ker(f)$, a la imagen rec\'iproca por $f$ del elemento neutro de $X'$. Es un
subgrupo normal (o un ideal, o un subm\'odulo) de $X$; para que $f$ sea inyectiva es necesario y
suficiente que $\ker(f)$ consista \'unicamente del elemento neutro de $X$. Llamamos \emph{imagen}
de $f$ al subconjunto $f(X)$ de $X'$; es un subgrupo (o un subanillo, o un subm\'odulo) de $X'$.

Dados dos conjuntos $X$, $X'$, una funci\'on $f$ de $X$ en $X'$ generalmente se nota
$f:X\to X'$. Cuando una funci\'on $f:X\to X'$ se describe dando el valor que
le asigna a un elemento arbitrario $x$ de $X$, utilizamos la notaci\'on $x\mapsto f(x)$.
Por ejemplo, la funci\'on seno, $\sin:\RR\to\RR$ puede definirse por
\begin{gather*}
x\mapsto\sum_{n=0}^{\infty}(-1)^{n}\frac{x^{2n+1}}{(2n+1)!}.
\end{gather*}
Utilizamos las notaciones usuales para los siguientes objetos matem\'aticos:

\begin{trivlist}\setlength{\itemindent}{\parindent}
\item[$\NN$:] conjunto de los enteros naturales $(0,1,2,\dots,n,\dots)$ (N por ``n\'umeros'').
\item[$\ZZ$:] anillo de los enteros racionales (enteros naturales o sus opuestos)
(Z por ``Zahlen'').
\item[$\QQ$:] cuerpo de n\'umeros racionales (cocientes de elementos de $\ZZ$) (Q por
``quotients'').
\item[$\RR$:] cuerpo de los n\'umeros reales (R por ``reales'').
\item[$\CC$:] cuerpo de los n\'umeros complejos (C por ``complejos'').
\item[$\FF_{q}$:] cuerpo finito de $q$ elementos (F por ``finito'' o ``field'').
\end{trivlist}

\mainmatter

\chapter{Dominios de ideales principales}\label{cap1}

\section{Divisibilidad en dominios de ideales principales}\label{sec1.1}

Sea $A$ un dominio \'integro, $K$ su cuerpo de fracciones, $x$ e $y$ dos elementos de $K$.
Decimos que \emph{$x$ divide a $y$} si existe $a\in A$ tal que $y = ax$. Tambi\'en utilizamos
las expresiones ``$x$ es un divisor de $y$'', ``$y$ es un m\'ultiplo de $x$'' y lo notamos $x\mid y$.
Esta relaci\'on entre elementos de $K$ depende de forma esencial del anillo $A$. Si es necesario
precisarlo, decimos que se trata de divisibilidad en $K$ \emph{respecto a $A$.}

Dado $x\in K$, el conjunto de multiplos de $x$ es, utilizando la notaci\'on cl\'asica, $Ax$. As\'i,
$x\mid y$ se puede escribir tambi\'en $y\in Ax$ o incluso $Ay\subset Ax$. El conjunto $Ax$
se llama \emph{ideal fraccionario principal} de $K$ respecto a $A$. Si $x\in A$, $Ax$ el el
ideal principal (usual) de $A$ generado por $x$. Como la relaci\'on de divisibilidad $x\mid y$
equivale a la relaci\'on de \emph{orden} $Ay\subset Ax$, se tienen las dos propiedades siguientes
que poseen todas las relaciones de orden.
\begin{gather}
\text{$x\mid x$;\quad si $x\mid y$ y $y\mid z$ entonces $x\mid z$}.
\end{gather}

Por otra parte, si $x\mid y$ y $y\mid x$, no podemos concluir en general que $x = y$; s\'olamente
se tiene que $Ax = Ay$, lo que quiere decir (si $y\neq 0$) que el cociente $xy^{-1}$ es un
elemento \emph{inversible} de $A$. Tales pares de elementos se dicen \emph{asociados;} son indistinguibles desde
el punto de vista de la divisibilidad.

\begin{example*}
Los elementos de $K$ asociados a $1$ son los elementos inversibles de $A$. Usualmente se llaman
\emph{unidades} de $A$ y forman un grupo con la multiplicaci\'on, que notaremos $A^{\times}$. El c\'alculo
de las unidades de un anillo $A$ es un problema interesante y nosotros lo trataremos cuando
$A$ es el anillo de enteros de un cuerpo de n\'umeros (ver c\'apitulo IV). Los siguientes son algunos
ejemplos f\'aciles:
\begin{enumerate}
\item[a)] Si $A$ es un cuerpo, $A^{\times}$ es el conjunto de elementos no nulos de $A$;
\item[b)] Si $A = \ZZ$, $A^{\times}$ consiste de $+1$ y $-1$.
\item[c)] Las unidades del anillo de polinomios $B = A[X_{1},\dots,X_{n}]$ son, si $A$ es un dominio
\'integro, las constantes inversibles. Es decir, $B^{\times} = A^{\times}$.
\item[d)] Las unidades del anillo de series formales $A[[X_{1},\dots,X_{n}]]$ son las series formales
cuyo t\'ermino constante es inversible.
\end{enumerate}
\end{example*}

\begin{definition}
Un anillo $A$ se dice un dominio de ideales principales si es un dominio \'integro y si todo ideal
es principal.
\end{definition}

En el curso de \'algebra b\'asica se demuestra que el anillo $\ZZ$ es un dominio de ideales
principales (Recordemos que todo ideal $\idl{a}\neq(0)$ de $\ZZ$ contiene un
entero $b > 0$ m\'inimo; utilizando divisi\'on euclidea de $x\in\idl{a}$ por $b$,
vemos que $x$ es un m\'ultiplo de $b$). Si $k$ es un cuerpo, sabemos igualmente
que el anillo $k[X]$ de polinomios en \emph{una} variable es un dominio de
ideales principales (mismo m\'etodo: tomamos un polinomio no nulo $b(X)$ de grado
m\'inimo en el ideal dado $\idl{a}$ y utilizamos divisi\'on euclidea por $b(X)$).
Este m\'etodo se generaliza a los anillos que se conocen como ``euclideos''
(\cite{Bourbaki1}, cap\'itulo~VIII, \S1, ejercicio; o \cite{ZariskiSamuel}, cap\'itulo~I).
Si $k$ es un cuerpo, es f\'acil ver que todo ideal no nulo del anillo de
series formales $A = k[[X]]$ es de la forma $AX^{n}$ con $n \geq 0$, de manera
que $A = k[[X]]$ es un dominio de ideales principales.

La divisibilidad en el cuerpo de fracciones $K$ de un \emph{dominio de ideales
principales} $A$ es particularmente simple. Como es una generalizaci\'on
inmediata del caso de los enteros usuales, hacemos una revisi\'on breve.

\setcounter{equation}{0}

\begin{trivlist}\setlength{\itemindent}{\parindent}
\item {I.} Dos elementos cualesquiera $u$, $v$ de $K$ tienen un \emph{m\'aximo com\'un divisor}
(m.c.d), es decir un element $d$ tal que las relaciones
\begin{gather}
\text{``$x\mid y$ y $x\mid v$'' y ``$x\mid d$''}
\end{gather}
son equivalentes. Equivalentemente, $Au$ y $Av$ tienen un \emph{supremo\footnote{N. del T.: cota superior minimal.}} en el
conjunto ordenado de los ideales fraccionarios principales; por ejemplo, el ideal
$Au+Av$, que es un ideal fraccionario principal pues el anillo $A$ es un dominio de
ideales principales (la afirmaci\'on es clara si $u, v\in A$; nos reducimos a este
caso multiplicando $u$ y $v$ por un denominador com\'un). De hecho, obtenemos un poco
m\'as (``\emph{la identidad de Bezout}''): existen elementos $a$, $b$ de $A$ tal que
el m.c.d. $d$ de $u$ y $v$ se escribe
\begin{gather}\label{eq-1.1-2}
d = au+bc.
\end{gather}
El m.c.d. de $u$ y $v$ est\'a determinado de forma \'unica a menos de un elemento
inversible de $A$.

\item {II.} Dos elementos cualesquiera $u$, $v$ de $K$ tienen
un \emph{m\'inimo com\'un m\'ultiplo}
(m.c.m.), es decir un elemento $m$ tal que las relaciones
\begin{gather}
\text{``$u\mid x$ y $v\mid x$'' y ``$m\mid x$''}
\end{gather}
son equivalentes. Esto se puede ver observando que pasar al inverso $t\mapsto t^{-1}$
invierte las relaciones de divisibilidad, lo que nos reduce al caso del m.c.d.; de esto
se sigue que
\begin{gather}
\mcm(u,v) = \mcd(u^{-1},v^{-1})^{-1}\quad (\text{si $u,v\neq 0$}),
\end{gather}
de donde deducimos la f\'ormula usual
\begin{gather}
\mcm(u,v)\cdot\mcd(u,v) = uv.
\end{gather}

Tambi\'en podr\'iamos haber procedido como en I) y observar que la existencia del
m.c.m. de $u$ y $v$ equivale a la existencia de un
\emph{\'infimo\footnote{N. del T.: cota inferior maximal.}} para $Au$ y
$Av$ en el conjunto ordenado de los ideales fraccionarios principales; esta no es
otra que $Au\cap Av$.

\item {III.} Dos elementos $a$, $b$ de $A$ se dicen \emph{coprimos} si $1$ es uno de sus
m.c.d. Recordemos el fundamental \emph{Lema de Euclides.} Sean $a$, $b$, $c$
elementos de un dominio de ideales principales $A$; si $a$ divide a $bc$ y es
coprimo con $b$, entonces $a$ divide a $c$.

\begin{comm}
Demostraci\'on breve: por Bezout~\eqref{eq-1.1-2}, existen $a'$ y $b'\in A$
tales que $1 = a'a+b'b$, de donde $c = a'ac+b'bc$. Como $a$ divide a cada uno de los
t\'erminos de la derecha, tambi\'en divide a $c$.
\end{comm}

\item {IV.} Por \'ultimo, tenemos la importante ``descomposici\'on en factores primos:''
\begin{theorem*}
Sea $A$ un dominio de ideales principales con cuerpo de fracciones $K$. Existe
un subconjunto $P$ de $A$ tal que todo $x\in K$ se escribe de forma \'unica
\begin{gather}
x = u\prod_{p\in P}p^{v_{p}(x)}
\end{gather}
donde $u$ es un elemento inversible de $A$ y los exponentes $v_{p}(x)$ son
elementos de $\ZZ$, todos nulos salvo un n\'umero finito.
\end{theorem*}
\end{trivlist}

\begin{comm}
Para una presentaci\'on m\'as sistem\'atica de estas cuestiones, enviamos al lector
a \cite{Bourbaki1}, {\itshape Alg\`ebre,} cap\'itulo~VI, \S1 y cap\'itulo~VII, \S1.
Una parte de la teor\'ia (m\'as precisamente, todo aquello que no depende de la identidad de Bezout)
se extiende a anillos m\'as generales que los dominios de ideales principales, a saber
los \emph{dominios de factorizaci\'on \'unica;} ver \cite{Samuel2}, o \cite{Bourbaki2} Alg\`ebre
commutative, cap\'itulo~VII, \S3.
\end{comm}

\section{Un ejemplo: las ecuaciones $x^{2}+y^{2}=z^{2}$ y $x^{4}+y^{4}=z^{4}$}\label{sec1.2}

Una de las partes m\'as atractivas de la Teor\'ia de N\'umeros es el estudio de
las \emph{ecuaciones diof\'anticas.} Se consideran ecuaciones polinomiales
$P(x_{1},\dots,x_{n}) = 0$ con coefficientes en $\ZZ$ (resp. en $\QQ$) de las cuales
se buscan soluciones $(x_{i})$ enteras (resp. racionales). Podemos reemplazar $\ZZ$
(resp. $\QQ$) por anillos $A$ (resp. cuerpos $K$) m\'as generales. Veremos un
ejemplo de esto m\'as tarde (\S\ref{sec1.6}).

Estudiaremos aqu\'i dos casos particulares de la famosa \emph{ecuaci\'on de Fermat:}
\begin{gather}\label{eq-1.2-1}
x^{n}+y^{n} = z^{n}.
\end{gather}
Fermat afirm\'o haber demostrado que, si $n\geq 3$, esta ecuaci\'on no tiene soluciones
$(x,y,z)$ con $x,y,z$ n\'umeros enteros no nulos; su demostraci\'on nunca se encontr\'o.
Numerosos matem\'aticos trabajaron intensamente desde entonces en este problema y mostraron
que la afirmaci\'on de Fermat es verdadera para un gran n\'umero de valores del exponente $n$.
Sin embargo, todav\'ia no se ha encontrado una demostraci\'on general (i.e. v\'alida para
todo $n$).

\begin{comm}
La opini\'on actual m\'as usual es que, en su ``demostraci\'on'', Fermat hab\'ia cometido
un error, pero un error digno de un matem\'atico de primer orden. Por ejemplo, tal vez
tuvo le idea (genial para su \'epoca) de trabajar en el anillo de enteros del cuerpo de
ra\'ices $n$-\'esimas de la unidad y crey\'o que este anillo era siempre un dominio de ideales
principales. Efectivamente, se sabe demostrar la afirmaci\'on de Fermat para todo exponente $n$
tal que este anillos es un dominio de ideales principales. Pero no lo es para todo $n$;
de hecho,
si $n$ es primo, este anillo s\'olo es un dominio de ideales principales para un n\'umero
finito de valores de $n$\footnote{Ver C.L. Siegel --- ``Gesamelte Weke'', t.~III, p.~436--442.}.
\end{comm}

Si $n=2$, la ecuaci\'on~\eqref{eq-1.2-1} tiene soluciones enteras, por
ejemplo $(3,4,5)$. Podemos describir completamente todas las soluciones:

\begin{theorem}\label{teo1.2.1}
Si $x$, $y$, $z$ son enteros $\geq 1$ tales que $x^{2}+y^{2}=z^{2}$, existe un entero
$d$ y enteros coprimos $u$, $v$ tales que (salvo una permutaci\'on de $x$ e $y$) se tiene:
\begin{gather}\label{eq-1.2-2}
x = d(u^{2}-v^{2})\quad y = 2duv\quad z = d(u^{2}+v^{2})
\end{gather}
\end{theorem}

Un c\'alculo f\'acil muestra que las f\'ormulas~\eqref{eq-1.2-2} dan soluciones
de $x^{2}+y^{2}=z^{2}$.
Rec\'iprocamente, sean $x$, $y$, $z$ enteros $\geq 1$ tales que $x^{2}+y^{2}=z^{2}$. Dividiendo
$x$, $y$, $z$ por su m.c.d. podemos suponer que son coprimos entre s\'i. En este caso tambi\'en
son coprimos dos a dos. En efecto, si, por ejemplo, $x$ y $z$ tiene un factor com\'un $p$, entonces
$p$ divide a $y^{2} = z^{2}-x^{2}$ y por lo tanto divide a $y$. En particular, dos de
los n\'umeros
$x$, $y$, $z$ son impares y el tercero es necesariamente par. Los n\'umeros $x$ e $y$ no
pueden ser ambos impares, pues sino se tendr\'ia $x^{2}\equiv 1\pmod 4$, $y^{2}\equiv 1\pmod 4$
de donde $z^{2}\equiv 2\pmod 4$, lo que contradice el hecho de que $z^{2}$ es un cuadrado.
Por lo tanto se tiene, eventualmente intercambiando $x$ e $y$, que
\begin{gather}\label{eq-1.2-3}
\text{$x$ es impar, $y$ es par, $z$ es impar.}
\end{gather}

Escribamos la ecuaci\'on de la siguiente forma
\begin{gather}\label{eq-1.2-4}
y^{2} = z^{2}-x^{2} = (z-x)(z+x).
\end{gather}
Como el m.c.d. de $2x$ y $2z$ es $2$, y $2x = (z+x)-(z-x)$ y $2z = (z+x)+(z-x)$, el
m.c.d. de $z-x$ y $z+x$ no puede ser otro que $2$. Sea $y = 2y'$, $z+x = 2x'$,
$z-x = 2z'$, donde $y'$, $x'$, $z'$ son enteros, pues $y$, $z+x$ y $z-x$ son pares
por~\eqref{eq-1.2-3}. Se tiene luego que $y'^{2} = x'z'$. Como $x'$ y $z'$ son coprimos, la descomposici\'on
en factores primos de $y'^{2}$ muestra que $x'$ y $z'$ son \emph{cuadrados} $u^{2}$ y $v^{2}$:
en efecto, todo factor primo de $y'^{2}$ aparece completamente, con su exponente par,
o bien en $x'$ o bien en $z'$. Se tiene por lo tanto que $z+x = 2u^{2}$, $z-x = 2v^{2}$,
$y^{2} = 2u^{2}\cdot 2v^{2}$, de donde $x = u^{2}-v^{2}$, $y=2uv$, $z=u^{2}+v^{2}$. Aqu\'i,
$u$ y $v$ son coprimos, pues sino $x$, $y$, $z$ tendr\'ian un factor primo en com\'un. Las
f\'ormulas~\eqref{eq-1.2-2} se deducen multiplicando de nuevo el m.c.d. por $d$.

\begin{theorem}\label{teo2.2.2}
La ecuaci\'on $x^{4}+y^{4}=z^{2}$ no tiene soluciones en n\'umeros enteros $x, y, z \geq 1$.
\end{theorem}

Razonemos por el absurdo. Se tiene entonces una soluci\'on $(x,y,z)$ donde $z$ es \emph{minimal.}
En este caso, $x$, $y$ y $z$ son coprimos dos a dos. En efecto, si, por ejemplo, $x$ e $y$ tuvieran
un factor primo en com\'un $p$, entonces $p^{4}$ dividir\'ia a $z^{2}$, por lo que $p^{2}$ dividir\'ia
a $z$, y $\left(\frac{x}{p},\frac{y}{p},\frac{z}{p^{2}}\right)$ ser\'ia otra soluci\'on, contradiciendo
la minimalidad de $z$. Los otros dos casos son an\'alogos e incluso m\'as f\'aciles.

Como nuestra
ecuaci\'on se puede escribir $(x^{2})^{2}+(y^{2})^{2}=z^{2}$, podemos aplicar el teorema~\ref{teo1.2.1}: despu\'es
de, eventualmente, permutar $x$ e $y$, existen enteros $u, v\geq 1$ coprimos tales que
\begin{gather}\label{eq-1.2-5}
x^{2} = u^{2}-v^{2},\quad y^{2} = 2uv,\quad z = u^{2}+v^{2}.
\end{gather}
Como $4\mid y^{2}$, la relaci\'on $y^{2} = 2uv$ muestra que uno de los n\'umeros $u$ y $v$ es
par; el otro es necesariamente impar. La condici\'on ``$u$ par, $v$ impar'' implica
$u^{2}\equiv 0\pmod 4$, $v^{2}\equiv 1\pmod 4$, de donde $x^{2}=u^{2}-v^{2}\equiv -1\pmod 4$,
lo que es absurdo. Luego, $u$ es impar y $v = 2v'$. La relaci\'on $y^{2} = 4uv'$ y el hecho
de que $u$ y $v'$ son coprimos muestran que $u$ y $v'$ son dos cuadrados $a^{2}$ y $b^{2}$.
Apliquemos de nuevo el teorema~\ref{teo1.2.1}, esta vez a la ecuaci\'on $x^{2}+v^{2} = u^{2}$ (cf.~\eqref{eq-1.2-5});
como $x$ y $u$ son impares, $v$ par, y $x$, $v$, $u$ coprimos dos a dos, existen enteros coprimos
$c,d\geq 1$ tales que
\begin{gather}\label{eq-1.2-6}
x = c^{2}+d^{2},\quad v = 2cd,\quad u=c^{2}=d^{2}.
\end{gather}
Luego, de $v = 2v' = 2b^{2}$, deducimos $cd = b^{2}$, de manera que $c$ y $d$ son nuevamente
cuadrados $x'^{2}$ e $y'^{2}$, pues son coprimos. Como $u=a^{2}$, la \'ultima ecuaci\'on de~\eqref{eq-1.2-6}
se escribe
\begin{gather}
a^{2} = x'^{4}+y'^{4}
\end{gather}
que tiene \emph{la misma forma} que la ecuaci\'on original. Por otra parte, se tiene, por~\eqref{eq-1.2-5},
$z = u^{2}+v^{2} = a^{4}+4b^{4} > a^{4}$, de donde $z > a$, lo que contradice el caracter minimal
de $z$. El teorema est\'a demostrado.

\begin{comm}
Una ligera variante de nuestra demostraci\'on muestra como, dada una soluci\'on $(x,y,z)$
en enteros $\geq 1$ de $x^{4}+y^{4}=z^{2}$, construir una sucesi\'on $(x_{n},y_{n},z_{n})$
infinita de tales soluciones, donde la sucesi\'on $z_{n}$ es estrictamente decreciente, lo
que es un absurdo. Este es el m\'etodo de \emph{descenso infinito} de Fermat.
\end{comm}

\begin{corollary*}
La ecuaci\'on $x^{4}+y^{4}=z^{4}$ no admite soluciones enteras $x,y,z\geq 1$.
\end{corollary*}

En efecto, esta ecuaci\'on puede escribirse $x^{4}+y^{4}=(z^{2})^{2}$ y podemos aplicar el teorema~\ref{teo2.2.2}.

\section{Algunos lemas sobre ideales; la funci\'on $\varphi$ de Euler}\label{sec1.3}

Sea $n\geq 1$ un entero natural. Llamamos \emph{funci\'on de Euler} de
$n$, y notamos $\varphi(n)$, al n\'umero de enteros $q$, coprimos con $n$, tales que
$0\leq q\leq n$ (equivalentemente, $1\leq q\leq n-1$, pues $0$ y $n$ no son
coprimos con $n$). Si $p$ es un n\'umero primo, es claro que
\begin{gather}
\varphi(p) = p-1.
\end{gather}

Si $n = p^{s}$, una potencia de un n\'umero primo, los enteros coprimos a $p^{s}$ son
aquellos que no son m\'ultiplos de $p$. Como hay $p^{s-1}$ m\'ultiples de $p$ entre $1$ y
$p^{s}$, se tiene
\begin{gather}\label{eq-1.3-2}
\varphi(p^{s}) = p^{s} - p^{s-1} = p^{s-1}(p-1).
\end{gather}

Nos proponemos ahora calcular $\varphi(n)$ utilizando la descomposici\'on de $n$
en factores primos. Para ello, nos hace falta una caracterizaci\'on de $\varphi(n)$ y
algunos lemas sobre los ideales que tambi\'en nos ser\'an \'utiles m\'as adelante.

\begin{proposition}\label{prop1.3.1}
Sea $n\geq 1$ un entero natural. El valor de la funci\'on de Euler $\varphi(n)$ es igual al n\'umero
de generadores de $\ZZ/n\ZZ$ y tambi\'en igual al n\'umero de elementos inversibles del
anillo $\ZZ/n\ZZ$.
\end{proposition}

Recordemos que cada clase de congruencia mod $n\ZZ$ contiene un \'unico entero $q$ tal que
$0\leq q\leq n-1$. Para un tal entero $q$, notemos $\oline q$ su clase mod $n\ZZ$.
Razonando ``en c\'irculo'', basta demostrar las implicaciones: $q$ coprimo con $n$ \then $\oline q$ inversible
\then $\oline q$ genera $\ZZ/n\ZZ$ \then $q$ coprimo con $n$.

Si $q$ es coprimo con $n$, la
identidad de Bezout (\S\ref{sec1.1},~\eqref{eq-1.1-2}) muestra que existen enteros $x$ e $y$ tales que
$qx+ny = 1$, de donde $\oline q\cdot\oline x = \oline 1$ y $\oline q$ es inversible.

Si $\oline q$ es inversible, notemos $x$ a un entero tal que $\oline q\cdot\oline x = 1$.
Si $\oline a$ es un elemento cualquier de $\ZZ/n\ZZ$ y si $a$ es un representante de $\oline a$,
tenemos $\oline a = \oline a\oline x\oline q$ (en el anillo $\ZZ/n\ZZ$), de donde
$\oline a = (ax)\cdot\oline q$ (en el grupo aditivo $\ZZ/n\ZZ$). Por lo tanto $\oline q$
genera el grupo $\ZZ/n\ZZ$.

Finalmente, si $\oline q$ genera $\ZZ/n\ZZ$, existe un entero
$x$ tal que $x\cdot\oline q = \oline 1$. Luego, tal que $xq\equiv 1\pmod n$. Es decir,
existe un entero $y$ tal que $xq-1=yn$, de donde $1 = xq-yn$. Esta es una identidad de Bezout
que muestra que $q$ es coprimo con $n$.

\begin{lemma}\label{lem1.3.1}
Sean $A$ un anillo, $\idl{a}$ y $\idl{b}$ dos ideal de $A$ tales que $\idl{a}+\idl{b}=A$.
Luego, $\idl{a}\cap\idl{b} = \idl{a}\idl{b}$ y el homomorfismo can\'onico
$\varphi:A\to A/\idl{a}\times A/\idl{b}$ define un isomorfismo $\theta:A/\idl{a}\idl{b}\to A/\idl{a}\times
A/\idl{b}$.
\end{lemma}

\begin{comm}
Recordemos que el homomorfismo $\varphi$ le hace corresponder a cada $x\in A$ el par formado
por la clase de $x\bmod\idl{a}$ y la clase de $y\bmod\idl{b}$.
\end{comm}

En general, se tiene que $\idl{a}\idl{b}\subset\idl{a}$ y $\idl{a}\idl{b}\subset\idl{b}$, de
donde $\idl{a}\idl{b}\subset\idl{a}\cap\idl{b}$. Sea ahora $x\in\idl{a}\cap\idl{b}$.
Como $\idl{a}+\idl{b}=A$, existen elementos $a\in\idl{a}$ y $b\in\idl{b}$ tales
que $a+b=1$. Luego, $x = ax+xb$ es suma de dos elementos de $\idl{a}\idl{b}$, de donde
$x\in\idl{a}\idl{b}$ y $\idl{a}\cap\idl{b}\subset\idl{a}\idl{b}$. Luego,
$\idl{a}\cap\idl{b} = \idl{a}\idl{b}$.

Est\'a claro que el n\'ucleo de $\varphi$ es $\idl{a}\cap\idl{b}$. Como $\idl{a}\cap\idl{b}
=\idl{a}\idl{b}$, $\varphi$ es constante en cada clase de equivalencia mod $\idl{a}\idl{b}$,
por lo que se tiene la funci\'on $\theta : A/\idl{a}\idl{b}\to A/\idl{a}\times A/\idl{b}$.
Esta funci\'on es evidentemente un homomorfismo. Como $\varphi^{-1}(0) = \idl{a}\idl{b}$,
tenemos que $\theta^{-1}(0) = (0)$ y luego $\theta$ es inyectiva. Falta mostrar que $\theta$
es sobreyectiva.

\begin{comm}
Hemos explicado con detalle este argumento de ``pasaje al cociente'' a modo de ilustraci\'on.
En lo que sigue, seremos bastante m\'as breves al realizar razonamientos an\'alogos.
\end{comm}

Para mostrar la sobreyectividad de $\theta$ (o, lo que es lo mismo, de $\varphi$) debemos construir
un elemento $x$ de $A$ tal que su clase m\'odulo $\idl{a}$ y su clase m\'odulo $\idl{b}$ puedan
elegirse arbitrariamente. Sean $y$ y $z$ dos representantes de tales clases. Existen elementos
$x\in\idl{a}$ y $b\in\idl{b}$ tales que $a+b=1$. Definimos $x = az+by$. Modulo $\idl{a}$,
se tiene $x\equiv by\equiv (1-a)y\equiv y-ay\equiv y$; intercambiando $x$ e $y$, deducimos
que $x\equiv z\bmod\idl{b}$. \QED

\begin{lemma}\label{lem1.3.2}
Sean $A$ un anillo y $(\idl{a}_{i})_{1\leq i\leq r}$ una colecci\'on finita de ideales de $A$
tal que $\idl{a}_{i}+\idl{a}_{j} = A$ si $i\neq j$. Se tiene entonces un isomorfismo can\'onico
de $A/\idl{a}_{1}\cdots\idl{a}_{r}$ sobre $\prod_{i=1}^{r}A/\idl{a}_{i}$.
\end{lemma}

El lema~\ref{lem1.3.1} es el caso $r=2$ del lema~\ref{lem1.3.2}. Procedemos por induci\'on en $r$ a partir de este
caso. Sea $\idl{b} = \idl{a}_{2}\cdots\idl{a}_{r}$ y mostremos que $\idl{a}_{1}+\idl{b}=A$.
En efecto, si $i\geq 2$, tenemos $\idl{a}_{1}+\idl{a}_{i} = A$ por lo que existen
elementos $c_{i}\in\idl{a}_{1}$ y $a_{i}\in\idl{a}_{i}$ tales que $c_{i}+a_{i} = 1$. Multiplicando
miembro a miembro, se tiene que $c+a_{2}\dots a_{n} = 1$, donde $c$ es una suma de t\'erminos
donde cada t\'ermino contiene al menos un $c_{i}$ como factor. Luego, $c\in\idl{a}_{1}$. Como
$a_{2}\cdots a_{r}\in\idl{b}$, obtenemos que $\idl{a}_{1}+\idl{b} =A$.

Por el lema~\ref{lem1.3.1}, se tiene un isomorfismo
$A/\idl{a}_{1}\idl{b}\sim A/\idl{a}_{1}\times A/\idl{b}$.
Por la hip\'otesis inductiva, se tiene un isomorfismo
\begin{gather*}
A/\idl{b} = A/\idl{a}_{2}\cdots\idl{a}_{r}\sim (A/\idl{a}_{2})\times\dots\times(A/\idl{a}_{r}).
\end{gather*}
Componiendo estos isomorfismos se obtiene el resultado. \QED

Ahora aplicamos estos resultados al anillo $\ZZ$:

\begin{proposition}
Sean $n$ y $n'$ dos enteros coprimos. Entonces el anillo $\ZZ/nn'\ZZ$ es isomorfo al anillo
producto $\ZZ/n\ZZ\times\ZZ/n'\ZZ$.
\end{proposition}

Este es un caso particular del lema~\ref{lem1.3.2}, ya que la hip\'otesis $n\ZZ+n'\ZZ=\ZZ$
no es otra cosa que la identidad de Bezout.

\begin{corollary}\label{coro1.3.1}
Si $n$ y $n'$ son dos enteros $\geq 1$ coprimos, se tiene que
$\varphi(nn') = \varphi(n)\varphi(n')$.
\end{corollary}

En efecto, $\varphi(nn')$ es el n\'umero de elementos inversibles del anillo $\ZZ/nn'\ZZ$
(proposici\'on~\ref{prop1.3.1}), que es isomorfo a $\ZZ/n\ZZ\times\ZZ/n'\ZZ$. Ahora
bien, un elemento
$(\alpha,\beta)$ de un anillo producto es inversible si y s\'olo si cada uno de sus componentes
$\alpha$ y $\beta$ es inversible. Aplicando la proposici\'on~\ref{prop1.3.1} se
obtiene el resultado.

\begin{corollary}
Sea $n$ un entero $\geq 1$ y $n = p_{1}^{\alpha_{1}}\dots p_{r}^{\alpha_{r}}$ su
descomposici\'on
en factores primos. Entonces
$\varphi(n) = n\left(1-\frac{1}{p_{1}}\right)\dots\left(1-\frac{1}{p_{r}}\right)$.
\end{corollary}

Por el corolario~\ref{coro1.3.1}, se tiene
$\varphi(n) = \varphi(p_{1}^{\alpha_{1}})\dots\varphi(p_{r}^{\alpha_{r}})$.
Por~\eqref{eq-1.3-2}, se tiene $\varphi(p_{i}^{\alpha_{i}}) = p_{i}^{\alpha_{i}-1}(p_{i}-1)
=p_{i}^{\alpha_{i}}\left(1-\frac{1}{p_{i}}\right)$. Multiplicando obtenemos
la f\'ormula deseada.

\section{Algunos preliminares sobre m\'odulos}\label{sec1.4}

Para poder estudiar los m\'odulo sobre un dominio de ideales principales, nos har\'an
falta algunos preliminares.

Dado un anillo $A$ y un conjunto $I$ notamos $A^{(I)}$ el conjunto de las
familias $(a_{i})_{i\in I}$,
indexadas por $I$, de elementos de $A$ \emph{tales que} $a_{i} = 0$ salvo un
n\'umero \emph{finito} de \'indices
$i\in I$. De manera que $A^{(I)}$ es un subconjunto del conjunto producto $A^{I}$,
y un subm\'odulo de $A^{I}$ si
dotamos a $A^{I}$ de la estructura de $A$-m\'odulo inducida por sus factores.

Si $I$ is finito, se tiene $A^{(I)} = A^{I}$.

Para $j\in I$, la familia $(\delta_{ji})_{i\in I}$ definida por
$\delta_{jj} = 1$ y $\delta_{ji} = 0$ si
$i\neq j$, es un elemento $e_{j}$ de $A^{(I)}$. Todo elemento $(a_{j})_{j\in I}$
de $A^{(I)}$ se escribe
de una forma \'unica como combinaci\'on lineal (finita) de los $e_{j}$. M\'as precisamente:
\begin{gather}
(a_{j})_{j\in I} = \sum_{j\in I}a_{j}e_{j}
\end{gather}
(observemos que, en la suma de la derecha, casi todos los t\'erminos son cero salvo un
n\'umero finito, de manera
que la suma tiene sentido). Decimos que $(e_{j})_{j\in I}$ es la
\emph{base can\'onica} de $A^{(I)}$.

Sea $A$ un anillo, $M$ un $A$-m\'odulo y $(x_{i})_{i\in I}$ una familia de elementos
de $M$. A todo elemento
$(a_{i})_{i\in I}$ de $A^{(I)}$ le hacemos corresponder el elemento $\sum_{i}a_{i}x_{i}$
de $M$ (como antes, la
suma tiene sentido). De esta manera hemos definido una aplicaci\'on
$\varphi:A^{(I)}\to M$ que es evidentemente
\emph{lineal.} Si $(e_{i})_{i\in I}$ es la base can\'onica de $A^{(I)}$, se
tiene $\varphi(e_{i}) = x_{i}$ para todo
$i\in I$.

Las equivalencias siguientes son inmediatas:
\begin{align}
\text{los $x_{i}$ son linealmente equivalentes} &\iff \text{$\varphi$ es inyectiva.}\\
\text{$(x_{i})_{i\in I}$ son un sistema de generadores} &\iff \text{$\varphi$ es sobreyectiva.}
\end{align}

Si $\varphi$ es \emph{biyectiva}, decimos que $(x_{i})_{i\in I}$ es una \emph{base}
de $M$. Esto quiere decir que
todo elemento $x$ de $M$ se escribe, \emph{de manera \'unica,} como combinaci\'on lineal
de los $x_{i}$. Un m\'odulo
$M$ que admite una base se llama un \emph{m\'odulo libre.}

\begin{comm}
Contrariamente a lo que ocurre con los espacios vectoriales sobre un cuerpo, un
m\'odulo sobre un
anillo no admite necesariamente una base. Por ejemplo, el $\ZZ$-m\'odulo $\ZZ/n\ZZ$
con $n\neq 0,1$
no es libre.
En lo que sigue demostraremos que ciertos m\'odulos son libres; un resultado de este
tipo es raramente trivial.
\end{comm}

Un m\'odulo se dir\'a \emph{de tipo finito} si admite un sistema finito de
generadores. El siguiente teorema
es b\'asico para el estudio de los anillos y m\'odulos noetherianos, que estudiaremos
m\'as a fondo en el cap\'itulo~\ref{cap3}.

\begin{theorem}\label{teo1.4.1}
Sea $A$ un anillo, $M$ un $A$-m\'odulo. Las siguientes condiciones son equivalentes:
\begin{enumerate}
\item Toda familia no vac\'ia de subm\'odulos de $M$ posee un elemento maximal
(para la relaci\'on de inclusi\'on);
\item Toda sucesi\'on creciente $(M_{n})_{n\geq 0}$ (para la relaci\'on de inclusi\'on)
de subm\'odulos
de $M$ se estaciona (es decir, existe $n_{0}$ tal que $M_{n} = M_{n_{0}}$ para
todo $n\geq n_{0}$);
\item Todo subm\'odulo de $M$ es de tipo finito.
\end{enumerate}
\end{theorem}

Mostremos que a) implica c). Sea $E$ un subm\'odulo de $M$ y
sea $\Phi$ la familia de subm\'odulos de tipo finito de $E$.
$\Phi$ no es vac\'ia pues $\{0\}\in\Phi$. Por a), $\Phi$ admite
un elemento maximal $F$. Si $x\in E$, $F+Ax$ es un subm\'odulo
de tipo finito de $E$ (generado por la uni\'on de $\{x\}$ y
un sistema finito de generadores de $F$). Luego tenemos
$F+Ax = F$ pues $F+Ax\supset F$ y $F$ es maximal. Por lo tanto,
$x\in F$, $E\subset F$, $E = F$ y $E$ es de tipo finito.

Probemos ahora que c) implica b). Sea $(M_{n})_{n\geq 0}$
una sucesi\'on creciente de subm\'odulos de $M$. Luego
$E = \bigcup_{n\geq 0}M_{n}$ es un subm\'odulo de $M$.
Por c), $E$ admite un sistema finito de generadores
$(x_{1},\dots,x_{q})$. Para todo $i$, hay un \'indice
$n(i)$ tal que $x_{i}\in M_{n(i)}$. Sea $n_{0}$ el m\'as
grande de los $n(i)$. Tenemos $x_{i}\in M_{n_{0}}$ para todo
$i$, de donde $E\subset E_{n_{0}}$ y $E = M_{n_{0}}$. Para
$n\geq n_{0}$, las inclusiones $M_{n_{0}}\subset M_{n}
\subset E$ y la igualdad $M_{n_{0}} = E$ muestran que
$M_{n_{0}} = M_{n}$. Luego la sucesi\'on $(M_{n})$ se
estaciona a partir de $n_{0}$.

Falta demostrar que b) implica a). La equivalencia
de a) y b) es un caso particular del siguiente lema sobre
los conjuntos ordenados:

\begin{lemma}
Sea $T$ un conjunto ordenado. Las siguientes condiciones son
equivalentes:
\begin{enumerate}
\item[a)] Toda familia no vac\'ia de elementos de $T$ admite un
elemento maximal;
\item[b)] Toda sucesi\'on creciente $(t_{n})_{n\geq 0}$ de
elementos de $T$ se estaciona.
\end{enumerate}
\end{lemma}

{\itshape a}) \then{ \itshape b}): Sea $t_{q}$ un elemento maximal de la sucesi\'on
creciente $(t_{n})$. Si $n\geq q$ tenemos $t_{n}\geq t_{q}$
(la sucesi\'on es creciente), luego $t_{n} = t_{q}$ (el elemento
$t_{q}$ es maximal).

{\itshape b}) \then{ \itshape a}). Supongamos que $S$ es un subconjunto no vac\'io
de $T$ sin ning\'un elemento maximal. Luego, si $x\in S$, el conjunto
de elementos de $S$ estrictamente superiores a $x$ es no
vac\'io. Por el axioma de elecci\'on, existe una aplicaci\'on
$f:S\to S$ tal que $f(x) > x$ para todo $x\in S$. Como $S$ es
no vac\'io, podemos elegir $t_{0}\in S$ y definir por recurrencia
la sucesi\'on $(t_{n})_{n\geq 0}$ por $t_{n+1}=f(t_{n})$.
Esta sucesi\'on es estrictamente creciente, luego no se estaciona.
De esta manera, hemos demostrado la implicaci\'on b) \then a)
por el contrapositivo. \QED

\begin{named}{Corolario del teorema 1}
En un dominio de ideales principales $A$ toda familia no vac\'ia
de ideales admite un elemento maximal.
\end{named}

En efecto, si consideramos a $A$ como un m\'odulo sobre si
mismo, sus subm\'odulos son precisamente sus ideales. Como estos
son todos principales, son $A$-m\'odulos generados por un
s\'olo elemento, luego son de tipo finito. Podemos aplicar
entonces la implicaci\'on c) \then a) del teorema~\ref{teo1.4.1}.

\section{M\'odulos sobre dominios de ideales principales}\label{sec1.5}

Sea $A$ un dominio \'integro y $K$ su cuerpo de fracciones. Un $A$-m\'odulo
libre, en particular isomorfo a un $A^{(I)}$, puede incluirse en
un espacio vectorial sobre $K$ ($K^{(I)}$ en el caso de $A^{(I)}$).
Por lo tanto lo mismo es cierto para todo subm\'odulo $M$ de un
$A$-m\'odulo libre. La dimensi\'on del subespacio generado por
$M$ se llama el \emph{rango} de $M$. Es el n\'umero m\'aximo de
elementos linealmente independientes de $M$. Si el propio $M$ es libre
y admite una base con $n$ elementos, entonces el rango de $M$ es
igual a $n$.

\begin{theorem}\label{teo1.5.1}
Sea $A$ un dominio de ideales principales, $M$ un $A$-m\'odulo
libre de rango finito $n$ y $M'$ un subm\'odulo de $M$. Entonces:
\begin{enumerate}
\item $M'$ es libre, de rango $\leq n$;
\item Existe una base $(e_{1},\dots,e_{n})$ de $M$, un entero
$q\leq n$ y elementos no nulos $a_{1},\dots,a_{q}$ de $A$ tales que
$(a_{1}e_{1},\dots, a_{q}e_{q})$ es una base de $M'$ y tal que
$a_{i}$ divide a $a_{i+1}$ para todo $1\leq i\leq q-1$.
\end{enumerate}
\end{theorem}

Como el teorema es trivial si $M' = \{0\}$, podemos suponer que $M' \neq \{0\}$.
Sea $L(M,A)$ el conjunto de formas lineales de $M$. Si $u\in L(M,A)$, $u(M')$
es un sub-$A$-m\'odulo de $A$, es decir un ideal de $A$. Podemos escribir $u(M') = Aa_{u}$
para alg\'un $a_{u}\in A$ pues todos los ideales son principales. Sea $u\in L(M,A)$
tal que $Aa_{u}$ es \emph{maximal} entre los $Aa_{v}$ ($v\in L(M,A)$) (\S\ref{sec1.4},
corolario del teorema~\ref{teo1.4.1}).
Tomemos una base $(x_{1},\dots,x_{n})$ de $M$ de manera de identificar
$M$ con $A^{n}$. Sea $\pr_{i}:M\to A$ la $i$-\'esima proyecci\'on coordenada, definida
por $\pr_{i}(x_{j}) = \delta_{ij}$. Como $M'\neq\{0\}$, alguno de los $\pr_{i}(M')$ es
$\neq\{0\}$. Por lo tanto $a_{u}\neq 0$. Por construcci\'on, existe $e'\in M'$ tal que
$u(e') = a_{u}$. Mostremos que, \emph{para todo} $v\in L(M,A)$, $a_{u}$ \emph{divide a} $v(e')$.
En efecto, si $d$ es el m.c.d. de $a_{u}$ y $v(e')$, se tiene $d = ba_{u}+cv(e')$ para
algunos $b,c\in A$, de donde $d = (bu+cv)(e')$. Como $bu+cv$ es una forma lineal $w$ en $M$,
se sigue que $Aa_{u}\subset Ad\subset w(M')$. La maximalidad de $Aa_{u}$ implica que
$Ad = Aa_{u}$ de manera que $a_{u}$ divide a $v(e')$.

En particular, $a_{u}$ divide a todos los $\pr_{i}(e')$ y podemos escribir $\pr(e') = a_{u}b_{i}$
para alg\'un
$b_{i}\in A$. Sea $c = \sum_{i=1}^{n}b_{i}x_{i}$. Se tiene $e' = a_{u}e$. Como
$u(e') = a_{u} = a_{u}\cdot u(e)$, se sigue que $u(e) = 1$ (recordar que $a_{u}\neq 0$).
{\itshape Mostremos ahora que
\begin{align}
M &= Ae+\Ker(u)\\
M' &= Ae'+(M'\cap \Ker(u))\quad(\text{donde $e' = a_{u}e$})
\end{align}
donde las sumas son directas.} En efecto, todo $x\in M$ se escribe $x = u(x)e+(x-u(x)e)$
y se tiene $u(x-u(x)e) = u(x) - u(x)u(e) = 0$, lo que demuestra (1). Si $y\in M'$, se
tiene $u(y) = ba_{u}$ para alg\'un $b\in A$, y por lo tanto
\begin{gather*}
y = ba_{u}e + (y-u(y)e) = be'+(y-u(y)e).
\end{gather*}
Adem\'as, $y-u(y)e\in\Ker(u)$ y tambi\'en $y-u(y)e = y-be'\in M'$, lo que demuestra (2).
Finalmente, para mostrar que las sumas son directas, basta ver que $Ae\cap\Ker(u) = \{0\}$.
Pero, si $x = ce$ es un elemento de $Ae$ ($c\in A$) y si adem\'as $u(x) = 0$, se tiene $c = cu(e) = u(ce)
= u(x) = 0$, de donde $x = 0$.

Ahora demostraremos a) haciendo inducci\'on en el rango $q$ de $M'$. Si $q = 0$, se tiene
$M' = \{0\}$ y todo es trivial. Si $q > 0$, $M'\cap\Ker(u)$ es de rango $q-1$ por (2), y por lo
tanto es libre por la hip\'otesis inductiva. Como en (2) la suma es directa, obtenemos una
base de $M'$ agregando $e'$ a una base de $M'\cap\Ker(u)$. Por lo tanto $M'$ es libre y vale a).

A continuaci\'on, demostraremos b) haciendo inducci\'on en el rango $n$ de $M$. Todo es trivial
si $n = 0$. Por a), $\Ker(u)$ es libre y de rango $n-1$ pues, en (1), la suma es directa. Apliquemos
la hip\'otesis inductiva al m\'odulo libre $\Ker(u)$ y a su subm\'odulo $M'\cap\Ker(u)$:
existe $q\leq n$, una base $(e_{2},\dots,e_{n})$ de $\Ker(u)$ y elementos no nulos
$a_{2},\dots,a_{q}$ de $A$ tales que $(a_{2}e_{2},\dots,a_{q}e_{q})$ es una base de
$M'\cap\Ker(u)$ y $a_{i}$ divide a $a_{i+1}$ para $2\leq i\leq q-1$. Utilizando la notaci\'on
anterior, ponemos $a_{1} = a_{u}$ y $e_{1} = e$. Luego $(e_{1},e_{2},\dots,e_{n})$ es una base
de $M$ por (1) y $(a_{1}e_{1},\dots,a_{q}e_{q})$ es una base de $M'$ (por (2) y porque
$e' = a_{1}e_{1}$). S\'olo resta mostrar la divisibilidad $a_{1}\mid a_{2}$. Sea $v$ la
forma lineal en $M$ definida por $v(e_{1}) = v(e_{2}) = 1$, $v(e_{i}) = 0$ para $i\geq 3$.
Se tiene $a_{1} = a_{u} = v(a_{u}e_{1}) = v(e')\in v(M')$, de donde $Aa_{u}\subset v(M')$.
Por la maximalidad de $Aa_{u}$ se deduce que $v(M') = Aa_{u} = Aa_{1}$; como $a_{2} = v(a_{2}e_{2})
\in v(M')$, se tiene que $a_{2}\in Aa_{1}$, es decir, $a_{1}\mid a_{2}$. \QED

\begin{comm}
Los ideales $Aa_{i}$ del Teorema~\ref{teo1.5.1} se llaman \emph{factores
invariantes} de $M'$ en $M$. Puede demostrarse que est\'an un\'ivocamente determinados
por $M$ y $M'$ (\cite{Bourbaki1}, Cap\'itulo VII, \S3).
\end{comm}

\begin{corollary}\label{cor1.5.1}
Sea $A$ un dominio de ideales principales y $E$ un $A$-m\'odulo
de tipo finito. Entonces $E$ es isomorfo a un producto
$(A/\idl{a}_{1})\times(A/\idl{a}_{2})\times\dots\times
(A/\idl{a}_{n})$, donde los $\idl{a}_{i}$ son ideales de $A$
tales que $\idl{a}_{1}\supset\idl{a}_{2}\supset\dots\supset
\idl{a}_{n}$.
\end{corollary}

Sea, en efecto, $(x_{1},\dots,x_{n})$ un sistema de generadores de $E$. Por lo visto
al comienzo de \S\ref{sec1.4}, se tiene un homomorfismo sobreyectivo $\varphi:A^{n}\to E$, de manera
que $E$ es isomorfo a $A^{n}/\Ker(\varphi)$. Por el teorema~\ref{teo1.5.1},
existe una base $(e_{1},\dots,e_{n})$
de $A^{n}$, un entero $q\leq n$, y elementos no nulos $a_{1},\dots,a_{q}$ de $A$ tales que
$(a_{1}e_{1},\dots,a_{q}e_{q})$ es una base de $\Ker(\varphi)$ y $a_{i}$ divide a $a_{i+1}$ para
todo $1\leq i\leq q-1$. Definimos $a_{p} = 0$ si $q+1\leq p\leq n$. Luego $A^{n}/\Ker(\varphi)$
es isomorfo al producto de los $Ae_{i}/Aa_{i}e_{i}$ ($1\leq i\leq n$) y $Ae_{i}/Aa_{i}e_{i}$
es isomorfo a $A/Aa_{i}$. Poniendo $\idl{a}_{i} = Aa_{i}$, se obtiene el resultado. \QED

Diremos que un m\'odulo $E$ sobre un dominio \'integro $A$ es
\emph{libre de torsi\'on} si la relaci\'on $ax = 0$
($a\in A$, $x\in E$) implica $a = 0$ o $x = 0$.

\begin{corollary}
Todo m\'odulo $E$ libre de torsion y de tipo finito
sobre un dominio de ideales principales es libre.
\end{corollary}

Aplicamos el corolario~\ref{cor1.5.1}: $E \sim (A/\idl{a}_{1})\times\dots
\times (A/\idl{a}_{n})$. Suprimiendo los factores nulos, podemos
suponer que $\idl{a}_{i}\neq A$ para todo $i$. Si $\idl{a}_{1}\neq(0)$,
$a$ es un elemento no nulo de $\idl{a}_{1}$,
$x_{1}$ es un elemento no nulo de $A/\idl{a}_{1}$ y
$x = (x_{1},0,\dots,0)$, se sigue que $ax = 0$, lo que contradice
el hecho de que $E$ es libre de torsi\'on. Luego,
$\idl{a}_{1} = (0)$, $\idl{a}_{i} = (0)$ para todo $i$ (pues
$\idl{a}_{i}\subset\idl{a}_{1}$) y $E$ es isomorfo a $A^{n}$.

\begin{comm}
La hip\'otesis de que $E$ es de tipo finito es imprescindible:
por ejemplo, $\QQ$ es un $\ZZ$-m\'odulo libre de torsion que no
es libre.
\end{comm}

\begin{corollary}
Sobre un dominio de ideales principales, todo m\'odulo $E$
de tipo finito es isomorfo a un producto finito de m\'odulos
$M_{i}$, donde cada $M_{i}$ es igual a $A$ o a un cociente
$A/Ap^{s}$ con $p$ primo.
\end{corollary}

Utilizamos el corolario~\ref{cor1.5.1}, y descomponemos cada factor
$A/Aa$ con $a\neq 0$ mediante \S\ref{sec1.3}, lema~\ref{lem1.3.2}: si
$a = up_{1}^{s_{1}}\dots p_{r}^{s_{r}}$ es la
descomposici\'on en factores primos de $a$, $A/Aa$ es
isomorfo al producto de los $A/Ap_{i}^{s_{i}}$.

\begin{corollary}\label{cor1.5.4}
Sea $G$ un grupo conmutative finito. Existe $x\in G$ tal que
su orden el el m.c.m. de los \'ordenes de los elementos de $G$.
\end{corollary}

Un grupo conmutativo es un $\ZZ$-m\'odulo (si lo escribimos
aditivamente). Por el corolario~\ref{cor1.5.1}, tenemos $G \simeq
\ZZ/a_{1}\ZZ\times\dots\times\ZZ/a_{n}\ZZ$ con
$a_{1}\mid a_{2}\mid\dots\mid a_{n}$. Ninguno de los $a_{i}$
es nulo, pues sino $G$ ser\'ia infinito. Notemos $y$ la
clase de $1$ en $\ZZ/a_{n}\ZZ$ y pongamos $x = (0,\dots,0,y)$.
El orden de $x$ es evidentemente $a_{n}$. Si $z = (z_{1},\dots,z_{n})\in G$,
tenemos $a_{n}z = 0$ pues $a_{i}$ divide $a_{n}$ para todo $i$.
Luego, $a_{n}$ es un m\'ultiplo del orden de $z$. Por lo tanto, el
elemento buscado es $x$.

\section{Ra\'ices de la unidad en un cuerpo}\label{sec1.6}

\begin{theorem}\label{teo1.6.1}
Sea $K$ un cuerpo. Todo subgrupo finito $G$ del grupo
multiplicativo $K^{\times}$ consiste de ra\'ices de la unidad y
es c\'iclico.
\end{theorem}

En efecto, por el corolario~\ref{cor1.5.4} al teorema~\ref{teo1.5.1}
de la \S\ref{sec1.5}, existe $z\in G$ cuyo orden $n$ es tal que $y^{n}=1$
para todo $y\in G$. Como un polinomio de grado $n$ sobre un cuerpo
(por ejemplo $X^{n}-1$) tiene a lo sumo $n$ ra\'ices en el
cuerpo, el n\'umero de elementos de $G$ es a lo sumo $n$.
Pero, como $z$ es de orden $n$, $G$ contiene los $n$ elementos
$z, z^{2},\dots,z^{n}=1$, que son todos distintos. Luego, $G$
consiste de estos elementos y es c\'iclico.% \QED

Si un cuerpo $K$ contiene $n$ ra\'ices $n$-\'esimas de la unidad,
\'estas forman un grupo c\'iclico de orden $n$ (isomorfo a
$\ZZ/n\ZZ$). Un generador de este grupo se llama una
\emph{raiz primitiva $n$-\'esima de la unidad;} toda raiz $n$-\'esima
de la unidad es por lo tanto una potencia de una tal raiz primitiva.
De la proposici\'on~\ref{prop1.3.1} de la \S\ref{sec1.3}, el n\'umero de estas ra\'ices
es $\varphi(n)$.

\section{Cuerpos finitos}\label{sec1.7}

Sea $K$ un cuerpo. Existe un \'unico homomorfismo de anillos
$\varphi:\ZZ\to K$ (definido por $\varphi(n)=1+1+\dots+1$,
$n$ veces, si $n\geq 0$ y $\varphi(-n) = -\varphi(n)$).

\begin{itemize}
\item Si $\varphi$ es inyectiva, $\ZZ$ se identifica con un
subanillo de $K$. Luego $K$ contiene el cuerpo de fracciones
$\QQ$ de $\ZZ$; decimos que $K$ es \emph{de caracter\'istica
$0$.}
\item Si $\varphi$ no es inyectiva, su n\'ucleo es un ideal
$p\ZZ$ con $p > 0$. Luego $\ZZ/p\ZZ$ se identifica con un
subanillo de $K$, necesariamente un dominio \'integro, por
lo que $p$ es un \emph{n\'umero primo.} Decimos que $K$ es
\emph{de caracter\'istica $p$.} En este caso $\ZZ/p\ZZ$ es un
cuerpo, que notamos $\FF_{p}$.
\end{itemize}

\begin{comm}
El subcuerpo, $\QQ$ o $\FF_{p}$, es el subcuerpo m\'as
peque\~no de $K$. Lo llamamos el \emph{subcuerpo primo} de $K$.

Para todo n\'umero primo $p$ existen cuerpos de caract\'eristica $p$,
por ejemplo $\FF_{p} = \ZZ/p\ZZ$.
\end{comm}

\begin{proposition}\label{prop1.7.1}
Si $K$ es un cuerpo de caracter\'istica $p\neq 0$, se tiene
$px = 0$ para todo $x\in K$ y $(x+y)^{p} = x^{p}+y^{p}$ para
todo $x, y\in K$.
\end{proposition}

Si $x\in K$, se tiene $p\cdot x = (p\cdot 1)\cdot x = 0\cdot x = 0$. Por otra parte,
de la f\'ormula del binomio, se tiene $(x+y)^{p} = x^{p}+y^{p}+\sum_{j=1}^{p-1}\binom{p}{j}x^{j}y^{p-j}$;
el coeficiente binomial $\binom{p}{j}$ es el entero $\frac{p!}{j!(p-j)!}$. Como el n\'umero primo
$p$ aparece como factor en el numerador y no aparece en el denominador, $\binom{p}{j}$ es m\'ultiplo
de $p$ para $1\leq j\leq p-1$. Luego el t\'ermino correspondiente es nulo.

Por inducci\'on en $n$ se tiene $(x+y)^{p^{n}}=x^{p^{n}}+y^{p^{n}}$ para todo $n\geq 0$.

\begin{theorem}\label{teo1.7.1}
Sea $K$ un cuerpo finito. Sea $q = \card(K)$. Entonces:
\begin{enumerate}%[a\upshape{)}]
\item[a)] La caracter\'istica de $K$ es un n\'umero primo $p$, $K$ es un espacio vectorial de
dimensi\'on finita $s$ sobre $\FF_{p}$ y se tiene $q = p^{s}$.
\item[b)] El grupo multiplicativo $K^{*}$ es c\'iclico de orden $q-1$.
\item[c)] Se tiene $x^{q-1} = 1$ para todo $x\in K^{*}$ y $x^{q} = x$ para todo $x\in K$.
\end{enumerate}
\end{theorem}

Efectivamente, como $\ZZ$ es infinito, $K$ no puede ser de caracter\'istica $0$. Por lo tanto,
$K$ contiene $\FF_{p}$ para alg\'un primo $p$. Por lo tanto $K$ es un espacio vectorial sobre
$\FF_{p}$. Su dimensi\'on $s$ es finita pues sino $K$ ser\'ia infinito. Como espacio vectorial,
$K$ es isomorfo a $(\FF_{p})^{s}$, luego tiene $p^{s}$ elementos. La parte b) se sigue del
teorema~\ref{teo1.6.1} de la \S\ref{sec1.6}. La parte c) es inmediata.

\begin{example*}
Apliquemos b) a $\FF_{p}$, donde $p$ es primo: existe un entero $x\in\ZZ$ tal que $0\leq x\leq p-1$ y tal
que todo entero $y$ que no es m\'ultiplo de $p$ es congruente a una potencia de $x$ m\'odulo $p$. Decimos
que $x$ es \emph{una ra\'iz primitiva m\'odulo $p$.} La determinaci\'on de ra\'ices primitivas m\'odulo
$p$ no es un asunto completamente trivial. Por ejemplo, hay $\varphi(6) = 2$ ra\'ices primitivas m\'odulo $7$:
el $3$ y el $5$ (en efecto, se tiene $1^{2}\equiv 6^{2}\equiv 1\pmod 7$ y $2^{3}\equiv 4^{3}\equiv 1\pmod 7$
y las \'unicas posibilidades restantes son el $3$ y el $5$).
\end{example*}

\begin{remark*}
Se sigue de c) que un cuerpo finito $K$ de $q$ elementos es el conjunto de ra\'ices del polinomio $X^{q}-X$
(que tiene exactamente $q$ ra\'ices). Se puede demostrar que dos cuerpo finitos de $q$ elementos son isomorfos.
A menudo se nota $\FF_{q}$ el cuerpo finito de $q$ elementos.
\end{remark*}

A manera de ejercicio y de intermedio, ahora demostraremos un elegante teorema concerniente a las
ecuaciones diof\'anticas sobre un cuerpo finito:

\begin{theorem}[Chevalley]\label{teo1.7.2}
Sea $K$ un cuerpo finito y $F(X_{1},\dots,X_{n})$ un polinomio homog\'eneo de grado $d$ sobre $K$.
Si $d < n$, existe un punto $(x_{1},\dots,x_{n})\in K^{n}$ distinto al origen $(0,\dots,0)$ tal que
$F(x_{1},\dots,x_{n}) = 0$.
\end{theorem}

\begin{comm}
Dado un cuerpo $K$ y un entero $j$, se dice que $K$ es un \emph{cuerpo} $C_{j}$ si todo polinomio
homog\'eneo sobre $K$ de grado $d$ en $n$ variables \emph{tal que} $n > d^{j}$, admite un cero no
trivial (i.e. distinto al origen) en $K^{n}$. Los cuerpos $C_{0}$ son exactamente los cuerpos algebraicamente
cerrados. El teorema de Chevalley dice que los cuerpos finitos son $C_{1}$ (tambi\'en llamados
cuerpos ``quasi-algebraicamente cerrados''). Se puede demostrar que, si $K$ es un cuerpo $C_{j}$, el
cuerpo $K(T)$ de funciones racionales en una variable sobre $K$ y el cuerpo $K((T))$ de series formales
en una variable sobre $K$ son cuerpos $C_{j+1}$ (\cite{Lang}). Por mucho tiempo se ignoraba si los
cuerpos $p$-\'adicos eran $C_{2}$, pero recientemente se demostr\'o que esto no es as\'i (\cite{Terjanian}).
\end{comm}

Demostremos el teorema~\ref{teo1.7.2}. Sea $q$ el cardinal de $K$ y $p$ su caracter\'istica (de forma que
$q = p^{s}$). Sea $V\subset K^{n}$ el conjunto de ceros de $F$, i.e. los puntos $(x_{1},\dots,x_{n})\in K^{n}$
tales que $F(x) = 0$ (de ahora en m\'as, utilizamos la notaci\'on vectorial en la cual $x$ denota
un punto $(x_{1},\dots,x_{n})$ de $K^{n}$). Por el teorema~\ref{teo1.7.1}, c),
se tiene $F(x)^{q-1} = 0$ si $x\in V$,
y $F(x)^{q-1} = 1$ si $x\in K^{n}-V$. Es decir, el polinomio $G(x) = F(x)^{q-1}$ es la
\emph{funci\'on caracter\'istica} de $K^{n}-V$, con valores in $\FF_{p}$. El n\'umero, m\'odulo $p$,
de puntos de $K^{n}-V$ puede escribirse como la suma $\sum_{x\in K^{n}}G(x)$. Nosotros calcularemos esta
suma y mostraremos que es \emph{cero.} Luego $\card(K^{n}-V)$ ser\'a un m\'ultiplo de $p$; como $\card(K^{n})
=q^{n}=p^{ns}$ tambi\'en es un m\'ultiplo de $p$, $\card(V)$ ser\'a un m\'ultiplo de $p$. Como $V$ contiene
al origen, deber\'a conteger necesariamente otros puntos, pues $p\geq 2$ y de esta manera habremos demostrado
el teorema~\ref{teo1.7.2}.

Calculemos entonces $\sum_{x\in K^{n}}G(x)$. El polinomio $G$ es combinaci\'on lineal de monomios
$M_{\alpha}(X) = X_{1}^{\alpha_{1}}\cdots X_{n}^{\alpha_{n}}$; por lo que basta calcular
$\sum_{x\in K^{n}}M_{\alpha}(x) = \sum_{x\in K^{n}}x_{1}^{\alpha_{1}}\cdots x_{n}^{\alpha_{n}}
= \left(\sum_{x_{1}\in K}x_{1}^{\alpha_{1}}\right)\cdots\left(\sum_{x_{n}\in K}x_{n}^{\alpha_{n}}\right)$.
Se trata entonces de calcular sumas de la forma $\sum_{z\in K}z^{\beta}$ ($\beta\in\NN$).

\begin{enumerate}
\item[(a)] Si $\beta = 0$, se tiene $z^{\beta} =1$ para todo $z\in K$, y la suma vale $q = 0$;

\item[(b)] Si $\beta > 0$, el t\'ermino $0^{\beta}$ es cero y la suma se reduce a $\sum_{z\in K^{*}}z^{\beta}$. Recordemos
que $K^{*}$ es un grupo c\'iclico de orden $q-1$ (teorema~\ref{teo1.7.1}, b));
sea $\omega$ un generador. Luego,
$\sum_{z\in K^{*}}z^{\beta} = \sum_{j=0}^{q-2}\omega^{\beta j}$, que es una serie geom\'etrica. Por lo tanto:

\item[(b')] Si la raz\'on $\omega^{\beta}$ es $\neq 1$, es decir, si $\beta$ no es m\'ultiplo de $q-1$, se tiene
$\sum_{j=0}^{q-2}\omega^{\beta j} = \frac{\omega^{\beta(q-1)}-1}{\omega^{\beta}-1} = 0$ (pues $\omega^{q-1} = 1$).

\item[(b'')] Si $\omega^{\beta} = 1$, es decir si $\beta$ es un m\'ultiplo de $q-1$, se tiene
\begin{gather*}
\sum_{j=0}^{q-2}\omega^{\beta j} = q-1.
\end{gather*}
\end{enumerate}
Resulta de (a), (b') y (b'') que $\sum_{x\in K^{n}}x_{1}^{\alpha_{1}}\cdots x_{n}^{\alpha_{n}}$ se anula \emph{salvo
si} todos los $\alpha_{i}$ son $>0$ y divisibles por $q-1$. El grado $\alpha_{1}+\dots+\alpha_{n}$ del
es, en ese caso, $\geq (q-1)n$. Pero, como $G = F^{q-1}$, $G$ tiene grado $(q-1)d$, y se tiene
$(q-1)d < (q-1)n$ por hip\'otesis. Por lo tanto, se tiene que $\sum_{x\in K^{n}}M_{\alpha}(x) = 0$ para todo
monomio que figura en $G$ con un coeficiente no nulo. Sumando, resulta que $\sum_{x\in K^{n}}G(x) = 0$. Ya vimos
que esta relaci\'on implica la conclusi\'on deseada.

\begin{comm}
Observemos que en vez de suponer que $F$ es homog\'eneo, es suficiente suponer que $F$ no tiene t\'ermino constante.
Por otra parte, la desigualdad \emph{estricta} $d < n$ entre el grado y el n\'umero de variables es esencial.
Por ejemplo, la \emph{norma} de $\FF_{q^{n}}$ en $\FF_{q}$ (cf. cap\'itulo~\ref{cap2}, \S\ref{sec2.6}) es un polinomio
homog\'eneo de grado $n$ en $n$ variables sobre $\FF_{q}$ que s\'olo se anula en el origen.
\end{comm}

\begin{example*}
{\itshape Un ejemplo.} Una forma cuadr\'atica en $3$ variables sobre un cuerpo \emph{finito} $K$
``representa al $0$'' (i.e. tiene un cero no trivial). Pasando de $K^{3}$ al plano proyectivo
$P_{2}(K)$, esto quiere decir que una \emph{c\'onica} sobre $K$ admite un punto racional sobre $K$
(i.e. tal que sus coordenadas homog\'eneas pueden elegirse en $K$). El ejemplo de la c\'onica
$x^{2}+y^{2}+z^{2}=0$ sobre $\RR$ (resp. $x^{2}+y^{2}-3y^{2}=0$ sobre $\QQ$: para verificar que
$x^{2}+y^{2}-3z^{2}=0$ no admite soluciones no triviales en $\QQ$ se reduce al caso donde $x$, $y$, $z$
son enteros coprimos entre s\'i, y luego se reduce m\'odulo $4$) muestra que el teorema no es verdad para
todo los cuerpos.
\end{example*}

\chapter[Elementos enteros sobre un anillo]
{Elementos enteros sobre un anillo; elementos algebraicos sobre un cuerpo}
\label{cap2}

Entre todos los n\'umeros complejos, en este libro nos ocuparemos de los n\'umeros
\emph{algebraicos,} es decir, aquellos que satisfacen una ecuaci\'on de la forma
\begin{gather*}
x^{n}+a_{n-1}x^{n-1}+\dots+a_{1}x+a_{0} = 0
\end{gather*}
donde los $a_{i}$ son n\'umeros racionales. Cuando los $a_{i}$ son enteros
($a_{i}\in\ZZ$), el n\'umero algebraico $x$ se dice \emph{entero algebraico.}
Por ejemplo, $\sqrt{2}$, $\sqrt{3}$, $i$ y $e^{2i\pi/5}$ son enteros algebraicos. No es evidente
a priori que sumas y productos de n\'umeros algebraicos (resp. enteros algebraicos) son
de nuevo n\'umeros algebraicos (resp. enteros algebraicos). Consideremos, por ejemplo,
$x = \sqrt{2}+\sqrt{3}$. Elevando al cuadrado, se tiene $x^{2} = 2+3+2\sqrt{6}$.
Separando la ra\'iz cuadrada de los dem\'as t\'erminos se obtiene la igualdad $x^{2}-6=2\sqrt{6}$ y
elevando al cuadrado nuevamente se obtiene finalmente $(x^{2}-5)^{2}=24$, lo que muestra que
$x$ es un entero algebraico. El lector tendr\'a que esforzarse para hacer lo mismo con
$\sqrt[3]{5}+\sqrt[5]{7}$ y se convencer\'a que la serie de trucos utilizandos en la demostraci\'on de
que este n\'umero es algebraico no se pueden generalizar f\'acilmente.

Para superar esta dificultad, los algebristas del \'ultimo siglo, Dedekind en particular, tuvieron
la idea de ``linearizar'' el problema, es decir, de introducir la noci\'on de m\'odulo. Esto es lo que haremos nosotros
tambi\'en. Reemplazar $\ZZ$ (o $\QQ$) por un anillo conmutativo cualquiera no require m\'as esfuerzo
y nos ser\'a muy \'util en lo que sige. Comenzaremos estudiando el caso general de elementos enteros sobre un
anillo y despu\'es estudiaremos el caso particular de elementos algebraicos sobre un cuerpo.

\section{Elementos enteros sobre un anillo}\label{sec2.1}

\begin{theorem}\label{teo2.1.1}
Sea $R$ un anillo, $A$ un subanillo de $R$ y $x$ un elemento de $R$. Las siguientes propiedades son equivalentes:
\begin{enumerate}
\item Existen $a_{0},\dots,a_{n-1}\in A$ tales que
\begin{gather}\label{eq-2.1-1}
x^{n}+a_{n-1}x^{n-1}+\dots+a_{1}x+a_{0} = 0
\end{gather}
(es decir, $x$ es ra\'iz de un polinomio m\'onico sobre $A$)
\item El anillo $A[x]$ es un $A$-m\'odulo de tipo finito.
\item Existe un subanillo $B$ de $R$ que contiene a $A$ y a $x$ y que es un $A$-m\'odulo
de tipo finito.
\end{enumerate}
\end{theorem}

Mostremos que {\itshape a}) implica {\itshape b}). Sea $M$ el sub-$A$-m\'odulo de $R$ generado por $1, x, \dots, x^{n-1}$.
Por {\itshape a}) se tiene $x^{n}\in M$. Mostremos que $x^{n+j}\in M$ haciendo inducci\'on en $j$. En efecto,
multiplicando~\eqref{eq-2.1-1} por $x^{j}$ se obtiene $x^{n+j}=-a_{n-1}x^{n+j-1}-\dots-a_{0}x^{j}$.
Como $A[x]$ es el $A$-m\'odulo generado por los $x^{k}$ ($k\geq 0$), se deduce que $A[x] = M$, lo que demuestra
{\itshape b}).

La implicaci\'on {\itshape b}) $\then$ {\itshape c}) es trivial. Mostremos que {\itshape c}) implica {\itshape a}).
Sea $(y_{1},\dots,y_{n})$ un sistema finito de generadores del $A$-m\'odulo $B$. Es decir,
$B = Ay_{1}+\dots+Ay_{n}$. Como $x\in B$, $y_{i}\in B$
y $B$ es un subanillo de $R$, se sigue que $xy_{i}\in B$ de manera que existen elementos $a_{ij}$ de $A$
tales que $xy_{i} = \sum_{j=1}^{n}a_{ij}y_{j}$. Esto se puede reescribir
\begin{gather*}
\sum_{j=1}^{n}(\delta_{ij}x-a_{ij})y_{j} = 0\quad(i=1,\dots,n)
\end{gather*}
Se obtiene as\'i un sistema de $n$ ecuaciones lineales homog\'eneas en $(y_{1},\dots,y_{n})$.

Si $d$ es el determinante $\det(\delta_{ij}x-a_{ij})$, la f\'ormula de
Cramer muestra que $dy_{i} = 0$ para todo $i$. Como $B = \sum_{i}Ay_{i}$, se deduce que
$dB = 0$, de donde $d = d\cdot 1 = 0$ pues $B$ posee un elemento neutro. Ahora bien, si desarollamos el
determinante
\begin{gather*}
d = \det(\delta_{ij}x - a_{ij}),
\end{gather*}
obtenemos una ecuaci\'on de la forma $P(x) = 0$ con $P$ un polinomio de grado $n$ sobre $A$. Este polinomio
es m\'onico pues el coeficiente de $x^{n}$ proviene \'unicamente del producto $\prod_{i=1}^{n}(x-a_{ii})$
de los elementos de la diagonal principal. Por lo tanto {\itshape a}) es verdad.

\begin{definition}\label{def2.1.1}
Sea $R$ un anillo y $A$ un subanillo de $R$. Un elemento $x$ de $R$ se dice entero sobre $A$ si satisface
las condiciones equivalentes a, b), c) del teorema~\ref{teo2.1.1}.
Sea $P\in A[X]$ un polinomio m\'onico tal que
$P(x) = 0$ (la existencia de este polinomio se sigue de a)); la relaci\'on $P(x) = 0$ se llama una
ecuaci\'on de dependencia integral de $x$ sobre $A$.
\end{definition}

\begin{example*}
El elemento $x = \sqrt{2}$ de $\RR$ es entero sobre $\ZZ$; una ecuaci\'on de dependencia
integral est\'a dada por $x^{2}-2 = 0$.
\end{example*}

\begin{proposition}\label{prop2.1.1}
Sea $R$ un anillo, $A$ un subanillo de $R$ y $(x_{i})_{1\leq i\leq n}$ una colecci\'on finita de elementos
de $R$. Si, para todo $i$, $x_{i}$ es entero sobre $A[x_{1},\dots,x_{i-1}]$ (en particular, si todos los
$x_{i}$ son enteros sobre $A$), entonces $A[x_{1},\dots,x_{n}]$ es un $A$-m\'odulo de tipo finito.
\end{proposition}

Razonemos por inducci\'on en $n$. Si $n=1$, se trata de la afirmaci\'on {\itshape b}) del teorema~\ref{teo2.1.1}. Supongamos que la
proposici\'on es verdad para $n-1$. Entonces $B = A[x_{1},\dots,x_{n-1}]$ es un $A$-m\'odulo de tipo finito
y podemos escribir $B = \sum_{j=1}^{p}Ab_{j}$. Aplicando el caso $n=1$ se sigue que $A[x_{1},\dots,x_{n}]
=B[x_{n}]$ es un $B$-m\'odulo de tipo finito, que lo podemos escribir $\sum_{k=1}^{q}Bc_{k}$. Luego se tiene
\begin{gather*}
A[x_{1},\dots,x_{n}] = \sum_{k=1}^{q}Bc_{k} = \sum_{k=1}^{q}\left(\sum_{j=1}^{p}Ab_{j}\right)c_{k} = \sum_{j,k}
Ab_{j}c_{k},
\end{gather*}
de manera que $(b_{j}c_{k})$ es un sistema finito de generadores del $A$-m\'odulo $A[x_{1},\dots,x_{n}]$.

\begin{corollary}\label{cor2.1.1}
Sea $R$ un anillo, $A$ un subanillo de $R$, $x$ e $y$ dos elementos de $R$ enteros sobre $A$. Entonces
$x+y$, $x-y$ y $xy$ son enteros sobre $A$.
\end{corollary}

En efecto, se tiene que $x+y, x-y, xy\in A[x,y]$. Por la proposici\'on~\ref{prop2.1.1},
$A[x,y]$ es un $A$-m\'odulo
de tipo finito; luego, por la parte {\itshape c}) del teorema~\ref{teo2.1.1},
$x+y$, $x-y$ y $xy$ son enteros sobre $A$.

\begin{corollary}\label{cor2.1.2}
Sea $R$ un anillo y $A$ un subanillo de $R$. El conjunto $A'$ de elementos de $R$ enteros sobre $A$
es un subanillo de $R$ que contiene a $A$.
\end{corollary}

Efectivamente, $A'$ es un subanillo de $R$ por el corolario~\ref{cor2.1.1}; contiene a $A$ pues todo $a\in A$ es
ra\'iz del polinomio m\'onico $X-a$ y por lo tanto es entero.

\begin{definition}
Sea $R$ un anillo, $A$ un subanillo de $R$. El anillo $A'$ de los elementos de $R$ enteros sobre $A$ se
llama la \emph{clausura \'integra} de $A$ en $R$. Sea $A$ un dominio \'integro y $K$ su cuerpo de fracciones. La clausura
\'integra de $A$ en $K$ se llama la clausura \'integra de $A$.
Sea $B$ un anillo y $A$ un subanillo de $B$. Decimos
que $B$ es \emph{entero} sobre $A$ si todo elemento de $B$ es entero sobre $A$ (es decir, si la clausura
entera de $A$ en $B$ es todo $B$).
\end{definition}

\begin{proposition}[de transitividad]\label{prop2.1.2}
Sea $C$ un anillo, $B$ un subanillo de $C$ y $A$ un subanillo de $B$. Si $B$ es entero sobre $A$ y $C$ es
entero sobre $B$, entonces $C$ es entero sobre $A$.
\end{proposition}

En efecto, sea $x\in C$; $x$ es entero sobre $B$ y por lo tanto se tiene una ecuaci\'on de dependencia entera
$x^{n}+b_{n-1}x^{n-1}+\dots+b_{0} = 0$ con $b_{i}\in B$. Sea $B' = A[b_{0},\dots,b_{n-1}]$, de manera que $x$
tambi\'en es entero sobre $B'$. Como $B$ es entero sobre $A$, los $b_{i}$ son enteros sobre
$A$. Por lo tanto, por la proposici\'on~\ref{prop2.1.1},
$B'[x] = A[b_{0},\dots,b_{n-1},x]$ es un $A$-m\'odulo de tipo finito. Por la parte
{\itshape c}) del teorema~\ref{teo2.1.1}
concluimos que $x$ es entero sobre $A$. Por lo tanto $C$ es entero sobre $A$.

\begin{proposition}\label{prop2.1.3}
Sea $B$ un dominio \'integro y $A$ un subanillo de $B$ tal que $B$ es entero sobre $A$. Para que $B$ sea
un cuerpo es necesario y suficiente que $A$ lo sea.
\end{proposition}

Supongamos que $A$ es un cuerpo y sea $b\in B$, $b\neq 0$. Luego $A[b]$ es un espacio vectorial de
dimension \emph{finita} sobre $A$ (teorema~\ref{teo2.1.1}, {\itshape b})).
Por otra parte, $y\mapsto by$ es una aplicaci\'on
$A$-lineal de $A[b]$ en si mismo, que es inyectiva pues $A[b]$ es un dominio \'integro y $b\neq 0$. Por
lo tanto, tambi\'en es sobreyectiva y existe $b'\in A[b]$ tal que $bb' =1$. Luego $b$ es inversible en $B$
y $B$ es un cuerpo\footnote{El mismo tipo de razonamiento, utilizando la homotecia $y\mapsto by$, muestra
que todo dominio \'integro \emph{finito} es un cuerpo.}.

Reciprocamente, supongamos que $B$ es un cuerpo y sea $a\in A$, $a\neq 0$. Luego $a$ admite una inversa
$a^{-1}\in B$ que satisface une ecuaci\'on de dependencia entera
\begin{gather*}
a^{-n}+a_{n-1}a^{-n+1}+\dots+a_{1}a^{-1}+a_{0} =0\quad(a_{i}\in A)
\end{gather*}
Multiplicando por $a^{n-1}$, obtenemos $a^{-1}=-(a_{n-1}+\dots+a_{1}a^{n-2}+a_{0}a^{n-1})$, de donde
$a^{-1}\in A$, de manera que $A$ es un cuerpo.

\section{Anillos \'integramente cerrados}\label{sec2.2}

\begin{definition*}
Decimos que un anillo $A$ es \'integramente cerrado si es un dominio \'integro y su clausura \'integra es
el mismo $A$.
\end{definition*}

En otras palabras, si todo elemento $x$ del cuerpo de fracciones $K$ de $A$ que es entero sobre $A$
es un elemento de $A$.

\begin{example}
Sea $A$ un dominio \'integro y $K$ su cuerpo de fracciones. Entonces la \emph{clausura \'integra} $A'$
de $A$ (es decir, la clausura \'integra de $A$ en $K$) es un anillo \'integramente cerrado. En efecto, la
clausura \'integra de $A'$ es entera sobre $A'$, y por lo tanto sobre $A$ (\S\ref{sec2.1}, proposici\'on~\ref{prop2.1.2}).
Luego coincide con $A'$.
\end{example}

\begin{example}{\itshape Todo dominio de ideales principales es \'integramente cerrado.}
Un dominio de ideales principales $A$ es un dominio \'integro por definici\'on. Sea $x$
un elemento entero sobre $A$ de su cuerpo de fracciones. Se tiene una ecuaci\'on de dependencia
entera
\begin{gather}\label{eq-2.2-1}
x^{n}+a_{n-1}x^{n-1}+\dots+a_{1}x+a_{0}=0\quad(a_{i}\in A)
\end{gather}

Ahora bien, podemos escribir $x = a/b$ con $a,b\in A$ \emph{coprimos.} Multiplicando \eqref{eq-2.2-1}
por $b^{n}$ obtenemos
\begin{gather*}
a^{n}+b(a_{n-1}a^{n-1}+\dots+a_{1}ab^{n-2}+a_{0}b^{n-1}) = 0.
\end{gather*}

Por lo tanto $b$ divide a $a^{n}$. Como es coprimo con $a$, la aplicaci\'on repetida del lema de Euclides
muestra que $b$ divide a $a$. Luego $x = a/b\in A$ y $A$ es \'integramente cerrado.
\end{example}

\begin{comm}
Observemos que solamente utilizamos las propiedades multiplicativas de los dominios de ideales principales
(elementos coprimos, lema de Euclides). El mismo razonamiento muestra que todo anillo de \emph{factorizaci\'on
\'unica} es \'integramente cerrado.
\end{comm}

\section[Elementos algebraicos sobre un cuerpo]{Elementos algebraicos sobre un cuerpo. Extensiones algebraicas}
\label{sec2.3}

\begin{definition*}
Sea $R$ un anillo y $K$ un subcuerpo de $R$. Decimos que un elemento $x$ de $R$ es \emph{algebraico sobre $K$} si
existen elementos $a_{0},\dots,a_{n}$ de $K$, no todos nulos, tales que $a_{n}x^{n}+\dots+a_{1}x+a_{0}=0$.
\end{definition*}

En otras palabras, los \emph{monomios} $(x^{j})_{j\in\NN}$ son linealmente dependientes sobre $K$.
Un elemento que no es algebraico sobre $K$ se dice \emph{trascendente} sobre $K$; equivalentemente, los
monomios $(x^{j})_{j\in\NN}$ son linealmente independientes sobre $K$.

En la condici\'on de la definici\'on~\ref{def2.1.1},
podemos suponer que $a_{n}$ es no nulo; luego, admite un inverso
$a_{n}^{-1}$ pues $K$ es un cuerpo. Multiplicando por $a_{n}^{-1}$ se obtiene una
ecuaci\'on de dependencia entera. Por lo tanto:
\begin{gather}
\text{\itshape Sobre un cuerpo, algebraico = entero.}
\end{gather}

Por lo tanto podemos aplicar la teor\'ia de los elementos enteros; por ejemplo, si $K\subset R$ y $x\in R$,
el teorema~\ref{teo2.1.1}, {\itshape b}) de la \S\ref{sec2.1} implica:
\begin{gather}
\text{\itshape $x$ algebraico sobre $K$ $\iff$ $[K[x]:K]$ es finito.}
\end{gather}

Decimos que un anillo $R$ que contiene un cuerpo $K$ es \emph{algebraico} sobre $K$ si todo elemento de $R$
es algebraico sobre $K$. Cuando el mismo $R$ es un cuerpo, decimos que $R$ es una \emph{extensi\'on algebraica}
de $K$.

Dado un cuerpo $L$ y un subcuerpo $K$ de $L$, la dimensi\'on $[L:K]$ se llama \emph{grado} de $L$ sobre $K$.
El teorema~\ref{teo2.1.1}, {\itshape c}) de la \S\ref{sec2.1} muestra entonces:
\begin{gather}
\text{\itshape Si el grado de $L$ sobre $K$ es finito, $L$ es una extensi\'on algebraica de $K$.}
\end{gather}
Llamamos \emph{cuerpo de n\'umeros algebraicos} (o \emph{cuerpo de n\'umeros}) a toda extensi\'on de $\QQ$
de grado finito.

\begin{proposition}\label{prop2.3.1}
Sea $K$ un cuerpo, $L$ una extensi\'on algebraica de $K$ y $M$ una extensi\'on algebraica de $L$. Entonces
$M$ es una extensi\'on algebraica de $L$. M\'as a\'un, $[M:K] = [M:L][L:K]$
(``multiplicatividad del grado'').
\end{proposition}

La primera afirmaci\'on es un caso particular de la proposici\'on~\ref{prop2.1.2} de la \S\ref{sec2.1}.
M\'as a\'un, si $(x_{i})_{i\in I}$
es una base de $L$ sobre $K$ y $(y_{j})_{j\in J}$ es una base de $M$ sobre $L$, entonces $(x_{i}y_{j})_{(i,j)\in I\times J}$
es una \emph{base} de $M$ sobre $K$: en efecto, es un sistema de generadores, tal como se vi\'o en la
proposici\'on~\ref{prop2.1.1} de la \S\ref{sec2.1}.
Por otra parte, una relaci\'on $\sum_{i,j}a_{ij}x_{i}y_{j} = 0$ con $a_{ij}\in K$ implica
$\sum_{j}\left(\sum_{i}a_{ij}x_{i}\right)y_{j} = 0$, de donde se sigue que
$\sum_{i}a_{ij}x_{i} = 0$ para todo $j$ (pues
$\sum a_{ij}x_{i}\in L$) y en consecuencia que $a_{ij} = 0$ para todo $i$, $j$. Esto demuestra la f\'ormula
de la multiplicatividad del grado.

\begin{proposition}
Sea $R$ un anillo y $K$ un subcuerpo de $R$. Entonces
\begin{enumerate}
\item[\upshape a)] El conjunto $K'$ de los elementos de $R$ algebraicos sobre $K$ es un subanillo de $R$ que contiene a $K$;
\item[\upshape b)] Si $R$ es un dominio \'integro, $K'$ es un subcuerpo de $R$.
\end{enumerate}
\end{proposition}

En efecto, {\itshape a}) es un caso particular del corolario~\ref{cor2.1.2} de la proposici\'on~\ref{prop2.1.1}, \S\ref{sec2.1}
y {\itshape b}) resulta de la proposici\'on~\ref{prop2.1.3} de \S\ref{sec2.1}.

Ahora estudiaremos m\'as de cerca los elementos algebraicos sobre un cuerpo. Sea $R$ un anillo, $K$ un subcuerpo
de $R$ y $x$ un elemento de $R$. Existe un \'unico homomorfismo $\varphi$ del anillo de polinomios $K[X]$ en
$R$ tal que $\varphi(X) = x$ y tal que $\varphi(a) = a$ para todo $a\in K$. La imagen de $\varphi$ es $K[x]$.
La definici\'on de elemento algebraico se traduce en:
\begin{gather}\label{eq-2.3-4}
\text{\itshape $x$ es algebraico sobre $K$ $\iff$ $\Ker(\varphi)\neq(0)$.}
\end{gather}

Si $x$ es algebraico sobre $K$, el ideal $\Ker(\varphi)$ es un ideal \emph{principal} $(F(X))$
(pues $K[X]$ es un dominio de ideales principales) generado por un polinomio no nulo $F(X)$ que podemos
suponer \emph{m\'onico} pues $K$ es un cuerpo. Este polinomio m\'onico est\'a determinado de forma \'unica por
$K$ y $x$ y se llama el \emph{polinomio minimal} de $x$ sobre $K$. Traduciendo la definici\'on, obtenemos:
\begin{gather}\label{eq-2.3-5}
\text{\parbox{0.9\textwidth}{\itshape Sea $F(X)$ el polinomio minimal de $x$ sobre $K$ y sea $G(X)\in K[X]$. Para que
$G(x) = 0$ es necesario y suficiente que $F(X)$ divida a $G(X)$ en $K[X]$.}}
\end{gather}
M\'as a\'un, pasando al cociente, obtenemos un \emph{isomorfismo can\'onico}
\begin{gather}\label{eq-2.3-6}
K[X]/(F(X))\to[\sim] K[x].
\end{gather}

Con la misma notaci\'on de antes, supongamos ahora que $x$ es algebraico sobre $K$ y sea $F(X)$ su polinomio
minimal. Aplicando \eqref{eq-2.3-5} y la proposici\'on~\ref{prop2.1.3} de la \S\ref{sec2.1}, se obtienen las siguientes equivalencias:
\begin{align}
\text{\itshape $K[X]$ es un cuerpo} &\iff \text{\itshape $K[x]$ es un dominio \'integro}\\
&\iff \text{\itshape $F(X)$ es irreducible.}\notag
\end{align}

Reciprocamente, sea $K$ un cuerpo y $F(X)\in K[X]$ un polinomio irreducible. Luego $K[X]/(F(X))$ es un cuerpo
que contiene a $K$ y tal que, si notamos $x$ la clase de $X$ en este cuerpo, se tiene $F(x) = 0$, de manera que
$F(X)$ es divisible por $X-x$ en el cuerpo $K[x]$. M\'as generalmente:

\begin{proposition}\label{prop2.3.3}
Sea $K$ un cuerpo y $P(X)\in K[X]$ un polinomio no constante. Existe una extension algebraica $K'$ de $K$ de
grado finito tal que $P(X)$ se descompone en factores lineales en $K'[X]$.
\end{proposition}

Razonamos por inducci\'on sobre el grado $d$ de $P(X)$. El resultado es evidente si $d=1$. Supongamos que el enunciado est\'a
demostrado para grados no mayores a $d-1$.

Sea $F(X)$ un factor irreducible de $P(X)$. Acabamos de ver que existe
una extensi\'on $K''$ de $K$ de grado finito (concretamente, $K[X]/(F(X))$) y un elemento $x\in K''$ tal que
$F(X)$ es multiplo de $X-x$ en $K''[X]$. Se tiene entonces que $P(X) = (X-x)P_{1}(X)$ con $P_{1}(X)\in K''[X]$.
De la hip\'otesis inductiva se sigue que $P_{1}(X)$ se descompone en factores lineales en una extensi\'on
$K'$ de $K''$ de grado finito. Luego $K'$ es una extensi\'on de grado finito de $K$
(proposici\'on~\ref{prop2.3.1}) y
$P(X)$ se descompone en factores lineales en $K'[X]$.

\begin{remark*}[cuerpos algebraicamente cerrados]
Decimos que $K$ es \emph{algebraicamente cerrado}
si \emph{todo} polinomio no constante $P(X)\in K[X]$ se descompone en factores lineales en $K[X]$. Para verificar
esta propieadad basta, por inducci\'on en el grado, que todo polinomio no constante $P(X)\in K[X]$ admita una
ra\'iz $x\in K$. Aplicando una versi\'on ``transfinita'' de la proposici\'on~\ref{prop2.3.3} (es decir,
combinando la proposici\'on~\ref{prop2.3.3} con el
teorema de Zorn; cf.~\cite{Bourbaki1}, cap\'itulo~V y \cite{ZariskiSamuel}, cap\'itulo~II), se demuestra que todo cuerpo
es un subcuerpo de un cuerpo algebraicamanete cerrado.

Utilizando las t\'ecnicas del An\'alisis puede demostrarse, de varias maneras\footnote{Para una demostraci\'on que utiliza las
propiedades de las funciones continuas sobre un espacio compacto, ver \cite{Choquet}. Para una demostraci\'on que utiliza
las propiedades de las funciones holomorfas de una variable compleja, ver \cite{Cartan}. Nostoros presentaremos
en el ap\'endice a este cap\'itulo una demostraci\'on m\'as algebraica, que s\'olo utilizar\'a las propiedades
m\'as simples de los n\'umeros reales.}, que el cuerpo $\CC$ de los \emph{n\'umeros complejos}
es algebraicamente cerrado. Esto es todo lo que nosostros necesitaremos.
\end{remark*}

\section{Elementos conjugados, cuerpos conjugados}\label{sec2.4}

Dados dos cuerpos $L$ y $L'$ conteniendo un cuerpo $K$, llamamos \emph{$K$-isomorfismo} de $L$ en $L'$
a todo isomorfismo $\varphi:L\to L'$ tal que $\varphi(a) = a$ para todo $a\in K$. En este caso, decimos que
$L$ y $L'$ son \emph{$K$-isomorfos,} o (si $L$ y $L'$ son algebraicos sobre $K$) que son \emph{cuerpos conjugados
sobre $K$.}

Dadas dos extensiones $L$ y $L'$ de $K$, decimos que dos elementos $x\in L$ y $x'\in L'$ son \emph{conjugados}
sobre $K$ si existe un $K$-isomorfismo $\varphi:K(x)\to K(x')$ tal que $\varphi(x) = x'$. En tal caso,
$\varphi$ es \'unica. Esto quiere decir que o bien $x$ y $x'$ son ambos trascendentes sobre $K$,
o bien $x$ y $x'$ son ambos algebraicos sobre $K$ y tienen el mismo polinomio minimal (cf.~\eqref{eq-2.3-5}, \S\ref{sec2.3}).

\begin{example*}
Sea $F(X)$ un polinomio \emph{irreducible} de grado $n$ sobre $K$ y sean
$x_{1},\dots,x_{n}$ sus ra\'ices en una extensi\'on $K'$ de $K$ (\S\ref{sec2.3}, proposici\'on~\ref{prop2.3.3}). Luego los $x_{i}$
son conjugados dos a dos sobre $K$ (\S\ref{sec2.3}, \eqref{eq-2.3-6}) y los cuerpos $K[x_{i}]$ tambi\'en son conjugados
dos a dos sobre $K$.
\end{example*}

\begin{lemma*}
Sea $K$ un cuerpo de caracter\'istica cero o un cuerpo finito, $F(X)\in K[x]$ un polinomio m\'onico
irreducible y $F(X) = \prod_{i=1}^{n}(X-x_{i})$ su descomposic\'on en factores lineales en una extensi\'on
$K'$ de $K$ (\S\ref{sec2.3}, proposici\'on~\ref{prop2.3.3}). Entonces
las $n$ ra\'ices $x_{1},\dots,x_{n}$ de $F(X)$ son todas distintas.
\end{lemma*}

Razonemos por el absurdo. En el caso contrario, $F(X)$ admite una ra\'iz \emph{m\'ultiple} $x$, que
tendr\'ia que ser entonces tambi\'en ra\'iz del polinomio derivado $F'(X)$, en cuyo caso $F(X)$
dividir\'ia a $F'(X)$ (\eqref{eq-2.3-4}, \S\ref{sec2.3}). Como $d^{\circ}F' < d^{\circ}F$, esto implica que $F'(X)$ es el
polinomio nulo. Ahora bien, si
\begin{gather*}
F(X) = X^{n}+a_{n-1}X^{n-1}+\dots+a_{0}\quad(a_{i}\in K)
\end{gather*}
se tiene
\begin{gather*}
F'(X) = nX^{n-1}+(n-1)a_{n-1}X^{n-2}+\dots+a_{1}.
\end{gather*}
Deducimos que $n\cdot 1 = 0$ y $j\cdot a_{j} = 0$ para $j=1,\dots,n-1$. Esto es imposible en caracter\'istica
cero.

En caracter\'istica $p\neq 0$, la implicaci\'on es que $p$ divide a $n$ y que $a_{j} = 0$ si $j$ no es
un m\'ultiplo de $p$ (recordemos que $p$ es un n\'umero primo). Luego, $F(X)$ es de la forma
\begin{gather*}
F(X) = X^{qp}+b_{q-1}X^{(q-1)p}+\dots+b_{1}X^{p}+b_{0}\quad(b_{i}\in K).
\end{gather*}
Si cada uno de los $b_{i}$ es una \emph{potencia $p$-\'esima,} es decir $b_{i} = c_{i}^{p}$ para
alg\'un $c_{i}\in K$, se sigue que $F(X) = (X^{q}+c_{q-1}X^{q-1}+\dots+c_{0})^{p}$ (cap\'itulo~\ref{cap1}, \S\ref{sec1.7},
proposici\'on~\ref{prop1.7.1}) y $F(X)$ no es irreducible. Por otra parte, si $K$ es un cuerpo finito de caracter\'istica $p$
($\neq 0$), la aplicaci\'on $x\mapsto x^{p}$ de $K$ en $K$ es inyectiva (pues
$x^{p}=y^{p}$ implica $x^{p}-y^{p} = (x-y)^{p} = 0$ y $x-y=0$). Por lo tanto tambi\'en es sobreyectiva
pues $K$ es finito. Hemos obtenido una una contradicci\'on.

\begin{comm}
Los cuerpos $K$ de caracter\'istica $p\neq 0$ tales que $x\mapsto x^{p}$ es sobreyectiva
(i.e. tales que todo elemento de $K$ es una potencia $p$-\'esima) se llaman cuerpos \emph{perfectos.}
Acabamos de demostrar que todo cuerpo finito es perfecto. Por convenci\'on, un cuerpo de caracter\'istica
cero es perfecto. De hecho, hemos demostrado que la conclusi\'on del teorema vale bajo la hip\'otesis
de que $K$ sea un cuerpo perfecto. El cuerpo $\FF_{p}(T)$ de las funciones racionales en une variable
sobre $\FF_{p}$ no es perfecto, pues la variable $T$ no es una potencia $p$-\'esima en $\FF_{p}(T)$.
\end{comm}

\begin{theorem}\label{teo2.4.1}
Sea $K$ un cuerpo de caracter\'istica cero o finito, $K'$ una extensi\'on de $K$ de grado finito $n$
y $C$ un cuerpo algebraicamente cerrado que contiene a $K$. Entonces existen $n$ $K$-isomorfismos distintos
de $K'$ en $C$.
\end{theorem}

El resultado es cierto para una extensi\'on \emph{unigenerada} $K'$, es decir de la forma $K' = K[x]$
($x\in K$). En este caso el polinomio minimal $F(X)$ de $x$ sobre $K$ tiene grado $n$ y admite
$n$ ra\'ices $x_{1},\dots, x_{n}$ en $C$ que son distinas por el lema anterior. Para cada $i=1,\dots,n$ se
tiene entonces un $K$-isomorfismo $\sigma_{i}:K'\to C$ tal que $\sigma_{i}(x) = x_{i}$.

En el caso general, procedemos por inducci\'on en el grado $n$ de $K'$. Sea $x\in K'$; consideremos los cuerpos
intermedios $K\subset K[x]\subset K'$ y sea $q = [K[x]:K]$. Podemos suponer que $q > 1$. Por el caso unigenerado,
se tienen $q$ $K$-isomorfismos distintos $\sigma_{1},\dots,\sigma_{q}$ de $K[x]$ en $C$. Como
$K[\sigma_{i}(x)]$ y $K[x]$ son isomorfos, podemos construir una extension $K'_{i}$ de $K[\sigma_{i}(x)]$
y un isomorfismo $\tau_{i}:K'\to K'_{i}$ que prolonga a $\sigma_{i}$. Ahora bien, $K[\sigma_{i}(x)]$
es un cuerpo de caracter\'istica cero o finito. Como
\begin{gather*}
[K'_{i}:K[\sigma_{i}(x)]] = [K':K[x]] = \frac{n}{q} < n,
\end{gather*}
la hip\'otesis inductiva implica que se tienen $\frac{n}{q}$ $K[\sigma_{i}(x)]$-isomorfismos distintos
$\theta_{ij}$ de $K'_{i}$ en $C$. Luego las $n$ composiciones $\theta_{ij}\circ\tau_{i}$ son los
$q\cdot\frac{n}{q} = n$ $K$-isomorfismos deseados de $K'$ en $C$. Son distintos pues
$\theta_{ij}\circ\tau_{i}$ y $\theta_{i'j'}\circ\tau_{i'}$ son distintos cuando se los restringe a
$K[x]$ si $i\neq i'$ y si $j\neq j'$ entonces $\theta_{ij}$ y $\theta_{i'j'}$ son distintos cuando
se los restringe a $K'_{i}$. \QED

\begin{comm}
El teorema~\ref{teo2.4.1} se extiende a un cuerpo perfecto $K$: puede demostrarse que
toda extension algebraica de
un cuerpo perfecto (en particular, $K[\sigma_{i}(x)]$) es un cuerpo perfecto; el resto de
la demostraci\'on es igual.
\end{comm}

\begin{corollary*}[``teorema del elemento primitivo'']
Sea $K$ un cuerpo finito o de caracter\'istica cero y sea $K'$ una extension de $K$ de grado finito $n$.
Existe entonces un elemento $x$ en $K'$ (llamado ``primitivo'') tal que $K' = K[x]$.
\end{corollary*}

Si $K$ es finito, $K'$ es finito y su grupo multiplicativo $K'^{*}$ consiste en las potencias de un
mismo elemento $x$ (cap\'itulo~\ref{cap1}, \S\ref{sec1.7}, teorema~\ref{teo1.7.1}, {\itshape b})). Entonces se tiene $K' = K[x]$.

Supongamos ahora que $K$ es de caracter\'istica cero, en particular infinito. Por el
teorema~\ref{teo2.4.1}, existen
$n$ $K$-isomorfismos $\sigma_{i}$ de $K'$ en un cuerpo algebraicamente cerrado $C$ que contiene a $K$.
Si $i\neq j$, la ecuaci\'on $\sigma_{i}(y) = \sigma_{j}(y)$ ($y\in K'$) define un subconjunto $V_{ij}$
de $K'$ que es evidentemente un sub-$K$-espacio vectorial de $K'$ y que es distinto de $K'$ pues
$\sigma_{i}\neq\sigma_{j}$. Como $K$ es infinito, el \'algebra lineal muestra que la uni\'on de los
$V_{ij}$ no es todo $K'$.

Sea $x$ en el complemento de dicha uni\'on. Los $\sigma_{i}(x)$ son entonces
distintos dos a dos, de manera que el polinomio minimal $F(X)$ de $x$ sobre $K$ tiene al menos
$n$ ra\'ices distintas (los $\sigma_{i}(x)$) en $C$. Tenemos por lo tanto que $d^{\circ}F\geq n$,
es decir que $[K[x]:K]\geq n$. Como $K[x]\subset K'$ y $[K':K]=n$ deducimos que $K' = K[x]$. \QED

\section{Enteros de un cuerpo cuadr\'atico}\label{sec2.5}

Interrumpimos ahora la teor\'ia general para dar un ejemplo.

\begin{definition*}
Llamamos cuerpo cuadr\'atico a toda extensi\'on de grado $2$ del cuerpo $\QQ$ de los
n\'umeros racionales.
\end{definition*}

Si $K$ es un cuerpo cuadr\'atico, cualquier elemento $x$ de $K-\QQ$ es de grado $2$ sobre $\QQ$
por lo que es un elemento primitivo de $K$ (i.e. $K = \QQ[x]$ y $(1,x)$ es una base de $K$ sobre
$\QQ$). Sea $F(X) = X^{2}+bX+c$ ($b,c\in\QQ$) el polinomio minimal de un tal elemento $x\in K$.
La resoluci\'on de la ecuaci\'on de segundo grado $x^{2}+bx+c= 0$ muestra que
$2x = -b\pm\sqrt{b^{2}-4c}$, por lo que $K=\QQ(\sqrt{b^{2}-4c})\footnote{Por $\sqrt{b^{2}-4c}$
se entiende uno de los dos elementos de $K$ cuyo cuadrado es $b^{2}-4ac$.}$.
Como $b^{2}-4ac$ es un n\'umero racional $\frac{u}{v} = \frac{uv}{v^{2}}$ con $u,v\in\ZZ$, se sigue que
$K = \QQ(\sqrt{uv})$ con $uv\in\ZZ$. An\'alogamente vemos que se tiene
$K = \QQ(\sqrt{d})$ donde $d$ es un entero \emph{sin factores cuadrados} en su descomposici\'on factores
primos. As\'i hemos probado:

\begin{proposition}
Todo cuerpo cuadr\'atico es de la forma $\QQ(\sqrt{d})$, donde $d$ es un entero libre de cuadrados.
\end{proposition}

El elemento $\sqrt{d}$ es una ra\'iz del polinomio irreducible $X^{2}-d$. Esta ra\'iz admite un
\emph{conjugado} en $K$, concretamente $-\sqrt{d}$. Por lo tanto existe un automorfismo $\sigma$
de $K$ que env\'ia $\sqrt{d}$ en $-\sqrt{d}$ (\S\ref{sec2.4}). Un elemento general de $K$ se escribe de
la forma $a+b\sqrt{d}$ con $a,b\in\QQ$ y por lo tanto se tiene
\begin{gather}\label{eq-2.5-1}
\sigma(a+b\sqrt{d}) = a-b\sqrt{d}
\end{gather}

Nos proponemos estudiar ahora el anillo $A$ de \emph{enteros} de $K$, es decir, el conjunto de
los $x\in K$ que son enteros sobre $\ZZ$ (\S\ref{sec2.1}, corolario~\ref{cor2.1.2} de la proposici\'on~\ref{prop2.1.2}).
Si $x\in A$, $\sigma(x)$
es ra\'iz de la misma ecuaci\'on de dependencia entera que $x$, por lo que $\sigma(x)\in A$.
Luego, se tiene $x+\sigma(x)\in A$ y $x\cdot\sigma(x)\in A$. Ahora bien, si $x = a+b\sqrt{d}$
con $a,b\in\QQ$, se sigue de~\eqref{eq-2.5-1} que
\begin{gather}
x+\sigma(x) = 2a\in\QQ,\quad x\sigma(x) = a^{2}-db^{2}\in\QQ.
\end{gather}
Como $\ZZ$ es un dominio de ideales principales, y por lo tanto \'integramente cerrado (\S\ref{sec2.2}, ex.~2),
se concluye que
\begin{gather}\label{eq-2.5-3}
2a\in\ZZ,\quad a^{2}-db^{2}\in\ZZ.
\end{gather}
Estas condiciones~\eqref{eq-2.5-3} son necesarias para que $x=a+b\sqrt{d}$ sea entero sobre $\ZZ$. Pero tambi\'en son
suficientes pues en caso de valer, $x$ es ra\'iz de
\begin{gather*}
x^{2}-2ax+a^{2}-db^{2}=0.
\end{gather*}
De~\eqref{eq-2.5-3} se deduce que $(2a)^{2}-d(2b)^{2}\in\ZZ$. Como $2a\in\ZZ$, se sigue que $d(2b)^{2}\in\ZZ$.
Recordemos que $d$ es libre de cuadrados. Si $2b$ no fuera entero, su denominador contendr\'ia un
factor primo $p$. Este factor aparecer\'ia como $p^{2}$ en $(2b)^{2}$ y la multiplicaci\'on por $d$
no podr\'ia producir un entero. Por lo tanto, $2b\in\ZZ$.

En res\'umen, podemos escribir $a = \frac{u}{2}$, $b = \frac{v}{2}$ con $u,v\in\ZZ$. La condici\'on~\eqref{eq-2.5-3}
se traduce en:
\begin{gather}\label{eq-2.5-4}
u^{2}-dv^{2}\in 4\ZZ.
\end{gather}
Si $v$ es par,~\eqref{eq-2.5-4} muestra que $u$ tambi\'en lo es y por lo tanto $a,b\in\ZZ$. Si $v$
es impar, necesariamente $v^{2}\equiv 1\bmod 4$. Recordemos que la clase de $u^{2}\bmod 4$ es $0$ \'o $1$
(escribir la tabla de los cuadrados mod $4$). Como $d$ es libre de cuadrados, no es un m\'ultiplo
de $4$ y por lo tanto necesariamente se tiene que $u^{2}\equiv 1\bmod 4$ y $d\equiv 1\bmod 4$. Hemos
demostrado el siguiente teorema:

\begin{theorem}\label{teo2.5.1}
Sea $K = \QQ(\sqrt{d})$ un cuerpo cuadr\'atico con $d\in\ZZ$ libre de cuadrados (y por lo tanto
$\not\equiv 0\bmod 4$).
\begin{enumerate}
\item[\upshape a)] Si $d\equiv 2$ \'o $d\equiv 3\bmod 4$, el anillo $A$ de enteros de $K$ es el conjunto de los
elementos $a+b\sqrt{d}$ con $a,b\in\ZZ$.
\item[\upshape b)] Si $d\equiv 1\bmod 4$, $A$ es el conjunto de los $\frac{1}{2}(u+v\sqrt{d})$ con $u,v\in\ZZ$
de la misma paridad.
\end{enumerate}
\end{theorem}

En el caso $d\equiv 2\text{ \'o }3\bmod 4$, una base del $\ZZ$-m\'odulo $A$ es evidentemente
$(1,\sqrt{d})$. En el caso $d\equiv 1\bmod 4$, una base del $\ZZ$-m\'odulo $A$ es
$\left(1,\frac{1}{2}(1+\sqrt{d})\right)$. En efecto, por {\itshape b}), los elementos $1$ y
$\frac{1}{2}(1+\sqrt{d})$ pertenecen a $A$. Reciprocamente, para mostrar que
$\frac{1}{2}(u+v\sqrt{d})$ (con $u,v\in\ZZ$ de la misma paridad) es combinaci\'on $\ZZ$-linear
de $1$ y $\frac{1}{2}(1+\sqrt{d})$, podemos suponer, restando de ser necesario $\sqrt{1}{2}(1+\sqrt{d})$,
que $u$ y $v$ son pares, en cuyo caso $\frac{1}{2}(u+v\sqrt{d})
=\left(\frac{u}{2}-\frac{v}{2}\right)\cdot 1 + v\cdot\frac{1}{2}(1+\sqrt{d})$.

Para terminar, un poco de \emph{terminolog\'ia.} Si $d > 0$, decimos que $\QQ(\sqrt{d})$ es un \emph{cuerpo
cuadr\'atico real} (pues existe un subcuerpo de $\RR$ conjugado a $\QQ(\sqrt{d})$ sobre $\QQ$). Si $d < 0$,
decimos que $\QQ(\sqrt{d})$ es un \emph{cuerpo cuadr\'atico imaginario.}

\section{Norma y traza}\label{sec2.6}

\subsection*{a) Repaso de \'algebra lineal}

Sea $A$ un anillo, $E$ un $A$-m\'odulo \emph{libre} de rango finito y $u$ un endomorfismo de $E$.
En \'algebra lineal se define la \emph{traza,} el \emph{determinante} y el \emph{polinomio caracter\'istico}
de $u$. Si se elige una base $(e_{i})$ de $E$ y si $(a_{ij})$ es la matriz de $u$ en esta base, estas
cantidades est\'an dadas, respectivamente, por las expresiones
\begin{gather}\label{eq-2.6-1}
\Tr u =\sum_{i=1}^{n}a_{ii},\quad\det(u) = \det(a_{ij}),\quad\text{y}\quad\det(X\cdot I_{E}-u)
\end{gather}
NB. Estas cantidades son independientes de la base elegida.

Las f\'ormulas~\eqref{eq-2.6-1} muestran que valen las siguientes propiedades:
\begin{align}\label{eq-2.6-2}
\Tr(u+u') &= \Tr(u)+\Tr(u') \notag\\
\det(uu') &= \det(u)\det(u')\\
\det(X\cdot I_{E}-u) &= X^{n}-(\Tr u)\cdot X^{n-1}+\dots+(-1)^{n}\det u.\notag
\end{align}

\subsection*{b) Norma y traza de una extensi\'on}

Sea $B$ un anillo y $A$ un subanillo de $B$ tal que $B$ es un $A$-m\'odulo libre de rango finito
$n$ (por ejemplo, $A$ puede ser un cuerpo y $B$ una extensi\'on de $A$ de grado $n$). Si $x\in B$,
la multiplicaci\'on $m_{x}$ por $x$ (es decir, $y\mapsto xy$) es un endomorfismo del $A$-m\'odulo $B$.

\begin{definition}\label{def2.6.1}
Llamamos traza (resp. norma, polinomio caracter\'istico) de $x\in B$ relativa a $B$ y $A$ a
la traza (resp. determinante, polinomio caracter\'istico) del endomorfismo $m_{x}$ de multiplicaci\'on
por $x$.
\end{definition}

La traza (resp. norma) de $x$ se nota $\Tr_{A/B}(x)$ (resp. $N_{A/B}(x)$), o bien $\Tr(x)$ (resp. $N(x)$)
si no hay riesgo de confusi\'on. Es un elemento de $A$. El polinomio caracter\'istico de $x$ es un polinomio
m\'onico con coeficientes en $A$.

Si $x, x'\in B$ y $a\in A$, es evidente que $m_{x}+m_{x'}=m_{x+x'}$, $m_{x}\circ m_{x'}=m_{xx'}$ y
$m_{ax} = am_{x}$. Por otra parte, la matriz de $m_{a}$ en cualquier base de $B$ sobre $A$ es una matriz
diagonal con todos sus elementos diagonales iguales a $a$. Se sigue entonces de las f\'ormulas~\eqref{eq-2.6-1} y~\eqref{eq-2.6-2}
que:
\begin{align}
\Tr(x+x') &=\Tr(x)+\Tr(x'), & \Tr(ax) &= a\Tr(x), & \Tr(a) &= n\cdot a\\
N(xx') &= N(x)N(x'), & N(a) &= a^{n}, & N(ax) &= a^{n}N(x).\notag
\end{align}

\begin{proposition}\label{prop2.6.1}
Sea $K$ un cuerpo de caracter\'istica cero o finito, $L$ una extensi\'on algebraica
de $K$ de grado $n$,
$x$ un elemento de $L$ y $x_{1},\dots,x_{n}$ las ra\'ices del polinomio minimal de
$x$ sobre $K$ (en una
extensi\'on conveniente de $K$; cf.~\S\ref{sec2.3}, proposici\'on~\ref{prop2.3.3}),
cada una repetida $[L:K[x]]$ veces.
Entonces $\Tr_{L/K}(x) = x_{1}+\dots+x_{n}$, $N_{L/K}(x) = x_{1}\cdots x_{n}$ y el
polinomio caracter\'istico de $x$
relativo a $L$ y $K$ es $(X-x_{1})\cdots(X-x_{n})$.
\end{proposition}

\begin{comm}
En particular, el polinomio caracter\'istico es la potencia $[L:K[x]]$-\'esima del
polinomio minimal de $x$
sobre $K$.
\end{comm}

Tratemos primero el caso cuando $x$ es un \emph{elemento primitivo} de $L$ sobre $K$
(cf. \S\ref{sec2.4}, corolario del teorema~\ref{teo2.4.1}).
Sea $F(X)$ el polinomio minimal de $x$ sobre $K$. Entonces $L$ es $K$-isomorfo a
$K[X]/(F(X))$ (\S\ref{sec2.3}, f\'ormula~\eqref{eq-2.3-5})
y $(1,x,\dots,x^{n-1})$ es una base de $L$ sobre $K$. Si
$F(X) = X^{n}+a_{n-1}X^{n-1}+\dots+a_{1}X+a_{0}$,
la matriz del endomorfismo $m_{x}$ en esta base es:
\begin{gather*}
\begin{vmatrix}
0 & 0 & \cdots & 0 & -a_{0}\\
1 & 0 & \cdots & 0 & -a_{1}\\
0 & 1 & \cdots & 0 & \vdots\\
\vdots & 0 & & \vdots & \vdots\\
\vdots & \vdots & & \vdots & \vdots\\
0 & 0 & \cdots & 1 & -a_{n-1}
\end{vmatrix}
\end{gather*}
El determinante de $X\cdot 1_{L}-m_{x}$ es por lo tanto
\begin{gather*}
\begin{vmatrix}
X & 0 & \cdots & 0 & a_{0}\\
-1 & X & & 0 & a_{1}\\
0 & -1 & & 0 & \vdots\\
\vdots & \vdots & & \vdots & \vdots\\
\vdots & \vdots & & X & a_{n-2}\\
0 & 0 & & -1 & X+a_{n-1}
\end{vmatrix}
\end{gather*}
Desarollando este determinante se obtiene el polinomio caracter\'istico de $x$, que es por lo tanto igual
al polinomio minimal $X^{n}+a_{n-1}X^{n-1}+\dots+a_{0}$. Por~\eqref{eq-2.6-2}, se deduce que $\Tr(x) = -a_{n-1}$
y $N(x) = (-1)^{n}a_{0}$. Como $x$ es primitivo, se tiene adem\'as que $F(X) = (X-x_{1})\cdots(X-x_{n})$, de
donde, comparando coeficientes, se deduce que
\begin{gather*}
\Tr(x) = x_{1}+\dots+x_{n}\quad\text{y}\quad N(x) = x_{1}\cdots x_{n}.
\end{gather*}

Pasemos ahora el \emph{caso general,} y sea $r = [L:K[x]]$. Basta demostrar que el polinomio caracter\'istico
$P(X)$ de $x$ relativo a $L$ y $K$ es igual a la potencia $r$-\'esima del polinomio minimal de $x$ sobre $K$.

Sea $(y_{i})_{i=1,\dots,q}$ una base de $K[x]$ sobre $K$ y $(z_{j})_{j=1,\dots,r}$ una base de $L$ sobre $K[x]$.
Luego $(y_{i}z_{j})$ es una base de $L$ sobre $K$ y se tiene $n = qr$ (\S\ref{sec2.3}, proposici\'on~\ref{prop2.3.1}). Sea $M = (a_{ih})$ la
matriz de la multiplicaci\'on por $x$ en $K[x]$ respecto a la base $(y_{i})$: es decir, se tiene que
$xy_{i} = \sum_{h}a_{ih}y_{h}$. Se sigue que $x(y_{i}z_{j}) = \left(\sum_{h}a_{ih}y_{h}\right)z_{j}
= \sum_{h} a_{ih}(y_{h}z_{j})$. Ordenando la base $(y_{i}z_{j})$ de $L$ sobre $K$ de forma lexicogr\'afica,
vemos que la matriz $M'$ de la multiplicaci\'on por $x$ en $L$ respecto a esta base es una matriz
diagonal por bloques:
\begin{gather*}
M_{1} = \begin{vmatrix}
M & 0 & \cdots & 0\\
0 & M & \cdots & 0\\
\vdots & \vdots & \ddots & \vdots\\
0 & 0 & \cdots & M
\end{vmatrix}
\end{gather*}
Como la matriz $X\cdot I-M_{1}$ es una matriz diagonal por bloques de la forma $X\cdot I_{q}-M$, se sigue que
$\det(X\cdot I_{n}-M_{1}) = (\det(X\cdot I_{q}-M))^{r}$. Como el miembro de la izquierda es $P(X)$
y $\det(X\cdot I_{q}-M)$ es el polinomio minimal de $x$ sobre $K$ por lo demostrado en la primera parte,
la proposici\'on queda demostrada. \QED

Para terminar, veamos un resultado sobre la traza y la norma de elementos enteros.

\begin{proposition}\label{prop2.6.2}
Sea $A$ un dominio \'integro, $K$ su cuerpo de fracciones, $L$ una extensi\'on de $K$ de grado finito
y $x$ un elemento de $L$ entero sobre $A$. Supongamos que $K$ es de caracter\'istica cero. Entonces los
coeficientes del polinomio caracter\'istico $P(X)$ de $x$ respecto a $L$ y $K$ (en particular, $\Tr_{L/K}(x)$
y $N_{L/K}(x)$), son enteros sobre $A$.
\end{proposition}

Utilizamos la proposici\'on~\ref{prop2.6.1}: se tiene $P(X) = (X-x_{1})\cdots(X-x_{n})$. Los coeficientes de $P(X)$ son, salvo
el signo, sumas de productos de los $x_{i}$. Basta demostrar que los $x_{i}$ son enteros sobre $A$
(\S\ref{sec2.1}, corolario~\ref{cor2.1.1} de la proposici\'on~\ref{prop2.1.1}).
Como cada $x_{i}$ es conjugado con $x$ sobre $K$ (\S\ref{sec2.4}), se tienen
$K$-isomorfismos $\sigma_{i}:K[x]\to K[x_{i}]$ tales que $\sigma_{i}(x) = x_{i}$. Aplicando $\sigma_{i}$
a una ecuaci\'on de dependencia entera de $x$ sobre $A$, se obtiene una ecuaci\'on de dependencia
entera de $x_{i}$ sobre $A$.

\begin{corollary*}
Supongamos adem\'as que $A$ es \'integramente cerrado. Entonces los coeficientes del polinomio caracter\'istico
de $x$ (en particular $\Tr_{L/K}(x)$ y $N_{L/K}(x)$) pertenecen a $A$.
\end{corollary*}

En efecto, los coeficientes son elementos de $K$ por definici\'on y son enteros sobre $A$ por la proposici\'on~\ref{prop2.6.2}.

\begin{comm}
Observemos que las cantidades $x+\sigma(x)$ y $x\cdot\sigma(x)$ que utilizamos en el estudio de los
cuerpos cuadr\'aticos (\S\ref{sec2.5}) son la traza y la norma de $x$. De hecho, probamos all\'i (\S\ref{sec2.5},~\eqref{eq-2.5-3})
un caso particular del corolario anterior.
\end{comm}

\section{Discriminante}\label{sec2.7}

\begin{definition}\label{def2.7.1}
Sea $B$ un anillo y $A$ un subanillo de $B$ tal que $B$ es un $A$-m\'odulo libre de rango finito $n$.
Si $(x_{1},\dots,x_{n})\in B^{n}$, llamamos discriminante del sistema $(x_{1},\dots,x_{n})$ al elemento
de $A$ definido por
\begin{gather}
D(x_{1},\dots,x_{n}) = \det(\Tr_{B/A}(x_{i}x_{j})).
\end{gather}
\end{definition}

\begin{proposition}\label{prop2.7.1}
Si $(y_{1},\dots,y_{n})\in B^{n}$ es otro sistema de elementos de $B$ tal que $y_{i} = \sum_{j=1}^{n}a_{ij}x_{j}$
con $a_{ij}\in A$, se tiene
\begin{gather}
D(y_{1},\dots,y_{n}) = \det(a_{ij})^{2}D(x_{1},\dots,x_{n}).
\end{gather}
\end{proposition}

En efecto, se tiene $\Tr(y_{p}y_{q}) = \Tr\left(\sum_{i,j}a_{pi}a_{qj}x_{i}x_{j}\right) = \sum_{i,j}a_{pi}a_{qj}
\Tr(x_{i}y_{j})$, de donde se sigue la igualdad de matrices
$(\Tr(y_{p}y_{q})) = (a_{pi})\cdot(\Tr(x_{i}x_{j}))\cdot{}^{t}(a_{qj})$ (donde ${}^{t}M$ denota le transpuesta de la
matriz $M$). Para terminar basta tomar determinantes en esta igualdad. \QED

Se sigue de la proposici\'on~\ref{prop2.7.1} que los discriminantes de dos bases cualesquiera de $B$ sobre $A$ son \emph{asociados}
en $A$: en efecto, la matriz de cambio de base $(a_{ij})$ es inversible por lo que su determinante es inveresible.
Por lo tanto, podemos hacer la

\begin{definition}\label{def2.7.2}
Bajo las hip\'otesis de la definici\'on~\ref{def2.7.1}, llamamos discriminante de $B$ sobre $A$, y lo notamos $\disc_{B/A}$,
al ideal principal de $A$ generado por el discriminante de una base cualquiera de $B$ sobre $A$.
\end{definition}

\begin{proposition}
Supongamos que $\disc_{B/A}$ contiene un elemento que no es divisor de cero. Entonces, para que un sistema
$(x_{1},\dots,x_{n})\in B^{n}$ sea una base de $B$ sobre $A$ es necesario y sufciente que $\disc_{B/A}$ est\'e
generado por $D(x_{1},\dots,x_{n})$.
\end{proposition}

La necesidad fue demostrada m\'as arriba. Supongamos que $d = D(x_{1},\dots,x_{n})$ genera
$\disc_{B/A}$. Sea $(e_{1},\dots,e_{n})$ une base cualquiera de $B$ sobre $A$.
Escribamos $d' = D(e_{1},\dots,e_{n})$ y $x_{i} = \sum_{j=1}^{n}a_{ij}e_{j}$ con $a_{ij}\in A$. Se tiene
$d = \det(a_{ij})^{2}d'$. Por hip\'otesis se tiene $Ad = \disc_{B/A} = Ad'$. Luego, existe $b\in A$ tal que
$d' = bd$, de donde $d(1-b\det(a_{ij})^{2}) = 0$. Se sigue que $d$ no puede ser un divisor de cero, pues sino
todo los elementos de $Ad = \disc_{B/A}$ ser\'ian divisores de cero. Por lo tanto, deducimos que
$1-b\det(a_{ij})^{2} = 0$, lo que muestra que $\det(a_{ij})$ es inversible y por lo tanto tambi\'en lo es
la matriz $(a_{ij})$. En consecuencia, $(x_{1},\dots,x_{n})$ es una base de $B$ sobre $A$.

\begin{proposition}\label{prop2.7.3}
Sea $K$ un cuerpo finito o de caracter\'istica cero, $L$ une extensi\'on de $K$ de grado finito $n$ y
$\sigma_{1},\dots,\sigma_{n}$ los $n$ $K$-isomorfismos distintos de $L$ en un cuerpo algebraicamente cerrado
$C$ que contiene a $K$ (\S\ref{sec2.4}, teorema~\ref{teo2.4.1}). Entonces si $(x_{1},\dots,x_{n})$ es una base de $L$ sobre $K$, se tiene
\begin{gather}
D(x_{1},\dots,x_{n}) = \det(\sigma_{i}(x_{j}))^{2}\neq 0.
\end{gather}
\end{proposition}

La primera igualdad resulta de un c\'alculo f\'acil:
\begin{align*}
D(x_{1},\dots,x_{n}) &= \det(\Tr(x_{i}x_{j})) = \det\left(\sum_{k}\sigma_{k}(x_{i}y_{j})\right)
= \det\left(\sum_{k}\sigma_{k}(x_{i})\sigma_{k}(x_{j})\right)\\
&= \det(\sigma_{k}(x_{i}))\cdot\det(\sigma_{k}(x_{j})) = \det(\sigma_{i}(x_{j}))^{2}.
\end{align*}

S\'olo resta demostrar que $\det(\sigma_{i}(x_{j}))\neq 0$. Razonemos por el absurdo. Si $\det(\sigma_{i}(x_{j})) = 0$,
existen $u_{1},\dots,u_{n}\in C$, no todos nulos, tales que $\sum_{i=1}^{n}u_{i}\sigma_{i}(x_{j}) = 0$ para todo $j$.
Por linealidad, se deduce $\sum_{i=1}^{n}u_{i}\sigma_{i}(x) = 0$ para todo $x\in L$. Esto contradice el siguiente
resultado:

\begin{named}{Lema de Dedekind}
Sea $G$ un grupo, $C$ un cuerpo y $\sigma_{1},\dots,\sigma_{n}$ homomorfismos distintos de $G$ en el grupo
multiplicativo $C^{*}$. Entonces los $\sigma_{i}$ son linealmente independientes sobre $C$ (i.e.
$\sum_{i}u_{i}\sigma_{i}(g) = 0$ para todo $g\in G$ implica que todos los $u_{i}$ son nulos{\upshape).}
\end{named}

Supongamos que los $\sigma_{i}$ son linealmente dependientes y consideremos una relaci\'on no trivial
$\sum_{i}u_{i}\sigma_{i} = 0$ ($u_{i}\in C$) tal que el n\'umero $q$ de los $u_{i}$ no nulos sea {\em
m\'inimo.} Renumerando si es necesario, podemos suponer que
\begin{gather}\label{eq-2.7-4}
u_{1}\sigma_{1}(g) +\dots+u_{q}\sigma_{q}(g) = 0\quad\text{para todo $g\in G$}.
\end{gather}
Necesariamente $q\geq 2$ pues los $\sigma_{i}$ no son todos nulos. Si $g$ y $h$ son dos elementos
cualesquiera de $G$, se tiene
\begin{gather*}
u_{1}\sigma_{1}(hg) + \dots + u_{q}\sigma_{q}(hg) = u_{1}\sigma_{1}(h)\sigma_{1}(g)+\dots+u_{q}\sigma_{q}(h)\sigma_{q}(g) = 0.
\end{gather*}
Multiplicando~\eqref{eq-2.7-4} por $\sigma_{1}(h)$ y restando de la \'ultima ecuaci\'on, obtenemos
\begin{gather*}
u_{2}(\sigma_{1}(h)-\sigma_{2}(h))\sigma_{1}(g)+\dots+u_{q}(\sigma_{1}(h)-\sigma_{q}(h))\sigma_{q}(g) = 0.
\end{gather*}
Como esta ecuaci\'on vale para todo $g\in G$ y $q$ se eligi\'o m\'inimo, se sigue que
$u_{2}(\sigma_{1}(h)-\sigma_{2}(h)) = 0$, de donde $\sigma_{1}(h) = \sigma_{2}(h)$ pues $u_{2}\neq 0$.
Esto contradice la hip\'otesis de que $\sigma_{i}$ son distintos. \QED

\begin{remark*}
En las condiciones de la proposici\'on~\ref{prop2.7.3}, la relaci\'on $D(x_{1},\dots,x_{n})\neq 0$ significa que
la forma bilineal $(x,y)\mapsto\Tr_{L/K}(xy)$ es \emph{no degenerada,} es decir que $\Tr_{L/K}(xy) = 0$
para todo $y\in L$ implica que $x = 0$. De esta manera, la aplicaci\'on $K$-lineal que a cada $x\in L$
le hace corresponder la forma $K$-lineal $s_{x}:y\mapsto\Tr_{L/K}(xy)$ es una inyecci\'on de $L$ en su
dual $\Hom_{K}(L,K)$ (con la estructura natural de espacio vectorial sobre $K$). Como $L$ y $\Hom_{K}(L,K)$
tienen la misma dimensi\'on finita $n$ sobre $K$ se sigue que $x\mapsto s_{x}$ es un isomorfismo. La
existencia de \emph{``bases duales''} en un espacios vectorial y su dual muestra entonces que, para toda
base $(x_{1},\dots,x_{n})$ de $L$ sobre $K$, existe otra base $(y_{1},\dots,y_{n})$ tal que
\begin{gather}\label{eq-2.7-5}
\Tr_{L/K}(x_{y}y_{j}) = \delta_{ij}\quad(1\leq i, j\leq n).
\end{gather}
Esta observaci\'on nos ser\'a \'util en lo que sigue.
\end{remark*}

\begin{theorem}\label{teo2.7.1}
Sea $A$ un anillo \'integramente cerrado, $K$ su cuerpo de fracciones, $L$ una extensi\'on de $K$ de grado
finito $n$ y $A'$ la clausura \'integra de $A$ en $L$. Supongamos que $K$ es de caracter\'istica cero.
Entonces $A'$ es un sub-$A$-m\'odulo  de un $A$-m\'odulo libre de rango $n$.
\end{theorem}

Sea $(x_{1},\dots,x_{n})$ una base de $L$ sobre $K$. Cada $x_{i}$ es algebraico sobre $K$, por lo que se tiene
$a_{n}x_{i}^{n}+a_{n-1}x_{i}^{n-1}+\dots+a_{0}$ con $a_{j}\in A$ para $j=0,\dots,n$. Multiplicando por
una potencia de $x_{i}$ podemos suponer que $a_{n}\neq 0$. Multiplicando por $a_{n}^{n-1}$ vemos que $a_{n}x_{i}$
es entero sobre $A$. Sea $x'_{i} = a_{n}x_{i}$. Entonces $(x'_{1},\dots,x'_{n})$ es una base de $L$ sobre $K$
contenida en $A'$.

Por la observaci\'on de m\'as arriba, existe otra base $(y_{1},\dots,y_{n})$ de $L$ sobre $K$ tal que
$\Tr(x'_{i}y_{j}) = \delta_{ij}$~\eqref{eq-2.7-5}. Sea $z\in A'$. Como $(y_{1},\dots,y_{n})$ es una base de $L$
sobre $K$, podemos escribir $z = \sum_{j=1}^{n}b_{j}y_{j}$ con $b_{j}\in K$. Para todo $i$ se tiene
$x'_{i}z\in A'$ (pues $x'_{i}\in A'$), de donde $\Tr(x'_{i}z)\in A$ (\S\ref{sec2.6}, corolario de la
proposici\'on~\ref{prop2.6.2}). Como
\begin{gather*}
\Tr(x'_{i}z) = \Tr\left(\sum_{j}b_{j}x'_{i}y_{j}\right) = \sum_{j}b_{j}\Tr(x'_{i}y_{j}) = \sum b_{j}\delta_{ij} = b_{i}
\end{gather*}
se sigue que $b_{i}\in A$ para todo $i$. Por lo tanto $A'$ est\'a contenido en el $A$-m\'odulo libre
$\sum_{j=1}^{n}Ay_{j}$. \QED

\begin{corollary*}
Bajo las hip\'otesis del teorema~\ref{teo2.7.1}, supongamos adem\'as que $A$ es un dominio de ideales principales. Entonces
$A'$ es un $A$-m\'odulo libre de rango $n$.
\end{corollary*}

En efecto, en este caso un subm\'odulo de un $A$-m\'odulo libre es de nuevo libre
(cap\'itulo~\ref{cap1}, \S\ref{sec1.5}, teorema~\ref{teo1.5.1}, {\itshape b}))
y de rango $\leq n$. Por otra parte, vimos en la demostraci\'on del teorema~\ref{teo2.7.1} que $A'$ contiene una base de $L$ sobre
$K$ y por lo tanto es de rango $n$.

\begin{comm}
A modo de ejercicio, el lector que no est\'e familiarizado con el contenido de la observaci\'on que
precede al teorema~\ref{teo2.7.1} puede tratar de encontrar una demostraci\'on m\'as calculadora ??? del siguiente teorema:
con la notaci\'on de arriba, sea $d = D(x'_{1},\dots,x'_{n})$ y supongamos que $z = \sum_{i}c_{i}x'_{i}$
($c_{i}\in K$) es entero sobre $A$. Entonces $dc_{i}\in A$ (calcular $\Tr(zx'_{j})$ y utilizar la f\'ormula de
Cramer).
\end{comm}

\subsection*{Un ejemplo de c\'alculo de discriminante}

Sea $K$ un cuerpo finito o de caracter\'istica cero, $L = K[x]$ una extension de $K$ de grado finito $n$ y
$F(X)$ el polinomio minimal de $x$ sobre $K$. Entonces
\begin{gather}\label{eq2.7.6}
D(1,x,\dots,x^{n-1})=(-1)^{\frac{n(n-1}{2}}N_{L/K}(F'(x))
\end{gather}
(donde $F'(X)$ denote el polinomio derivado de $F(X)$). En efecto, sean $x_{1},\dots,x_{n}$ las ra\'ices de
$F(X)$ en una extensi\'on de $K$. Estos son los conjugados de $x$
(\S\ref{sec2.3}, proposici\'on~\ref{prop2.3.3} y~\S\ref{sec2.4}). Se tiene
\begin{align*}
D(1,x,\dots,x^{n-1}) &= \det(\sigma_{i}(x^{j}))^{2}\text{ (proposici\'on~\ref{prop2.7.3})} = \det(x_{i}^{j})^{2}\\
(-1)^{\frac{n(n-1)}{2}}\det(x_{i}^{j})^{2} &= \prod_{i\neq j}(x_{i}-x_{j})\text{ (Vandermonde)} =
\prod_{i}\left(\prod_{j\neq i}(x_{i}-x_{j})\right)\\
&=\prod_{i}F'(x_{i}) = N_{L/K}(F'(x))
\end{align*}
(pues los $F'(x_{i})$ son los conjugados de $F'(x)$).

En particular, apliquemos~\eqref{eq2.7.6} al caso donde $F(X)$ es un \emph{trinomio}
$X^{n}+aX+b$ ($a,b\in K$). Sea $y = F'(x)$. Se tiene
\begin{gather*}
y = nx^{n-1} + x = -(n-1)a-nbx^{-1}
\end{gather*}
(pues $x^{n}+ax+b = 0$, de donde $nx^{n-1}=-na-nbx^{-1}$). Se deduce que
$x = -nb(y+(n-1)a)^{-1}$. El polinomio minimal de $y$ sobre $K$ es el numerador
de $F(-nb(Y+(n-1)a)^{-1})$. Haciendo las cuentas, calculamos que es
$(Y+(n-1)a)^{n}-na(Y+(n-1)a)^{n-1}+(-1)^{n}b^{n-1}$. La norma de $y$ es el producto
de $(-1)^{n}$ y el t\'ermino constante de este polinomio, es decir,
\begin{gather*}
n^{n}b^{n-1}+(-1)^{n-1}(n-1)^{n-1}a^{n}.
\end{gather*}
Se sigue que
\begin{gather}\label{eq2.7.7}
D(1,x,\dots,x^{n-1}) = \left[n^{n}b^{n-1}+(-1)^{n-1}(n-1)^{n-1}a^{n}\right](-1)^{\frac{n(n-1)}{2}}.
\end{gather}
Si $n=2$ (resp. $3$), recobramos la conocida f\'ormula $4b-a^{2}$ (resp. $-4a^{3}-27b^{2}$).

\section{Terminolog\'ia de los cuerpos de n\'umeros}\label{sec2.8}

Llamamos \emph{cuerpo de n\'umeros algebraicos} (o \emph{cuerpo de n\'umeros}) a toda extensi\'on de
$\QQ$ de grado finito (y por lo tanto algebraica). Dado un cuerpo de n\'umeros $K$, el grado
$[K:\QQ]$ se llama \emph{grado} de $K$. Un cuerpo de n\'umeros de grado $2$ (resp. $3$) se llama
\emph{cuerpo cuadr\'atico} (cf.~\S\ref{sec2.5}) (resp. \emph{cuerpo c\'ubico}). Un cuerpo de n\'umeros
tiene caracter\'istica cero.

Dado un cuerpo de n\'umeros $K$, los elementos de $K$ que son enteros sobre $\ZZ$ se llaman
\emph{enteros} de $K$. Forman un \emph{subanillo} $A$ de $K$ (\S\ref{sec2.1}, corolario~\ref{cor2.1.2}
de la proposici\'on~\ref{prop2.1.1}) que es un
$\ZZ$-m\'odulo \emph{libre} de rango $[K:\QQ]$ (\S\ref{sec2.7}, corolario del teorema~\ref{teo2.7.1}). Los discriminantes de
las diferentes bases del $\ZZ$-m\'odulo $A$ difieren en un elementos inversible de $\ZZ$ (\S\ref{sec2.7}, definici\'on~\ref{def2.7.2}),
que adem\'as es un cuadrado (\S\ref{sec2.7}, proposici\'on~\ref{prop2.7.1}). Luego este elemento es necesariamente $+1$, de manera
que los discriminantes de dos bases cualesquiera del $\ZZ$-m\'odulo $A$ son \emph{iguales;} su valor
en com\'un se llama \emph{discriminante absoluto,} o \emph{discriminante,} de $K$.

Como un cuerpo de n\'umeros $K$ determina de forma un\'ivoca al anillo $A$ de los enteros de $K$, a veces
haremos un abuso del lenguaje y le atribuiremos a $K$ propiedades de $A$. As\'i, cuando hablemos
de ideales (o de unidades) de $K$, se trata de ideales (o unidades) de $A$.

\section{Cuerpos ciclot\'omicos}\label{sec2.9}

Llamamos \emph{cuerpo ciclot\'omico} a todo cuerpo generado sobre $\QQ$ por ra\'ices de la unidad. Dado
un n\'umero primo $p$, denotamos $z$ una ra\'iz primitiva $p$-\'esima de la unidad (como elemento
de $\CC$ por ejemplo). Ahora estudiaremos el cuerpo ciclot\'omico $\QQ[z]$.
El n\'umero $z$ es ra\'iz del polinomio $X^{p}-1$. Como $z$ es $\neq 1$, tambi\'en es ra\'iz del polinomio
$\frac{X^{p}-1}{X-1} = X^{p-1}+X^{p-2}+\dots+X+1$, llamado \emph{polinomio ciclot\'omico.} No es inmediatamente
evidente el hecho de que este polinomio es irreducible sobre $\QQ$ (equivalentemente, $\QQ[z]$ tiene
grado $p-1$). Para demostrarlo nos har\'a falta el

\begin{named}{Criterio de Eisenstein}
Sea $A$ un anillo principal, $p\in A$ un elemento primo de $A$ y $F(X) = X^{n}+a_{n-1}x^{n-1}+\dots+a_{1}X+a_{0}$
un elemento de $A[X]$ tal que $p$ divide a todos los $a_{i}$ ($0\leq i\leq n-1${\upshape),} pero tal que $p^{2}$ no
divide a $a_{0}$. Entonces $F(X)$ es irreducible sobre el cuerpo de fracciones $K$ de $A$.
\end{named}

Supongamos en efecto que se tiene $F = G\cdot H$ con $G, H\in K[X]$, $G$ y $H$ \emph{m\'onicos.} Las
ra\'ices de $F$ son \emph{enteras} sobre $A$. Como todas las ra\'ices de $G$ (resp. $H$) son ra\'ices de $F$, \'estas
son todas enteras sobre $A$ (\S\ref{sec2.1}, corolario~\ref{cor2.1.1} de la proposici\'on~\ref{prop2.1.1}). Como $A$ es un dominio de ideales principales, y por
lo tanto \'integramente cerrado (\S\ref{sec2.2}, ex.~2), se tiene que $G\in A[X]$ y $H\in A[X]$.

Sean ahora $\oline F$, $\oline G$, $\oline H$ las im\'agenes de $F$, $G$, $H$ en $(A/Ap)[X]$, de manera que
$\oline F = \oline G\cdot\oline H$. Por la hip\'otesis en los $a_{i}$, se tiene $\oline F = X^{n}$.
Como $A/Ap$ es un \emph{dominio \'integro,} la descomposici\'on $X^{n}=\oline G\cdot\oline H$ es necesariamente
de la forma $X^{n}=X^{q}\cdot X^{n-q}$ (pues $\oline G$ y $\oline H$ son m\'onicos), de donde
$\oline G = X^{q}$ y $\oline H = X^{n-q}$. Si $G$ y $H$ son ambos no constantes, se sigue que
$p$ divide los t\'erminos constantes de $G$ y $H$ y por lo tanto $p^{2}$ divide al t\'ermino constante
$a_{0}$ de $F$, lo que contradice la hip\'otesis. Por lo tanto, o bien $G$ o bien $H$ es constante y por
lo tanto $F$ es irreducible. \QED

\begin{example*}
El polinomio $X^{3}-2X+6$ es irreducible sobre $\QQ$ (tomar $p=2$, $A=\ZZ$).
\end{example*}

\begin{theorem}\label{teo2.9.1}
Para todo n\'umero primo $p$, el polinomio ciclot\'omico $X^{p-1}+X^{p-2}+\dots+X+1$ es irreducible
en $\QQ[X]$.
\end{theorem}

En efecto, pongamos $X = Y+1$. Se tiene
\begin{align*}
X^{p-1}+\dots+1 = \frac{X^{p}-1}{X-1} &= \frac{(Y+1)^{p}-1}{Y}\\
&=Y^{p-1}+\sum_{j=p-1}^{1}\binom{p}{j}Y^{j-1}=F_{1}(Y).
\end{align*}
Como $p$ divide a todos los coeficientes binomiales $\binom{p}{j}$ pero $p^{2}$ no divide al
t\'ermino constante $\binom{p}{1} = p$, $F_{1}(Y)$ es irreducible por el criterio de Eisenstein y
por lo tanto tambi\'en lo es el polinomio ciclot\'omico. \QED

Seguimos denotando por $z$ una ra\'iz $p$-\'esima primitiva de la unidad. Resulta del
teorema~\ref{teo2.9.1} que el cuerpo
$\QQ[z]$ tiene grado $p-1$ y $(1,z,\dots,z^{p-2})$ es una base de  $\QQ[z]$ sobre $\QQ$. Ahora investigaremos el
anillo de enteros de $\QQ[z]$ y demostraremos que es $\ZZ[z]$.

Para ello, nos har\'a falta calcular algunas \emph{trazas y normas} (escribiremos $\Tr(x)$ y $N(x)$ en vez
de $\Tr_{\QQ[z]/\QQ}(x)$ y $N_{\QQ(z)/\QQ}(x)$). Observemos que los conjugados de $z$ sobre $\QQ$ son los
$z^{j}$, ($j=1,\dots,p-1$) (teorema~\ref{teo2.9.1}).

La irreducibilidad del polinomio ciclot\'omica implica:
\begin{gather}\label{eq-2.9-1}
\Tr(z) = -1,\quad\Tr(1)=p-1,
\end{gather}
de donde se sigue $\Tr(z^{j}) = -1$ para $j=1,\dots,p-1$ y por lo tanto
\begin{gather}\label{eq-2.9-2}
\Tr(1-z) = \Tr(1-z^{2})=\dots=\Tr(1-z^{p-1}) = p.
\end{gather}
Por otra parte el c\'alculo hecho en la demostraci\'on del teorema~\ref{teo2.9.1} muestra que $N(z-1)=(-1)^{p-1}p$, de donde
$N(1-z)=p$. Como la norma de $1-z$ es el producto de todos los conjugados de $1-z$, obtenemos
\begin{gather}\label{eq-2.9-3}
p=(1-z)(1-z^{2})\dots(1-z^{p-1}).
\end{gather}

Notemos $A$ al anillo de enteros de $\QQ[z]$. Evidentemente $A$ contiene a $z$ y a sus potencias. Ahora
demostraremos que
\begin{gather}\label{eq-2.9-4}
A(1-z)\cap\ZZ = p\ZZ.
\end{gather}
En efecto se tiene $p\in Z(1-z)$ por~\eqref{eq-2.9-3}, de donde $A(1-z)\cap\ZZ\supset p\ZZ$. Como $p\ZZ$
es un ideal maximal de $\ZZ$, la relaci\'on $A(1-z)\cap\ZZ\neq p\ZZ$ implicar\'ia $A(1-z)\cap\ZZ=\ZZ$ y
$1-z$ ser\'ia inversible en $A$. Entonces sus conjugados tambi\'en lo ser\'ian y por lo tanto $p$ ser\'ia
inversible por~\eqref{eq-2.9-4}. Es decir, $\frac{1}{p}$ ser\'ia entero sobre $\ZZ$, lo que es absurdo
(\S\ref{sec2.2}, ex.~2).

Mostremos por \'ultimo que, para todo $y\in A$, se tiene
\begin{gather}\label{eq-2.9-5}
\Tr(y(1-z))\in p\ZZ.
\end{gather}
En efecto cada conjugado $y_{j}(1-z^{j})$ de $y(1-z)$ es m\'ultiplo (en $A$) de $1-z^{j}$, que es a su
vez m\'ultiplo de $1-z$ pues
\begin{gather*}
1-z^{j} = (1-z)(1+z+\dots+z^{j-1}).
\end{gather*}
Como la traza es la suma de los conjugados se sigue que
\begin{gather*}
\Tr(y(1-z))\in A(1-z),
\end{gather*}
de donde se sigue~\eqref{eq-2.9-5} utilizando~\eqref{eq-2.9-4} y el hecho de que la traza de un entero
pertenece a $\ZZ$ (\S\ref{sec2.6}, corolario de la proposici\'on~\ref{prop2.6.2}).

Ahora estamos listos para determinar el anillo de enteros de $\QQ[z]$.

\begin{theorem}\label{teo2.9.2}
Sea $p$ un n\'umero primo y $z$ una ra\'iz primitiva $p$-\'esima de la unidad (en $\CC$). Entonces el
anillo $A$ de enteros del cuerpo ciclot\'omico $\QQ[z]$ es $\ZZ[z]$ y $(1,z,\dots,z^{p-2})$ es una base
del $\ZZ$-m\'odulo $A$.
\end{theorem}

En efecto, sea $x = a_{0}+a_{1}z+\dots+a_{p-2}z^{p-2}$ ($a_{i}\in\QQ$) un elemento de $A$. Entonces
se tiene
\begin{gather*}
x(1-z) = a_{0}(1-z)+a_{1}(z-z^{2})+\dots+a_{p-2}(x^{p-2}-z^{p-1})
\end{gather*}
Tomando trazas en ambos lados de la igualdad, resulta de~\eqref{eq-2.9-1} y~\eqref{eq-2.9-2} que
\begin{gather*}
\Tr(x(1-z)) = a_{0}\Tr(1-z)=a_{0}p,
\end{gather*}
de donde, utilizando~\eqref{eq-2.9-5}, $pa_{0}\in p\ZZ$ y por lo tanto $a_{0}\in\ZZ$. Como
$z^{-1}=z^{p-1}$ se sigue que $z^{-1}\in A$, de donde $(x-a_{0})z^{-1}=a_{1}+a_{2}z+\dots+a_{p-2}z^{p-3}\in A$.
Aplicando la primer parte del argumento a este elemento deducimos que $a_{1}\in\ZZ$. Aplicando sucesivamente
este procedimiento, vemos que todos los $a_{i}\in\ZZ$. \QED

\begin{remark*}
Lo que hicimios en esta \S{} se extiende sin dificultad a cuerpos ciclot\'omicos $\QQ[t]$ donde $t$ es una
ra\'iz primitiva $p^{r}$-\'esima de la unidad ($p$ primo). Un tal cuerpo tiene grado $p^{r-1}(p-1)$ y su
anillo de enteros es $\ZZ[t]$. El polinomio minimal de $t$ sobre $\QQ$ es
\begin{gather*}
X^{p^{r-1}(p-1)}+X^{p^{r-1}(p-2)}+\dots+X^{p^{r-1}}+1=\frac{X^{p^{r}}-1}{X^{p^{r-1}}-1}.
\end{gather*}
\end{remark*}

\section*{Ap\'endice. El cuerpo de los n\'umeros complejos es algebraicamente cerrado}%

Dado un cuerpo $K$, consideremos las siguientes propiedades:
\begin{enumerate}
\item[(a)] Todo polinomio de grado $>0$ sobre $K$ es un producto de polinomios lineales.
\item[(b)] Todo polinomio de grado $>0$ sobre $K$ admite una ra\'iz en $K$.
\end{enumerate}
Es claro que {\itshape a}) implica {\itshape b}). Reciprocamente, si {\itshape b}) es verdad,
$P(X)$ es un polinomio de grado $d\geq 1$ sobre $K$ y $a\in K$ es una ra\'iz de $P(X)$, entonces
$P(X)$ es un m\'ultiplo de $X-a$, y una inducci\'on en el grado $d$ muestra que {\itshape a}) es verdad.
Un cuerpo $K$ que satisface las condiciones equivalentes {\itshape a}) y {\itshape b}) se dice
\emph{algebraicamente cerrado.}

Mostraremos ahora que $\CC$ ($=\RR[i]$, $i^{2}=-1$) es algebraicamente cerrado utilizando un m\'etodo
que se debe esencialmente a Lagrange. De los n\'umeros reales s\'olamente utilizaremos
las siguientes propiedades:
\begin{enumerate}
\item[1.] Todo polinomio de grado impar sobre $\RR$ admite una ra\'iz en $\RR$; este es un caso f\'acil del teorema
del valor intermedio.
\item[2.] Todo polinomio de grado dos sobre $\CC$ tiene sus ra\'ices en $\CC$. Un c\'alculo f\'acil con
``$ax^{2}+bx+c=0$'' muestra que es suficiente demostrar que todo $z=a+bi\in\CC$ ($a,b\in\RR$) posee una
ra\'iz cuadrado en $\CC$. Como $(x+iy)^{2}=a+ib$ ($x,y\in\RR$) es equivalente a $x^{2}-y^{2}=a$,
$2xy = b$, se debe tener que $a^{2}+b^{2}=(x^{2}+y^{2})^{2}$ o $x^{2}+y^{2} = \sqrt{a^{2}+b^{2}}$. Deducimos
los valores de $x^{2}$ e $y^{2}$, de donde obtenemos aquellos de $x$ e $y$.
\item[3.] Dado un polinomio no constante $P(X)\in K[X]$, existe una extensi\'on $K'$ de $K$ tal que
$P(X)$ se descompone en factores lineales en $K'[X]$. Esto fue demostrado de manera sencilla en la
proposici\'on~\ref{prop2.3.3} de \S\ref{sec2.3} (aquella demostraci\'on es casi completamente independiente del material que la precede,
basta saber que, si $F(X)$ es irreducible, $K[X]/(F(X))$ es un cuerpo y hacer un demostraci\'on por inducci\'on).
\item[4.] Las relaciones entre los coeficientes y las ra\'ices de un polinomio.
\item[5.] El hecho de que un polinomio sim\'etrico $G(X_{1},\dots,X_{n})\in K[X_{1},\dots,X_{n}]$ es un polinomio
en las funciones sim\'etricas elementales $\sum X_{i}$, $\sum X_{i}X_{j}$, ..., $X_{1}\cdots X_{n}$ de las
$X_{i}$.
\end{enumerate}

Finalmente, tenemos el

\begin{theorem*}
El cuerpo de los n\'umeros complejos es algebraicamente cerrado.
\end{theorem*}

Demostraremos la parte (b), que todo polinomio no constante $P(X)\in\CC[X]$ admite una ra\'iz en
$\CC$. Considerando $F(X) = P(X)\oline P(X)$ ($\oline P$: el polinomio cuyos coeficientes son los
conjugados complejos de los coeficientes correspondientes de $P$) podemos suponer que $P(X)$ tiene
coeficientes reales: en efecto, si $a\in\CC$ es una ra\'iz de $F(X)$, entonces o bien $a$ es ra\'iz
de $P(X)$ o bien $a$ es ra\'iz de $\oline P(X)$ y en este caso $\oline a$ es ra\'iz de $P(X)$.
Ahora escribimos el grado de $F(X)$ ($\in\RR[X]$) en la manera $d = 2^{n}q$, donde $q$ es impar.
Razonaremos por inducci\'on en el \emph{exponente} $n$ de $2$.

Si $n = 0$, $d$ es impar y
$F(X)$ posee una ra\'iz en $\RR$. (cf.~1)). Supongamos que $n \geq 1$. Por 3) existe una extensi\'on
$K'$ de $\CC$ y elementos $x_{1},\dots,x_{d}\in K'$ tales que $F(X) = \prod_{i=1}^{d}(X-x_{i})$
(suponiendo que $F(X)$ es m\'onico, lo que est\'a permitido). Sea $c$ un elemento arbitrario de $\RR$;
consideremos los elementos $y_{ij} = x_{i}+x_{j}+cx_{i}x_{j}$ de $K'$ ($i\leq j$). Hay
$\frac{1}{2}d(d+1) = 2^{n-1}q(d+1)$ de estos n\'umeros y $q(d+1)$ es \emph{impar.} El polinomio
$G(X) = \prod_{i\leq j}(X-y_{ij})$ tiene coeficientes que son polinomios sim\'etricos el los $x_{i}$
con coeficientes reales. Por 5) son entonces necesariamente polinomios con coeficientes reales
en las funciones sim\'etricas elementales de los $x_{i}$ y por lo tanto los coeficientes de $G(X)$
son reales por 4). Como el grado es de la forma $2^{n-1}\times(\text{impar})$, le hip\'otesis
inductiva muestar que $G(X)$ admite una ra\'iz $z_{c}\in\CC$. Luego, alguno de los $y_{ij}$,
por ejemplo $y_{i(c),j(c)} = x_{i(c)}+y_{j(c)}+cx_{i(c)}x_{j(c)}$ es igual a $z_{c}$.

Como $\RR$ es \emph{infinito} y el conjunto de los pares $(i,j)$ ($i\leq j$) es finito, existen
dos n\'umeros reales distintos $c$, $c'$ tales que $i(c) = i(c')$ y $j(c) = j(c')$. Sean $r$, $s$
estos \'indices. Luego $x_{r}+x_{s}+cx_{r}x_{s} = z_{c}\in\CC$ y
\begin{gather*}
x_{r}+x_{s}+c'x_{r}x_{s}=z_{c'}\in\CC.
\end{gather*}
Restando esta ecuaci\'on de la an\'aloga con $c$ se deduce que $x_{r}+x_{s}\in\CC$ y $x_{r}x_{s}\in\CC$.
Luego, por 4), $x_{r}$ y $x_{s}$ son ra\'ices de una ecuaci\'on de segundo grado con coeficientes
en $\CC$. Como $\CC\subset K'$, se tiene que $x_{r},x_{s}\in\CC$ por 2). As\'i, hemos demostrado que
$F(X)$ posee una ra\'iz en $\CC$ y el teorema est\'a demostrado.

\begin{comm}
La demostraci\'on dada parece apropiada para un curso opcional o de primer ciclo, y tambi\'en
en una lecci\'on...
\end{comm}

\chapter{Anillos noetherianos y anillos de Dedekind}\label{cap3}

El lector que quiere saber porqu\'e introducimos los anillos de Dedekind puede dirigirse
a \S\ref{sec3.4} y leer el ejemplo y la discusi\'on que siguen al teorema~\ref{teo3.4.1}.
Los anillos noetherianos, de los cuales estudiaremos un m\'inimo de propiedades, son m\'as generales que los
anillos de Dedekind. Los introducimos para poder enunciar estas propiedades en su
nivel natural de generalidad y tambi\'en porque juegan un rol fundamental en otras aplicaciones
del \'algebra, por ejemplo en en Geometr\'ia Algebraica. Por \'ultimo, el pasaje de los anillos noetherianos
a los m\'odulos del mismo nombre es otro caso de ``linearizaci\'on'', una t\'ecnica cuya eficacia
el lector ya a podido comprobar.

\section{M\'odulos y anillos noetherianos}\label{sec3.1}

En el cap\'itulo~\ref{cap1}, \S\ref{sec1.4}, teorema~\ref{teo1.4.1} demostramos el siguiente resultado:

\begin{theorem}\label{teo3.1.1}
Sea $A$ un anillo y $M$ un $A$-m\'odulo. Las siguientes condiciones son equivalentes:
\begin{enumerate}
\item[a)] Toda familia no vac\'ia de subm\'odulos de $M$ posee un elemento maximal.
\item[b)] Toda sucesi\'on creciete de subm\'odulos de $M$ se estaciona.
\item[c)] Todo subm\'odulo de $M$ es de tipo finito.
\end{enumerate}
\end{theorem}

\begin{definition}
Un $A$-m\'odulo $M$ se dice noetheriano si satisface las condiciones equivalentes del
teorema~\ref{teo1.4.1}. Un anillo $A$ se dice noetheriano si, considerado como un $A$-m\'odulo, es
un m\'odulo noetheriano.
\end{definition}

Vimos (cap\'itulo~\ref{cap1}, \S\ref{sec1.4}, corolario del teorema~\ref{teo1.4.1}) que un
dominio de ideales principales es noetheriano.

\begin{proposition}\label{prop3.1.1}
Sea $A$ un anillo, $E$ un $A$-m\'odulo y $E'$ un subm\'odulo de $E$. Para que $E$ sea
noetheriano es necesario y suficiente que $E'$ y $E/E'$ sean noetherianos.
\end{proposition}

Demostremos la necesidad. Supongamos que $E$ es noetheriano. El conjunto ordenado de los
subm\'odulos de $E'$ (resp. $E/E'$) es isomorfo al conjunto ordenado de los subm\'odulos de $E$
contenidos en $E'$ (resp. conteniendo $E'$). Por lo tanto, $E'$ y $E/E'$ son noetherianos por {\itshape a})
o {\itshape b}).

Reciprocamente, supongamos que $E'$ y $E/E'$ son noetherianos. Sea $(F_{n})_{n\geq 0}$ una
sucesi\'on creciente de subm\'odulos de $E$. Como $E'$ es noetheriano, existe un entero
$n_{0}$ tal que $F_{n}\cap E' = F_{n+1}\cap E'$ para todo $n\geq n_{0}$. Como $E/E'$ es noetheriano,
existe un entero $n_{1}$ tal que
\begin{gather*}
(F_{n}+E')/E' = (F_{n+1}+E')/E'\text{ para todo }n\geq n_{1}.
\end{gather*}
Por lo tanto se tiene $F_{n}+E'=  F_{n+1}+E'$. Tomemos $n\geq\sup(n_{0},n_{1})$ y mostremos
que se tiene $F_{n} = F_{n+1}$. Bata ver que $F_{n+1}\subset F_{n}$. Sea $x\in F_{n+1}$.
Como $F_{n+1}+E' = F_{n}+E'$, existen $y\in F_{n}$ y $z',z''\in E'$ tales que
$x+z' = y+z''$. Luego, $x-y = z''-z'\in F_{n+1}\cap E'$. Como $F_{n+1}\cap E' = F_{n}\cap E'$,
se tiene $x-y\in F_{n}$, de donde $x\in F_{n}$ pues $y\in F_{n}$. As\'i $F_{n+1} = F_{n}$ para
todo $n\geq \sup(n_{0},n_{1})$ y $E$ es noetheriano por {\itshape b}).

\begin{corollary}\label{cor3.1.1}
Sea $A$ un anillo, $E_{1},\dots,E_{n}$ $A$-m\'odulos noetherianos. Entonces el $A$-m\'odulo
producto $\prod_{i=1}^{n}E_{i}$ es noetheriano.
\end{corollary}

Si $n=2$, $E_{1}$ se identifica con el subm\'odulo $E_{1}\times(0)$ de $E_{1}\times E_{2}$ y el
cociente correspondiente es isomorfo a $E_{2}$, por lo que el resultado se sigue en este caso
de la proposici\'on~\ref{prop3.1.1}. El caso general se deduce haciendo inducci\'on en $n$.

\begin{corollary}\label{cor3.1.2}
Sea $A$ un anillo noetheriano y $E$ un $A$-m\'odulo de tipo finito. Entonces $E$ es un
$A$-m\'odulo noetheriano {\upshape(}y por lo tanto todo subm\'odulo es de tipo finito{\upshape).}
\end{corollary}

En efecto (cap\'itulo~\ref{cap1}, \S\ref{sec1.4}), $E$ es isomorfo a un m\'odulo cociente $A^{n}/R$ (donde
$n$ es el cardinal de un sistema finito de generadores de $E$). Como $A^{n}$ es noetheriano
por el corolario~\ref{cor3.1.1}, $A^{n}/R$ tambi\'en lo es por la proposici\'on~\ref{prop3.1.1}.

\section{Aplicaci\'on a los elementos enteros}\label{sec3.2}

\begin{proposition}
Sea $A$ un anillo noetheriano \'integramente cerrado, $K$ su cuerpo de fracciones,
$L$ una extensi\'on de $K$ de grado finito $n$ y $A'$ la clausura \'integra de $A$ en
$L$. Supongamos que $K$ es de caracter\'istica cero. Entonce $A'$ es un $A$-m\'odulo de tipo finito
y un anillo noetheriano.
\end{proposition}

En efecto, sabemos que $A'$ es un subm\'odulo de un $A$-m\'odulo libre de rango $n$
(cap\'itulo~\ref{cap2}, \S\ref{sec2.7}, teorema~\ref{teo2.7.1}). Por lo tanto, $A'$ es un $A$-m\'odulo de tipo finito
(\S\ref{sec3.1}, corolario~\ref{cor3.1.2} de la proposici\'on~\ref{prop3.1.1}) y por lo tanto noetheriano (ibid). Por otra parte los ideales
de $A'$ son casos particulares de sub-$A$-m\'odulos de $A'$. Por lo tanto satisfacen la condici\'on
de maximalidad (\S\ref{sec3.1}, teorema~\ref{teo3.1.1}, {\itshape a}) de manera que $A'$ es un anillo noetheriano.

\begin{example*}
El anillo de los enteros de un cuerpo de n\'umeros $L$ es \emph{noetheriano} (poner $A = \ZZ$,
$K = \QQ$).
\end{example*}

\section{Algunos preliminares sobre ideales}\label{sec3.3}

Un ideal $\idl{p}$ de un anillo $A$ se dice \emph{primo} si el anillo cociente $A/\idl{p}$
es un \emph{dominio \'integro.} Equivalentemente, las condiciones $x\in A-\idl{p}$,
$y\in A-\idl{p}$ implican $xy\in A-\idl{p}$ o, en otras palabras, que el complemento
$A-\idl{p}$ de $\idl{p}$ es estable por multiplicaci\'on.

Para que un ideal $\idl{m}$ de $A$ sea \emph{maximal} (es decir, maximal entre los ideales de
$A$ distintos a $A$), es necesario y suficiente que $A/\idl{m}$ no tenga otros ideales que \'el mismo
y $(0)$. Es decir que $A/\idl{m}$ sea un \emph{cuerpo.}  En particular,
\emph{todo ideal maximal es primo.} La rec\'iproca es falsa, por ejemplo el ideal $(0)$ de $\ZZ$
es primo pero no maximal.

\begin{lemma}\label{lem3.3.1}
Sea $A$ un anillo, $\idl{p}$ un ideal primo de $A$ y $A'$ un subanillo de $A$. Entonces
$p\cap A'$ es un ideal primo de $A'$.
\end{lemma}

En efecto, $\idl{p}\cap A'$ es el n\'ucleo del homomorfismo compuesto $A'\to A\to A/\idl{p}$,
de manera que se tiene un homomorfismo inyectivo $A'/\idl{p}\cap A'\to A/\idl{p}$. Como un
subanillo de un dominio \'integro es un dominio \'integro, el resultado est\'a demostrado.

Dados dos ideales $\idl{a}$ y $\idl{b}$ de un anillo $A$, llamamos \emph{producto} de $\idl{a}$
y $\idl{b}$, y lo notamos $\idl{a}\idl{b}$, no s\'olamente al conjunto de los productos
$ab$, donde $a\in\idl{a}$ y $b\in\idl{b}$ (conjunto que en general no es un ideal), pero al
conjunto de \emph{sumas finitas} $\sum a_{i}b_{i}$ de tales productos. Es f\'acil ver que
$\idl{a}\idl{b}$ es un \emph{ideal} de $A$ y que se tiene
\begin{gather*}
\idl{a}\idl{b}\subset\idl{a}\cap\idl{b}.
\end{gather*}

\begin{comm}
En general no hay una igualdad: en un dominio de ideales principales el miembro de la izquierda
corresponde al producto, y el de la derecha a su m.c.m.
\end{comm}
El producto de ideales es asociativo y conmutativo y $A$ funciona como elemento neutro.

\begin{comm}
Dados un $A$-m\'odulo $E$, un subm\'odulo $F$ y un ideal $\idl{a}$ de $A$, definimos
de la misma manera el producto $\idl{a}F$; es un subm\'odulo de $E$.
\end{comm}

\begin{lemma}\label{lem3.3.2}
Si un ideal primo $\idl{p}$ de un anillo $A$ contiene un producto de ideales
$\idl{a}_{1}\idl{a}_{2}\cdots\idl{a}_{n}$, entonces $\idl{p}$ contiene uno de ellos.
\end{lemma}

En efecto, si $\idl{a}_{i}\not\subset\idl{p}$ para todo $i$, existe $a_{i}\in\idl{a}_{i}$ tal que
$a_{i}\notin\idl{p}$. Se tiene entonces que $a_{1}\cdots a_{n}\notin\idl{p}$ pues $\idl{p}$
es primo. Pero tambi\'en $a_{1}\cdots a_{n}\in\idl{a}_{1}\cdots\idl{a}_{n}$. Contradicci\'on.

\begin{lemma}\label{lem3.3.3}
En un anillo noetheriano, todo ideal contiene un producto de ideales primos. En un dominio \'integro noetheriano
$A$, todo ideal no nulo contiene un producto de ideales primos no nulos.
\end{lemma}

Utilizaremos un argumento t\'ipico de la teor\'ia de anillos noetherianos. Demostremos la segunda
afirmaci\'on (la demostraci\'on de la primera es an\'aloga: simplemente hay que borrar tres
veces ``no nulos''). Razonemos por el absurdo. Entonces la familia $\Phi$ de ideales no nulos de $A$ que
no contienen ning\'un producto de ideales primos no nulos es \emph{no vac\'ia.} Como $A$ es noetheriano,
$\Phi$ admite un elemento \emph{maximal} $\idl{b}$ (\S\ref{sec3.1}, teorema~\ref{teo3.1.1}, {\itshape a})).
El ideal $\idl{b}$ no es primo pues
sino contendr\'ia el producto consistente en s\'i mismo. Luego, existen $x, y\in A-\idl{b}$ tal que
$xy\in\idl{b}$. Los ideales $\idl{b}+Ax$ y $\idl{b}+Ay$ contienen estrictamente a $\idl{b}$, por lo que
no pertenecen a $\Phi$ pues $\idl{b}$ es maximal en $\Phi$. Por lo tanto contienen productos de ideales
primos no nulos:
\begin{gather*}
\idl{b}+Ax\supset\idl{p}_{1}\cdots\idl{p}_{n},\quad\idl{b}+Ay\supset\idl{q}_{1}\cdots\idl{q}_{r}
\end{gather*}
Como $xy\in\idl{b}$, se tiene que
\begin{gather*}
(\idl{b}+Ax)(\idl{b}+Ay)\subset\idl{b};\quad\text{de donde}\quad\idl{p}_{1}\cdots\idl{p}_{n}\idl{q}_{1}\cdots
\idl{q}_{r}\subset\idl{b},
\end{gather*}
lo que es una contradicci\'on.

Sea ahora $A$ un \emph{dominio \'integro} y $K$ su cuerpo de fracciones. Llamamos \emph{ideal fraccionario} de $A$
(o de $K$ respecto a $A$) a todo sub-$A$-m\'odulo $I$ de $K$ tal que existe $d\in A$, $d\neq 0$ que satisface
$I\subset d^{-1}A$. Esto quiere decir que los elementos de $I$ tienen un ``denominador com\'un'' $d\in A$.
Los ideales ordinarios de $A$ son ideales fraccionarios (con $d=1$). Los llamamos \emph{ideales enteros} si
hay riezgo de confusi\'on. Todo sub-$A$-m\'odulo \emph{de tipo finito} $I$ de $K$ es un ideal fraccionario.
En efecto, si $(x_{1},\dots,x_{n})$ es un sistema finito de generadores de $I$, los $x_{i}$ tienen un denominador
com\'un $d$ (por ejemplo el producto de los denominadores $d_{i}$, donde $x_{i} = a_{i}d_{i}^{-1}$ con
$a_{i},d_{i}\in A$) y $d$ sirve de denominador com\'un de $I$. Rec\'iprocamente, si $A$ es \emph{noetheriano,}
todo ideal fraccionario $I$ es un $A$-m\'odulo de \emph{tipo finito:} en efecto, se tiene $I\subset d^{-1}A$ y
$d^{-1}A$ es un $A$-m\'odulo isomorfo a $A$, luego noetheriano.

Definimos el \emph{producto} $II'$ de dos ideales fraccionario $I$, $I'$ como el conjunto de sumas finitas
$\sum x_{i}y_{i}$ donde $x_{i}\in I$ y $y_{i}\in I'$. Si $I$ e $I'$ son ideales fraccionarios con denominadores
comunes $d$ y $d'$, entonces los conjuntos
\begin{gather*}
I\cap I',\quad I+I',\quad II'
\end{gather*}
son \emph{ideales fraccionarios.} En efecto, es claro que son sub-$A$-m\'odulos de $K$ y admiten como denominador
com\'un, respectivamente, a $d$ (o $d'$), $dd'$ y $dd'$. Los ideales fraccionarios no nulos de $A$ forman un
\emph{monoide} conmutativo con la multiplicaci\'on.

\section{Anillos de Dedekind}\label{sec3.4}

\begin{definition}
Un anillo $A$ se llama \emph{anillo de Dedekind} si es noetheriano
e \'integramente cerrado {\upshape(}en particular,
un dominio \'integro{\upshape)} y si todo ideal primo no nulo de $A$ es maximal.
\end{definition}

El anillo $\ZZ$ y, m\'as generalmente, todo dominio de ideales principales es un anillo de Dedekind.
El anillo de enteros de un cuerpo de n\'umeros es un anillo de Dedekind como consecuencia del teorema
siguiente:

\begin{theorem}\label{teo3.4.1}
Sea $A$ un anillo de Dedekind, $K$ su cuerpo de fracciones, $L$ una extensi\'on de $K$ de grado finito y
$A'$ la clausura \'integra de $A$ en $L$. Supongamos que $K$ tiene caracter\'istica cero. Luego,
$A'$ es un anillo de Dedekind y un $A$-m\'odulo de tipo finito.
\end{theorem}

En efecto, $A'$ es \'integramente cerrado por construcci\'on, noetheriano y $A$-m\'odulo de tipo finito
por la proposici\'on de \S\ref{sec3.2}. Resta demostrar que todo idel primo $\idl{p}'\neq(0)$ de $A'$ es maximal. Tomemos
un elemento $x\neq 0$ de $\idl{p}'$ y consideremos una ecuaci\'on de dependencia entera de $x$ sobre $A$ de
grado m\'inimo:
\begin{gather}\label{eq-3.4-1}
x^{n}+a_{n-1}x^{n-1}+\dots+a_{1}x+a_{0} = 0\quad(a_{i}\in A)
\end{gather}
Necesariamente $a_{0}\neq 0$, pues sino podr\'iamos dividir por $x$ y obtendr\'iamos una ecuaci\'on de dependencia
\'integral de grado $n-1$. Por~\eqref{eq-3.4-1}, se tiene que $a_{0}\in A'x\cap A\subset\idl{p}'\cap A$. Por lo tanto,
$\idl{p}'\cap A\neq(0)$. Como $\idl{p}'\cap A$ es un ideal primo de $A$ (\S\ref{sec3.3}, lema~\ref{lem3.3.1}), $\idl{p}'\cap A$
es un ideal maximal de $A$, y $A/\idl{p'}\cap A$ es un cuerpo. Pero $A/\idl{p}'\cap A$ se identifica con un
subanillo de $A'/\idl{p}'$ y $A'/\idl{p'}$ es \emph{entero} sobre $A/\idl{p}'\cap A$ pues $A'$ es entero
sobre $A$. Luego, $A'/\idl{p}'$ es un cuerpo (cap\'itulo~\ref{cap2}, \S\ref{sec2.1}, proposici\'on~\ref{prop2.1.3}), de manera que $\idl{p}'$
es maximal. \QED

El interes en los anillos de Dedekind proviene de que el anillo de enteros de un cuerpo de n\'umeros es un
anillo de Dedekind pero no siempre un dominio de ideales principales.

\begin{example*}
Consideremos el anillos de enteros $A = \ZZ[\sqrt{-5}]$ de $\QQ[\sqrt{-5}]$
(cap\'itulo~\ref{cap2}, \S\ref{sec2.5}, teorema~\ref{teo2.5.1}). Se tiene
\begin{gather}\label{eq-3.4-2}
(1+\sqrt{-5})(1-\sqrt{-5}) = 2\cdot 3.
\end{gather}
Las normas de los cuatro factores son, respectivamente, $6$, $6$, $4$ y $9$. Luego
$1+\sqrt{-5}$ no puede tener un divisor no trivial en $A$, pues la norma de un tal divisor
ser\'ia un divisor no trivial de $6$, pero las ecuaciones
\begin{gather*}
a^{2}+5b^{2}=2\quad\text{y}\quad a^{2}+5b^{2}=3
\end{gather*}
no tienen soluciones en $\ZZ$. Si $A$ es un dominio de ideales principales, el elemento $1+\sqrt{-5}$
que divide al producto $2\cdot 3$ por~\eqref{eq-3.4-2} (BOOK IS WRONG?), deber\'ia dividir a alguno de los factores, pero entonces,
tomando normas, $6$ dividir\'ia a $4$ o a $9$, lo que es imposible.
\end{example*}

Hist\'oricamente, el te\'orico de n\'umeros Kummer (1810--1893) se di\'o cuenta de que
ciertos anillos de enteros de cuerpos de n\'umeros (de hecho, de cuerpos ciclot\'omicos, relacionados
con sus trabajos sobre la ecuaci\'on de Fermat; cf.~\ref{cap1}, \S\ref{sec1.2}) no eran dominios de ideales principales. Para
superar, al menos en parte, estos inconvenientes, Kummer y Dedekind (1831--1916) introdujeron la noci\'on
de \emph{ideal,} y Dedekind estudi\'o los anillos que hoy llevan su nombre. El principal inter\'es de los dominios
de ideales principales es la existencia de descomposici\'on \'unica en factores primos. En los anillos de
Dedekind, esto se generaliza felizmente a una descomposici\'on \'unica en \emph{ideales primos,} que es \'util
en muchos sentidos y que describiremos a continuaci\'on:

\begin{theorem}\label{teo3.4.2}
Sea $A$ un anillo de Dedekind que no es un cuerpo. Todo ideal maximal de $A$ es inversible en el monoide de
ideales fraccionarios de $A$.
\end{theorem}

Sea $\idl{m}$ un ideal maximal de $A$. Se tiene $\idl{m}\neq(0)$ pues $A$ no es un cuerpo. Sea
\begin{gather}\label{eq-3.4-3}
\idl{m}' = \{x\in K\mid x\idl{m}\subset A\}
\end{gather}
Es claro que $\idl{m}'$ es un sub-$A$-m\'odulo de $K$ y que admite como denominador com\'un cualquier
elemento no nulo de $\idl{m}$. Por lo tanto, $\idl{m}'$ es un ideal fraccionario de $A$. Basta
demostrar que $\idl{m'}\idl{m} = A$.  Como $\idl{m}'\idl{m}\subset A$ por~\eqref{eq-3.4-3} y es claro que
$A\subset\idl{m}'$ (pues $\idl{m}$ es un ideal) se sigue que $\idl{m}=A\idl{m}\subset\idl{m'}\idl{m}$.
Como $\idl{m}$ es maximal y $\idl{m}\subset\idl{m'}\idl{m}\subset A$ se deduce que, o bien $\idl{m'}\idl{m} = A$,
o bien $\idl{m}'\idl{m} = \idl{m}$. Solo queda probar que $\idl{m}'\idl{m}=\idl{m}$ es imposible.

Si $\idl{m}'\idl{m} = \idl{m}$ y $x\in\idl{m}'$ se tiene $x\idl{m}\subset\idl{m}$ de donde $x^{2}\idl{m}\subset
x\idl{m}\subset\idl{m}$ y, por inducci\'on, $x^{n}\idl{m}\subset\idl{m}$ para todo $n\in\NN$. Luego, cualquier
elemento no nulo $d$ de $\idl{m}$ sirve de denominador com\'un para todos los $x^{n}$, de manera que
$A[x]$ es un ideal fraccionario de $A$. Como $A$ es noetheriano, $A[x]$ es un $A$-m\'odulo de tipo finito
(\S\ref{sec3.3}, al final), por lo que $x$ es \emph{entero} sobre $A$ (cap\'itulo~\ref{cap2}, \S\ref{sec2.1}, teorema~\ref{teo2.1.1}).
Como $A$ es \'integramente
cerrado, se tiene $x\in A$. Hemos demostrado que $\idl{m}'\idl{m}=\idl{m}$ implica $\idl{m}'=A$. Resta ver que
$\idl{m}'=A$ es imposible.

En efecto, tomemos un elemento no nulo $a\in\idl{m}$. El ideal $Aa$ contiene un producto
$\idl{p}_{1}\idl{p}_{2}\cdots\idl{p}_{n}$ de ideales primos no nulos (\S\ref{sec3.3}, lema~\ref{lem3.3.3}). Podemos suponer que $n$
es m\'inimo. Se tiene $\idl{m}\supset Aa\supset\idl{p}_{1}\cdots\idl{p}_{n}$ por lo que $\idl{m}$ debe
contener algun $\idl{p}_{i}$ (\S\ref{sec3.3}, lema~\ref{lem3.3.2}), por ejemplo $\idl{p}_{1}$. Como $\idl{p}_{1}$ es maximal por
hip\'otesis, se sigue que $\idl{m} = \idl{p}_{1}$. Sea $\idl{b} = \idl{p}_{2}\cdots\idl{p}_{n}$, de manera
que $Aa\supset\idl{m}\idl{b}$, pero $Aa\not\supset\idl{b}$ pues $n$ es minimal. Existe entonces un elemento
$b\in\idl{b}$ tal que $b\notin Aa$. Como $\idl{m}\idl{b}\subset Aa$, se tiene $\idl{m}b\subset Aa$, de donde
$\idl{m}ba^{-1}\subset A$. La definici\'on~\eqref{eq-3.4-3} de $\idl{m}'$ implica que $ba^{-1}\in\idl{m}'$. Pero como
$b\notin Aa$, se sigue que $ba^{-1}\notin A$ y por lo tanto $\idl{m}'\neq A$. \QED

\begin{theorem}\label{teo3.4.3}
Sea $A$ un anillo de Dedekind, $P$ el conjunto de ideales primos no nulos de $A$.
\begin{enumerate}
\item Todo ideal fraccionario no nulo $\idl{b}$ de $A$ se escribe de una \'unica manera de la forma
\begin{gather}\label{eq-3.4-4}
\idl{b} = \prod_{\idl{p}\in P}\idl{p}^{n_{\idl{p}}(\idl{b})}
\end{gather}
donde los $n_{\idl{p}}(\idl{b})$ son n\'umeros enteros, casi todos nulos.
\item El monoide de ideales fraccionarios no nulos de $A$ es un grupos.
\end{enumerate}
\end{theorem}

Demostremos primero la afirmaci\'on de \emph{existencia} de {\itshape a}), es decir que todo ideal fraccionario
$\idl{b}$ es producto de potencias ($\geq 0$ o $\leq 0$) de ideles primos. Sabemos que existe un
elemento no nulo de $A$ tal que $d\idl{b}\subset A$, i.e. tal que $d\idl{b}$ es un ideal entero de $A$.
Como $\idl{b} = (d\idl{b})\cdot(Ad)^{-1}$, podemos suponer que $\idl{b}$ es un ideal entero. Ahora
argumentamos como en el lema~\ref{lem3.3.3} de la \S\ref{sec3.3} y consideramos la familia $\Phi$ de los ideales $\neq(0)$ de
$A$ que no son productos de ideales primos. Supongamos, por el absurdo, que $\Phi$ es no vac\'ia. Entonces
admite un elemento maximal $\idl{a}$ pues $A$ es noetheriano. Se tiene $\idl{a}\neq A$ pues $A$ es el
producto de la familia vac\'ia de ideales primos. Luego $\idl{a}$ est\'a contenido en un ideal maximal
$\idl{p}$, a saber un elemento maximal de la familia de los ideales no triviales de $A$ que contienen
a $\idl{a}$. Sea $\idl{p}'$ el ideal (fraccionario) inverso a $\idl{p}$ Como $\idl{a}\subset\idl{p}$ se deduce que
$\idl{a}\idl{p}'\subset\idl{p}\idl{p}' = A$. Como $\idl{p}'\supset A$, se tiene $\idl{a}\idl{p}'\supset\idl{a}$,
e incluso $\idl{a}\idl{p}'\neq\idl{a}$: en efecto, si $\idl{a}\idl{p}' = \idl{a}$ y $x\in\idl{p}'$, se tiene
$x\idl{a}\subset\idl{a}$, $x^{n}\idl{a}\subset\idl{a}$ para todo $n$, $x$ es entero sobre $A$ y luego $x\in A$
(como en el teorema~\ref{teo3.4.2}). Pero esto es imposible pues $\idl{p}'\neq A$ (sino $\idl{p}' = A$ y $\idl{p}\idl{p}'=\idl{p}$).
Como $\idl{a}$ es maximal en $\Phi$, se sigue que $\idl{a}\idl{p}'\in\Phi$ y por lo tanto $\idl{a}\idl{p}'$ debe
ser un producto $\idl{p}_{1}\cdots\idl{p}_{n}$ de ideales primos. Multiplicando por $\idl{p}$, vemos que
$\idl{a} = \idl{p}\idl{p}_{1}\cdots\idl{p}_{n}$. Es decir, todo ideal entero de $A$ es un producto de ideales primos.

Pasemos ahora a la afirmaci\'on de \emph{unicidad} de {\itshape a}). Supongamos que se tiene $\prod_{p\in P}\idl{p}^{n(\idl{p})}
=\prod_{\idl{p}\in P}\idl{p}^{m(\idl{p})}$, es decir $\prod_{\idl{p}\in A}\idl{p}^{n(\idl{p})-m(\idl{p})}=A$.
Si los $n(\idl{p})-m(\idl{p})$ no son todos nulos, podemos separar los exponentes en positivos y negativos y
obtener:
\begin{gather}
\idl{p}_{1}^{\alpha_{1}}\cdots\idl{p}_{r}^{\alpha_{r}} = \idl{q}_{1}^{\beta_{1}}\cdots\idl{q}_{s}^{\beta_{s}}
\end{gather}
con $\idl{p}_{i},\idl{q}_{j}\in P$, $\alpha_{i} > 0$, $\beta_{j} > 0$, $\idl{p}_{i}\neq\idl{q}_{j}$ para todo
$i, j$. Como $\idl{p}_{1}$ contiene a $\idl{q}_{1}^{\beta_{1}}\cdots\idl{q}_{s}^{\beta_{s}}$, contiene
a unos de los $\idl{q}_{j}$ (\S\ref{sec3.3}, lema~\ref{lem3.3.2}), supongamos $\idl{p}_{1}\supset\idl{q}_{1}$. Como
$\idl{p}_{1}$ y $\idl{q}_{1}$ son ambos maximales, esto implica que $\idl{p}_{1}=\idl{q}_{1}$, lo que
es una contradicci\'on.

Por \'ultimo,~\eqref{eq-3.4-4} muestra que $\prod_{\idl{p}\in P}\idl{p}^{-n_{\idl{p}}(\idl{b})}$ es el inverso
de $\idl{b}$, lo que demuestra {\itshape b}).

\begin{remark*}
Acabamos de ver que el monoide $I(A)$ de ideales fraccionarios no nulos de un anillo de Dedekind $A$
es un grupo. Los ideales fraccionarios \emph{principales} (es decir, los de la forma $Ax$, $x\in K^{*}$)
forman un subgrupo $F(A)$ de $I(A)$ (pues $(Ax)\cdot(Ay)^{-1} = Axy^{-1}$). El grupo cociente
$C(A) = I(A)/F(A)$ se llama el \emph{grupo de clases de ideales} de $A$. Para que $A$ sea un dominio
de ideales principales es necesario y suficiente que $C(A)$ consista \'unicamente del elemento neutro.
\end{remark*}

Terminemos con un \emph{formulario,} en el cual $n_{\idl{p}}(\idl{b})$ denote el exponente de $\idl{p}$
en la descomposici\'on de $\idl{b}$ como producto de ideales primos (cf.~\eqref{eq-3.4-4}).
\begin{gather}\label{eq-3.4-6}
n_{\idl{p}}(\idl{a}\idl{b}) = n_{\idl{p}}(\idl{a})+n_{\idl{p}}(\idl{b})\quad(\text{trivial})
\end{gather}
\begin{gather}\label{eq-3.4-7}
\idl{b}\subset A\iff n_{\idl{p}}(\idl{b})\geq 0\text{ para todo }\idl{p}\in P
\end{gather}
($\implies$ se sigue de la demostraci\'on del teorema~\ref{teo3.4.3}; $\Leftarrow$ es trivial)
\begin{gather}\label{eq-3.4-8}
\idl{a}\subset\idl{b}\iff n_{\idl{p}}(\idl{a})\geq n_{\idl{p}}(\idl{b})\text{ para todo }\idl{p}\in P.
\end{gather}
(en efecto, $\idl{a}\subset\idl{b}$ equivale a $\idl{a}\idl{b}^{-1}\subset A$; aplicamos~\eqref{eq-3.4-6} y~\eqref{eq-3.4-7})
\begin{gather}
n_{\idl{p}}(\idl{a}+\idl{b}) = \inf(n_{\idl{p}}(\idl{a}),n_{\idl{p}}(\idl{b}))
\end{gather}
(pues $\idl{a}+\idl{b}$ es el supremo de $\idl{a}$ y $\idl{b}$ para la inclusi\'on de ideales; para
terminar aplicar~\eqref{eq-3.4-8})
\begin{gather}
n_{\idl{p}}(\idl{a}\cap\idl{b}) = \sup(n_{\idl{p}}(\idl{a}),n_{\idl{p}}(\idl{b}))
\end{gather}
(raz\'on an\'aloga, las desigualdades se invierten en~\eqref{eq-3.4-8})

\section{Norma de un ideal}\label{sec3.5}

\begin{trivlist}\item
{\itshape En toda esta \S, $K$ denota un cuerpo de n\'umeros, $n$ su grado y $A$ el anillo
de enteros de $K$.} Escribimos $N(x)$ en lugar de $N_{K/\QQ}(x)$.
\end{trivlist}

\begin{proposition}\label{prop3.5.1}
Si $x$ es un elemento no nulo de $A$, se tiene $\abs{N(x)} = \card(A/Ax)$.
\end{proposition}

\begin{comm}
Notemos que, como $x\in A$, se tiene $N(x)\in\ZZ$ (cap\'itulo~\ref{cap2}, \S\ref{sec2.6}, corolario de la
proposici\'on~\ref{prop2.6.2}), de manera que
la f\'ormula anterior tiene sentido.
\end{comm}

Sabemos que $A$ es un $\ZZ$-m\'odulo libre de rango $n$ (cap\'itulo~\ref{cap2}, \S\ref{sec2.8}) y $Ax$ es un
sub-$\ZZ$-m\'odulo de $A$. Tambi\'en es de rango $n$ pues la multiplicaci\'on por $x$
es un isomorfismo $A\to Ax$. Por lo visto en el cap\'itulo~\ref{cap1}, \S\ref{sec1.5}, teorema~\ref{teo1.5.1}, existe una base
$(e_{1},\dots,e_{n})$ del $\ZZ$-m\'odulo $A$ y elementos $c_{i}$ de $\NN$ tales que
$(c_{1}e_{1},\dots,c_{n}e_{n})$ es una base de $Ax$. Luego $A/Ax$ es isomorfo a
$\prod_{i=1}^{n}\ZZ/c_{i}\ZZ$ y su cardinal es $c_{1}c_{2}\cdots c_{n}$. Sea $u$ la
aplicaci\'on $\ZZ$-lineal de $A$ sobre $Ax$ definida por $u(e_{i}) = c_{i}e_{i}$ para
$i=1,\dots,n$. Se tiene $\det(u) = c_{1}\cdots c_{n}$.

Por otra parte, $(xe_{1},\dots,xe_{n})$ es otra base de $Ax$. Por lo tanto tenemos un automorfismo
$v$ del $\ZZ$-m\'odulo $Ax$ tal que $v(c_{i}e_{i}) = xe_{i}$. Como $\det(v)$ es inversible en $\ZZ$,
se tiene que $\det(v)=\pm 1$. Pero entonces $v\circ u$ no es otra cosa que multiplicaci\'on por $x$
y su determinante es, por definici\'on, $N(x)$ (cap\'itulo~\ref{cap2}, \S\ref{sec2.6},
definici\'on~\ref{def2.6.1}). Como $\det(v\circ u)=
\det(v)\cdot\det(u)$, deducimos que $N(x) = \pm c_{1}c_{2}\cdots c_{n}=\pm \card(A/Ax)$.% \QED

\begin{definition}\label{def3.5.1}
Dado un ideal entero no nulo $\idl{a}$ de $A$, llamamos norma de $\idl{a}$, y notamos $N(\idl{a})$,
al n\'umero $\card(A/\idl{a})$.
\end{definition}

Observemos que $N(\idl{a})$ es \emph{finito.} En efecto, si $a$ es un elemento no nulo de $\idl{a}$,
se tiene $Aa\subset\idl{a}$ y $A/\idl{a}$ se identifica con un cociente de $A/Aa$. De esto que
$\card(A/\idl{a})\leq\card(A/Aa)$, que es finito por la proposici\'on~\ref{prop3.5.1}. Al mismo tiempo, esto demuestra que
para un ideal principal $Ab$ se tiene $N(Ab) = \abs{N(b)}$.

\begin{proposition}\label{prop3.5.2}
Si $\idl{a}$ y $\idl{b}$ son dos ideales enteros no nulos de $A$, se tiene
$N(\idl{a}\idl{b}) = N(\idl{a})N(\idl{b})$.
\end{proposition}

Descomponiendo $\idl{b}$ en un producto de ideales maximales (\S\ref{sec3.4}, teorema~\ref{teo3.4.3}), vemos que
basta demostrar que se tiene $N(\idl{a}\idl{m}) = N(\idl{a})N(\idl{m})$ para $\idl{m}$ maximal.
Como $\idl{a}\idl{m}\subset\idl{a}$, tenemos $\card(A/\idl{a}\idl{m}) =
\card(A/\idl{a})\card(\idl{a}/\idl{a}\idl{m})$. Luego, basta probar que $\card(\idl{a}/\idl{a}\idl{m})
=\card(A/\idl{m})$. Ahora bien, $\idl{a}/\idl{a}\idl{m}$ es un $A$-m\'odulo anulado por $\idl{m}$, y por
lo tanto, un espacio vectorial sobre $A/\idl{m}$. Sus subespacios vectoriales son sub-$A$-m\'odulos y por
lo tanto son de la forma $\idl{q}/\idl{a}\idl{m}$, donde $\idl{q}$ es un ideal tal que
$\idl{a}\idl{m}\subset\idl{q}\subset\idl{a}$. Pero la f\'ormula~\eqref{eq-3.4-8} de \S\ref{sec3.4}
muestra que no hay
ning\'un ideal contenido estrictamente entre $\idl{a}\idl{m}$ y $\idl{a}$. Luego el espacio vectorial
$\idl{a}/\idl{a}\idl{m}$ es de dimensi\'on uno sobre $A/\idl{m}$ y se tiene en consecuencia que
$\card(\idl{a}/\idl{a}\idl{m}) = \card(A/\idl{m})$. \QED

\chapter{Clases de ideales y el teorema de las unidades}\label{cap4}

El presente cap\'itulo est\'a consagrado a dos teorema importantes de finitud. Nos ser\'an
\'utiles algunas herramientas del An\'alisis (tomadas prestadas de la Topolog\'ia y la integraci\'on
en $\RR^{n}$).

\section{Preliminares sobre subgrupos discretos de $\RR^{n}$}\label{sec4.1}

Un subgrupo aditivo $H$ de $\RR^{n}$ es discreto si y s\'olo si para todo compacto $K$ de $\RR^{n}$,
la intersecci\'on $H\cap K$ es finita. Un ejemplo t\'ipico de subgrupo discreto de $\RR^{n}$ es $\ZZ^{n}$.
Ahora demostraremos que este es esencialmente el \'unico:

\begin{theorem}\label{teo4.1.1}
Sea $H$ un subgrupo discreto de $\RR^{n}$. Entonces $H$ est\'a generado {\upshape(}como $\ZZ$-m\'odulo\/{\upshape)} por
$r$ vectores linealmente independientes sobre $\RR$ {\upshape(}en particular, $r\leq n${\upshape).}
\end{theorem}

Elijamos, en efecto, un sistema $(e_{1},\dots,e_{r})$ de elementos de $H$ que sean linealmente independientes
sobre $\RR$ y tal que $r$ sea m\'aximo. Sea
\begin{gather}
P = \left\{\sum_{i=1}^{r}\alpha_{i}e_{i}\mid 0\leq\alpha_{i}\leq 1\right\}\subset\RR^{n}
\end{gather}
el paralelotopo ???? construido con estos vectores. Es claro que $P$ es compacto y por lo tanto
$P\cap H$ es finito. Sea $x\in H$. Como $r$ es maximal, $x$ se escribe $x = \sum_{i=1}^{r}\lambda_{i}e_{i}$
con $\lambda_{i}\in\RR$. Consideremos entonces, para $j\in\ZZ$, el elemento
\begin{gather}\label{eq4.1.2}
x_{j} = jx-\sum_{i=1}^{r}[j\lambda_{i}]e_{i}
\end{gather}
(donde $[\mu]$ denota la parte entera de $\mu\in\RR$). Se tiene que
\begin{gather*}
x_{j} = \sum_{i=1}^{r}(j\lambda_{i}-[j\lambda_{i}])e_{i},
\end{gather*}
de donde $x_{j}\in P$ y $x_{j}\in P\cap H$ por~\eqref{eq4.1.2}. Si observamos que $x = x_{1}+\sum_{i=1}^{r}[\lambda_{i}]e_{i}$,
vemos que el $\ZZ$-m\'odulo $H$ est\'a generado por $P\cap H$, y por lo tanto es de \emph{tipo finito.}

Por otra parte, como $P\cap H$ es finito y $\ZZ$ es infinito, existen dos enteros distintos $j$ y $k$
tales que $x_{j} = x_{k}$. Se sigue entonces de~\eqref{eq4.1.2} que, para estos enteros, se tiene
$(j-k)\lambda_{i} = [j\lambda_{i}]-[k\lambda_{i}]$, lo que muestra que los $\lambda_{i}$ son
\emph{racionales.} As\'i el $\ZZ$-m\'odulo $H$ est\'a generado por un
n\'umero \emph{finito} de elementos,
todos ellos combinaciones lineales de los $(e_{i})$ con coeficientes racionales.
Sea $d$ un denominador
com\'un ($d\in\ZZ$, $d\neq 0$) de estos coeficientes. Se tiene entonces
que $dH\subset\sum_{i=1}^{r}\ZZ e_{i}$.
Por lo tanto existe una base $(f_{i})$ del $\ZZ$-m\'odulo $\sum_{i=1}^{r}\ZZ e_{i}$ y elementos
$\alpha_{i}\in\ZZ$ tales que $(\alpha_{1}f_{1},\dots,\alpha_{r}f_{r})$ generan $dH$
(cap\'itulo~\ref{cap1}, \S\ref{sec1.5}, teorema~\ref{teo1.5.1}).
Como el $\ZZ$-m\'odulo tiene el mismo rango que $H$ y $H\supset\sum_{i=1}^{r}\ZZ e_{i}$,
el rango de
$dH$ es $\geq r$ y por lo tanto es igual a $r$ y los $\alpha_{i}$ son todos no nulos.
Como los $(f_{i})$
son, al igual que los $(e_{i})$, linealmente independientes sobre $\RR$, vemos que $dH$,
y por lo tanto
tambi\'en $H$, est\'a generado (sobre $\ZZ$) por $r$ elementos linealmente independientes
sobre $\RR$.

\begin{remark*}[Ejemplo de aplicaci\'on]
Sea $t = (\theta_{1},\dots,\theta_{n})\in\RR^{n}$ tal que alguno de
los $\theta_{i}$ es \emph{irracional.} Sea $(e_{1},\dots,e_{n})$ la base can\'onica
de $\RR^{n}$ y $H$
el subgrupo de $\RR^{n}$ generado sobre $\ZZ$ por $(e_{1},\dots,e_{n},t)$. $H$ no es
discreto pues sino,
aplicando el teorema~\ref{teo4.1.1}, $t$ ser\'ia combinaci\'on lineal racional de
los $e_{i}$. Por lo tanto, dado
$\varepsilon > 0$, existen un elemento no nulo de $H$ a distancia menor que $\varepsilon$
del $0$. Por
lo tanto existen enteros $p_{i}\in\ZZ$, $q\in\NN$, $q\neq 0$, tales que
$\abs{q\theta_{i}-p_{i}}\leq\varepsilon$,
es decir $\abs{\theta_{i}-\frac{p_{1}}{q}}\leq\frac{\varepsilon}{q}$. Observemos que
la estimaci\'on simplista de $\theta_{i}$ entre dos m\'ultiplos consecutivos de
$\frac{1}{q}$ s\'olo
produce la aproximaci\'on $\abs{\theta_{i}-\frac{n_{i}}{q}}\leq\frac{1}{2q}$ ($n_{i}\in\ZZ$).
Este resultado en uno de los primeros resultados de la rica teor\'ia de aproximaci\'on de
n\'umeros
irracionales por n\'umeros racionales. Para m\'as detalles sobre esta teor\'ia, ver
Koksma, ``Diophantische Approximationen'', Berlin (Springer), 1936.
\end{remark*}

\begin{definition}
Un subgrupo discreto de rango $n$ de $\RR^{n}$ se llama un reticulado de $\RR^{n}$.
\end{definition}

Por el teorema~\ref{teo4.1.1}, un reticulado est\'a generado sobre $\ZZ$ por una base
de $\RR^{n}$ que es entonces
una $\ZZ$-base del mismo. Para cada $\ZZ$-base $e = (e_{1},\dots,e_{n})$ de un
reticulado $H$, denotamos
por $P_{e}$ el paralep\'ipedo semi-abierto
$P_{e} = \left\{\sum_{i=1}^{n}\alpha_{i}e_{i}\mid 0\leq \alpha_{i}<1\right\}$;
As\'i todo punto de $\RR^{n}$ es congruente m\'odulo $H$ a un \'unico punto de $P_{e}$
(decimos en este
caso que $P_{e}$ es un \emph{dominio fundamental} para $H$). Escribiremos $\mu$ para
la \emph{medida de
Lebesgue} en $\RR^{n}$, de forma que, para todo subconjunto medible $S$ de $\RR^{n}$,
$\mu(S)$ denotar\'a
su medida (que nosotros llamaremos tambi\'en su vol\'umen).

\begin{lemma}
El vol\'umen $\mu(P_{e})$ es independiente de la base $e$ elegida.
\end{lemma}

En efecto, sea $(f_{1},\dots,f_{n})$ otra base de $H$. Se tiene
$f_{i} = \sum_{j=1}^{n}\alpha_{ij}e_{j}$
para algunos $\alpha_{ij}\in\ZZ$. Es bien sabido el efecto de una transformaci\'on lineal
en el vol\'umen y
se tiene $\mu(P_{f}) = \abs{\det(\alpha_{ij})}\mu(P_{e})$. Pero, como es un determinante de
un cambio de base,
$\det(\alpha_{ij})$ es inversible en $\ZZ$, por lo tanto igual a $\pm 1$. Se concluye
que $\mu(P_{f}) = \mu(P_{e})$.

El vol\'umen de uno cualquiera de los $P_{e}$ se llama el \emph{vol\'umen del reticulado} $H$
y se nota
$v(H)$ (la palabra ``vol\'umen'' aqu\'i es un abuso del lenguaje, pues $\mu(H) = 0$. ?`Tal
vez ser\'ia mejor decir ``malla'' del reticulado $H$?).

\begin{theorem}[Minkowski]\label{teo4.1.2}
Sea $H$ un reticulado de $\RR^{n}$ y $S$ un subconjunto medible de $\RR^{n}$ tal
que $\mu(S) > v(H)$. Entonces existen
dos elementos distintos $x$, $y$ de $S$ tales que $x-y\in H$.
\end{theorem}

En efecto, sea $e = (e_{1},\dots,e_{n})$ una $\ZZ$-base de $H$ y $P_{e}$ el
paralep\'ipedo semi-abierto construido
con $e$. Como $P_{e}$ es un dominio fundamental para $H$, $S$ es la uni\'on disjunta de
los $S\cap (h+P_{e})$
($h\in H$). Se sigue que
\begin{gather}\label{eq-4.1-3}
\mu(S) = \sum_{h\in H}\mu(S\cap (h+P_{e}))
\end{gather}
Como $\mu$ es invariante por translaciones, se tiene
\begin{gather*}
\mu(S\cap (h+P_{e})) = \mu((-h+S)\cap P_{e})
\end{gather*}
Por otra parte, los conjuntos $(-h+S)\cap P_{e}$ ($h\in H$) no pueden ser disjuntos dos a
dos, pues de ser as\'i,
$\mu(P_{e})\geq\sum_{h\in H}\mu((-h+S)\cap P_{e})$, que contradice~\eqref{eq-4.1-3} y la
hip\'otesis $\mu(P_{e})=v(H)<\mu(S)$.
Luego existen dos elementos distintos $h$, $h'$ de $H$ tales que
$P_{e}\cap (-h+S)\cap (-h'+S)\neq\emptyset$.
Por lo tanto se tienen dos elementos $x$, $y$ de $S$ tales que $-h+x=-h'+y$, de donde
$x-y=h-h'\in H$ y
$x\neq y$ pues $h\neq h'$. \QED

\begin{corollary*}
Sea $H$ un reticulado de $\RR^{n}$, $S$ un subconjunto medible, sim\'etrico respecto a
$0$ y convexo en $\RR^{n}$.
Supongamos que una de las siguientes condiciones es verdad:
\begin{enumerate}
\item[a)] se tiene $\mu(S) > 2^{n}v(H)$
\item[b)] se tiene $\mu(S)\geq 2^{n}v(H)$ y $S$ es compacto.
\end{enumerate}
Entonces $S\cap H$ contiene un punto distinto a $0$.
\end{corollary*}

En el caso a), aplicamos el teorema~\ref{teo4.1.1} a
\begin{gather*}
S'=\frac{1}{2}S\quad\left(\text{pues }\mu(S')=\frac{1}{2^{n}}\mu(S)>v(H)\right);
\end{gather*}
luego existen dos puntos distintos $z$, $y$ de $S'$ tales que $y-z\in H$. Entonces
$x=y-z=\frac{1}{2}(2y+(-2z))$ es un punto de $S$ (pues $S$ es sim\'etrico y convexo),
que satisface la conclusi\'on. En el caso b), aplicamos el caso a) a
$(1+\varepsilon)S)$ ($\varepsilon > 0$). Poniendo $H' = H-\{0\}$, vemos que
$H'\cap (1+\varepsilon)S$ es no vac\'io, y es finito pues es compacto y discreto. Entonces
$\bigcap_{\varepsilon>0}H'\cap (1+\varepsilon)S$ es no vac\'io. Un elemento de esta intersecci\'on pertenece
a $\bigcap_{\varepsilon>0}(1+\varepsilon)S$, conjunto que es igual a $S$ pues $S$ es compact. \QED

\begin{comm}
La hip\'otesis de compacidad es necesaria en b), como lo muestra el paralep\'ipedo abierto
$\left\{\sum_{i=1}^{n}\lambda_{i}e_{i}\mid-1<\lambda_{i}<+1\right\}$ y el reticulado de base $(e_{i})$.
\end{comm}

\section{La inmersi\'on can\'onica de un cuerpo de n\'umeros}\label{sec4.2}

Sea $K$ un cuerpo de n\'umeros y $n$ su grado. Vimos
(cap\'itulo~\ref{cap2}, \S\ref{sec2.4}, teorema~\ref{teo2.4.1}) que se tienen $n$
isomorfismos distintos $\sigma_{i}:K\to\CC$. Tenemos exactamente $n$ pues el polinomio minimal de
un elemento primitovo de $K$ sobre $\QQ$ (\emph{ibid.}, corolario del teorema~\ref{teo2.4.1}) tiene $n$ ra\'ices en $\CC$. Sea
$\alpha:\CC\to\CC$ la conjugaci\'on compleja. Entonces, para todo $i$, $\alpha\circ\sigma_{i}$ es uno
de los $\sigma_{j}$ y es igual a $\sigma_{i}$ si y s\'olo si $\sigma_{i}(K)\subset\RR$. Sea $r_{1}$ el n\'umero
de aquellos $i$ tales que $\sigma_{i}(K)\subset K$. Los restantes son un n\'umero par $2r_{2}$ y
se tiene
\begin{gather}\label{eq-4.2-1}
r_{1}+2r_{2}=n.
\end{gather}
Numeraremos los $\sigma_{i}$ de manera que $\sigma_{i}(K)\subset\RR$ si $1\leq i\leq r_{1}$ y
$\sigma_{j+r_{2}}(x) = \oline{\sigma_{j}(x)}$ si $r_{1}+1\leq j\leq r_{1}+r_{2}$. As\'i, los $r_{1}+r_{2}$
primeros $\sigma_{i}$ determinan los $r_{2}$ restantes. Si $x\in K$, escribimos
\begin{gather}
\sigma(x) = (\sigma_{1}(x),\dots,\sigma_{r_{1}+r_{2}}(x))\in\RR^{r_{1}}\times\CC^{r_{2}}
\end{gather}
Llamaremos a $\sigma$ la \emph{inclusi\'on can\'onica} de $K$ en $\RR^{r_{1}}\times\CC^{r_{2}}$. Es un
homomorfismo inyectivo para las correspondientes estructuras de anillos. Generalmente identificaremos
$\RR^{r_{1}}\times\CC^{r_{2}}$ con $\RR^{n}$ (cf.~\eqref{eq-4.2-1}). Las notaciones $\sigma$, $K$, $n$, $r_{1}$, $r_{2}$
ser\'an utlizadas en el resto de esta \S.

\begin{proposition}\label{prop4.2.1}
Si $M$ es un sub-$\ZZ$-m\'odulo libre de rango $n$ de $K$ y si $(x_{i})_{1\leq i\leq n}$ es una $\ZZ$-base
de $M$, luego $\sigma(M)$ es un reticulado de $\RR^{n}$, cuyo vol\'umen est\'a dado por
\begin{gather}\label{eq-4.2-3}
v(\sigma(M)) = 2^{-r_{2}}\abs{\det_{1\leq i,j\leq n}(\sigma_{i}(x_{j}))}
\end{gather}
\end{proposition}

En efecto, para $i$ fijo, las coordenadas de $\sigma(x_{i})$ respecto a la base can\'onica de $\RR^{n}$ est\'an
dadas por
\begin{gather}\label{eq-4.2-4}
\sigma_{1}(x_{i}),\dots,\sigma_{r_{1}}(x_{i}),R(\sigma_{r_{1}+1}(x_{i})),I(\sigma_{r_{1}+1}(x_{i})),\dots,
R(\sigma_{r_{1}+r_{2}}(x_{i})),I(\sigma_{r_{1}+r_{2}}(x_{i}))
\end{gather}
donde $R$ e $I$ denotan la parte real y la parte imaginaria. Calculemos el determinante $D$ cuyo $i$-\'esima
columna est\'a dada por~\eqref{eq-4.2-4}. Utilizando las f\'ormulas $R(z) = \frac{1}{2}(z+\oline z)$ y
$I(z) = \frac{1}{2i}(z-\oline z)$ ($z\in\CC$) y la linearidad respcto a las filas, obtenemos $D = \pm (2i)^{-r_{2}}
\det(\sigma_{j}(x_{i}))$. Como los $x_{i}$ forman una base de $K$ sobre $\QQ$ se tiene $\det(\sigma_{j}(x_{i}))\neq 0$
(cap\'itulo~\ref{cap2}, \S\ref{sec2.7}, proposici\'on~\ref{prop2.7.3})
y por lo tanto $D\neq 0$. Por lo tanto los vectores $\sigma(x_{i})$ son linealmente
independientes en $\RR^{n}$, de manera que el $\ZZ$-m\'odulo que general (es decir, $\sigma(M)$) es un
reticulado de $\RR^{n}$. El c\'alculo de $D$ que acabamos de hacer muestra que su vol\'umen est\'a dado
por~\eqref{eq-4.2-3}.

\begin{proposition}\label{prop4.2.2}
Sea $d$ el discriminante absoluto de $K$, $A$ su anillo de enteros y $\idl{a}$ un ideal entero no nulo de $A$.
Entonces $\sigma(A)$ y $\sigma(\idl{a})$ son reticulados y se tiene
\begin{gather}\label{eq-4.2-5}
v(\sigma(A)) = 2^{-r_{2}}\abs{d}^{1/2}\quad v(\sigma(\idl{a})) = 2^{-r_{2}}\abs{d}^{1/2}N(\idl{a}).
\end{gather}
\end{proposition}

En efecto, sabemos que $A$ y $\idl{a}$ son $\ZZ$-m\'odulos libres de rango $n$, de manera que podemos aplicar
la proposici\'on~\ref{prop4.2.1}.
Por otra parte, si $(x_{i})$ es una $\ZZ$-base de $A$, se tiene
$d = \det(\sigma_{i}(x_{j}))^{2}$
(cap\'itulo~\ref{cap2}, \S\ref{sec2.7}, proposici\'on~\ref{prop2.7.3}).
De esto se sigue la primer f\'ormula~\eqref{eq-4.2-5}. La segunda se deduce observando que
$\sigma(\idl{a})$ es un subgrupo de $\sigma(A)$ de \'indice $N(\idl{a})$
(cap\'itulo~\ref{cap3}, \S\ref{sec3.5}, definici\'on~\ref{def3.5.1}) y que por
lo tanto un dominio fundamental para $\sigma(\idl{a})$ se obtiene como uni\'on disjunta de $N(\idl{a})$ copias
de un dominio fundamental de $\sigma(A)$.

\section{Finitud del grupo de clases de ideales}\label{sec4.3}

\begin{proposition}\label{prop4.3.1}
Sea $K$ un cuerpo de n\'umeros, $n$ su grado, $r_{1}$ y $r_{2}$ los enteros definidos al comienzo de
\S\ref{sec4.2}, $d$ su discriminante absoluto y $\idl{a}$ un ideal entero no nulo de $K$. Entonces $\idl{a}$
contiene un elemento no nulo $x$ tal que
\begin{gather}\label{eq-4.3-1}
\abs{N_{K/\QQ}(x)}\leq\left(\frac{4}{\pi}\right)^{r_{2}}\frac{n!}{n^{n}}\abs{d}^{1/2}N(\idl{a}).
\end{gather}
\end{proposition}

En efecto, sea $\sigma$ la inmersi\'on can\'onica de $K$ en $\RR^{r_{1}}\times\CC^{r_{2}}$ (\S\ref{sec4.2}).
Sea $t$ un n\'umero real $>0$ y $B_{t}$ el conjunto de aquellos
\begin{gather*}
(y_{1},\dots,y_{r_{1}},z_{1},\dots,z_{r_{2}})\in\RR^{r_{1}}\times\CC^{r_{2}}
\end{gather*}
tales que
\begin{gather}\label{eq-4.3-2}
\sum_{i=1}^{r_{1}}\abs{y_{i}}+2\sum_{j=1}^{r_{2}}\abs{z_{j}}\leq t.
\end{gather}
Entonces $B_{t}$ es un conjunto compact, convexo, sim\'etrico respecto al origen y veremos en el ap\'endice
que su vol\'umen es
\begin{gather}
\mu(B_{t}) = 2^{r_{1}}\left(\frac{\pi}{2}\right)^{r_{r}}\frac{t^{n}}{n!}.
\end{gather}
Elijamos $t$ tal que $\mu(B_{t}) = 2^{n}v(\sigma(\idl{a}))$, es decir tal que
\begin{gather*}
2^{r_{1}}\left(\frac{\pi}{2}\right)^{r_{2}}\frac{t^{n}}{n!} = 2^{n-r_{2}}\abs{d}^{1/2}N(\idl{a})
\end{gather*}
(\S\ref{sec4.2}, proposici\'on~\ref{prop4.2.2}) o, equivalentemente, $t^{n} = 2^{n-r_{1}}\pi^{-r_{2}}n!\abs{d}^{1/2}N(\idl{a})$.
Por el corolario del teorema~\ref{teo4.1.2}, \S\ref{sec4.1},
existe un elemento $x$ de $\idl{a}$ no nulo tal que $\sigma(x)\in B_{t}$.
Ahora estimemos la norma $\abs{N(x)} = \prod_{i=1}^{r_{1}}\abs{\sigma_{i}(x)}\prod_{j=r_{1}+1}^{r_{1}+r_{2}}\abs{\sigma_{j}(x)}^{2}$.
La desigualdad de la media geom\'etrica muestra que se tiene
\begin{gather*}
\abs{N(x)}\leq\left[\frac{1}{n}\sum_{i=1}^{r_{1}}\abs{\sigma_{i}(x)}+\frac{2}{n}\sum_{j=r_{1}+1}^{r_{1}+r_{2}}
\abs{\sigma_{j}(x)}\right]^{n}\leq\frac{t^{n}}{n^{n}}\quad(\text{por~\eqref{eq-4.3-2}})
\end{gather*}
de donde $\abs{N(x)}\leq\frac{1}{n^{n}}2^{n-r_{1}}\pi^{-r_{2}}n!\abs{d}^{1/2}N(\idl{a})$, lo que es equivalente
a~\eqref{eq-4.3-1} ya que $r_{1}+2r_{2} = n$. \QED

\begin{corollary}\label{cor4.3.1}
Con la misma notaci\'on que antes, toda clase de ideales de $K$ (cap\'itulo~\ref{cap3}, \S\ref{sec3.4})
contiene un ideal entero $\idl{b}$ tal que
\begin{gather}\label{eq-4.3-4}
N(\idl{b})\leq\left(\frac{4}{\pi}\right)^{r_{2}}\frac{n!}{n^{n}}\abs{d}^{1/2}.
\end{gather}
\end{corollary}

En efecto, sea $\idl{a}'$ un ideal perteneciente a la clase dada. Multiplicando por un escalar, podemos suponer
que $\idl{a} = \idl{a}'^{-1}$ es un ideal entero. Tomemos un elemento $x$ no nulo de $\idl{a}$ tal que~\eqref{eq-4.3-1}
sea verdad. Entonces $\idl{b} = x\idl{a}^{-1}$ es un ideal entero perteneciente a la clase dada cuya norma
satisface~\eqref{eq-4.3-4} utilizando la multiplicatividad de las normas (cap\'itulo~\ref{cap3}, \S\ref{sec3.5},
proposici\'on~\ref{prop3.5.2}).

\begin{corollary}\label{cor4.3.2}
Sea $K$ un cuerpo de n\'umeros, $n$ su grado y $d$ su discriminante absoluto. Entonces si $n\geq 2$, se tiene
$\abs{d}\geq\frac{\pi}{3}\left(\frac{3\pi}{4}\right)^{n-1}$ y $\frac{n}{\log\abs{d}}$ est\'a acotado por una
constante independiente de $K$.
\end{corollary}

En efecto, como $N(\idl{b})\geq 1$, se tiene $\abs{d}^{1/2}\geq\left(\frac{\pi}{4}\right)^{r_{2}}\frac{n^{n}}{n!}$.
Como $\frac{\pi}{4} < 1$ y $2r_{2}\leq n$, se tiene $\abs{d}\geq a_{n}$, donde
$a_{n} = \left(\frac{\pi}{4}\right)^{r_{2}}\frac{n^{2n}}{(n!)^{2}}$. Ahora bien, se tiene $a_{2} = \frac{\pi^{2}}{4}$
y $\frac{a_{n+1}}{a_{n}} = \frac{\pi}{4}\left(1+\frac{1}{n}\right)^{2n} = \frac{\pi}{4}(1+2+\text{t\'erminos positivos})
\text{ (por la f\'ormula del binomio)} \geq \frac{3\pi}{4}$. De donde $n\geq 2$,
$\abs{d}\geq\frac{\pi^{2}}{4}\left(\frac{3\pi}{4}\right)^{n-2}$,
o que implica la desigualdad deseada. La estimaci\'on uniforme de $\frac{n}{\log\abs{d}}$ se sigue tomando logaritmos.

\begin{theorem}[Hermite-Minkowski]
Para todo cuerpo de n\'umeros $K\neq\QQ$, el discriminante absoluto $d$ de $K$ es $\neq\pm 1$.
\end{theorem}

En efecto, por el corolario~\ref{cor4.3.2}, se tiene $\abs{d}\geq\frac{\pi}{3}\left(\frac{3\pi}{4}\right)^{n-1}$ y
$\frac{\pi}{3} >1$, $\frac{3\pi}{4} > 1$ de donde se sigue que $\abs{d} > 1$.

\begin{theorem}[Dirichlet]\label{teo4.3.2}
Para todo cuerpo de n\'umeros $K$, el grupo de clases de ideales de $K$ es finito (cap\'itulo~\ref{cap3}, \S\ref{sec3.4}).
\end{theorem}

En virtud del corolario~\ref{cor4.3.1} de la proposici\'on~\ref{prop4.3.1},
basta demostrar que el conjunto de ideales enteros $\idl{b}$
de $K$, cuya norma es un entero dado $q$, es finito. Como para un tal ideal $\idl{b}$ se tiene que
$\card(A/\idl{b}) = q$ (cap\'itulo~\ref{cap3}, \S\ref{sec3.5}), se sigue que $q\in\idl{b}$ pues en un grupo el orden de un elemento
divide al orden del grupo. Por lo tanto los ideales $\idl{b}$ considerados son algunos de aquellos que contienen
a $Aq$, y hay un n\'umero finito de estos \'ultimos (cap\'itulo~\ref{cap3}, \S\ref{sec3.4}, f\'ormula~\eqref{eq-3.4-8},
o bien por la finitud de
$A/Aq$).% \QED

\begin{theorem}[Hermite]\label{teo4.3.3}
En $\CC$ hay s\'olo un n\'umero finito de cuerpos de n\'umeros de discriminante $d$ dado.
\end{theorem}

En efecto, por el corolario~\ref{cor4.3.2} de la proposici\'on~\ref{prop4.3.1}, el grado de un tal cuerpo est\'a acotado. Luego podemos suponer
que $n$, y $r_{1}$ y $r_{2}$ est\'an fijos.

En $\RR^{r_{1}}\times\CC^{r_{2}}$ consideramos el siguiente conjunto $B$:
\begin{enumerate}
\item Si $r_{1} > 0$, $B$ es el conjunto de los $(y_{1},\dots,y_{r_{1}},z_{1},\dots,z_{r_{2}})\in\RR^{r_{1}}
\times\CC^{r_{2}}$ tales que
\begin{align}\label{eq-4.3-5}
\abs{y_{1}}\leq 2^{n}\left(\frac{\pi}{2}\right)^{-r_{2}}\abs{d}^{1/2},\quad\abs{y_{i}} &\leq\frac{1}{2}\text{ para }
i=2,\dots,r_{1},\\
\abs{z_{j}} &\leq\frac{1}{2}\text{ para }j=1,\dots,r_{2}.\notag
\end{align}
\item Si $r_{1} = 0$, $B$ es el conjunto de los $(z_{1},\dots,z_{r_{2}})\in\CC^{r_{2}}$ tales que
\begin{align}\label{eq-4.3-6}
\abs{z_{1}-\oline z_{1}} \leq 2^{n}\frac{8}{\pi}\left(\frac{\pi}{2}\right)^{-r_{2}}\abs{d}^{1/2},\quad
\abs{z_{1}+\oline z_{1}} &\leq \frac{1}{2},\\
\abs{z_{j}} &\leq\frac{1}{2}\text{ para }j=2,\dots,r_{2}.\notag
\end{align}
\end{enumerate}
Entonces $B$ es un conjunto compacto, convexo y sim\'etrico respecto al origen, cuyo vol\'umen es
exactamente $2^{n}2^{-r_{2}}\abs{d}^{1/2}\footnote{Este vol\'umen se calcula, de manera f\'acil, observando
que $B$ es un producto de intervalos, discos y un rect\'angulo en el caso b).}$. Si $\sigma$ es la
inmersi\'on can\'onica de $K$ (\S\ref{sec4.2}), la proposici\'on~\ref{prop4.2.2} de la \S\ref{sec4.2} y el cor. del teorema~\ref{teo4.1.2}
de la \S\ref{sec4.1} muestran que
existe un entero $x\neq 0$ de $K$ tal que $\sigma(x)\in B$.

Mostremos que $x$ es un elemento \emph{primitivo} de $K$ sobre $\QQ$. En efecto, en el caso a),~\eqref{eq-4.3-5}
muestra que se tiene $\abs{\sigma_{i}(x)}\leq\frac{1}{2}$ si $i\neq 1$. Como
\begin{gather*}
\abs{N(x)} = \prod_{i=1}^{n}\abs{\sigma_{i}(x)}
\end{gather*}
es un entero $\neq 0$ (cap\'itulo~\ref{cap2}, \S\ref{sec2.6}, corolario de la proposici\'on~\ref{prop2.6.2}),
deducimos que $\abs{\sigma_{1}(x)}\geq 1$,
de donde $\sigma_{1}(x)\neq\sigma_{i}(x)$ para todo $i\neq 1$. Pero si $x$ no fuera primitivo,
$\sigma_{1}(x)$ coincidir\'ia con alguno de los $\sigma_{i}(x)$ con $i\neq 1$
(cap\'itulo~\ref{cap2}, \S\ref{sec2.6}, proposici\'on~\ref{prop2.6.1}),
lo que termina la demostraci\'on en este caso. ???? En el caso b), se tiene de la misma manera que
$\abs{\sigma_{1}(x)}=\oline{\abs{\sigma_{1}(x)}}\geq 1$, de donde $\sigma_{1}(x)\neq\sigma_{j}(x)$
cuando $\sigma_{j}$ es distinto de $\sigma_{1}$ y $\oline\sigma_{1}$. Por otra parte,~\eqref{eq-4.3-6}
muestra que la parte real $\abs{R(\sigma_{1}(x))}$ es $\leq\frac{1}{4}$, de manera que $\sigma_{1}(x)$
no es real y $\sigma(x)\neq\oline{\sigma_{1}(x)}$. Como en el caso a), concluimos que $x$ no es primitivo.

Ahora las f\'ormulas~\eqref{eq-4.3-5} y~\eqref{eq-4.3-6} muestran que los conjugados $\sigma_{i}(x)$ de $x$ son
\emph{acotados,} y por lo tanto tambi\'en lo son las funciones sim\'etricas elementales en los $\sigma_{i}(x)$,
es decir, los coeficientes del polinomio minimal de $x$. Como estos tambi\'en son elementos de $\ZZ$
(cap\'itulo~\ref{cap2}, \S\ref{sec2.6}, corolario de la proposici\'on~\ref{prop2.6.2}),
s\'olo pueden tomar en este caso un n\'umero finito de valores.
Por lo tanto, s\'olo hay un n\'umero finito de posibles polinomios minimales de $x$ y en consecuencia
solamente un n\'umero finito de valores posibles para $x$ en $\CC$. Como $x$ genera $K$, el teorema~\ref{teo4.3.3} queda
demostrado LQQD.% \QED

\section{El teorema de las unidades}\label{sec4.4}

Por abuso del lenguaje, llamamos \emph{unidades} de un cuerpo de n\'umeros $K$ a los elementos inversibles
del anillo de enteros $A$ de $K$. Est\'as unidades forman un grupo multiplicativo notado $A^{*}$. El
siguiente resultado nos ser\'a \'util.

\begin{proposition}\label{prop4.4.1}
Sea $K$ un cuerpo de n\'umeros y $x\in K$. Para que $x$ sea una unidad de $K$ es
necesario y suficiente que $x$ sea un entero de $K$ de norma $\pm 1$.
\end{proposition}

En efecto, si $x$ es una unidad de $K$, $N(x)$ y $N(x^{-1})$ son elementos de $\ZZ$ cuyo producto es
$N(1) = 1$. Luego, $N(x) = \pm 1$. Rec\'iprocamente, sea $x$ un entero de $K$ de norma $\pm 1$. Su
polinomio caracter\'istico se escribe $x^{n}+a_{n-1}x^{n-1}+\dots+a_{1}x\pm 1= 0$ para algunos
$a_{0}\in\ZZ$ (cap\'itulo~\ref{cap2}, \S\ref{sec2.6}). Luego $\pm(x^{n-1}+\dots+a_{1})$ es el inverso de $x$ y es un entero de
$K$, por lo que $x$ es una unidad de $K$.

\begin{theorem}[Dirichlet]\label{teo4.4.1}
Sea $K$ un cuerpo de n\'umeros, $n$ su grado, $r_{1}$ y $r_{2}$ los enteros definidos en \S\ref{sec4.2}
y $r = r_{1}+r_{2}-1$. El grupo $A^{*}$ de las unidades de $K$ es isomorfo a $\ZZ^{r}\times G$, donde
$G$ es un grupo c\'iclico finito, formado por las ra\'ices de la unidad contenidas en $K$.
\end{theorem}

Probaremos primero que nada que $A^{*}$ es un grupo conmutativo de tipo finito y calcularemos su rango.
Consdireemos la inmersi\'on can\'onica (\S\ref{sec4.2}) $x\mapsto(\sigma_{1}(x),\dots,\sigma_{r_{1}+r_{2}}(x))$
de $K$ en $\RR^{r_{1}}\times\CC^{r_{2}}$ y la aplicaci\'on
\begin{gather}
x\mapsto L(x) = (\log\abs{\sigma_{1}(x)},\dots,\log\abs{\sigma_{r_{1}+r_{2}}(x)})
\end{gather}
de $K^{*}$ en $\RR^{r_{1}+r_{2}}$. Es un homomorfismo (i.e. $L(xy) = L(x)+L(y)$), que llamaremos
\emph{inmersi\'on logar\'itmica} de $K^{*}$. Sea $B$ un subespacio compacto de $\RR^{r_{1}+r_{2}}$ y
mostremos que el conjunto $B'$ de las unidades $x\in A^{*}$ tales que $L(x)\in B$ es \emph{finito.} En
efecto, como $B$ est\'a acotado, existe un n\'umero real $\alpha > 1$ tal que, para todo $x\in B'$, se tiene
\begin{gather*}
\frac{1}{\alpha}\leq\abs{\sigma_{i}(x)}\leq\alpha\quad(i=1,\dots,n).
\end{gather*}
Entonces, las funciones sim\'etricas elementales en los $\sigma_{i}(x)$ est\'an acotadas en valor absoluto
y, como toman valores en $\ZZ$ (pues $x\in A$), s\'olo pueden tomar un n\'umero finito de valores. Luego
s\'olo hay un n\'umero finito de polinomio caracter\'isticos posibles para $x$ y por lo tanto s\'olo un n\'umero
finito de valores posibles para $x$. La finitud de $B'$ tiene entonces las siguientes consecuencias:
\begin{enumerate}
\item El n\'ucleo $G$ de la restricci\'on de $L$ a $A^{*}$ es un grupo finito. Luego consiste en ra\'ices de
la unidad y es \emph{c\'iclico} (cap\'itulo~\ref{cap1}, \S\ref{sec1.6}, teorema~\ref{teo1.6.1}). Toda \emph{ra\'iz de la unidad} en $K$ pertenece
a este n\'ucleo, ya que son enteros de $K$ y $\abs{\sigma_{i}(x)}^{q}=\abs{\sigma_{i}(x^{q})}=\abs{1}=1$
implica $\abs{\sigma_{i}(x)}=1$.
\item La imagen $L(A^{*})$ es un subgrupo discreto de $\RR^{r_{1}+r_{2}}$ (\S\ref{sec4.1}) y por lo tanto un
$\ZZ$-m\'odulo libre de rango $s\leq r_{1}+r_{2}$ (\S\ref{sec4.1}, teorema~\ref{teo4.1.1}). Como $L(A^{*})$ es libre, $A^{*}$
es isomorfo a $G\times L(A^{*}) = G\times\ZZ^{s}$. S\'olo nos resta demostrar que el rango $s$ de $L(A^{*})$
es igual a $r_{1}+r_{2}-1$.
\end{enumerate}
La desigualdad $s\leq r_{1}+r_{2}-1$ es f\'acil. En efecto, si $x\in A^{*}$, la igualdad
$\pm 1=N(x) = \prod_{i=1}^{n}\sigma_{i}(x) = \prod_{i=1}^{r_{1}}\sigma_{i}(x)\prod_{j=r_{1}+1}^{r_{1}+r_{2}}
\sigma_{j}(x)\oline{\sigma_{j}(x)}$ (proposici\'on~\ref{prop4.4.1})
implica que el vector $L(x) = (y_{1},\dots,y_{r_{1}+r_{2}})$
pertenece al hiperplano $W$ de ecuaci\'on
\begin{gather}\label{eq-4.4-2}
\sum_{i=1}^{r_{1}}y_{i}+2\sum_{j=r_{1}+1}^{r_{1}+r_{2}}y_{j} = 0,
\end{gather}
y por lo tanto $L(A^{*})$ es un subgrupo discreto de $W$, de donde $s\leq r_{1}+r_{2}-1$.

S\'olo queda demostrar que $L(A^{*})$ contiene $r = r_{1}+r_{2}-1$ vectores linealmente independientes,
lo que es m\'as delicado. Se trata de mostrar que, para toda forma lineal $f\neq 0$ en $W$, existe una unidad
$u$ tal que $f(L(u))\neq 0$. Como la proyecci\'on de $W$ sobre $\RR^{r}$ es un isomorfismo
(por~\eqref{eq-4.4-2}), podemos escribir, para todo $y=(y_{1},\dots,y_{r+1})\in W\subset\RR^{r+1}$,
\begin{gather}\label{eq-4.4-3}
f(y) = c_{1}y_{1}+\dots+c_{r}y_{r}\quad\text{con}\quad c_{i}\in\RR
\end{gather}
Fijemos un n\'umero real $\alpha$ suficientemente grande, m\'as precisamente tal que
\begin{gather*}
\alpha\geq 2^{n}\left(\frac{1}{2\pi}\right)^{r_{2}}\abs{d}^{1/2}.
\end{gather*}
Para todo sistema $\lambda = (\lambda_{1},\dots,\lambda_{r})$ de $r$ n\'umeros reales $>0$, sea
$\lambda_{r+1}$ el n\'umero real $>0$ tal que $\prod_{i=1}^{r_{1}}\lambda_{i}\prod_{j=r_{1}+1}^{r_{1}+r_{2}}\lambda_{j}^{2}=\alpha$.
En $\RR^{r_{1}}\times\CC^{r_{2}}$, el conjunto $B$ de los $(y_{1},\dots,y_{r_{1}},z_{1},\dots,z_{r_{2}})$
($y_{i}\in\RR$, $z_{j}\in\CC$) tales que $\abs{y_{i}}\leq\lambda_{i}$ y $\abs{z_{j}}\leq\lambda_{j}$
es compacto, convexo, sim\'etrico respecto al origen y su vol\'umen es $\prod_{i=1}^{r_{1}}2\lambda_{i}
\prod_{r=r_{1}+1}^{r_{2}}\pi\lambda_{j}^{2}=2^{r}\pi^{r_{2}}\alpha\geq 2^{n}2^{-r_{2}}\abs{d}^{1/2}$.
Luego, por la proposici\'on~\ref{prop4.2.2} de la \S\ref{sec4.2} y
el corolario del teorema~\ref{teo4.1.2} de \S\ref{sec4.1}, existe un \emph{entero $x_{\lambda}$
de $K$} tal que $\sigma(x_{\lambda})\in B$; dicho de otra menra, se tiene $\abs{\sigma_{i}(x_{\lambda})}
\leq\lambda_{i}$ para todo $i=1,\dots,n$ (poniendo $\lambda_{j+r_{2}}=\lambda_{j}$ para $j=r_{1}+1,\dots,r_{1}+r_{2}$).
Como $x_{\lambda}$ es un entero, se tiene
\begin{gather*}
1\leq\abs{N(x_{\lambda})}=\prod_{i=1}^{n}\abs{\sigma_{i}(x_{\lambda})}\leq\prod_{i=1}^{r_{1}}\lambda_{i}
\prod_{j=r_{1}+1}^{r_{1}+r_{2}}\lambda_{j}^{2}=\alpha
\end{gather*}
Por otra parte, para todo $i$, se tiene
\begin{gather*}
\abs{\sigma_{i}(x_{\lambda})}=\abs{N(x_{\lambda})}\prod_{j\neq i}\abs{\sigma_{j}(x_{\lambda})}^{-1}
\geq\prod_{j\neq i}\lambda_{j}^{-1}=\lambda_{i}\alpha^{-1}
\end{gather*}
De donde $\lambda_{i}\alpha^{-1}\leq\abs{\sigma_{i}(x_{\lambda})}\leq\lambda_{i}$ para todo $i$, de manera que
se tiene
\begin{gather}
0\leq\log\lambda_{i}-\log\abs{\sigma_{i}(x_{\lambda})}\leq\log\alpha.
\end{gather}
Entonces por~\eqref{eq-4.4-3}, se tiene tambi\'en
\begin{gather}\label{eq-4.4-5}
\abs{f(L(x_{\lambda}))-\sum_{i=1}^{r}c_{i}\log\lambda_{i}}\leq\left(\sum_{i=1}^{r}\abs{c_{i}}\right)\log\alpha
\end{gather}
Sea $\beta$ una constante estrictamente mayor al miembro de la derecha de~\eqref{eq-4.4-5} y, para todo entero $h>0$,
elijamos $r$ n\'umeros reales $\lambda_{i,h}>0$ ($i=1,\dots,r$) tales que $\sum_{i=1}^{r}c_{i}\log\lambda_{i,h}
=2\beta h$. Sea
\begin{gather*}
\lambda(h)= (\lambda_{1,h},\dots,\lambda_{r,h}),
\end{gather*}
y sea $x_{h}$ el entero $x_{\lambda(h)}$ correspondiente. Por~???? se tiene
\begin{gather*}
\abs{f(L(x_{h}))-2\beta h} < \beta,
\end{gather*}
de donde
\begin{gather}\label{eq-4.4-6}
(2h-1)\beta < f(L(x_{h})) < (2h+1)\beta.
\end{gather}
Resulta de~\eqref{eq-4.4-6} que los n\'umeros $f(L(x_{h}))$ ($h\geq 0$) son todos \emph{disintos.} Por otra parte,
como $\abs{N(x_{h})}\leq\alpha$, hay un n\'umero finito de los ideales $Ax_{h}$ (cf.~\S\ref{sec4.3}, demostraci\'on
del teorema~\ref{teo4.3.2}). Luego existen dos \'indices $h$ y $k$ distintos tales que $Ax_{h}=Ax_{k}$ y por lo tanto
exite una \emph{unidad} $u$ de $A$ tal que $x_{k} = ux_{h}$. Por lo tanto (como $f$ es lineal) se tiene que
$f(L(u)) = f(L(x_{k})) - f(L(x_{h})) \neq 0$, y $u$ es la unidad buscada. \QED

\begin{remark*}
El teorema~\ref{teo4.4.1} (llamado ``teorema de las unidades'') muestra que existen $r$ ($=r_{1}+r_{2}-1$) y unidades $(u_{i})$
de $K$ tal que toda unidad $u$ de $K$ se escribe, de manera \'unica, en la forma
\begin{gather}
u = zu_{1}^{n_{1}}\cdots u_{r}^{n_{r}}
\end{gather}
donde los $n_{i}\in\ZZ$ y $z$ es una ra\'iz de la unidad. Entones $(u_{i})$ se llama un \emph{sitema de unidades
fundamentales} de $K$.
\end{remark*}

\begin{example*}[Ejemplo de los cuerpos ciclot\'omicos]
Sea $p$ un n\'umero primo $\neq 2$, $z$ una
ra\'iz primitiva $p$-\'esima de la unidad en $\CC$ y $K$ el cuerpo ciclot\'omico $\QQ[z]$
(cf.~cap\'itulo~\ref{cap2}, \S\ref{sec2.9}). Se tiene $[K:\QQ] = p-1$ ({\itshape ibid.}, teorema~\ref{teo2.9.1}). Como ning\'un conjugado de
$z$ en $\CC$ es real, se tiene $r_{1}=0$, $2r_{1} = p-1$, de donde $r = \frac{p-3}{2}$.
\end{example*}

\section{Las unidades de un cuerpo cuadr\'atico imaginario}

Sea $K$ un cuerpo cuadr\'atico imaginario (cap\'itulo~\ref{cap2}, \S\ref{sec2.5}).
Entonces tenemos $r_{1}=0$, $2r_{2}=2$,
$r_{2}=1$ y $r_{1}+r_{2}-1=0$. As\'i, las \'unicas unidades de $K$ son las ra\'ices de la unidad contenidas en $K$
(\S\ref{sec4.4}, teorema~\ref{teo4.4.1}). Estas forman un grupo \emph{finito c\'iclico.} Redemostraremos este resultado con un sencillo
c\'alculo que adem\'as nos dar\'a un poco m\'as de precisi\'on.

Sea $K=\QQ[\sqrt{-m}]$, donde $m$ es un entero $>0$ libre de cuadrados. Recordemos que las unidades de $K$
son los enteros de norma $\pm 1$ de $K$ (\S\ref{sec4.4}, proposici\'on~\ref{prop4.4.1}).

1) Si $m\equiv 2\text{ \'o }2\pmod 4$, el anillo de enteros de $K$ es $\ZZ+\ZZ\sqrt{-m}$ (cap\'itulo~\ref{cap2},
\S\ref{sec2.5}, teorema~\ref{teo2.5.1}). Si $x= a+b\sqrt{-m}$ ($a,b\in\ZZ$), tenemos
\begin{gather*}
N(x) = a^{2}+mb^{2}\geq 0.
\end{gather*}
Por lo tanto, para que $x$ sea una unidad es necesario y suficiente que $a^{2}+mb^{2}=1$. Si $m\geq 2$, esto
implica que $b=0$ y $a=\pm 1$, de donde $x=\pm 1$. Si $m=1$, adem\'as de las soluciones $x=\pm 1$ est\'an
las soluciones $a=0$, $b=\pm 1$, $x=\pm i$ ($i^{2}=-1$).

2) Si $m\equiv 3\pmod 4$, el anillo de enteros de $K$ es $\ZZ+\ZZ\frac{1+\sqrt{-m}}{2}$ (cap\'itulo~\ref{cap2},
\S\ref{sec2.5}, teorema~\ref{teo2.5.1}). Si $x=a+\frac{b}{2}(1+\sqrt{-m})$ ($a,b\in\ZZ$), tenemos
$N(x) = \left(a+\frac{b}{2}\right)^{2}+\frac{mb^{2}}{4}$. Por lo tanto, para que $x$ sea una unidad es necesario y
suficiente que $(2a+b)^{2}+mb^{2}=4$. Si $m\geq 7$, esto implica que $b=0$, de donde $(2a)^{2}=4$,
$a=\pm 1$, y $x=\pm 1$. Si $m=3$, tambi\'en obtenemos las soluciones $b=\pm 1$, de donde
$(2a\pm 1)^{2}=\pm 1$, es decir
\begin{gather*}
x=\frac{1}{2}(\pm 1\pm \sqrt{-3})
\end{gather*}
(los signos $\pm 1$ son independientes).

En res\'umen, hemos obtenido el siguiente resultado:

\begin{proposition}
Si $K$ es un cuerpo cuadr\'atico imaginario, el grupo $G$ de las unidades de $K$ est\'a formado por $+1$ y
$-1$, excepto en los dos casos siguientes:
\begin{enumerate}
\item[1)] si $K=\QQ[i]$ ($i^{2}=-1$), $G$ est\'a formado de las ra\'ices cuartas de la unidad $i$, $-1$, $-i$, $1$.
\item[2)] Si $K=\QQ[\sqrt{-3}]$, $G$ est\'a formado de las ra\'ices sextas de la unidad $\left(\frac{1+\sqrt{-3}}{2}\right)^{j}$,
$j=0,1,\dots,5$.
\end{enumerate}
\end{proposition}

\section{Las unidades de un cuerpo cuadr\'atico real}\label{sec4.6}

Este {\S} ser\'a decididamente m\'as divertido que el anterior. Sea $K$ un cuerpo cuadr\'atico real. Con la
notaci\'on habitual, se tiene $r_{1}=2$, $r_{2}=0$, de donde $r=r_{1}+r_{2}-1=1$. El teorema de las unidades
(\S\ref{sec4.4}, teorema~\ref{teo4.4.1}) muestra que el grupo de las unidades de $K$ es isomorfo al producto de $\ZZ$ por el grupo de
las ra\'ices de la unidad contenidas en $K$. Como $K$ admite una inmersi\'on en $\RR$, est\'as son $1$ y $-1$.
Por lo tanto, suponiendo  que $K$ est\'a inmerso en $\RR$, tenemos:

\begin{proposition}
Las unidades positivas de un cuerpo cuadr\'atico real $K\subset\RR$ forman un grupo (multiplicativo) isomorfo
a $\ZZ$.
\end{proposition}

Este grupo admite un s\'olo generador $>1$. Lo llamamos \emph{la unidad fundamental de $K$.}

Sea $K = \QQ[\sqrt{d}]$ donde $d$ es un entero $\geq 2$ libre de cuadrados y sea $x = a+b\sqrt{d}$
($a,b\in\QQ$) una unidad de $K$. Los n\'umeros $x$, $x^{-1}$, $-x$, $-x^{-1}$ son todas unidades de $K$ y,
como $N(x) = (a+b\sqrt{d})(a-b\sqrt{d})=\pm 1$ (\S\ref{sec4.4}, proposici\'on~\ref{prop4.4.1}), estos cuatro n\'umeros son precisamente
$\pm a\pm b\sqrt{d}$. Si $x\neq\pm 1$, exactamente uno de los cuatro n\'umeros $x$, $x^{-1}$, $-x$, $-x^{-1}$
es $>1$ y es el m\'as grande de los cuatro. Por lo tanto \emph{las unidades $>1$ de $K$ son las unidades de la
forma $a+b\sqrt{d}$ con $a,b>0$.}

a) Supongamos primero que $d\equiv 2\text{ \'o }3\pmod 4$. Entonces el anillo de enteros de $K$ es
$\ZZ+\ZZ\sqrt{d}$ (cap\'itulo~\ref{cap2}, \S\ref{sec2.5}, teorema~\ref{teo2.5.1}). Como las unidades de $K$ son los enteros de norma
$\pm 1$ (\S\ref{sec4.4}, proposici\'on~\ref{prop4.4.1}), las unidades $>1$ de $K$ son los n\'umeros $a+b\sqrt{d}$ con $a,b\in\ZZ$, $a,b>0$ tales que
\begin{gather}\label{eq-4.6-1}
a^{2}-db^{2}=\pm 1.
\end{gather}
Vemos entonces que las soluciones ``en n\'umeros enteros naturales'' $(a,b)$ de la ecuaci\'on~\eqref{eq-4.6-1}
(llamada \emph{``ecuaci\'on de Pell-Fermat''}\/) se obtienen de la siguiente manera: tomamos la unidad fundamental
$a_{1}+b_{1}\sqrt{d}$ de $K$ y ponemos
\begin{gather}\label{eq-4.6-2}
a_{n}+b_{n}\sqrt{d} = (a_{1}+b_{1}\sqrt{d})^{n}\quad(n\geq 1).
\end{gather}
La sucesi\'on $(a_{n},b_{n})$ provee ???? entonces \emph{todas las soluciones} de~\eqref{eq-4.6-1}.

\begin{remark*}[Observaciones]
1) Resulta de~\eqref{eq-4.6-2} que $b_{n+1}=a_{1}b_{n}+b_{1}a_{n}$. Como $a_{1}, b_{1}, a_{n}, b_{n}>0$,
la sucesi\'on $(b_{n})$ es estrictamente creciente. As\'i, para \emph{calcular} expl\'icitamente la unidad
fundamental $a_{1}+b_{1}\sqrt{d}$, podemos comenzar escribiendo la sucesi\'on de los $db^{2}$ ($b\in\NN$, $b\geq 1$) y
detenernos en el primer t\'ermino $db_{1}^{2}$ de esta sucesi\'on que difiera de un cuadrado $a_{1}^{2}$
por $\pm 1$. Entonces $a_{1}+b_{1}\sqrt{d}$ es la unidad fundamental de $K$. Por ejemplo, si $d=7$, la
sucesi\'on de los $db^{2}$ es $7, 28, 63=64-1=8^{2}-1$. Por lo tanto tenemos $b_{1}=3$, $a_{1}=8$ y
la unidad fundamental de $\QQ[\sqrt{7}]$ es $8+3\sqrt{7}$. Vemos de la misma manera que las unidades
fundamentales de $\QQ[\sqrt{2}$, $\QQ[\sqrt{3}]$ y $\QQ[\sqrt{6}]$ son $1+\sqrt{2}$, $2+\sqrt{3}$, $5+2\sqrt{6}$.
Hay otras maneras de calcularlas, m\'as eficaces, relacionadas con la teor\'ia de fracciones continuas.

2) Si la unidad fundamental es de norma $1$, los $(a_{n},b_{n})$ son todos soluciones de
(1') $a^{2}-db^{2}=1$. Entonces (1'') $a^{2}-db^{2}=-1$ no posee soluciones. Si la unidad fundamental es de norma
$-1$, las soluciones de (1') son los $(a_{2n},b_{2n})$ y las de (1'') son los $(a_{2n+1},b_{2n+1})$. El primer
caso ocurre por ejemplo si $d=3$, $d=6$ y $d=7$ y el segundo si $d=2$ y $d=10$.
\end{remark*}

b) Supongamos ahora que $d\equiv 1\pmod 4$. Los enteros de $K=\QQ[\sqrt{d}]$ son entonces los n\'umeros
$\frac{1}{2}(a+b\sqrt{d})$ con $a,b\in\ZZ$ de la misma paridad (cap\'itulo~\ref{cap2}, \S\ref{sec2.5}, teorema~\ref{teo2.5.1}).
Por lo tanto, si $\frac{1}{2}(a+b\sqrt{d})$ es una unidad de $K$, se tiene (\S\ref{sec4.4}, proposici\'on~\ref{prop4.4.1})
\begin{gather}\label{eq-4.6-3}
a^{2}-db^{2}=\pm 4.
\end{gather}
Reciprocamente para todo soluci\'on $(a,b)$ en n\'umeros enteros de~\eqref{eq-4.6-3}, $\frac{1}{2}(a+b\sqrt{d})$
es un entero de $K$ (pues su traza es $a$ y su norma es $\pm 1$ por~\eqref{eq-4.6-3}) y por lo tanto una unidad de $K$.
Como en {\itshape a}) vemos que, si $\frac{1}{2}(a_{1}+b_{1}\sqrt{d})$ denota la unidad fundamental de $K$, las
soluciones $(a,b)$ de~\eqref{eq-4.6-3} en n\'umeros enteros $>0$ forman una sucesi\'on $(a_{n},b_{n})$ ($n\geq 1$)
definida por
\begin{gather}
a_{n}+b_{n}\sqrt{d}=2^{1-n}(a_{1}+b_{1}\sqrt{d})^{n}.
\end{gather}
El c\'alculo de $a_{1}+b_{1}\sqrt{d}$ puede efectuarse como en {\itshape a}); las unidades fundamentales
de $\QQ[\sqrt{5}]$, $\QQ[\sqrt{13}]$ y $\QQ[\sqrt{17}]$ son $\frac{1}{2}(1+\sqrt{5})$, $\frac{1}{2}(3+\sqrt{13})$,
$4+\sqrt{17}$. Estas tres unidades son de norma $-1$. Para la elecci\'on de signo $\pm$ en~\eqref{eq-4.6-3} se tienen
los mismos resultados que en el caso {\itshape a}).

\begin{remark*}
%\begin{comm}
%{\itshape Observaci\'on.}
En el caso $d\equiv 1\pmod 4$, las soluciones de la ecuaci\'on de Pell-Fermat propiamente dichas
\begin{gather}
a^{2}-db^{2}=\pm 1
\end{gather}
corresponden a las unidades $a+b\sqrt{d}$ ($a,b>0$) del anillo $B=\ZZ[\sqrt{d}]$, que es un subanillo
del anillo $A$ de enteros de $K$. Ahora bien, las unidades $>0$ de $B$ forman un subgrupo $G$ del grupo de
unidades positivas de $A$. Sea $u=\frac{1}{2}(a+b\sqrt{d})$ la unidad fundamental de $K$. Si $a$ y $b$ son
ambos \emph{pares,} se sigue que $u\in B$, de manera que $G$ \emph{est\'a formado de las potencias de $u$} (este
es le caso si $d=17$). Si $a$ y $b$ son ambos \emph{impares, se sigue que $u^{3}\in B$:} en efecto se
tiene $8u^{3}=a(a^{2}+3b^{2}d)+b(3a^{2}+b^{2}d)\sqrt{d}$. Como $a^{2}-db^{2}=\pm 4$, tenemos
$a^{2}+3b^{2}d=4(b^{2}d\pm 1)$, que es un m\'ultiplo de $8$ pues $b$ y $d$ son impares. En este caso $G$
\emph{est\'a formado de las potencias de $u^{3}$} (en efecto, necesariamente $u^{2}\notin B$ pues sino
$u=u^{3}/u^{2}\in B$). Este es el caso si $d=5$ (resp. $d=13$), en cuyo caso $u^{3}=2+\sqrt{5}$
(resp. $u^{3}=18+5\sqrt{13}$).
\end{remark*}

\section{Una generalizaci\'on del teorema de las unidades}

\begin{proposition}
Sea $A$ un anillo que es un $\ZZ$-m\'odulo de tipo finito. Entonces el grupo multiplicativo
$A^{*}$ de los elementos inversibles de $A$ es un grupo multiplicativo de tipo finito.
\end{proposition}

Para un grupo conmutativo $G$, ``de tipo finito'' quiere decir ``de tipo finito para la estructura
de \emph{$\ZZ$-m\'odulo} de $G$''. Un subgrupo de un grupo conmutativo de tipo finito es de tipo finito
(cap\'itulo~\ref{cap3}, \S\ref{sec3.1}, corolario~\ref{cor3.1.2} del teorema~\ref{teo3.1.1}).
Observemos primero que nada que $A$ es un anillo \emph{noetheriano}
y que los ideales de $A$ son los sub-$\ZZ$-m\'odulos de $A$.

Primero trataremos el caso donde $A$ es un \emph{dominio \'integro.} Si su cuerpo de fracciones $K$ es
de caracter\'istica cero, es un $\QQ$-espacio vectorial de tipo finito, luego un cuerpo de n\'umeros.
Por otra parte $A$ es entero sobre $\ZZ$ (pues es un $\ZZ$-m\'odulo de tipo finito,
cf.~cap\'itulo~\ref{cap2}, \S\ref{sec2.1}, teorema~\ref{teo2.1.1})
y por lo tanto es un subanillo del anillo $B$ de enteros de $K$. Luego, $A^{*}\subset B^{*}$ y $B^{*}$ es
de tipo finito por el teorema de las unidades (\S\ref{sec4.4}, teorema~\ref{teo4.4.1}). Si $K$ es de caracter\'istica $p\neq 0$, $K$
es una extensi\'on finita de $\FF_{p}$, en particular un cuerpo finito, en cuyo caso $A^{*}$ es finito.

Pasemos ahora al caso cuando $A$ es \emph{reducido} (lo que significa, por definici\'on, que $0$ es el
\'unico elemento nilpotente de $A$). Nos har\'a falta el siguiente lema.

\begin{lemma*}
En un anillo noetheriano reducido $A$, el ideal $(0)$ es intersecci\'on finita de ideales primos.
\end{lemma*}

En efecto, sabemos que en un anillo noetheriano, todo ideal contiene un producto de ideales primos
(cap\'itulo~\ref{cap3}, \S\ref{sec3.3}, lema~\ref{lem3.3.3}).
Como $(0)$ es el ideal m\'as peque\~no, debe \emph{ser} un producto de ideales primos:
$(0) = \idl{p}_{1}^{n_{1}}\cdots\idl{p}_{q}^{n_{q}}$. Sea $x\in\idl{p}_{1}\cap\dots\cap\idl{p}_{q}$. Se
tiene $x^{n_{1}+\dots+n_{q}}\in\idl{p}_{1}^{n_{1}}\cdots\idl{p}_{q}^{n_{q}} = (0)$, de manera que
$x^{n_{1}+\dots+n_{q}}=0$, de donde $x = 0$ pues $A$ es reducido. Por lo tanto, $(0) = \idl{p}_{1}\cap\dots\cap\idl{p}_{q}$.

En res\'umen, $(0) = \idl{p}_{1}\cap\dots\idl{p}_{1}$ con los $\idl{p}_{i}$ ideales primos. Deducimos que
el homomorfismo can\'onico $\varphi:A\to\prod_{i=1}^{q}A/\idl{p}_{i}$ es inyectivo. Como un elemento de un anillo
producto es inversible si y s\'olo si todas sus componentes son inversibles, se sigue que
$\left(\prod_{i}A/\idl{p}_{i}\right)^{*}=\prod_{i}(A/\idl{p}_{i})^{*}$. Por el caso de un dominio \'integro, cada
$(A/\idl{p}_{i})^{*}$ es de tipo finito y por lo tanto tambi\'en lo es $\prod(A/\idl{p}_{i})^{*}$ y por lo tanto
tambi\'en $\varphi(A^{*})$ (recordemos que $\ZZ$ es noetheriano). Es decir, $A^{*}$ es de tipo finito ya que
$\varphi$ es inyectiva.

Pasemos finalmente al caso \emph{general.} Observemos que el conjunto $\idl{n}$ de elementos nilpotentes de $A$
es un \emph{ideal,} pues $x^{p}=0$, $y^{p}=0$ y $a\in A$ implica que $(x+y)^{p+q-1}=0$ y $(ax)^{p}=0$. Por otra parte,
existe un entero $s$ tal que $\idl{n}^{s}=(0)$: en efecto, como $A$ es noetheriano, $\idl{n}$ admite un sistema
finito de generadores $(x_{1},\dots,x_{r})$ con $x_{i}^{q_{i}}=0$ para todo $i$. Entonces, si $s = q_{1}+\dots+q_{r}$,
todo monomio en los $x_{i}$ de grado $s$ es nulo, de manera que $\idl{n}^{s}=(0)$. Procederemos por inducci\'on
en $s$. El caso $s=1$ es precisamente el caso de $A$ reducido, que ya tratamos. Supongamos por lo tanto que
$s>1$ y notemos $\varphi$ el homomorfismo can\'onico $\varphi:A\to A/\idl{n}^{s-1}$. Se tiene
$\varphi(A^{*})\subset(A/\idl{n}^{s-1})^{*}$, de manera que $\varphi(A^{*})$ es un grupo de tipo finito. Por
otra parte el n\'ucleo de la restricci\'on de $\varphi$ a $A^{*}$ est\'a contenido en $1+\idl{n}^{s-1}$ y
por lo tanto es \emph{igual} a $1+\idl{n}^{s-1}$, pues como $s>1$, se tiene $(\idl{n}^{s-1})^{2}\subset\idl{n}^{s}=(0)$,
y todo elemento $1+x$ de $1+\idl{n}^{s-1}$ es inversible en virtud de
$(1+x)(1-x) = 1-x^{2}=1$. S\'olo resta demostrar que el grupo multiplicativo $1+\idl{n}^{s-1}$ es de tipo finito.
Como $(\idl{n}^{s-1})^{2}=(0)$ se tiene $(1+x)(1+y) = 1+x+y$ para $x,y\in\idl{n}^{s-1}$, de manera que
$x\mapsto 1+x$ es un isomorfismo del grupo aditivo $\idl{n}^{s-1}$ sobre el grupo multiplicativo $1+\idl{n}^{s-1}$.
Pero, como $A$ es un $\ZZ$-m\'odulo de tipo finito, tambi\'en lo es $\idl{n}^{s-1}$. \QED

\begin{comm}
Utilizando m\'etodos pertenecientes a la Geometr\'ia Algebraica puede demostrarse que para todo anillo
\emph{reducido} $B$ de la forma $B = \ZZ[x_{1},\dots,x_{n}]$ (es decir, generado como anillo sobre $\ZZ$ por un
n\'umero finito de elementos), el grupo $B^{*}$ de los elementos inversibles es de tipo finito (\cite{Samuel1}).
\end{comm}

\section*[Un c\'alculo de vol\'umen]{Ap\'endice. Un c\'alculo de vol\'umen}

\begin{proposition*}
Sean $r_{1},r_{2}\in\NN$, $n=r_{1}+2r_{2}$, $t\in\RR$ y $B_{t}$ el conjunto de aquellos
$(y_{1},\dots,y_{r_{1}},z_{1},\dots,z_{r_{2}})\in\RR^{r_{1}}\times\CC^{r_{2}}$ tales que
\begin{gather}
\sum_{i=1}^{r_{1}}\abs{y_{i}}+2\sum_{j=1}^{r_{2}}\abs{z_{j}}\leq t.
\end{gather}
Entonces para la medidad de Lebesgue $\mu$ se tiene $\mu(B_{t})=0$ si $t < 0$ y
\begin{gather}\label{eq-4-ap-2}
\mu(B_{t}) = 2^{r_{1}}\left(\frac{\pi}{2}\right)^{r_{2}}\frac{t^{n}}{n!}\quad\text{si}\quad t\geq 0.
\end{gather}
\end{proposition*}

Podemos suponer que estamos en el caso $t\geq 0$ pues si $t < 0$ tenemos $B_{t}=\emptyset$ y $\mu(B_{t})=0$.
Escribimos $\mu(B_{t}) = V(r_{1},r_{2},t)$ y hacemos inducci\'on doble en $r_{1}$ y $r_{2}$. Se tiene
$V(1,0,t) = 2t$ (segmento $[-t,+t]$) y
\begin{gather*}
V(0,1,t)=\frac{\pi t^{2}}{4}
\end{gather*}
(disco de radio $\frac{t}{2}$), lo que coincide con~\eqref{eq-4-ap-2}.

Pasemos de $r_{1}$ a $r_{1}+1$. El conjunto $B_{t}\subset\RR\times\RR^{r_{1}}\times\CC^{r_{2}}$ correspondiente
a $r_{1}+1$ y $r_{2}$ est\'a definido por
\begin{gather*}
\abs{y}+\sum_{i=1}^{r_{1}}\abs{y_{i}}+2\sum_{j=1}^{r_{2}}\abs{z_{j}}\leq t\quad(y\in\RR)
\end{gather*}
La f\'ormula de integraci\'on ``por partes'' nos da ????
\begin{gather*}
V(r_{1}+1,r_{2},t) = \int_{\RR}V(r_{1},r_{2},t-\abs{y})\,dy=\int_{-t}^{+t}V(r_{1},r_{2},t-\abs{y})\,dy.
\end{gather*}
Por la hip\'otesis inductiva, esto coincide con ????
\begin{gather*}
V(r_{1}+1,r_{2},t) = 2\int_{0}^{t}2^{r_{1}}\left(\frac{\pi}{2}\right)^{r_{2}}\frac{(t-y)^{n}}{n!}dy
=2^{r_{1}+1}\left(\frac{\pi}{2}\right)^{r_{2}}\frac{t^{n+1}}{(n+1)!},
\end{gather*}
lo que de nuevo coincide con~\eqref{eq-4-ap-2}.

Pasemos finalmente de $r_{2}$ a $r_{2}+1$. El conjunto $B_{t}\subset\RR^{r_{1}}\times\CC^{r_{2}}\times\CC$
correspondiente a $r_{1}$ y $r_{2}+1$ est\'a definido por
\begin{gather*}
\sum_{i=1}^{r_{1}}\abs{y_{i}}+2\sum_{j=1}^{r_{2}}\abs{z_{j}}+2\abs{z}\leq t\quad(z\in\CC)
\end{gather*}
La f\'ormula de integraci\'on ``por partes'' nos da aqu\'i ??????
\begin{gather*}
V(r_{1},r_{2}+1,t)=\int_{\CC}V(r_{1},r_{2},t-2\abs{z})\,d\mu(z)=\int_{\abs{z}\leq\frac{t}{2}}V(r_{1},r_{2},t-2\abs{z})\,d\mu(z)
\end{gather*}
donde $d\mu(z)$ denota la medida de Lebesgue sobre $\CC$. Poniendo
\begin{gather*}
\text{$z = \rho e^{i\theta}$ ($\rho\in\RR_{+}$, $0\leq\theta\leq 2\pi$) se tiene $d\mu(z) = \rho\,d\rho\,d\theta$}
\end{gather*}
Utilizando la hip\'otesis inductiva, podemos reescribir esto ????
\begin{align*}
V(r_{1},r_{2}+1,t) = \int_{0}^{t/2}\int_{0}^{2\pi}2^{r_{1}}\left(\frac{\pi}{2}\right)^{r_{2}}
&\frac{(t-2\rho)^{n}}{n!}\rho\,d\rho\,d\theta\\
&=2^{r_{1}}\left(\frac{\pi}{2}\right)^{r_{2}}\frac{2\pi}{n!}\int_{0}^{t/2}(t-2\rho)^{n}\rho\,d\rho
\end{align*}
Para calcular $\int_{0}^{t/2}(t-2\rho)^{n}\rho\,d\rho$ ponemos $2\rho = x$ e integramos por partes.
Encontramos que esta integral vale $\frac{t^{n+2}}{4(n+1)(n+2)}$, de donde
$V(r_{1},r_{2}+1,t) = 2^{r_{1}}\left(\frac{\pi}{2}\right)^{r_{2}+1}\frac{t^{n+2}}{(n+2)!}$, lo que coincide
con~\eqref{eq-4-ap-2} pues $r_{1}+2(r_{2}+1)=n+2$.

\chapter{Descomposici\'on de ideales primos en extensiones}\label{cap5}

Sea $K$ un cuerpo de n\'umeros, $A$ el anillo de enteros de $K$, $L$ una extensi\'on finita de $K$ y $B$
la clausura \'integra de $A$ en $L$ (que no es otra cosa que el anillo de enteros de $L$). Dado un ideal primo
$\idl{p}\neq(0)$ de $A$, el ideal $B\idl{p}$ que genera en $B$ no es en general primo; pero se descompone en
producto de ideales primos (cap\'itulo~\ref{cap3}, \S\ref{sec3.4}, teorema~\ref{teo3.4.3}): $B\idl{p}=\prod_{i}\idl{P}_{i}^{e_{i}}$. En este
cap\'itulo nos proponemos estudiar esta descomposici\'on. El caso donde $B$ es un $A$-m\'odulo \emph{libre}
(p.ej. cuando $A$ es un dominio de ideales principales;
cf.~cap\'itulo~\ref{cap2}, \S\ref{sec2.7}, corolario del teorema~\ref{teo2.7.1}) es particularmente
simple. Expondremos en \S\ref{sec5.1} una t\'ecnica que nos permite situarnos en este caso.

\section{Preliminares sobre anillos de fracciones}\label{sec5.1}

\begin{definition}
Sea $A$ un dominio \'integro, $K$ su cuerpo de fracciones y $S$ un subconjunto de $A$ stable por multiplicaci\'on
que no cotiene $0$ y contiene $1$. Llamamos anillo de fracciones de $A$ respecto a $S$, y lo notamos $S^{-1}A$,
al conjunto de elementos $\frac{a}{s}\in K$ con $a\in A$ y $s\in S$.
\end{definition}

Es un anillo conmutativo (pues $\frac{a}{s}+\frac{a'}{s'} = \frac{s'a+sa'}{ss'}$ y $\frac{a}{s}\cdot\frac{a'}{s'}=
\frac{aa'}{ss'}$) y contiene a $A$ (pues $1\in S$). Si $S$ consiste \'unicamente del $1$, o si consiste \'unicamente
de elementos inversibles de $A$, se tiene $S^{-1}A=A$.

\begin{proposition}\label{prop5.1.1}
Sea $A$ un dominio \'integro, $S$ un subconjunto m\'ultiplicativamente estable de $A$ que contiene al $1$ y
que no contiene al $0$ y sea $A' = S^{-1}A$.
\begin{enumerate}
\item[1.]
Para todo ideal $\idl{b}'$ de $A'$, se tiene $(\idl{b}'\cap A)A' = \idl{b}'$ de manera que
$\idl{b}'\mapsto\idl{b}'\cap A$ es una inyecci\'on creciente (para la inclusi\'on) del conjunto de ideales de $A'$
en el conjunto de ideales de $A$.
\item[2.]
La aplicaci\'on $\idl{p}'\mapsto\idl{p}'\cap A$ es un isomorfismo del conjunto ordenado (para la inclusi\'on)
del conjunto de ideales primos de $A'$ sobre el conjunto de ideales primos $\idl{p}$ de $A$ tales que
$\idl{p}\cap S=\emptyset$. La aplicaci\'on inversa es $\idl{p}\mapsto\idl{p}A'$.
\end{enumerate}
\end{proposition}

Demostremos 1). Si $\idl{b}'$ es un ideal de $A'$, se tiene $\idl{b}'\cap A\subset\idl{b}'$, de donde
$(\idl{b}'\cap A)A'\subset\idl{b}'$ pues $\idl{b}'$ es un ideal. Para demostrar la inclusi\'on inversa, sea
$x\in\idl{b}'$; se tiene $x = \frac{a}{s}$ con $a\in A$ y $s\in S$. Luego $xs\in\idl{b}'$ pues
$A\subset A'$ y $\idl{b}'$ es un ideal, de donde $a\in\idl{b}'$ y $a\in\idl{b}'\cap A$. Entonces
$x=\frac{1}{s}\cdot a\in A'(\idl{b}'\cap A)$, de donde $\idl{b}'\subset A'(\idl{b}'\cap A)$ y
$\idl{b}'=A'(\idl{b}'\cap A)$. Esta f\'ormula asegura la inyectividad de la aplicaci\'on
$\varphi:\idl{b}'\mapsto\idl{b}'\cap A$ pues se tiene una aplicaci\'on $\theta:\idl{b}\mapsto A'\idl{b}$
tal que $\theta\circ\varphi = \text{identidad}$. Que $\varphi$ es creciente es evidente. Esto demuestra 1).

Pasemos a 2). Si $\idl{p}'$ es un ideal primo de $A'$, entonces $\idl{p} = \idl{p}'\cap A$ es un ideal primo de $A$
(cap\'itulo~\ref{cap3}, \S\ref{sec3.3}, lema~\ref{lem3.3.1}).
Adem\'as se tiene $\idl{p}\cap S=\emptyset$ pues, si
$s\in\idl{p}\cap S$, se tiene que $s\in\idl{p}'$ y $1=\frac{1}{s}\cdot s\in A'\idl{p}'=\idl{p}'$, lo que
es absurdo.

Reciprocamete, sea $\idl{p}$ un ideal primo de $A$ tal que $\idl{p}\cap S=\emptyset$. Mostraremos que
$\idl{p}A'$ es un ideal primo de $A'$ y que se tiene $\idl{p}A'\cap A = \idl{p}$.

Observemos primero que nada que $\idl{p}A'$ \emph{es el conjunto de los $\frac{p}{s}$ con
$p\in\idl{p}$ y $s\in S$:} en efecto todo elemento $x$ de $\idl{p}A'$ se escribe
$x=\sum_{i=1}^{n}\frac{a_{i}}{s_{i}}p_{i}$
($a_{i}\in A$, $s_{i}\in S$, $p_{i}\in\idl{p}$), por lo tanto $x=\sum_{i}\frac{b_{i}}{s}p_{i}$ utilizando
un denominador en com\'un ($b_{i}\in S$, $s\in S$) y por lo tanto $x = \frac{p}{s}$ con $p=\sum b_{i}p_{i}\in\idl{p}$.
Deducimos que $1\notin\idl{p}A'$ pues $\idl{p}\cap S=\emptyset$ y por lo tanto no podemos tener
$1=\frac{p}{s}$ con $p\in\idl{p}$ y $s\in S$. Mostremos que el ideal $\idl{p}A'$ es primo:
sean $\frac{a}{s}\in A'$ y $\frac{b}{t}\in A'$ tales que $\frac{a}{s}\cdot\frac{b}{t}\in\idl{p}A'$; entonces
tenemos $\frac{a}{s}\cdot\frac{b}{t} = \frac{p}{u}$ con $p\in\idl{p}$ y $u\in S$, de donde
$abu = pst\in\idl{p}$. Como $\idl{p}\cap S=\emptyset$, se tiene $u\notin\idl{p}$, de donde $ab\in\idl{p}$
(pues $\idl{p}$ es primo). As\'i, $a$ o $b$ pertenecen a $\idl{p}$, de manera que
$\frac{a}{s}$ o $\frac{b}{t}$ pertenecen a $\idl{p}A'$. Mostremos por \'ultimo que $\idl{p}=\idl{p}A'\cap A$.
La inclusi\'on $\idl{p}\subset\idl{p}A'\cap A$. Reciprocamente, si $x\in\idl{p}A'\cap A$, se tiene
$x = \frac{p}{s}$ ($p\in\idl{p}$, $s\in S$) pues $x\in\idl{p}A'$, de donde $sx = p\in\idl{p}$. Como
$s\notin\idl{p}$ (se tiene $\idl{p}\cap S = \emptyset$) y $\idl{p}$ es primo, deducimos que $x\in\idl{p}$.

Ahora bien, las f\'ormulas $\idl{p} = \idl{p}A'\cap A$ y $\idl{p}' = A'(\idl{p}\cap A)$ muestran que
las aplicaciones $\varphi:\idl{p}'\mapsto\idl{p}'\cap A$ y $\theta : \idl{p}\mapsto\idl{p}A$ (MAL EN EL LIBRO?)
(restrictas a los ideales primos descritos en el enunciado) son dos biyecciones inversas la una de la otra, pues sus
composiciones en los dos sentidos son la aplicaci\'on identidad. Que son crecientes es evidente. \QED

\begin{corollary*}
Si $A$ es un dominio \'integro noetheriano, todo anillo de fracciones $S^{-1}A$ es noetheriano.
\end{corollary*}

En efecto, el conjunto de ideales de $S^{-1}A$ se aplica, de manera inyectiva y creciente, en aquel de los
ideales de $A$ (proposici\'on~\ref{prop5.1.1}, 1).
Por lo tanto satisface tambi\'en la condici\'on de maximalidad.

\begin{proposition}\label{prop5.1.2}
Sea $R$ un dominio \'integro, $A$ un subanillo de $R$, $S$ una subconjunto multiplicativamente estable de $A$ con
$1\in S$ y $0\notin S$ y $B$ la clausura \'integra de $A$ en $R$. Entonces la clausura \'integra de $S^{-1}A$ en
$S^{-1}R$ es $S^{-1}B$.
\end{proposition}

En efecto, todo elemento de $S^{-1}B$ se escribe en la forma $\frac{b}{s}$ con $b\in B$ y
$s\in S$. Se tiene una ecuaci\'on de dependencia entera $b^{n}+a_{n-1}b^{n-1}+\dots+a_{0}=0$ con
$a_{0}=0$. Dividiendo por $s^{n}$ se obtiene $\left(\frac{b}{s}\right)^{n}+\frac{a_{n-1}}{s}\left(\frac{b}{s}\right)^{n-1}
+\dots+\frac{a_{0}}{s}=0$, lo que muestra que $\frac{b}{s}$ es entero sobre $S^{-1}A$. Reciprocamente, sea
$\frac{x}{s}$ ($x\in R$, $s\in S$) un elemento de $S^{-1}R$ entero sobre $S^{-1}A$. Se tiene una ecuaci\'on de
dependencia entera $\left(\frac{x}{s}\right)^{n}+\frac{a_{n-1}}{t_{n-1}}\left(\frac{x}{s}\right)^{n-1}+
\dots+\frac{a_{0}}{t_{0}}=0$ ($a_{i}\in A$, $t_{i}\in S$); multiplicando por $(t_{0}t_{1}\cdots t_{n-1})^{n}$
vemos que $xt_{0}\cdots t_{n-1}/s$ es entero sobre $A$ y por lo tanto un elemento de $B$. As\'i,
$\frac{x}{s} = \frac{a}{t_{0}\cdots t_{n-1}}\cdot\frac{xt_{0}\cdots t_{n-1}}{s}$ es un elemento de $S^{-1}B$.

\begin{corollary*}
Si $A$ es un anillo \'integramente cerrado, todo anillo de fracciones $S^{-1}A$ tambi\'en es \'integramente
cerrado.
\end{corollary*}

En efecto tomamos $R$ el cuerpo de fraccione de $A$ en la proposici\'on anterior.

\begin{proposition}\label{prop5.1.3}
Si $A$ es un anillo de Dedekind, todo anillo de fracciones $S^{-1}A$ es un anillo de Dedekind.
\end{proposition}

En efecto, $S^{-1}A$ es noetheriano (corolario de la proposici\'on~\ref{prop5.1.1})
e \'integramente cerrado (corolario de la proposici\'on~\ref{prop5.1.2}). Adem\'as,
como ``perdemos'' ideales primos al pasar de $A$ a $S^{-1}A$ (proposiciones~\ref{prop5.1.1},~\ref{prop5.1.2}), todo ideal primo no nulo de $S^{-1}A$
es maximal.

\begin{proposition}\label{prop5.1.4}
Sea $A$ un anillo de Dedekind, $\idl{p}$ un ideal primo no nulo de $A$ y sea $S = A-\idl{p}$. Entonces $S^{-1}A$
es un dominio de ideales principales y existe un elemento primo $p$ de $S^{-1}A$ tal que los \'unicos
ideales no nulos de $S^{-1}A$ son los $(p^{n})_{n\geq 0}$.
\end{proposition}

En efecto, como $\idl{p}$ es el \'unico ideal primo $\neq(0)$ de $A$ contenido en $\idl{p}$, es decir, disjunto con
$S$, el \'unico ideal primo no nulo de $S^{-1}A$ es $\idl{P}=\idl{p}S^{-1}A$
(proposici\'on~\ref{prop5.1.1},~2). Como $S^{-1}A$ es un
anillo de Dedekind (proposici\'on~\ref{prop5.1.3}), sus \'unicos ideales no nulos son los $\idl{P}^{n}$ ($n\geq 0$). Tomamos entonces
$p\in\idl{P}-\idl{P}^{2}$; el ideal $(p)$ que genera est\'a contenido en $\idl{P}$ y no contiene a $\idl{P}^{2}$,
por lo que necesariamente $(p) = \idl{P}$. Los \'unicos ideales no nulos de $S^{-1}A$ son as\'i los $(p^{n})$
y por lo tanto $S^{-1}A$ es un dominio de ideales principales.

\begin{proposition}\label{prop5.1.5}
Sea $A$ un dominio \'integro, $S$ un subconjunto multiplicativamente cerrado de $A$
{\upshape(}$1\in S$, $0\notin S${\upshape)} y $\idl{m}$ un ideal maximal tal que
$\idl{m}\cap S = \emptyset$. Entonces
\begin{gather*}
S^{-1}A/\idl{m}S^{-1}A\simeq A/\idl{m}.
\end{gather*}
\end{proposition}

M\'as precisamente el homomorfismo compuesto $A\to S^{-1}A\to S^{-1}A/\idl{m}S^{-1}A$ tiene n\'ucleo
$\idl{m}S^{-1}A\cap A=\idl{m}$ (proposici\'on~\ref{prop5.1.1},~2)),
de donde una inyecci\'on $\varphi:A/\idl{m}\to S^{-1}A/\idl{m}S^{-1}A$.
S\'olo resta mostrar que $\varphi$ es sobreyectivo. Sea $x = \frac{a}{s}\in S^{-1}A$ ($a\in A$, $s\in S$). Como
$s\notin\idl{m}$ (se tiene $\idl{m}\cap S=\emptyset$) y como $\idl{m}$ es maximal, $s$ es inversible mod $\idl{m}$
y existe $b$ tal que $bs\equiv 1\pmod\idl{m}$. Entonces $\frac{a}{s}-ab=\frac{a}{s}(1-bs)\in\idl{m}S^{-1}A$,
de manera que la imagen por $\varphi$ de la clase de $ab$ es igual a la clase de $\frac{a}{s} = x$. \QED

\section{Descomposici\'on de un ideal primo en una extensi\'on}\label{sec5.2}

\begin{trivlist}
\item\textit{En este \S, denotamos por $A$ un anillo de Dedekind de caracter\'istica cero, por $K$ su cuerpo
de fracciones, por $L$ una extensi\'on de $K$ de grado finito $n$ y por $B$ la clausura \'integra de $A$ en $L$.
Recordemos que $B$ es un anillo de Dedekind (cap\'itulo~\ref{cap3}, \S\ref{sec3.4}, teorema~\ref{teo3.4.1}).}
\end{trivlist}

Sea $\idl{p}$ un ideal primo no nulo de $A$. Entonces $B\idl{p}$ es un ideal de $B$ que tiene una descomposici\'on
\begin{gather}\label{eq-5.2-1}
B\idl{p} = \prod_{i=1}^{q}\idl{P}_{i}^{e_{i}},
\end{gather}
donde los $\idl{P}_{i}$ son ideales primos de $B$, dos a dos distintos, y donde los $e_{i}$ son enteros $\geq 1$.

\begin{proposition}\label{prop5.2.1}
Los $\idl{P}_{i}$ son exactamente los ideales primos $\idl{Q}$ de $B$ tales que $\idl{Q}\cap A=\idl{p}$.
\end{proposition}

En efecto, para un ideal primo $\idl{Q}$ de $B$, la relaci\'on $\idl{Q}\cap A=\idl{p}$ equivale a
$\idl{Q}\supset\idl{p}B$ ($\Rightarrow$ es evidente; $\Leftarrow$ pues $\idl{Q}\cap A$ es un ideal primo de
$A$ y $\idl{p}$ es maximal). La proposici\'on~\ref{prop5.2.1} resulta entonces del formulario de los anillos de Dedekind
(cap\'itulo~\ref{cap3}, \S\ref{sec3.4}).

As\'i $A/\idl{p}$ se identifica a un subm\'odulo de $B/\idl{P}_{i}$. Estos dos anillos son cuerpos. Como $B$
es un $A$-m\'odulo de tipo finito (cap\'itulo~\ref{cap3}, \S\ref{sec3.4}, teorema~\ref{teo3.4.1}),
$B/\idl{P}_{i}$ es un espacio vectorial
de dimensi\'on finita sobre $A/\idl{p}$. Notaremos esta dimensi\'on por $f_{i}$ y la llamaremos el
\emph{grado residual} de $\idl{P}_{i}$ sobre $A$. El exponente $e_{i}$ en~\eqref{eq-5.2-1} se llama el
\emph{\'indice de ramificaci\'on de $\idl{P}_{i}$} sobre $A$. Por \'ultimo observemos que se tiene
$B\idl{p}\cap A = \idl{p}$ ($\supset$ evidente; $\subset$ resulta de que $\idl{P}_{i}\cap A=\idl{p}$), de manera
que $B/B\idl{p}$ es un espacio vectorial sobre $A/\idl{p}$, de dimensi\'on finita sobre este??.

\begin{theorem}\label{teo5.2.1}
Con la notaci\'on de arriba, se tiene
\begin{gather}\label{eq-5.2-2}
\sum_{i=1}^{q}e_{i}f_{i}=[B/B\idl{p}:A/\idl{p}]=n.
\end{gather}
\end{theorem}

La primera igualdad es f\'acil. Consideremos la sucesi\'on de ideales
\begin{gather*}
B\supset\idl{P}_{1}\supset\idl{P}_{1}^{2}\supset\dots\supset\idl{P}_{1}^{e_{1}}\supset\idl{P}_{1}^{e_{1}}\idl{P}_{2}
\supset\dots\supset\idl{P}_{1}^{e_{1}}\idl{P}_{2}^{e_{2}}\supset\dots\supset\idl{P}_{1}^{e_{1}}\cdots\idl{P}_{q}^{e_{q}}
=B\idl{p}.
\end{gather*}
Dos t\'erminos consecutivos son de la forma $\idl{B}$ y $\idl{B}\idl{P}_{i}$; pero como no hay ideales estrictamente
contenidos entre $\idl{B}$ y $\idl{B}\idl{P}_{i}$, $\idl{B}/\idl{B}\idl{P}_{i}$ es un espacio vectorial de dimensi\'on
$1$ sobre $B/\idl{P}_{i}$ (cf. demostraci\'on de la proposici\'on~\ref{prop3.5.2}, \S\ref{sec3.5}, cap\'itulo~\ref{cap3}).
Por lo tanto es un
espacio vectorial de dimensi\'on $f_{i}$ sobre $A/\idl{p}$. Ahora bien, en la sucesi\'on de arriba, hay $e_{i}$
cocientes consecutivos de la forma $\idl{B}/\idl{P}_{i}$ con $i$ dado. La dimensi\'on total $[B:B\idl{p}:A/\idl{p}]$
es igual a la suma de las dimensiones de estos cocientes, es decir a $\sum_{i=1}^{q}e_{i}f_{i}$.

La segunda igualdad es f\'acil tambi\'en en el caso en el cual $B$ es un $A$-m\'odulo libre, en particular cuando
$A$ es un \emph{dominio de ideales principales}
(cap\'itulo~\ref{cap2}, \S\ref{sec2.7}, corolario del teorema~\ref{teo2.7.1}): en efecto una
base $(x_{1},\dots,x_{n})$ del $A$-m\'odulo $B$ da, por reducci\'on mod $B\idl{p}$, una base de $B/B\idl{p}$ sobre
$A/\idl{p}$. Nos reduciremos a este case considerando el subconjunto multiplicativo $S = A-\idl{p}$ de $A$ y
los anillos de fracciones $A' = S^{-1}A$ y $B' = S^{-1}B$. Sabemos que $A'$ es un dominio de ideales principales
en el cual $\idl{p}A'$ es el \'unico ideal maximal (\S\ref{sec5.1}, proposici\'on~\ref{prop5.1.4}) y que $B'$ es la clausura \'integra de $A'$ en $L$
(\S\ref{sec5.1}, proposici\'on~\ref{prop5.1.2}). Por el caso de dominios de ideales principales, se tiene $[B'/\idl{p}B':A'/\idl{p}A']=n$.
Consideremos entonces ls descomposici\'on del ideal $\idl{p}B'$ en el anillo de Dedekind $B'$:
de $\prod_{i=1}^{q}\idl{P}_{i}^{e_{i}}$, deducimos que $\idl{p}B' = \prod_{i=1}^{q}(B'\idl{P}_{i})^{e_{i}}$. Como
$\idl{P}_{i}\cap A=\idl{p}$, (proposici\'on~\ref{prop5.2.1}), se tiene $\idl{P}_{i}\cap S = \emptyset$ y $B'\idl{P}_{i}$ es un ideal primo
no nulo de $B'$ (\S\ref{sec5.1}, proposiciones~\ref{prop5.1.1},~\ref{prop5.1.2}). La primer parte de la demostraci\'on nos da ?????!!!! por lo tanto
\begin{gather*}
[B'/\idl{p}B':A'/\idl{p}A']=\sum_{i=1}^{q}e_{i}[B'/B'\idl{P}_{i}:A'/\idl{p}A'].
\end{gather*}
Ahora bien, se tiene $A'/\idl{p}A'\simeq A/\idl{p}$ y $B'/B'\idl{P}_{i}\simeq B/\idl{P}_{i}$ (\S\ref{sec5.1},
proposici\'on~\ref{prop5.1.5}),
de donde, combinando las igualdades, $n=[B'/\idl{p}B':A'/\idl{p}A']=\sum_{i=1}^{q}e_{i}f_{i}$, lo que
demuestra~\eqref{eq-5.2-2}.

\begin{proposition}\label{prop5.2.2}
Con la misma notaci\'on que antes, el anillo $B/B\idl{p}$ es isomorfo a $\prod_{i=1}^{q}B/\idl{P}_{i}^{e_{i}}$.
\end{proposition}

En efecto, como $\idl{P}_{i}$ es el \'unico ideal maximal de $B$ que contiene a $\idl{P}_{i}^{e_{i}}$, se tiene
$\idl{P}_{i}^{e_{i}}+\idl{P}_{j}^{e_{j}}=B$ si $i\neq j$. Aplicamos entonces~\eqref{eq-5.2-1} y el
lema de \S\ref{sec1.3}, cap\'itulo~\ref{cap1}.

\subsubsection*{Ejemplo de los cuerpos ciclot\'omicos}
%\minisec{Ejemplo de los cuerpos ciclot\'omicos}

Sea $p$ un n\'umero primo y $z\in\CC$ una ra\'iz primitiva $p$-\'esima de la unidad. Las ra\'ices $p^{r}$-\'esimas
de la unidad, en $\CC$, son entonces los $z^{j}$ ($j=1,\dots,p^{r})$. Entre ellas, las ra\'ices primitivas son los
$z^{j}$ tales que $j$ no es un m\'ultiplo de $p$ y por lo tanto su n\'umero es
\begin{gather*}
\varphi(p^{r}) = p^{r}-p^{r-1}=p^{r-1}(p-1)
\end{gather*}
(cf. cap\'itulo~\ref{cap1}, \S\ref{sec1.6}). Estas rac\'ices primitivas $p^{r}$-\'esimas de la unidad son las ra\'ices del
polinomio ciclot\'omico
\begin{gather}\label{eq-5.2-3}
F(X) = \frac{X^{p^{r}}-1}{X^{p^{r-1}}-1} = X^{p^{r-1}(p-1)}+X^{p^{r-1}(p-2)}+\dots+X^{p^{r-1}}+1.
\end{gather}
Nos proponemos ahora demostrar aqu\'i que se tiene $[\QQ[z]:\QQ] = p^{r-1}(p-1)$, es decir que $F(X)$ es irreducible
(cf. cap\'itulo~\ref{cap2}, \S\ref{sec2.9}). Pongamos $e=p^{r-1}(p-1)$ y sean $z_{1},\dots,x_{e}$ las ra\'ices primitivas
$p^{r}$-\'esimas de la unidad. Como el t\'ermino constante de $F(X+1)$ es $p$, se tiene
\begin{gather*}
\prod_{j=1}^{e}(z_{j}-1) = \pm p.
\end{gather*}
Sea $B$ el anillo de enteros de $\QQ[z]$; evidentemente se tiene $z_{j}\in B$, y tambi\'en $z_{j}-1\in B(z_{k}-1)$
para todo $j$, $k$ pues $z_{j}$ es una potencia $z_{k}^{q}$ de $z_{k}$ y se tiene $z_{k}^{q}-1=(z_{k}-1)(z_{k}^{q-1}+
\dots+z_{k}+1)$. As\'i todos los ideales $B(z_{k}-1)$ son iguales. Tenemos por lo tanto $Bp = B(z_{1}-1)^{e}$.

Escribimos $Bp = \prod_{i=1}^{q}\idl{P}_{i}^{e_{i}}$ donde los $\idl{P}_{i}$ son ideales primos de $B$. Los
$e_{i}$ son por lo tanto todos m\'ultiplos de $e$. Pero como $e\geq[\QQ[z]:\QQ]$ (por~\eqref{eq-5.2-3}), se sigue
que $e\geq \sum_{i=1}^{q}e_{i}f_{i}$ (teorema~\ref{teo5.2.1}). De estas desigualdades opuestas ????? deducimos que
$q=1$, $e=e_{1}$, $f_{1}=1$, $[\QQ[z]:\QQ]=e$. En res\'umen:
\begin{enumerate}
\item $[\QQ[z]:\QQ]=e=p^{r-1}(p-1)$
\item $B(z_{1}-1)$ es un ideal primo de $B$ de grado residual $1$.
\item $Bp = B(z_{1}-1)^{e}$.
\end{enumerate}

\section{Discriminante y ramificaci\'on}\label{sec5.3}

Con la notaci\'on de \S\ref{sec5.2} (sea $B\idl{p}=\prod_{i=1}^{q}\idl{P}_{i}^{e_{i}}$, decimos que un ideal primo
$\idl{p}$ de $A$ \emph{ramifica} en $B$ (o en $L$) si alguno de los \'indices de ramificaci\'on $e_{i}$ es
$\geq 2$. Por medio de la teor\'ia del discriminante (cap\'itulo~\ref{cap2}, \S\ref{sec2.7}) ahora determinaremos los
ideales primos de $A$ que ramifican en $B$, y veremos en particular que s\'olo hay \emph{un n\'umero finito.}
Algunos lemas sobre los discriminantes nos ser\'an \'utiles.

\begin{lemma}\label{lem5.3.1}
Sea $A$ un anillo, $B_{1},\dots, B_{q}$ anillos que contienen a $A$ y que son $A$-m\'odulos libres de rango finito
y $B = \prod_{i=1}^{q}B_{i}$ el anillo producto. Entonces $\disc_{B/A}=\prod_{i=1}^{q}\disc_{B_{i}/A}$
(cf. cap\'itulo~\ref{cap2}, \S\ref{sec2.7}, definici\'on~\ref{def2.7.2}).
\end{lemma}

En efecto una inducci\'on sobre $q$ nos permite suponer que $q=2$. Sean entonces $(x_{1},\dots,x_{m)})$,
$(y_{1},\dots,y_{n})$ dos bases de $B_{1}$ y $B_{2}$ sobre $A$. Con la identificaci\'on cl\'asica de $B_{1}$ y
$B_{2}$ con $B_{1}\times(0)$ y $(0)\times B_{2}$, $(x_{1},\dots,x_{m},y_{1},\dots,y_{n})$ es una base de
$B = B_{1}\times B_{2}$ sobre $A$. Se tiene $x_{i}y_{j}=0$ por definici\'on de la estructura de anillo producto,
de donde $\Tr(x_{i}y_{j}) = 0$. As\'i, el determinante $D(x_{1},\dots,x_{m},y_{1},\dots,y_{n})$ se escribe:
\begin{gather*}
\begin{vmatrix}
\Tr(x_{i}x_{i'}) & 0\\
0 & \Tr(y_{j}y_{j'})
\end{vmatrix}
\end{gather*}
Por lo tanto tiene el valor ????
\begin{gather*}
\det(\Tr(x_{i}x_{i'}))\cdot\det(\Tr(y_{j}y_{j'})),
\end{gather*}
de donde
\begin{gather*}
D(x_{1},\dots,x_{m},y_{1},\dots,y_{n}) = D(x_{1},\dots,x_{m})D(y_{1},\dots,y_{n}).\qedhere%\quad\QED
\end{gather*}

\begin{lemma}\label{lem5.3.2}
Sea $A$ un anillo, $B$ un anillo que contiene a $A$ y que admite una base finita $(x_{1},\dots,x_{n})$ y
$\idl{a}$ un ideal de $A$. Si $x\in B$, notamos $\oline x$ la clase de $x$ en $B/\idl{a}B$. Entonces
$(\oline x_{1},\dots,\oline x_{n})$ es una base de $B/\idl{a}B$ sobre $A/\idl{a}$ y se tiene
\begin{gather}\label{eq-5.3-1}
D(\oline x_{1},\dots,\oline x_{n}) = \oline{D(x_{1},\dots,x_{n})}.
\end{gather}
\end{lemma}

En efecto, sea $x\in B$. Si la matriz de la multiplicaci\'on por $x$, en la base $(x_{i})$ es $(a_{ij})$
($a_{ij}\in A$), la matriz de la multiplicaci\'on por $\oline x$ en la base $(\oline x_{i})$ es la matriz
$(\oline a_{ij})$. Por lo tanto se tiene $\Tr(\oline x) = \oline{\Tr(x)}$, de donde
\begin{gather*}
\Tr(\oline x_{i}\cdot\oline x_{j}) = \oline{\Tr(x_{i}x_{j})},
\end{gather*}
y por lo tanto se obtiene~\eqref{eq-5.3-1} tomando determinantes.

\begin{lemma}\label{lem5.3.3}
Sea $K$ un cuerpo finito o de caracter\'istica cero y $L$ una $K$-\'algebra de dimensi\'on finita sobre $K$.
Para que $L$ sea reducido es necesario y suficiente que $\disc_{L/K}\neq(0)$.
\end{lemma}

Supongamos primero que $L$ no es reducida y sea $x\in L$ un elemento nilpotente no nulo. Ponemos $x_{1}=x$ y
completamos este elemento de base en una base $(x_{1},\dots,x_{n})$ de $L$ sobre $K$. Entonces $x_{1}x_{j}$
es nilpotente y por lo tanto la multiplicaci\'on por $x_{1}x_{j}$ es un endomorfismo nilpotente; por lo tanto
todos sus autovalores son nulos, de donde $\Tr(x_{1}x_{j}) = 0$. La matriz $(\Tr(x_{i}x_{j}))$ tiene por lo tanto
una fila nula, de manera que su determinante $D(x_{1},\dots,x_{n})$ es nulo, de donde $\disc_{L/K}=(0)$.

Reciprocamente, supongamos que $L$ es reducido. Entonces el ideal $(0)$ de $L$ es una intersecci\'on finita
de ideales primos, $(0) = \bigcap_{i=1}^{q}\idl{P}_{i}$ (cap\'itulo~\ref{cap4}, \S\ref{sec4.6}, lema).
Como $L/\idl{P}_{i}$ es un dominio \'integro de dimensi\'on finita sobre $K$, es un cuerpo (cap\'itulo~\ref{cap2},
\S\ref{sec2.1}, proposici\'on~\ref{prop2.1.3}).
Por lo tanto $\idl{P}_{i}$ es un ideal maximal de $L$, de manera que $\idl{P}_{i}+\idl{P}_{j}=L$
si $i\neq j$. As\'i $L$ es isomorfo al producto $\prod_{i=1}^{q}L/\idl{P}_{i}$ (cap\'itulo~\ref{cap1}, \S\ref{sec1.3}, lema~\ref{lem1.3.1}).
Tenemos por lo tanto $\disc_{L/K}=\prod_{i=1}^{q}\disc_{(L/\idl{P}_{i})/K}$
(lema~\ref{lem5.3.1}). Ahora bien,
$\disc_{(L/\idl{P}_{i})/K}\neq(0)$ pues $K$ es finito o de caracter\'istica cero
(cap\'itulo~\ref{cap2}, \S\ref{sec2.7}, proposici\'on~\ref{prop2.7.3}), de donde $\disc_{L/K}\neq(0)$. \QED

\begin{definition}\label{defV.3.1}
Sean $K$ y $L$ dos cuerpos de n\'umeros con $K\subset L$, $A$ y $B$ los anillos de enteros de $K$ y $L$.
Llamamos ideal discriminante de $B$ sobre $A$ (o de $L$ sobre $K$), y lo notamos $\disc_{B/A}$ o $\disc_{L/K}$,
al ideal de $A$ generado por los discriminantes de las bases de $L$ sobre $K$ contenidas en $B$.
\end{definition}

\begin{remark}
Si $(x_{1},\dots,x_{n})$ es una base de $L$ sobre $K$ contenida en $B$,
se tiene $\Tr_{L/K}(x_{i}x_{j})\in A$ (cap\'itulo~\ref{cap2}, \S\ref{sec2.6}, corolario de la proposici\'on~\ref{prop2.6.2}), de donde
$D(x_{1},\dots,x_{n})\in A$. As\'i $\disc_{B/A}$ es un ideal \emph{entero} de $A$. Es \emph{no nulo} por
el cap\'itulo~\ref{cap2}, \S\ref{sec2.7}, proposici\'on~\ref{prop2.7.3}.
\end{remark}

\begin{remark}
Cuando $B$ es un $A$-m\'odulo \emph{libre} (por ejemplo si $A$ es principal)
ya hab\'iamos definido el ideal discriminante $\disc_{B/A}$ como aquel generado por $D(e_{1},\dots,e_{n})$,
donde $(e_{1},\dots,e_{n})$ es una base de $B$ sobre $A$ (cap\'itulo~\ref{cap2}, \S\ref{sec2.7}, definici\'on~\ref{def2.7.2}). Coincide
con el definido arriba pues, para toda base $(x_{i})$ de $L$ sobre $K$ contenida en $B$, se tiene
$x_{i} = \sum_{i}a_{ij}e_{j}$ con $a_{ij}\in A$, de donde
$D(x_{1},\dots,x_{n}) = \det(a_{ij})^{2}D(e_{1},\dots,e_{n})$
(cap\'itulo~\ref{cap2}, \S\ref{sec2.7}, proposici\'on~\ref{prop2.7.1}).
\end{remark}

\begin{theorem}\label{teo5.3.1}
Con la notaci\'on de la def., para que un ideal primo $\idl{p}$ de $A$ ramifique en $B$ es necesario y
suficiente que contena al ideal discriminante $\disc_{B/A}$. S\'olo hay un n\'umero finito de
ideales primos de $A$ que ramifican en $B$.
\end{theorem}

La segunda afirmaci\'on resulta de la primera pues vimos que $\disc_{B/A}\neq(0)$.

Demostremos la primera. Como $B/\idl{p}B\simeq\prod_{i=1}^{q}B/\idl{P}_{i}^{e_{i}}$ (\S\ref{sec5.2}, proposici\'on~\ref{prop5.2.2}),
``$\idl{p}$ ramifica'' es equivalente a ``$B/\idl{p}B$ no reducido'', es decir a
``$\disc_{(B/\idl{p}B)/(A/\idl{p})}=(0)$'' (CHEQUEAR LIBRO) pues $A/\idl{p}$ es un cuerpo finito (lema~\ref{lem5.3.3}). Ahora bien,
si ponemos $S = A-\idl{p}$, $A' = S^{-1}A$, $B' =S^{-1}B$ y $\idl{p}' = \idl{p}A'$, entonces
$A'$ es un dominio de ideales principales (\S\ref{sec5.1}, proposici\'on~\ref{prop5.1.4}),
$B'$ es un $A'$-m\'odulo libre y se
tiene $A/\idl{p}\simeq A'/\idl{p}'$ y $B/\idl{p}B\simeq B'/\idl{p}'B'$
(\S\ref{sec5.1}, proposici\'on~\ref{prop5.1.5}). Por lo tanto, si denotamos por $(e_{1},\dots,e_{n})$ una base de $B'$ sobre $A'$, la
relaci\'on $\disc_{(B/\idl{p}B)/(A/\idl{p})}=(0)$ es equivalente a $D(e_{1},\dots,e_{n})\in\idl{p}'$
(lema~\ref{lem5.3.2}). Por lo tanto, si $D(e_{1},\dots,e_{n})\in\idl{p}'$ y $(x_{1},\dots,x_{n})$ es una base de $L$
sobre $K$ contenida en $B$, se tiene $x_{i} = \sum a'_{ij}e_{j}$ con $a'_{ij}\in A'$
(pues $B\subset B'$), de donde $D(x_{1},\dots,x_{n})=\det(a'_{ij})^{2}D(e_{1},\dots,e_{n})\in\idl{p}'$.
Como $\idl{p}'\cap A = \idl{p}$ (\S\ref{sec5.1}, proposiciones~\ref{prop5.1.1},~\ref{prop5.1.2}),
deducimos que $D(x_{1},\dots,x_{n})\in\idl{p}$ y
$\disc_{B/A}\subset\idl{p}$. Reciprocamente, si $\disc_{B/A}\subset\idl{p}$, se tiene
$D(e_{1},\dots,e_{n})\in\idl{p}'$ pues podemos escribir $e_{i}=\frac{y_{i}}{s}$ con $y_{i}\in B$
y $s\in S$ para $1\leq i\leq n$. As\'i,
\begin{gather*}
D(e_{1},\dots,e_{n}) = s^{-2n}D(x_{1},\dots,x_{n})\in A'\disc_{B/A}\subset A'\idl{p}=\idl{p}'.\qedhere
\end{gather*}

\subsubsection*{Ejemplo de los cuerpos cuadr\'aticos}

Tomemos $K = \QQ$ y $L = \QQ[\sqrt{d}]$, donde $d$ es un entero libre de cuadrados (cap\'itulo~\ref{cap2},
\S\ref{sec2.5}).

a) Si $d\equiv 2\text{ \'o }3\pmod 4$, $(1,\sqrt{d})$ es una base del anillo de enteros de $L$. Como
$\Tr(1) = 2$, $\Tr(\sqrt{d}) = 0$ y $\Tr(d) = 2d$, se tiene $D(1,\sqrt{d}) = 4d$. Los n\'umeros primos
que ramifican en $L$ son por lo tanto $2$ y los divisores primos de $d$.

b) Si $d\equiv 1\pmod 4$, $\left(1,\frac{1+\sqrt{d}}{2}\right)$ es una base del anillo de enteros de $L$.
Se tiene
\begin{gather*}
\Tr(1) = 2,\quad\Tr\left(\frac{1+\sqrt{d}}{2}\right) = 1
\end{gather*}
y
\begin{gather*}
\Tr\left(\left(\frac{1+\sqrt{d}}{2}\right)^{2}\right) = \Tr\left(\frac{d+1}{4}+\frac{1}{2}\sqrt{d}\right)=\frac{d+1}{2}.
\end{gather*}
De donde se obtiene $D\left(1,\frac{1+\sqrt{d}}{2}\right) = 2\cdot\frac{d+1}{2}-1 = d$. Los n\'umeros primos
que ramifican en $L$ son por lo tanto los divisores de $d$.

Observamos que un cuerpo cuadr\'atico $\QQ[\sqrt{d}]$ est\'a un\'ivocamente determinado por su discriminante
$D$; en efecto $4d = D$ si $d\equiv 2\text{ \'o }3\pmod 4$ y $D\equiv 1\pmod 4$ es imposible. Tambi\'en
observamos que el discriminante de un cuerpo cuadr\'atico no es un entero arbitrario.

\subsubsection*{Ejemplo de los cuerpos ciclot\'omicos}

Sea $p$ un n\'umero primo, $z\in\CC$ una ra\'iz primitiva $p$-\'esima de la unidad y $L = \QQ[z]$
el cuerpo ciclot\'omico correspondiente. Sabemos que el anillo $B$ de enteros de $L$ admite $(1,z,\dots,z^{p-2})$
como base sobre $\ZZ$ (cap\'itulo~\ref{cap2}, \S\ref{sec2.9}, teorema~\ref{teo2.9.2})
y que el polinomio minimal $F(X)$ de $z$ sobre $\QQ$
satisface $(X-1)F(X) = X^{p}-1$ (\emph{ibid;} teorema~\ref{teo2.9.1}). Ahora calcularemos el discriminante $\disc_{B/\ZZ}$ utilizando
la f\'ormula $D(1,z,\dots,z^{p-2}) = N(F'(z))$ (cap\'itulo~\ref{cap2}, \S\ref{sec2.7}, f\'ormula~\eqref{eq2.7.6}).
Derivando $(X-1)F(X) = X^{p}-1$, obtenemos $(z-1)F'(z) = pz^{p-1}$ (pues $F(z) = 0$). Como
$N(p)=p^{p-1}$, $N(z) = \pm 1$, $N(z-1)=\pm p$ (cap\'itulo~\ref{cap2}, \S\ref{sec2.9}), tenemos por lo tanto
\begin{gather}
D(1,z,\dots,z^{p-2})=\pm p^{p-2}.
\end{gather}
Se sigue que $p$ es el \emph{\'unico} n\'umero primo que ramifica en $\QQ[z]$.

El siguiente resultado es a veces \'util para determinar el anillo de enteros de un cuerpo de n\'umeros:

\begin{proposition}
Sea $L$ un cuerpo de n\'umeros de grado $n$ sobre $\QQ$ y $(x_{1},\dots,x_{n})$ enteros de $L$
que forman una base de $L$ sobre $\QQ$. Si el discriminante $D(x_{1},\dots,x_{n})$ es libre de cuadrados
entonces $(x_{1},\dots,x_{n})$ es una base sobre $\ZZ$ del anillo $B$ de enteros de $L$.
\end{proposition}

En efecto, si $(e_{1},\dots,e_{n})$ es una base de $B$ sobre $\ZZ$, se tiene $x_{1}=\sum_{j=1}^{n}a_{ij}e_{j}$
con $a_{ij}\in\ZZ$. De donde se sigue que $D(x_{1},\dots,x_{n})=\det(a_{ij})^{2}D(e_{1},\dots,e_{n})$.
Como $D(x_{1},\dots,x_{n})$ es libre de cuadrados, deducimos que $\det(a_{ij})=\pm 1$, lo que implica que
$(x_{1},\dots,x_{n})$ tambi\'en es una base de $B$ sobre $\ZZ$.

\begin{comm}
El ejemplo de los cuerpos ciclot\'omicos (para $p\geq 5$), o de los cuerpos cuadr\'aticos, muestra que
la condici\'on suficiente anterior no es necesaria.
\end{comm}

\begin{example*}
El polinomio $X^{3}-X-1$ (resp. $X^{3}+X+1$, $X^{3}+10X+1$) es \emph{irreducible} sobre $\QQ$.
Sino, en efecto, tendr\'ia un factor lineal, y por lo tanto una ra\'iz $x\in\QQ$ y se tendr\'ia entonces
que $x\in\ZZ$ pues el polinomio es m\'onico. Como el t\'ermino constante es $1$, tendr\'iamos que
$x=\pm 1$ (pues todo factor de $x$ divide al t\'ermino constante). Pero esto no es verdad. Por lo tanto,
si denotamos por $x\in\CC$ una ra\'iz de este polinomio, el cuerpo $L = \QQ[x]$ es un \emph{cuerpo c\'ubico}
(i.e. de grado $3$). As\'i $(1,x,x^{3})$ es una base de $L$ sobre $\QQ$ y $x$ es evidentemente un entero de
$L$. Como, utilizando la f\'ormula~\eqref{eq2.7.7} del \S\ref{sec2.7}, cap\'itulo~\ref{cap2}, se tiene
que $D(1,x,x^{2}) = -4+27 = 23$ (resp.~$31$,~$4027$), que es un n\'umero primo.  Por lo tanto
$(1,x,x^{2})$ es una base sobre $\ZZ$ del anillo de enteros de $L$.
\end{example*}

\section{Descomposici\'on de un n\'umero primo en un cuerpo cuadr\'atico}\label{sec5.4}

Sea $d\in\ZZ$ un entero libre de cuadrados, $L$ el cuerpo cuadr\'atico $L = \QQ[\sqrt{d}]$,
$B$ el anillo de enteros de $L$ y $p$ un n\'umero primo. Estudiaremos la descomposici\'on en ideales
primos del ideal $pB$.

La f\'ormula $\sum_{i=1}^{q}e_{i}f_{i}=2$ (\S\ref{sec5.2}, teorema~\ref{teo5.2.1}) muestra que se tiene $q\leq 2$ y que
s\'olo puede producirse tres casos:
\begin{enumerate}
\item $q=2$, $e_{1}=e_{2}=1$, $f_{1}=f_{2}=1$. Decimos entonces que $p$ \emph{se descompone totalmente}
en $L$;
\item $q=1$, $e_{1}=1$, $f_{1}=2$; decimos entonces que $p$ es \emph{inerte} en $L$;
\item $q=1$, $e_{1}=2$, $f_{1}=1$; esto quiere decir que $p$ \emph{ramifica} en $L$.
\end{enumerate}
Examinemos primero el \emph{caso cuando $p$ es impar.} Sabemos (cap\'itulo~\ref{cap2}, \S\ref{sec2.5}) que
$B = \ZZ+\ZZ\sqrt{d}$ o $B = \ZZ+\ZZ\left(\frac{1+\sqrt{d}}{2}\right)$ seg\'un el valor de $d$. Pero,
si tomamos las clase de $B$ m\'odulo $Bp$, vemos, en el segundo caso, que $a+b\left(\frac{1+\sqrt{d}}{2}\right)$
(con $b$ impar) es congruente a $a+(b+p)\left(\frac{1+\sqrt{d}}{2}\right)$, que es un elemento de
$\ZZ+\ZZ\sqrt{d}$. Por lo tanto, en todos los casos, se tiene
\begin{gather*}
B/Bp\simeq(\ZZ+\ZZ\sqrt{d})/(p).
\end{gather*}
Ahora bien, $\ZZ+\ZZ\sqrt{d}\simeq\ZZ[X]/(X^{2}-d)$, de donde se sigue
\begin{gather*}
B/Bp\simeq\ZZ[X]/(p,X^{2}-d)\simeq(\ZZ[X]/(p))/(X^{2}-d)\simeq\FF_{p}[X]/(X^{2}-\oline d),
\end{gather*}
donde $\oline d$ denota la clase de $d$ m\'odulo $p$. Ahora bien, la afirmaci\'on de que $p$ se
descompone totalmente (resp. es inerte, ramifica) en $B$ significa que $B/Bp$ es un producto de dos cuerpos
(resp. es un cuerpo, tiene elementos nilpotentes) (cf. \S\ref{sec5.2}, proposici\'on~\ref{prop5.2.2}), esto significa por lo tnto que,
en $\FF_{p}[X]$, el polinomio $X^{2}-d$ es producto de dos factores distintos de grado uno (resp.
es irreducible, es un cuadrado). Y esto ocurre si $\oline d$ es un cuadrado no nulo en $\FF_{p}$
(resp. no es un cuadrado en $\FF_{p}$, es nulo en $\FF_{p}$). Cuando $\oline d$ es un cuadrado no nulo
en $\FF_{p}$ (resp. no es un cuadrado en $\FF_{p}$) decimos que $d$ es un \emph{residue cuadr\'atico}
(resp. un \emph{no residue cuadr\'atico}) m\'odulo $p$.

Tratemos ahora \emph{el caso} $p=2$. Si $d\equiv 2, 3\pmod 4$, se tiene $B=\ZZ+\ZZ\sqrt{d}$, de donde,
como m\'as arriba $B/2B\simeq\FF_{2}[X]/(X^{2}-\oline d)$. Ahora bien, $X^{2}-\oline d$ vale $X^{2}$
o $X^{2}+1 = (X+1)^{2}$ y es por lo tanto un cuadrado. As\'i $2$ ramifica en $B$. Si $d\equiv 1\pmod 4$,
$\frac{1+\sqrt{d}}{2}$ admite $X^{2}-X-\frac{d-1}{4}$ como polinomio minimal, de donde, como m\'as arriba,
$B/2B\simeq\FF_{2}[X]/(X^{2}-X-\delta)$ donde $\delta$ es la clase mod $2$ de $\frac{d-1}{4}$. Si
$d\equiv 1\pmod 8$ se tiene $\delta=0$ y $X^{2}-X-\delta=X(X-1)$, de manera que $2$ se descompone
totalmente. Si $d\equiv 5\pmod 8$, se tiene $\delta=1$ y $X^{2}-X-\delta=X^{2}+X+1$ es irreducible en
$\FF_{2}[X]$, de manera que $2$ es inerte.

En res\'umen, hemos demostrado los siguientes resultados:

\begin{proposition}\label{prop5.4.1}
Sea $L = \QQ[\sqrt{d}]$ un cuerpo cuadr\'atico, donde $d\in\ZZ$ es libre de cuadrados.
\begin{enumerate}
\item Se descomponent totalmente en $L$: los n\'umeros primos impares $p$ tales que
$d$ es un residuo cuadr\'atico m\'od $p$ y $2$ y $d\equiv 1\pmod 8$;
\item Son inertes en $L${\upshape:} los n\'umeros primos impares $p$ tales que $d$ no es un
residue cuadr\'atico mod $p$ y $2$ si $d\equiv 5\pmod 8$;
\item Ramifican en $L$: los divisores primos impares de $d$ y $2$ si $d\equiv 2\text{ \'o }3\pmod 4$.
\end{enumerate}
\end{proposition}

La afirmaci\'on {\itshape c}) ya hab\'ia sido demostrada en un ejemplo del \S\ref{sec5.3}.

\section{La ley de reciprocidad cuadr\'atica}\label{sec5.5}

Dados un n\'umero primo \emph{impar} $p$ y un entero $d$ coprimo con $p$, hemos introducido en el \S\ref{sec5.4}
la frase ``$d$ es \emph{un residue cuadr\'atico} mod $p$'' (resp. ``$d$ \emph{no es un residue cuadr\'atico}
mod $p$'') como queriendo decir que la clase de $d$ mod $p$ es un cuadrado (resp. no es un cuadrado) en
$\FF_{p}^{*}$. Ahora introducimos aqu\'i el \emph{s\'imbolo de Legendre} $\left(\frac{d}{p}\right)$ definido
as\'i:
\begin{gather}
\begin{cases}
\left(\frac{d}{p}\right) = +1  & \text{si $d$ es un residuo cuadr\'atico mod $p$.}\\
\left(\frac{d}{p}\right) = -1 & \text{si $d$ no es un residuo cuadr\'atico mod $p$.}
\end{cases}
\end{gather}
Por supuesto ???? $\leg{d}{p}$ s\'olo est\'a definido para $d$ coprimo con $p$, es decir para
$d\in\ZZ-p\ZZ$. Como el grupo multiplicativo $\FF_{p}^{*}$ es c\'iclico de orden par $p-1$
(cap\'itulo~\ref{cap1}, \S\ref{sec1.7}, teorema~\ref{teo1.7.1}), los cuadrados forman un subgrupo $\FF_{p}^{*2}$ de \'indice $2$ y
$\FF_{p}^{*}/\FF_{p}^{*2}$ es isomorfo a $\{+1,-1\}$. As\'i el s\'imbolo de Legendre se obtiene componiendo
los homomorfismo
\begin{gather*}
\ZZ-p\ZZ\to\FF_{p}^{*}\to\FF_{p}^{*}/\FF_{p}^{*2}\to[\sim]\{+1,-1\}.
\end{gather*}
Por lo tanto se tiene la f\'ormula de multiplicatividad
\begin{gather}\label{eq-5.5-2}
\leg{ab}{p} = \leg{a}{p}\leg{b}{p}.
\end{gather}

\begin{proposition}[Criterio de Euler]\label{prop5.5.1}
Si $p$ es un n\'umero primo impar y si $a\in\ZZ-p\ZZ$, se tiene $\leg{a}{p}\equiv a^{\frac{p-1}{2}}\pmod p$.
\end{proposition}

En efecto sea $w$ una ra\'iz primitiva mod $p$ (cap\'itulo~\ref{cap1}, \S\ref{sec1.7}). Se tiene
$a\equiv w^{j}\pmod p$ para alg\'un $0\leq j\leq p-2$ pues la clase $\oline w$ de $w$ es un generador
de $\FF_{p}^{*}$. Es claro que ``$a$ residue cuadr\'atico'' equivale a ``$j$ par''. Por lo tanto se tiene
$\leg{a}{p}=(-1)^{j}$. Por otra parte, $\FF_{p}^{*}$ tiene un s\'olo elemento de orden $2$, a saber
$\oline w^{\frac{p-1}{2}}$ y necesariamente debe ser igual a $-1$ pues su cuadrado es $1$. Por lo tanto,
en $\ZZ$, se tiene $-1\equiv w^{\frac{p-1}{2}}\pmod p$. As\'i,
\begin{gather*}
\leg{a}{p}=(-1)^{j}\equiv w^{j\cdot\frac{p-1}{2}}\equiv a^{\frac{p-1}{2}}\pmod p.\qedhere
\end{gather*}

Ahora demostraremos un resultado celebre, que muestra que las propiedades de congruencias m\'odulo dos
n\'umeros primos impares distintos no son independientes.

\begin{theorem}[``Ley de reciprocidad cuadr\'atica de Legendre-Gauss'']\label{teo5.5.1}
Si $p$ y $q$ son dos n\'umeros primos impares distintos, se tiene
\begin{gather*}
\leg{p}{q}\leg{q}{p}=(-1)^{\frac{(p-1)(q-1)}{4}}.
\end{gather*}
\end{theorem}

En efecto, consideremos, en una extensi\'on conveniente de $\FF_{q}$, una ra\'iz primitiva $p$-\'esima de
la unidad $w$. Como $w^{p}=1$, la notaci\'on $w^{x}$ tiene sentido para $x\in\FF_{p}$. Tambi\'en escribiremos
el s\'imbolo de Legendre $\leg{x}{p}$ para $x\in\FF_{p}^{*}$ pues $\leg{d}{p}$ s\'olo depende evidentemente
de la clase de $d$ mod $p$. Si $x\in\FF_{p}^{*}$, consideremos la \emph{``suma de Gauss''}.
\begin{gather}
\tau(a) = \sum_{x\in\FF_{p}^{*}}\leg{x}{p}w^{ax}.
\end{gather}
Es un elemento de una extensi\'on de $\FF_{q}$. Poniendo $ax = y$, se tiene
\begin{gather*}
\tau(a) = \sum_{y\in\FF_{p}^{*}}\leg{ya^{-1}}{p}w^{y} = \leg{a^{-1}}{p}\sum_{y\in\FF_{p}^{*}}\leg{y}{p}w^{y}
\end{gather*}
(por~\eqref{eq-5.5-2}), de donde
\begin{gather}\label{eq-5.5-4}
\tau(a) = \leg{a}{p}\tau(1).
\end{gather}
Por otra parte, como estamos calculando en caracter\'istica $q$ y $\leg{x}{p}\in\FF_{q}$, se tiene
$\tau(1)^{q}=\sum_{x\in\FF_{p}^{*}}\leg{x}{p}^{q}w^{qx}$, de donde, identificando $q$ con su clase mod $p$,
\begin{gather}\label{eq-5.5-5}
\tau(1)^{q}=\tau(q).
\end{gather}
Calculemos ahora $\tau(1)^{2}$. Se tiene
\begin{gather*}
\tau(1)^{2} = \sum_{x,y\in\FF_{p}^{*}}\leg{x}{p}\leg{y}{p}w^{x+y}.
\end{gather*}
Poniendo $y=tx$, esto se vuelve ????
\begin{gather*}
\tau(1)^{2}=\sum_{x,t\in\FF_{p}^{*}}\leg{x}{p}^{2}\leg{t}{p}w^{x(1+t)}=\sum_{x,t}\leg{t}{p}w^{x(1+t)}
=\sum_{t\in\FF_{p}^{*}}\left[\leg{t}{p}\sum_{x\in\FF_{p}^{*}}w^{x(1+t)}\right].
\end{gather*}
Si $w^{1+t}\neq 1$, se tiene $\sum_{j=0}^{p-1}(w^{1+t})^{j}=0$ por la f\'ormula de la sucesi\'on geom\'etrica,
pues $(w^{1+t})^{p}=1$, de donde $\sum_{x\in\FF_{p}^{*}}w^{x(1+t)}=-1$. Si $w^{1+t}=1$, se tiene
$\sum_{x\in\FF_{p}^{*}}w^{x(1+t)}=p-1$. Esto \'ultimo ocurre unicamente cuando $t=-1$, pues
$w$ es una ra\'iz primitiva $p$-\'esima de la unidad. Tenemos por lo tanto
\begin{gather*}
\tau(1)^{2}=\leg{-1}{p}(p-1)-\sum_{t\in\FF_{p}^{*}, t\neq -1}\leg{t}{p}.
\end{gather*}
Como hay tantos cuadrados como no cuadrados en $\FF_{p}^{*}$, se tiene
\begin{gather*}
\sum_{t\in\FF_{p}^{*}}\leg{t}{p}=0,\quad\text{de donde}\quad\tau(1)^{2}=\leg{-1}{p}(p-1)+\leg{-1}{p}=\leg{-1}{p}p.
\end{gather*}
Por el criterio de Euler (proposici\'on~\ref{prop5.5.1}), tenemos por lo tanto,
\begin{gather}\label{eq-5.5-6}
\tau(1)^{2}=(-1)^{\frac{p-1}{2}}p.
\end{gather}
Por \'ultimo, por~\eqref{eq-5.5-4} y~\eqref{eq-5.5-5}, tenemos $\tau(1)^{q}=\tau(q)=\leg{q}{p}\tau(1)$. Como
$\tau(1)$ es no nulo por~\eqref{eq-5.5-6}, podemos simplificar: $\tau(1)^{q-1}=\leg{q}{p}$. Por~\eqref{eq-5.5-6}
de nuevo tenemos
\begin{gather*}
\leg{q}{p}=(\tau(1)^{2})^{\frac{q-1}{2}}=(-1)^{\frac{(p-1)(q-1)}{4}}p^{\frac{q-1}{2}}.
\end{gather*}
Como $p^{\frac{q-1}{2}}=\leg{q}{p}$ (proposici\'on~\ref{prop5.5.1})
y $\leg{p}{q} = \leg{p}{q}^{-1}$, la ley de reciprocidad
cuadr\'atica queda demostrada.

\begin{proposition}[``f\'ormulas complementarias'']\label{prop5.5.2}
Si $p$ es un n\'umero primo impar se tiene
\begin{enumerate}
\item $\leg{-1}{p}=(-1)^{\frac{p-1}{2}}$.
\item $\leg{2}{p} = (-1)^{\frac{p^{2}-1}{8}}$. CHECK THIS.
\end{enumerate}
\end{proposition}

En efecto {\itshape a}) es un caso particular del criterio de Euler (proposici\'on~\ref{prop5.5.1}).
Demostremos por lo tanto
{\itshape b}). Observemos primero que nada que, como los cuadrados de $1, 3, 5, 7\bmod 8$ son $1,1,1,1$,
se tiene $p^{2}\equiv 1\pmod 8$ y la f\'ormula escrita tiene sentido. Obeservemos a continuaci\'on
que, en el grupo $H = \{1,3,5,7\}$ de elementos inversibles de $\ZZ/8\ZZ$, $\{1,7\}$ es un subgrupo
$H'$ de \'indice $2$. Pongamos $\theta(x) = 1$ si $x\in H'$ y $\theta(x) = -1$ si $x\in H-H'$, de manera
que $\theta(xy) = \theta(x)\theta(y)$ para $x,y\in H$. Sea entonces $w$ una ra\'iz primitiva $8$-\'esima
de la unidad en una extensi\'on de $\FF_{p}$. Como en el teorema~\ref{teo5.5.1},
consderemos, para $a\in H$, la \emph{``suma de Guass''}.
\begin{gather}
\tau(a) = \sum_{x\in H}\theta(x)w^{ax}.
\end{gather}
Como en el teorema~\ref{teo5.5.1}, se tiene $\tau(a) = \theta(a)\tau(1)$ y $\tau(1)^{p}=\tau(p)$ (identificando
$p$ con su clase mod $8$). Por la definici\'on de $\theta(x)$, se tiene
\begin{align*}
\tau(1) &= w-w^{3}-w^{5}+w^{7}=(1-w^{2})(w-w^{5})\\
&= w(1-w^{2})(1-w^{4})=2w(1-w^{2})
\end{align*}
(pues $w^{8}=1$ y $w^{4}=-1$), de donde
\begin{gather*}
\tau(1)^{2}=4w^{2}(1-2w^{2}+w^{4})=-8w^{4}=8.
\end{gather*}
Como en el teorema~\ref{teo5.5.1}, deducimos que $\tau(1)^{p}=\tau(p) = \theta(p)\tau(1)$, de donde, simplificando,
$\theta(p) = (\tau(1)^{2})^{\frac{p-1}{2}}=8^{\frac{p-1}{2}}=\leg{8}{p}
\text{ (proposici\'on~\ref{prop5.5.1})}=
\leg{2}{p}^{2}=\leg{2}{p}$. Tenemos por lo tanto que $\leg{2}{p} = \theta(p)$. Pero como podemos
constatar con un c\'alculo directo, si $x=1,3,5,7$ (o, m\'as eficazmente, si $x = 1,3,-3,-1$), se
tiene $\theta(x) = (-1)^{\frac{x^{2}-1}{2}}$ y $\frac{x^{2}-1}{2}$ s\'olo depende de la clase de $x\bmod 8$. \QED

%\begin{remark*}[Ejemplo de aplicaci\'on]
\begin{example*}
%\begin{labeling}{Ejemplo}
%\item[\bfseries\sffamily Ejemplo]
La ley de reciprocidad cuadr\'atica y las f\'romulas complementarias permiten calcular el s\'imbolo de Legendre por
reducci\'on sucesiva. Calculemos as\'i $\leg{23}{59}$, sin escribir la larga table de los cuadrados
m\'odulo $49$. Se tiene
\begin{align*}
\leg{23}{59} &= (-1)^{11\cdot 29}\leg{59}{23} = -\leg{13}{23} = -(-1)^{6\cdot 11}\leg{23}{13}=-\leg{10}{13}\\
&= -\leg{-3}{13} = -\leg{-1}{13}\leg{3}{13} =-(-1)^{6}\leg{3}{13}\\
&=-(-1)^{6\cdot 1}\leg{13}{3}=-\leg{1}{3}=-1.
\end{align*}
Por lo tanto $23$ no es un cuadrado m\'odulo $59$.
%\end{labeling}
%\end{remark*}
\end{example*}

\section{Teorema de los dos cuadrados}\label{sec5.6}

Ahora aplicaremos la proposici\'on~\ref{prop5.4.1} del \S\ref{sec5.4} al cuerpo $\QQ[i]$ donde $i^{2}=-1$. Como $-1\equiv 3\pmod 4$,
el anillo $B$ de enteros de $L$ es $\ZZ+\ZZ i$. Se llama \emph{el anillo de enteros de Gauss.} Su discriminante
es $-4$ (\S\ref{sec5.3}, ejemplo). Si $p$ es un n\'umero primo impar y si $u$ es un generador del grupo c\'iclico
$\FF_{p}^{*}$, se tiene $-1=u^{\frac{p-1}{2}}$. Por lo tanto $-1$ es un cuadrado en $\FF_{p}$ si y s\'olo si
$\frac{p-1}{2}$ es par. De esto se sigue la clasificaci\'on:
\begin{itemize}
\item $2$ ramifica en $\QQ[i]$;
\item los n\'umeros primos de la forma $4k+1$ se descomponen totalmente;
\item los n\'umeros primos de la forma $4k+3$ quedan intertes.
\end{itemize}
El siguiente resultado nos ser\'a muy \'util:

\begin{proposition}\label{prop5.6.1}
El anillo $B = \ZZ+\ZZ i$ de los enteros de Gauss es un dominio de ideales principales.
\end{proposition}

Aplastemos, en efecto, esta mosca con un ``gros pave'' ???? Con la notaci\'on del cap\'itulo~\ref{cap4},
\S\ref{sec4.3}, tenemos $n=2$, $r_{1}=0$, $r_{2}=1$, $d=-4$. Por lo tanto (cap\'itulo~\ref{cap4}, \S\ref{sec4.3},
corolario de la proposici\'on~\ref{prop4.3.1}),
toda clase de ideales de $B$ contiene un ideal entero de norma $\leq\frac{4}{\pi}\cdot\frac{2}{4}
\abs{4}^{1/2}=\frac{4}{\pi}$, por lo tanto contiene al ideal unidad $B$ (que es el \'unico ideal entero de
norma $1$) pues $\frac{4}{\pi}<2$. As\'i todo ideal de $B$ es equivalente al ideal principal $B$, y es por
lo tanto principal. \QED

\begin{comm}
\emph{Sketch???} de una demostraci\'on elemental: como los puntos de $B$ forman un ret\'iculo de $\CC$,
un poco de geometr\'ia muestra que, para todo $x\in\QQ[i]$, existe $z\in B$ tal que
$N(x-z)=\abs{x-z}^{2}\leq\left(\frac{1}{\sqrt{2}}\right)^{2}=\frac{1}{2}< 1$; entonces, si $\idl{a}$ es un
ideal no nulo de $B$, elegimos en $\idl{a}$ un elemento no nulo $u$ de norma minimal (NB: esta norma es un
entero $>0$); para $v\in\idl{a}$ aproximamos $\frac{v}{u}$ por un $z\in B$ tal que $N\left(\frac{v}{u}-z\right)<1$.
Entonces
\begin{gather*}
N(v-zu) < N(u),\quad\text{de donde}\quad v-zu=0
\end{gather*}
pues $v-zu\in\idl{a}$. En consecuencia $v\in Bu$ y $\idl{a}=Bu$. Observemos la analog\'ia con el proceso de
divisi\'on euclidea en $\ZZ$.
\end{comm}

\begin{proposition}[Fermat]\label{prop5.6.2}
Todo n\'umero primo $p\equiv 1\pmod 4$ es suma de dos cuadrados (i.e. es de la forma $p=a^{2}+b^{2}$ con
$a,b\in\NN$).
\end{proposition}

En efecto $Bp$ se descompone en un producto $\idl{p}_{1}\idl{p}_{2}$ de ideales primos distintos.
De esto se sigue que $p^{2}=N(Bp) = N(\idl{p}_{1})N(\idl{p}_{2})$ (cap\'itulo~\ref{cap3}, \S\ref{sec3.5},
proposici\'on~\ref{prop3.5.2}).
Como las normas de $\idl{p}_{1}$ y de $\idl{p}_{2}$ son distintas de $1$, necesariamente se tiene
\begin{gather*}
N(\idl{p}_{1}) = N(\idl{p}_{2}) = p.
\end{gather*}
Pero como $\idl{p}_{1}$ es un ideal principal $B(a+bi)$ ($a,b\in\ZZ$)
(proposici\'on~\ref{prop5.6.1}), se sigue que, tomando
normas, $p = N(a+bi)=a^{2}+b^{2}$. \QED

\begin{theorem}
Sea $x$ un entero natural y $x = \prod_{p}p^{v_{p}(x)}$ su descomposici\'on en factores primos. Para que $x$
sea suma de dos cuadrados es necesario y suficiente que, para todo $p\equiv 3\pmod 4$, el exponente
$v_{p}(x)$ sea par.
\end{theorem}

Para demostrar la suficiencia, observemos que una suma de dos cuadrados $a^{2}+b^{2}$ es la norma
$N(a+bi)$ de un elemento de $B$; por la multiplicatividad de las normas, el conjunto $S$ de sumas de dos
cuadrados es por lo tanto estable por multiplicaci\'on. Como $2 = 1^{2}+1^{2}\in S$ y como todo cuadrado
es elemento de $S$ ($x^{2}=x^{2}+0^{2}$), se sigue entonces de la
proposici\'on~\ref{prop5.6.2}, que nuestra condici\'on es suficiente.

Reciprocamente, sean $x=a^{2}+b^{2}$ una suma de dos cuadrados ($a,b\in\NN$) y $p$ un n\'umero primo
$\equiv 3\pmod 4$. Vimos que el ideal $Bp$ de $B$ es primo. Por otra parte, se tiene
$x = a^{2}+b^{2}=(a+bi)(a-bi)$. Sea $n$ el exponente de $Bp$ en la descomposici\'on de $B(a+bi)$ en factores
primos. Como $Bp$ es estable por el automorfismo $\sigma:u+iv\mapsto u-iv$ de $B$, y $\sigma(a+ib)=a-ib$,
el exponente de $Bp$ en la descomposici\'on de $B(a-ib$ tambi\'en es $n$. En la descomposici\'on de
$B(a^{2}+b^{2})$, el exponente de $Bp$ es por lo tanto $2n$. Como ning\'un n\'umero primo distinto de
$p$ pertenece a $Bp$ (pues $Bp\cap\ZZ = p\ZZ$), se tiene $v_{p}(x) = 2n$ y $v_{p}(x)$ es par. \QED

\section{Teorema de los cuatro cuadrados}\label{sec5.7}

En esta \S, nos proponemos demostrar el siguiente teorema:

\begin{theorem}[Lagrange]\label{teo5.7.1}
Todo entero natural es suma de cuatro cuadrados.
\end{theorem}

El m\'etodo empleado es an\'alogo de aquel de \S\ref{sec5.6}: en vez del anillo de enteros de Gauss, nosotros
trabajaremos con un anillo de \emph{quaterniones} convenientemente elegido.

Comencemos definiendo los quaterniones. Dadu un anillo $A$, notaremos $(1,i,j,k)$ la base can\'onica del
$A$-m\'odulo $A^{4}$, y definimos una multiplicaci\'on por:
\begin{gather}\label{eq-5.7-1}
\begin{cases}
\text{$1$ es el elemento unidad}\\
\text{$i^{2}=j^{2}=k^{2}=-1$.}\\
\text{$ij=-ji=k, jk=-kj=i, ki=-ik=j$.}
\end{cases}
\end{gather}
Extendemos esta multiplicaci\'on a los elementos $a+bi+cj+dk$ de $A^{4}$ por linearidad. La distributividad
es entonces evidente. En cuanto a la asociatividad basta verficarla en elementos de la base: as\'i
\begin{gather*}
i(jk) = i^{2} = -1 = k^{2} = (ij)k;
\end{gather*}
como las f\'ormulas donde figura $1$ son evidentes, s\'olo restan $3^{3}-1=26$ f\'ormulas para verificar;
el lector paciente e incr\'edulo observar\'a reducir\'a el n\'umero a verificar por permutaci\'on y los
dem\'as creer\'an la palabra del autor ???. Equipada con esta multiplicaci\'on, $A^{4}$ es por lo tanto
un \emph{anillo no necesariamente conmutativo,} e incluso una \emph{$A$-\'algebra,} que llamamos
\emph{el anillo de los cuaterniones} sobre $A$ y que notamos $\HH(A)$ ($\HH$ en honor a W.R. Hamilton, inventor
de los cuaterniones).

Dado un cuaterni\'on $z = a+bi+cj+dk$ sobre $A$ (escribimos $a$ en vez de $a\cdot 1$), llamamos
\emph{cuaterni\'on conjugado} de $z$, y notamos $\oline z$, al cuaterni\'on $\oline z = a-bi-cj-dk$.

\begin{lemma}
Se tiene $\oline{z+z'} = \oline z+\oline z'$, $\oline{zz'} = \oline z'\cdot\oline z$ y
$\oline{\oline z}=z$. En t\'erminos m\'as savants ????, $z\mapsto\oline z$ es un antihomomorfismo involutivo
de $\HH(A)$.
\end{lemma}

La primer y la tercer f\'ormula son evidentes. Para la segunda basta demostrar, por linealidad,
que $\oline{xy} = \oline{y}\oline x$ cuando $x,y\in\{1,i,j,k\}$.

Todo est\'a claro si $x=1$ o si $y=1$. Si $x=y=i$, tenemos $\oline{xy} = \oline{-1}=-1$ y
$\oline y\cdot\oline x = (-i)(-i) = i^{2}=-1$. Si $x=i$ y $j=j$ se tiene
\begin{gather*}
\oline{xy} = \oline k = -k\quad\text{y}\quad\oline y\cdot\oline x = (-j)(-i) = ji = -k.
\end{gather*}
Las dem\'as verificaciones se deducen permutando los t\'erminos. \QED

Dado un cuaterni\'on $z$ sobre $A$, llamamos \emph{norma reducida} de $z$, y la notamos
$N(z)$, al cuaterni\'on $z\oline z$.

\begin{lemma}\label{lem5.7.2}
a) Dado un cuaterni\'on $z = a+bi+cj+dk$ sobre $A$, se tiene
$N(z) = a^{2}+b^{2}+c^{2}+d^{2}$ (!`cuatro cuadrados!), por lo tanto $N(z)\in A$.

b) Dados dos cuaterniones $z$, $z'$ sobre $A$, se tiene $N(zz') = N(z)N(z')$.
\end{lemma}

Para {\itshape a}), desarollamos $(a+bi+cj+dk)(a-bi-cj-dk)$: por~\eqref{eq-5.7-1} los t\'erminos
``rect\'angulos'' ???? desaparecen, y lo que resta es $a^{2}+b^{2}+c^{2}+d^{2}$. Vemos tambi\'en que
$z\oline z = \oline z z$. Entonces
\begin{gather*}
N(zz') = zz'\cdot\oline{zz'} = zz'\oline z'\oline z = zN(z')\oline z = z\oline zN(z')
\end{gather*}
(pues el elemento $N(z')$ de $A$ conmuta con cualquier cueterni\'on), de donde $N(zz') = N(z)\cdot N(z')$. \QED

\begin{comm}
El lema~\ref{lem5.7.2} muestra que, en un anillo $A$ (conmutativo), el conjunto de las normas reducidas de cuaterniones,
es decir, las sumas de cuatro cuadrados, es estable por multiplicaci\'on.
\end{comm}

Ahora consideraremos, en $\HH(\QQ)$, el subanillo no conmutativo $\HH(\ZZ)$ y el conjunto $\HH$ de los
\emph{``cuaterniones de Hurwitz''} $a+bi+cj+dk$ donde los $a, b, c, d$ est\'an todos en $\ZZ$ o
bien est\'an todos en $\frac{1}{2}+\ZZ$.

\begin{lemma}
\begin{enumerate}
\item El conjunto $\HH$ de los cuaterniones de Hurwitz es un subanillo no conmutativo de $\HH(\QQ)$ que
contiene a $\HH(\ZZ)$ y estable por $z\mapsto\oline z$.
\item Para todo $z\in\HH$, se tiene $z+\oline z\in\ZZ$ y $N(z) = z\oline z\in\ZZ$
\item Para que $z\in H$ sea inversible, es neceseario y sufciente que $N(z) = 1$
\item Todo ideal a izquierda (reps. a drecha) $\idl{a}$ de $\HH$ es principal (i.e. es de la forma
$\HH z$ (resp. $z\HH$)).
\end{enumerate}
\end{lemma}

Para {\itshape a}), todas las afirmaciones son evidentes, salvo la estabilidad de $\HH$ bajo la multiplicaci\'on.
Para eso, basta verificar que, si ponemos
\begin{gather*}
u = \frac{1}{2}(1+i+j+k),
\end{gather*}
se tiene $u\cdot 1$, $u\cdot i$, $u\cdot j$, $u\cdot k$ y $u^{2}\in\HH$. Ahora bien,
$u\cdot 1 = \frac{1}{2}(1+i+j+k)$, $u\cdot i = \frac{1}{2}(-1+i+j-k)$, $u\cdot j=\frac{1}{2}(-1-i+j+k)$,
\begin{gather*}
u\cdot k = \frac{1}{2}(-1+i-j+k);
\end{gather*}
de donde, sumando, $2u^{2}=\frac{1}{2}(-2+2i+2j+2k)$ y $u^{2}\in\HH$.

Para {\itshape b}), si
\begin{gather*}
z = \frac{1}{2}+a+\left(\frac{1}{2}+b\right)i+\left(\frac{1}{2}+c\right)j+\left(\frac{1}{2}+d\right)k
\quad(a,b,c,d\in\ZZ),
\end{gather*}
se tiene $z+\oline z = 1+2a\in\ZZ$ y
\begin{gather*}
z\oline z = \left(\frac{1}{2}+a\right)^{2}+\left(\frac{1}{2}+b\right)^{2}+\left(\frac{1}{2}+c\right)^{2}
+\left(\frac{1}{2}+d\right)^{2}\in\frac{4}{4}+\ZZ\subset\ZZ
\end{gather*}
(lema~2, {\itshape b}).

Si $z$ es inversible en $\HH$, y si $z'$ es su inverso, tenemos
\begin{gather*}
N(z)N(z') = N(zz') = 1;
\end{gather*}
como $N(z)$ y $N(z')$ son enteros $>0$ ((b) y
lema~\ref{lem5.7.2}, {\itshape a})), necesariamente se sigue que
$N(z) = 1$. Reciprocamente si $z\in\HH$ y si $N(z) = 1$, se tiene
$z\oline z = \oline z\cdot z = N(z) = 1$
y $z$ es inversible pues $\oline z\in\HH$ por {\itshape a}). Esto demuestra {\itshape c}).

Demostremos por \'ultimo {\itshape d}). Dado un cuaterni\'on
\begin{gather*}
x=a+bi+cj+dk\in\HH(\QQ),
\end{gather*}
existe cuatro enteros $a',b',c',d'\in\ZZ$ tales que
\begin{gather*}
\abs{a-a'}\leq\frac{1}{2},\quad\abs{b-b'}\leq\frac{1}{2},\quad\abs{c-c'}\leq\frac{1}{2},\quad
\abs{d-d'}\leq\frac{1}{2}.
\end{gather*}
Pongamos $z = a'+b'i+c'j+d'k$. Entonces tenemos
\begin{gather*}
N(x-z) = (a-a')^{2}+(b-b')^{2}+(c-c')^{2}+(d-d')^{2}\leq 4\frac{1}{4}=1.
\end{gather*}
De hecho siempre tenemos la desigualdad estricta, a menos que $a,b,c,d$ son todos en $\frac{1}{2}+\ZZ$.
Pero en este caso $x\in\HH$. Por lo tanto, para todo cuaterni\'on $z\in\HH(\QQ)$, existe otro cuaterni\'on de
Hurwitz $z\in\HH$ tal que $N(x-z) < 1$ (es precisamente para tener la desigualdad estricta que introdujimos
los cuaterniones de Hurwitz; aquellos de $\HH(\ZZ)$ solamente no alcanzan). Sea ahora $\idl{a}$ un
ideal a izquierda de $\HH$. Para mostrar que es principal, podemos suponer que $\idl{a}\neq(0)$. Elijamos,
en $\idl{a}$, un elemento no nulo $u$ de norma reducida minimal (uno tal existe, pues las normas son
enteros $>0$ por {\itshape b})). Entonces $u$ es inversible en $\HH(\QQ)$ pues su inversa es
$\oline uN(u)^{-1}$ (esto muestra de paso que $\HH(\QQ)$ es un cuerpo no conmutativo). Para $y\in\idl{a}$,
formemos $yu^{-1}\in\HH(\QQ)$ y tomemos un elemento $z\in\HH$ tal que $N((yu^{-1}-z)u)<N(u)$.
Como $y-zu\in\idl{a}$ y $N(u)$ es minimal, deducimos que $y-zu=0$, $y\in\HH u$ y $\idl{a} = \HH u$. \QED

Como el conjunto de las sumas de cuatro cuadrados en $\ZZ$ es multiplicativamente estable
(cf. lema~\ref{lem5.7.2}), el teorema~\ref{teo5.7.1} se reduce a la:

\begin{proposition}
Todo n\'umero primo $p$ es suma de cuatro cuadrados.
\end{proposition}

Como $2 = 1^{2}+1^{2}+0^{2}+0^{2}$, podemos suponer que $p$ es imar. Como $p$ conmuta con todos los
cuaterniones, el ideal a izquierda $\HH p$ es bil\'atero. Luego podemos formar el anillo cociente
$\HH/\HH p$. Como $p$ es impar, todo $z\in\HH$ es congruente mod $\HH p$ a un elemento de $\HH(\ZZ)$
(si las componentes de $z$ son todas en $\frac{1}{2}+\ZZ$, formamos $z+p\cdot\frac{1}{2}(1+i+j+k)$);
por lo tanto $\HH/\HH p$ es isomorfo al cociente correspondiente de $\HH(\ZZ)$, es decir a $\HH(\FF_{p})$.

Ahora bien, como la forma $a^{2}+b^{2}+c^{2}+d^{2}$ representa $0$ en $\FF_{p}$ (cap\'itulo~\ref{cap1},
\S\ref{sec1.7}, teorema~\ref{teo1.7.2};
ver m\'as abajo para una demostraci\'on directa), $\HH(\FF_{p})$ admite elementos
no nulos cuyo norma reducida es nula. Un tal elemento no es inversible (lema~2, {\itshape b})), por lo
tanto general un ideal a izquierda no trivial. Volviendo a $\HH$, vemos que $\HH p$ est\'a
contenido en un ideal a izquierda $\HH z$ distinto de $\HH$ y de $\HH p$. En consecuencia, tenemos
$p = z'z$ con $z,z'\in\HH$ no inversibles. Entonces $p^{2}=N(p)=N(z)N(z')$ y, como
$N(z)$ y $N(z')$ son enteros $>1$ (lema~3, {\itshape b}) y {\itshape c})), se tiene $N(z) = N(z') = p$.

Pongamos $z = a+bi+cj+dk$ ($a,b,c,d\in\ZZ$ o $\in\frac{1}{2}+\ZZ$). Si $a,b,c,d\in\ZZ$, se tiene
$p = N(z) = a^{2}+b^{2}+c^{2}+d^{2}$ y ganamos. Falta mostrar que, si $a,b,c,d\in\frac{1}{2}+\ZZ$, podemos
reducirnos ?? al caos precedente multiplicando $z$ por un elemento de norma reducida $1$ de $\HH$, m\'as
precisamente por un elemento de la forma $\frac{1}{2}(\pm 1\pm i\pm j\pm k)$. En efecto consideremos
la clase $\eta$ de $2z$ en $\HH(\ZZ)/4\HH(\ZZ)\simeq\HH(\ZZ/4\ZZ)$. Como $N(z)\in\ZZ$, se tiene
$N(2z)\in 4\ZZ$, de donde $N(\eta) = 0$ y $\eta\oline\eta = 0$. Ahora bien, $\oline\eta$ es la clase
de un cuaterni\'on $z'$ de la forma $\pm 1\pm i\pm j\pm k$; entonces $u = \frac{1}{2}z'\in\HH$,
$u$ es de norma reducida $1$ y, como la clase de $(2z)\cdot(2u)$ es nula mod $4$, se tiene $zu\in\HH(\ZZ)$.
Como $p=N(z)=N(zu)$, la afirmaci\'on queda demostrada. \QED

\begin{remark*}
Aqu\'i, una demostraci\'on elemental del hecho de que, sobre un cuerpo finito $K$, la forma cuadr\'adtica
$a^{2}+b^{2}+c^{2}+d^{2}$ \emph{representa} $0$ (i.e. tiene un zero no trivial en $K^{4}$). Tomando
$c=1$, $d=0$, basta demostrar que la ecuaci\'on $a^{2}+b^{2}=0$ tiene una soluci\'on en $K^{2}$. Escribimos
la ecuaci\'on en la forma $b^{2}+1=-a^{2}$. En caracter\'istica $2$, podemos tomar $b=0$ y $a=1$.
Sino, si $q$ es el cardinal de $K$, hay $\frac{q+1}{2}$ cuadrados en $K$
($0$ y los $\frac{q-1}{2}$ cuadrados no nulos). Por lo tanto, el conjunto $T$ (resp. $T'$) de los elementos
de $K$ de la forma $b^{2}+1$ con $b\in K$ (resp. de la forma $-a^{2}$ con $a\in K$) tiene
$\frac{q+1}{2}$ elementos por traslaci\'on (resp. simetr\'ia). Como $\frac{q+1}{2}+\frac{q+1}{2}>q$,
se tiene $T\cap T'\neq\emptyset$, lo que significa que $b^{2}+1=-a^{2}$ tiene un soluci\'on \QED.
\end{remark*}

\chapter{Extensiones galoisianas de cuerpos de n\'umeros}

\section{Teor\'ia de Galois}\label{sec6.1}

Esta {\S} es un complemento a la teor\'ia general de cuerpos conmutativos
cf.~cap\'itulo~\ref{cap2}, \S\S\ref{sec2.3},~\ref{sec2.4},~\ref{sec2.6} y~\ref{sec2.7}).
Dados un cuerpo $L$ y un conjunto $G$ de automorfismos de $L$, el conjunto de los $x\in L$ tales que
$\sigma(x) = x$ para todo $\sigma\in G$ es, como se ve f\'acilmente, un \emph{subcuerpo} de $L$, que llamamos
el \emph{cuerpo de invariantes} de $G$. Por otra parte, dada una extensi\'on $L$ de un cuerpo $K$, el conjunto
de $K$-automorfismos de $L$ es un \emph{grupo} con la composici\'on de funciones.

\begin{theorem}\label{teo6.1.1}
Sea $L$ una extensi\'on de grado finito $n$ de un cuerpo $K$ finito o de caracter\'istico cero. Las siguientes
condiciones son equivalentes:
\begin{trivlist}\setlength{\itemindent}{\parindent}
\item c) $K$ es el cuerpo de invariantes del grupo $G$ de $K$-automorfismos de $L$;
\item b) para todo $x\in L$, el polinomio minimal de $x$ sobre $K$ tiene todas sus ra\'ices en $L$;
\item c) $L$ est\'a generado por las ra\'ices de un polinomio sobre $K$.
\end{trivlist}
Bajo estas condiciones, el grupo $G$ de $K$-automorfismos de $L$ tiene $n$ elementos.
\end{theorem}

Mostremos que {\itshape a}) implica {\itshape b}). En efecto, si $x\in L$, el polinomio
$\prod_{\sigma\in G}(X-\sigma(x))$ es invariante por $G$ (pues todo $\tau\in G$ permuta los factores
entre s\'i)\footnote{La finitud de $G$ resulta del cap\'itulo~\ref{cap2}, \S\ref{sec2.4}, teorema~\ref{teo2.4.1}.}. Por lo tanto
sus coeficientes pertenecen a $K$. Como el polinomio admite a $x$ como ra\'iz ($1\in G$), necesariamente
es un m\'ultiplo del polinomio minimal de $x$ sobre $K$ (cap\'itulo~\ref{cap2}, \S\ref{sec2.3},~\eqref{eq-3.4-4}), de
lo que se deduce {\itshape b}).

Para ver que \textit{b}) implica {\itshape c}), tomamos un elemento primitivo $x$ de $L$ sobre $K$
(cap\'itulo~\ref{cap2}, \S\ref{sec2.4}, cor. del teorema~\ref{teo2.4.1}). Su polinomio minimal sobre $K$ tiene todas sus ra\'ices en
$L$ por {\itshape b}) y evidentemente \'estas general $L$ sobre $K$.

Probemos por \'ultimo que {\itshape c}) implica {\itshape a}). Por hip\'otesis $L$ est\'a generado sobre $K$
por un n\'umero finito de elementos $(x^{(1)},\dots,x^{(q)})$ y por sus conjugados $(x_{j}^{(i)})$
(cap\'itulo~\ref{cap2}, \S\ref{sec2.4}). Entonces todo $K$-isomorfismo $\sigma$ de $L$ en una extensi\'on de $L$
env\'ia a cada uno de estos generadores a otro de estos generadores. Por lo tanto se tiene que $\sigma(L)\subset L$,
de donde $\sigma(L) =L$ por \'algebra lineal, ya que $\sigma$ es una aplicaci\'on $K$-lineal inyectiva. Es decir,
$\sigma$ es un \emph{$K$-automorfismo} de $L$.

En este caso el grupo $G$ de $K$-automorfismos de $L$ tiene $n$ elementos (cap\'itulo~\ref{cap2}, \S\ref{sec2.4}, teorema~\ref{teo2.4.1} y
el corolario). Sea entonces $x\in L$ invariante por $G$, de manera que todo $\sigma\in G$ es un $K[x]$-automorfismo
de $L$. Como (cap\'itulo~\ref{cap2}, \S\ref{sec2.4}) hay exactamente $[L:K[x]]$ $K[x]$-isomorfismos de $L$ en una
extensi\'on de $L$, se tiene por lo tanto $n\leq[L:K[x]]$, de donde $n=[L:K[x]]$, $K[x] = K$ y
$x\in K$. Esto demuestra {\itshape a}). La igualdad $\card(G) = n$ fue demostrada por el camino. \QED

\begin{definition}
Si las condiciones del teorema~\ref{teo4.1.1} se satisfacen, decimos que $L$ es una extensi\'on galoisiana de $K$ y que
$G$ es el grupo de Galois de $L$ sobre $K$. Si $G$ es abeliano (resp. c\'iclico) decimos que $L$ es una
extensi\'on abeliana (resp. c\'iclica) de $K$.
\end{definition}

\begin{named}{Corolario del teorema 1}
Sea $K$ un cuerpo finito o de caracter\'istica cero, $L$ una extensi\'on de $K$ de grado finito $n$ y
$H$ un grupo de automorfismos de $L$ que tiene a $K$ por su grupo de invariantes. Entonces $L$ es una
extensi\'on galoisiana de $K$ y $H$ es su grupo de Galois.
\end{named}

En efecto, si $x\in L$, el polinomio $\prod_{\sigma\in H}(X-\sigma(x))$ es invariante
por $H$ y por
lo tanto sus coeficientes est\'an en $K$. As\'i, por el
teorema~\ref{teo6.1.1}, {\itshape b}), $L$ es una extensi\'on
galoisiana de $K$. Si $G$ denota su grupo de Galois, se tiene $H\subset G$ y $\card(G) = n$
(teorema~\ref{teo6.1.1}). Tomemos entonces un elemento primitivo $x$ de $L$ sobre $K$
(cap\'itulo~\ref{cap2},
\S\ref{sec2.4}, corolario del teorema~\ref{teo2.4.1}) y consideremos el polinomio
\begin{gather*}
P(X) = \prod_{\sigma\in H}(X-\sigma(x)).
\end{gather*}
Como m\'as arriba, este polinomio tiene todos sus coeficientes en $K$ y es m\'ultiplo
del polinomio minimal
de $x$ sobre $K$, de donde $n\leq d^{\circ}(P)$. Como
\begin{gather*}
d^{\circ}(P) = \card(H)\leq\card(G) = n,
\end{gather*}
deducimos que $H = G$. \QED

\begin{theorem}\label{teo6.1.2}
Sea $K$ un cuerpo finito o de caracter\'istica cero, $L$ una extensi\'on galoisiana de $K$
y $G$ su grupo
de Galois. A todo subgrupo $G'$ de $G$ le asociamos  el cuerpo de invariantes $k(G')$
de $G'$ y a todo
subcuerpo $K'$ de $L$ que contiene a $K$ le asociamos el subgrupo $g(K')\subset G$ de
los $K'$-automorfismos
de $L$.
\begin{enumerate}%\setlength{\itemindent}{\parindent}
\item[a)] Las funciones $g$ y $k$ son biyecciones inversas una de la otra, decrecientes
para la relaci\'on de
inclusi\'on. M\'as a\'un, $L$ es una extensi\'on galoisiana de todo cuerpo intermedio $K'$
(i.e. $K\subset K'\subset L$).
\item[b)] Para que un cuerpo intermedio $K'$ sea una extensi\'on galoisiana de $K$ es
necesario y suficiente que
$g(K')$ sea un subgrupo normal de $G$. Entonces el grupo de Galois de $K'$ sobre $K$
se identifica con el
grupo cociente $G/g(K')$.
\end{enumerate}%{trivlist}
\end{theorem}

En efecto, para todo cuerpo intermedio $K'$ y todo $x\in L$, el polinomio minimal de $x$
sobre $K'$ divide
al polinomio minimal de $x$ sobre $K$. Por lo tanto tiene todas sus ra\'ices en $L$ por
el teorema~\ref{teo6.1.1}, {\itshape b}),
de manera que $L$ es una extensi\'on \emph{galoisiana} de $K'$ por el teorema~\ref{teo6.1.1},
{\itshape b}) nuevamente.
Entonces, $K'$ es el cuerpo de invariantes del grupo $g(K')$ de los $K'$-automorfismos de $L$
(teorema~\ref{teo6.1.1}, {\itshape a})). Es decir, $k(g(K')) = K'$. Sea ahora $G'$
un subgrupo de $G$. Entonces
$G'$ es el grupo de Galois de $L$ sobre $k(G')$ (cor. del teorema~\ref{teo6.1.1}).
Es decir, $G' = g(k(G'))$.
Las f\'ormulas $k(g(K')) = K'$ y $g(k(G')) = G'$ muestran que $k$ y $g$ son biyecciones
inversas una de la otra.
Que son decrecientes es evidente. Esto demuestra {\itshape a}).

Probemos ahora {\itshape b}). Sea $K'$ un cuerpo intermedio ($K\subset K'\subset L$). Si $x\in K'$, las
ra\'ices del polinomio minimal de $x$ sobre $K$ son las $\sigma(x)$ ($x\in G$). Se sigue
del teorema~\ref{teo6.1.1}, {\itshape b}),
que para que $K'$ sea una extensi\'on galoisiana de $K$ es necesario y suficiente
que $\sigma(x)\in K'$
para todo $x\in K'$ y $\sigma\in G$, es decir que $\sigma(K')\subset K'$ para
todo $\sigma\in G$. Ahora bien,
si $\sigma(K')\subset K'$, $\tau\in g(K')$ y si $x\in K'$, se tiene
$\sigma^{-1}\tau\sigma(x) = x$, de donde
$\sigma^{-1}\tau\sigma\in g(K')$. En otras palabras, ``$K'$ galoisiana sobre $K$''
implica ``$g(K')$ normal
en $G$''. Reciprocamente, supongamos que $g(K')$ es normal en $G$. Si $x\in K'$,
$\sigma\in G$ y si
$\tau\in g(K')$, se tiene
$\tau\sigma(x) = \sigma\cdot\sigma^{-1}\tau\sigma(x) = \sigma(x)$ pues
$\sigma^{-1}\tau\sigma\in g(K')$, de manera que $\sigma(x) \in K'$.
En consecuencia, ``$g(K')$ es normal
en $G$'' implica ``$\sigma(K')\subset K'$'' y por lo tanto que $K'$ es galoisiana sobre $K$.

Por \'ultimo, calculemos el grupo de Galois de $K'$ sobre $K$ en este caso. Como se
tiene $\sigma(K')\subset K'$
para todo $\sigma\in G$ (e incluso $\sigma(K') = K'$ por \'algebra lineal), la restricci\'on
$\sigma\mid K'$ de $\sigma$ a $K'$ es un $K$-automorfismo de $K'$. Por lo tanto, se tiene un
``homomorfismo de restricci\'on'' $\sigma\mapsto\sigma\mid K'$ de $G$ en el grupo de
Galois $H$ de $K'$ sobre $K$.
Su n\'ucleo es evidentemente $g(K')$. Como se tiene
\begin{align*}
\card(H) &= [K':K]=[L:K][L:K']^{-1}=\card(G)\cdot\card(g(K'))^{-1}\\
&= \card(G/g(K')),
\end{align*}
este homomorfismo es sobreyectivo y por eso $H\simeq G/g(K')$. \QED

%\subsubsection*{Ejemplo 1. Extensiones cuadr\'aticas}
\begin{example}[Extensiones cuadr\'aticas]

Sea $K$ un cuerpo de caracter\'istica cero y $L$ una extensi\'on cuadr\'atica (i.e. de
grado $2$) de $K$.
Como en el principio de \S\ref{sec2.5}, cap\'itulo~\ref{cap2} vemos que $L$ es de la
forma $K[x]$, donde $x$ es
ra\'iz de un polinomio $X^{2}-d$ ($d\in K$, $d$ no es un cuadrado en $K$). Como la otra
ra\'iz de este polinomio
es $-x$, $L$ admite un $K$-automorfismo no trivial $\sigma$ definido por $\sigma(x) = -x$, i.e.
\begin{gather*}
\sigma(a+bx) = a-bx\quad(a,b\in K).
\end{gather*}
Tenemos $\sigma^{2}=1$ y $K$ es el cuerpo de invariantes de $\sigma$.
Por lo tanto $L$ es una extensi\'on
\emph{galoisiana} de $K$ con el grupo \emph{c\'iclico} $\{1,\sigma\}$ como grupo de
Galois (teorema~\ref{teo6.1.1} y el corolario del teorema~\ref{teo6.1.1}).
\end{example}

%\subsubsection*{Ejemplo 2. Extensiones ciclot\'omicas}\label{ej6.1.2}
\begin{example}[Extensiones ciclot\'omicas]\label{ej6.1.2}
%\minisec{Ejemplo. Extensiones ciclot\'omicas}\label{ej6.1.2}

Sea $K$ un cuerpo de caracter\'istica cero, $z$ una ra\'iz primitiva $n$-\'esima de
la unidad en una
extensi\'on de $K$ y $L=K(z)$. Decimos que $L$ es una extensi\'on \emph{ciclot\'omica}
de $K$. El polinomio
minimal $F(X)$ de $z$ sobre $K$ divide a $X^{n}-1$
(cap\'itulo~\ref{cap2}, \S\ref{sec2.3},~\eqref{eq-2.3-4}) y por lo tanto
sus ra\'ices son ra\'ices $n$-\'esimas de la unidad y por lo tanto potencias
de $z$ (cap\'itulo~\ref{cap1}, \S\ref{sec1.6}).
As\'i, $L$ es una extensi\'on \emph{galoisiana} de $K$ por el
teorema~\ref{teo6.1.1}, {\itshape c}).

Sea $G$ el grupo de Galois. Todo $\sigma\in G$ est\'a determinado por $\sigma(z)$,
que es una potencia
$z^{j(\sigma)}$ de $z$, donde $j(\sigma)$ est\'a bien definido s\'olo mod $n$.
Si $\sigma,\tau\in G$, se tiene
$\sigma\tau(z) = \sigma(z^{j(\tau)}) = \sigma(z)^{j(\tau)} = z^{j(\sigma)j(\tau)}$, de donde
\begin{gather*}
j(\sigma\tau) \equiv j(\sigma)j(\tau)
\end{gather*}
(mod $n$). En otras palabras, podemos considerar $\sigma\mapsto j(\sigma)$ como
un \emph{homomorfismo}
$G\to(\ZZ/n\ZZ)^{*}$. Como $j(\sigma)$ determina a $\sigma$ un\'ivocamente, este
homomorfismo es \emph{inyectivo}
y $G$ es abaliano. As\'i \emph{toda extensi\'on ciclot\'omica es abeliana.} Si $n$ es
primo, esta extensi\'on
es incluso \emph{c\'iclica,} pues $G$ es isomorfo a un subgrupo de
$(\ZZ/n\ZZ)^{*} = \FF_{n}^{*}$ (cap\'itulo~\ref{cap1},
\S\ref{sec1.7}, teorema~\ref{teo1.7.1}, {\itshape b})).
\end{example}

\begin{comm}
Como todo subgrupo de un grupo abeliano es normal, todo cuerpo intermedio $K'$ de
una extensi\'on ciclot\'omica
$L$ de $K$ es una extensi\'on galoisiana (e incluso abeliana) de $K$
(teorema~\ref{teo6.1.2}, {\itshape b})). En particular, todo
subcuerpo de un cuerpo ciclot\'omica es una extensi\'on abeliana de $\QQ$. Reciprocamente,
se puede
demostrar (teorema de Kronecker-Weber) que toda extensi\'on abeliana de $\QQ$ es un
subcuerpo de un cuerpo ciclot\'omico.
\end{comm}

Observemos que, con la misma notaci\'on que antes, el automorfismo $\sigma$ \emph{eleva
todas las ra\'ices
$n$-\'esimas de la unidad a la potencia $j(\sigma)$,} pues todas ellas son potencias
de $z$. Luego, $\sigma\mapsto j(\sigma)$ es independiente de la elecci\'on de $z$.

%\subsubsection*{Ejemplo 3. Cuerpos finitos}
\begin{example}[Cuerpos finitos]\label{ej6.1.3}

%\minisec{Ejemplo 3. Cuerpos finitos}

Sea $\FF_{q}$ un cuerpo \emph{finito} ($q = p^{s}$ con $p$ primo). Toda extensi\'on de grado finito de $\FF_{q}$
es de la forma $\FF_{q^{n}}$; su grado es $n$ (cap\'itulo~\ref{cap1}, \S\ref{sec1.7}). Sabemos que
$\sigma:x\mapsto x^{q}$ es un automorfismo de $\FF_{q^{n}}$ ({\itshape ibid.}, proposici\'on~\ref{prop1.7.1}) y que
$\FF_{q}$ es su cuerpo de invariantes ({\itshape ibid.}, teorema~\ref{teo1.7.1}, {\itshape c})). Para todo $x\in\FF_{q^{n}}$,
se tiene $\sigma^{j}(x) = x^{q^{j}}$, de donde $\sigma^{n}=1$ pues $\FF^{q^{n}}$ es el conjunto de los
$x$ tales que $x^{q^{n}}=x$ ({\itshape ibid.},
teorema~\ref{teo1.7.1}, {\itshape c})). Por otra parte, para
$1\leq j\leq n-1$, se tiene $\sigma^{j}\neq 1$ pues $\FF_{q^{j}}\neq\FF_{q^{n}}$. Por lo tanto,
$\{1,\sigma,\dots,\sigma^{n-1}\}$ es un grupo c\'iclico de orden $n$. As\'i, por el corolario
del teorema~\ref{teo6.1.1}, $\FF_{q^{n}}$
\emph{es una extensi\'on c\'iclica de grado $n$ de $\FF_{q}$ y su grupo de Galois tiene un generador
distinguido, a saber $x\mapsto x^{q}$, que llamamos automorfismo de Frobenius.}
\end{example}

\section{Grupo de descomposici\'on y grupo de inercia}\label{sec6.2}

\begin{trivlist}
\item \textit{En esta \S, $A$ denota un anillo de Dedekind, $K$ su cuerpo de fracciones que suponemos
de caracter\'istica cero, $K'$ una extensi\'on galoisiana de $K$, $n$ su grado, $G$ su grupo de Galois y
$A'$ la clausura \'integra de $A$ en $K'$.}
\end{trivlist}
%\plainbreak{1}

Aplicando $\sigma\in G$ a una ecuaci\'on de dependencia entera (sobre $A$) de un elemento $x\in A'$, vemos
que $\sigma(x)\in A'$. Por lo tanto:
\begin{gather}\label{eq6.2.1}
\text{\itshape $A'$ es estable por $G$, i.e. $\sigma(A')=A'$ para todo $\sigma\in G$.}
\end{gather}

\begin{comm}
De hecho nosotros s\'olamente hemos demostrado que $\sigma(A')\subset A'$, pero entonces tambi\'en
tenemos $\sigma^{-1}(A')\subset A'$, de donde $A'=\sigma\sigma^{-1}(A')\subset\sigma(A')$. Nosotros
omitiremos en general en lo que sigue este sencillo razonamiento adicional.
\end{comm}

Por otra parte, si $\idl{p}$ es un ideal maximal de $A$ y $\idl{p}'$ es un ideal maximal de $A'$ tal que
$\idl{p}'\cap A=\idl{p}$ (es decir que figura en la descomposici\'on de $A'\idl{p}$ en ideal primos;
cf~cap\'itulo~\ref{cap5}, \S\ref{sec5.2}, proposici\'on~\ref{prop5.2.1}),
se tiene evidentemente $\sigma(\idl{p}')\cap A=\idl{p}$ y
$\sigma(\idl{p}')$ figura en la descomposici\'on de $A'\idl{p}$ con el \emph{mismo} exponente que $\idl{p}'$.
Diremos que $\idl{p}'$ y $\sigma(\idl{p}')$ son ideales primos \emph{conjugados} de $A'$. Ahora demostraremos
que no hay otros primos en la descomposici\'on de $A'\idl{p}$:

\begin{proposition}\label{prop6.2.1}
Sea $\idl{p}$ es un ideal maximal de $A$. Los ideales maximales $\idl{p}'_{i}$ de $A'$ que figuran en
la descomposici\'on de $A'\idl{p}$ (i.e. tales que $\idl{p}'_{i}\cap A=\idl{p}$) son conjugados dos a dos
y tiene el mismo grado residual $f$ y el mismo \'indice de ramificaci\'on $e$, de manera que
$A'\idl{p} = \left(\prod_{i=1}^{g}\idl{p}'_{i}\right)^{e}$ y $n=efg$.
\end{proposition}

La afirmaci\'on sobre el \'indice de ramificaci\'on y el grado residual son evidentes, pues el automorfismo
$\sigma$ preserva \emph{todas} las relaciones algebraicas. La f\'ormula $n=efg$ es entonces un caso particular
de $\sum e_{i}f_{i}=n$ (cap\'itulo~\ref{cap5}, \S\ref{sec5.2}, teorema~\ref{teo5.2.1}). Sea ahora $\idl{p}'$ uno de los $\idl{p}'_{i}$
y supongamos que otro de los $\idl{p}'_{i}$, que notaremos $\idl{q}'$, no es conjugado con $\idl{p}'$.
Como $\idl{q}'$ y $\sigma(\idl{p}')$ ($\sigma\in G$) son maximales y distintos, se tiene $\sigma(\idl{p}')\not\subset\idl{q}'$.
Tenemos el siguiente lema

\begin{lemma}[lema de ????? prima]
Sea $R$ un anillo, $\idl{p}_{1},\dots,\idl{p}_{q}$ una familia finita de ideales primos de $R$ y $\idl{b}$
un ideal de $R$ tal que $\idl{b}\not\subset\idl{p}_{i}$ para todo $i$. Entonces existe $b\in\idl{b}$ tal que
$b\notin\idl{p}_{i}$ para todo $i$.
\end{lemma}

En efecto, suprimiendo los $\idl{p}_{i}$ no maximales de
$\{\idl{p}_{1},\dots,\idl{p}_{q}\}$, podemos suponer
que se tiene que $\idl{p}_{j}\not\subset\idl{p}_{i}$ si $i\neq j$. Sea entonces
$x_{ij}\in\idl{p}_{j}$ tal que
$x_{ij}\notin\idl{p}_{i}$. Por otra parte, como $\idl{b}\not\subset\idl{p}_{i}$, existe
$a_{i}\in\idl{b}$
tal que $a_{i}\notin\idl{p}_{i}$. Sea entonces $b_{i} = a_{i}\prod_{j\neq i}x_{ij}$. Se tiene
$b_{i}\in\idl{b}$, $b_{i}\in\idl{p}_{j}$ si $j\neq i$ y $b_{i}\notin\idl{p}_{i}$ pues
$\idl{p}_{i}$ es primo.
Entonces $b=b_{1}+\dots+b_{q}$ satisface la conclusi\'on deseada, pues $b\in\idl{b}$ y,
para todo $i$, se tiene
$\sum_{j\neq i}b_{j}\in\idl{p}_{i}$, $b_{i}\notin\idl{p}_{i}$. \QED

Luego, el lema muestra que existe un elemento $x\in\idl{q}'$ tal que $x\notin\sigma(\idl{p}')$ para todo
$\sigma\in G$. Consderemos entonces $N(x) = \prod_{\tau\in G}\tau(x)$ (cap\'itulo~\ref{cap2}, \S\ref{sec2.6},
proposici\'on~\ref{prop2.6.1}).
Como $\tau(x)\in A'$ para todo $\tau\in G$ (por~\eqref{eq6.2.1}),
tenemos $N(x)\in\idl{q}'$, de donde
\begin{gather*}
N(x) \in \idl{q}'\cap A = \idl{p}.
\end{gather*}
Por otra parte, se tiene $x\notin\tau^{-1}(\idl{p}')$, de donde $\tau(x)\notin\idl{p}'$
para todo
$\tau\in G$. Como $\idl{p}'$ es primo, deducimos que $N(x)\notin\idl{p}'$, lo que
contradice $N(x)\in\idl{p}$. \QED

Sea ahora $\idl{p}'$ uno de los ideales maximales de $A'$ tales que $\idl{p}'\cap A = \idl{p}$. Los $\sigma\in G$
tal que $\sigma(\idl{p}') = \idl{p}'$ forman un subgrupo $D$ de $G$, que llamamos \emph{grupo de descomposici\'on de
$\idl{p}'$.} Si $g$ es el n\'umero de conjugados de $\idl{p}'$ tenemos por lo tanto
\begin{gather}\label{eq6.2.2}
g=\card(G)\cdot\card(D)^{-1}\quad\text{donde}\quad\card(D) = n/g = ef.
\end{gather}
Si $\sigma\in D$, la relaci\'on $\sigma(A') = A'$ y $\sigma(\idl{p}') = \idl{p}'$ muestra que $\sigma$ define,
pasando al cociente, un automorfismo $\oline\sigma$ de $A'/\idl{p}'$ (en efecto,
$x\equiv y\pmod{\idl{p}'}$ implica $\sigma(x)\equiv\sigma(y)\pmod{\idl{p}'}$). Es claro que $\oline\sigma$
es un $(A/\idl{p})$-automorfismo. LA aplicaci\'on $\sigma\mapsto\oline\sigma$ es un \emph{homomorfismo} de grupos,
cuyo \emph{n\'ucleo} es el conjunto $I$ de los $\sigma\in D$ tales que $\sigma(x)-x\in\idl{p}'$ para todo $x\in A'$.
Por lo tanto, $I$ es un subgrupo \emph{normal} de $D$, que llamamos \emph{subgrupo de inercia} de $\idl{p}'$.

\begin{proposition}\label{prop6.2.2}
Con la misma notaci\'on que antes, supongamos que $A/\idl{p}$ es finito o de caracter\'istica cero. Entonces
$A'/\idl{p}'$ es una extensi\'on galoisiana de grado $f$ de $A/\idl{p}$ y $\sigma\mapsto\oline\sigma$ es un
homomorfismo sobreyectivo de $D$ sobre su grupo de Galois. Adem\'as, $\card(I) = e$.
\end{proposition}

En efecto, sea $K_{D}$ el cuerpo de invariantes de $D$, $A_{D} = A'\cap K_{D}$ la clausura \'integral de
$A$ en $K_{D}$ y $\idl{p}_{D}$ el ideal primo $\idl{p}'\cap A_{D}$. Por la
proposici\'on~\ref{prop6.2.1} y la definici\'on
de $D$, $\idl{p}'$ es el \'unico factor primo de $A'\idl{p}_{D}$. Escribamos $A'\idl{p}_{D} = \idl{p}'^{e}$
y notemos $f'$ el grado residual $[A'/\idl{p}':A_{D}/\idl{p}_{D}]$. Por el teorema~\ref{teo5.2.1} de \S\ref{sec5.2},
cap\'itulo~\ref{cap5},
el teorema~\ref{teo6.1.1} de \S\ref{sec6.1} y~\eqref{eq6.2.2}, tenemos
\begin{gather*}
e'f'=[K':K_{D}]=\card(D)=ef.
\end{gather*}
Como $A/\idl{p}\subset A_{D}/\idl{p}_{D}\subset A'/\idl{p}'$, se tiene $f'\leq f$. Como $\idl{p}A_{D}\subset\idl{p}_{D}$,
se tiene $e'\leq e$. Junto a $e'f'=ef$, esto muestra que $e=e'$ y $f=f'$, de donde:
\begin{gather}\label{eq6.2.3}
A/\idl{p}\simeq A_{D}/\idl{p}_{D}.
\end{gather}
Sea ahora $\oline x$ un elemento primitivo de $A'/\idl{p}'$ sobre $A/\idl{p}$ y sea $x\in A'$ un
representante de $\oline x$. Sea $X^{r}+a_{r-1}X^{r-1}+\dots+a_{0}$ el polinomio minimal de $x$
\emph{sobre} $K_{D}$; se tiene $a_{i}\in A_{D}$ (cap\'itulo~\ref{cap2}, \S\ref{sec2.6},
corolario de la proposici\'on~\ref{prop2.6.2}).
El conjunto de sus ra\'ices consiste en los $\sigma(x)$ con $\sigma\in D$. El polinomio
``reducido'' $X^{r}+\oline a_{r-1}X^{r-1}+\dots+\oline a_{0}$ tiene sus coeficientes en $A/\idl{p}$
(por~\eqref{eq6.2.3}) y el conjunto de ra\'ices consiste en los $\oline\sigma(\oline x)$ con $\sigma\in D$. Resulta
primero que nada que $A'/\idl{p}'$ contiene a todos los conjugados de $\oline x$ sobre $A/\idl{p}$ y
$A'/\idl{p}'$ es por lo tanto una extensi\'on \emph{galoisiana} de $A/\idl{p}$
(\S\ref{sec6.1}, teorema~\ref{teo6.1.1}, {\itshape c})). Resulta por otra parte que, como todo conjugado $\oline x$ sobre $A/\idl{p}$
es un $\oline\sigma(\oline x)$ que todo $(A/\idl{p})$-automorfismo de $A'/\idl{p}'$ es un $\oline\sigma$.
As\'i el grupo de Galois de $A'/\idl{p}'$ sobre $A/\idl{p}$ se identifica a $D/I$. Como tiene orden
$[A'/\idl{p}':A/\idl{p}]=f$, tenemos que $\card(D)/\card(I) = f$, de donde $\card(I) = e$ por~\eqref{eq6.2.2}. \QED

\begin{corollary*}
Para que $\idl{p}$ no ramifique en $A'$ es necesario y suficiente que el grupo de inercia $I$ consista
\'unicamente de la identidad.
\end{corollary*}

\begin{remark*}
Si notamos $D_{\idl{p}'}$ y $I_{\idl{p}'}$ los grupos de descomposici\'on e inercia del ideal maximal
$\idl{p}'$, aquellos de su \emph{conjugado} $\sigma(\idl{p}')$ ($\sigma\in G$) son
\begin{gather}
D_{\sigma(\idl{p}')}=\sigma D_{\idl{p}'}\sigma^{-1},\quad I_{\sigma(\idl{p}')}=\sigma I_{\idl{p}'}\sigma^{-1}
\end{gather}
En efecto, si $\tau\in D_{\idl{p}'}$, se tiene $\sigma\tau\sigma^{-1}\cdot\sigma(\idl{p}')=\sigma\tau(\idl{p}')
=\sigma(\idl{p}')$, de donde $\sigma D_{\idl{p}'}\sigma^{-1}\subset D_{\sigma(\idl{p}')}$. Aplicando esto
a $\sigma^{-1}$ se obtiene la inclusi\'on inversa. De la misma manera, si $\tau\in I_{\idl{p}'}$ y $x\in A'$,
se tiene
\begin{gather*}
\sigma\tau\sigma^{-1}(x) - x = \sigma\tau(\sigma^{-1}(x))-\sigma\sigma^{-1}(x)=\sigma(\tau(\sigma^{-1}(x))
-\sigma^{-1}(\sigma(x))\in\sigma(\idl{p}'),
\end{gather*}
de donde $\sigma I_{\idl{p}'}\sigma^{-1}\subset I_{\sigma(\idl{p}')}$ y de hecho se tiene una igualdad
aplicando $\sigma^{-1}$ a $\sigma(\idl{p}')$.

Cuando $K'$ es una extensi\'on \emph{abeliana} de $K$, los grupos $D_{\sigma(\idl{p}')}$ (resp.
$I_{\sigma(\idl{p}')}$) ($\sigma\in G$) son por lo tanto todos \emph{iguales,} y no dependen del ideal
$\idl{p}$ del anillo de abajo. ????
\end{remark*}

\section{Caso de un cuerpo de n\'umeros. El automorfismo de Frobenius}\label{sec6.3}

Lo anterior se aplica a los cuerpos de n\'umeros y sus anillos de enteros. En efecto, estos cuerpos son de
caracter\'istica cero y los cuerpos residuales de estos anillos son finitos.

Conservamos la notaci\'on de arriba ($K\subset K'$ cuerpos de n\'umeros, $K'$ galoisiana sobre $K$, grupo $G$,
anillos $A$ y $A'$). Sea $\idl{p}$ un ideal maximal de $A$ que \emph{no ramifica} en $A'$ y sea $\idl{p}'$ un
factor primo de $A'\idl{p}$. Entonces el grupo de inercia de $\idl{p}'$ se reduce a la identidad
(\S\ref{sec6.2}, corolario de la proposici\'on~\ref{prop6.2.2})
y su grupo de descomposici\'on $D$ es por lo tanto can\'onicamente isomorfo al
grupo de Galois de $A'/\idl{p}'$ sobre $A/\idl{p}$ (\S\ref{sec6.2}, proposici\'on~\ref{prop6.2.2}).
Pero este \'ultimo es c\'iclico, con un
generador distinguido $\oline\sigma:\oline x\mapsto\oline x^{q}$, donde $q = \card(A/\idl{p})$
(\S\ref{sec6.1}, ejemplo~\ref{ej6.1.3}). Por lo tanto $D$ tambi\'en es \emph{c\'iclico,} con un generador distinguido $\sigma$ tal
que $\sigma(x) \equiv x^{q}\pmod{\idl{p}'}$ para todo $x\in A'$. Este generador se llama \emph{automorfismo de
Frobenius} de $\idl{p}$ y generalmente lo notamos $(\idl{p}',K'/K)$.

Si $\tau\in G$ tenemos, como en la observaci\'on al final de \S\ref{sec6.2}, que
\begin{gather}
(\tau(\idl{p}'),K'/L) = \tau\cdot(\idl{p}',K'/K)\cdot\tau^{-1}.
\end{gather}
En particular, si $K'$ es una extensi\'on \emph{abeliana,} $(\idl{p'},K'/K)$ depende \'unicamente del
ideal $\idl{p}$ de $A$. Entonces a veces lo notamos $\left(\frac{K'/L}{\idl{p}}\right)$.

\begin{proposition}\label{prop6.3.1}
Con las notaciones precedentes, sea $F$ un cuerpo intermedio {\upshape(}$K\subset F\subset K'${\upshape).}
Notemos $f$ el grado residual de $\idl{p}'\cap F$ sobre $K$. Entonces
\begin{enumerate}
\item se tiene $(\idl{p}',K'/F) = (\idl{p}',K'/K)^{f}$
\item si $F$ es galoisiana sobre $K$, la restricci\'on de $(\idl{p}',K'/K)$ a $F$ es igual a
$(\idl{p}'\cap F,F/K)$.
\end{enumerate}
\end{proposition}

En efecto, pongamos $\sigma = (\idl{p}',K'/K)$. Por definici\'on, tenemos $\sigma(\idl{p}')=\idl{p}'$ y
$\sigma(x) \equiv x^{q}\pmod{\idl{p}'}$ para todo $x\in A'$ (aqu\'i, $q = \card(A/\idl{p})$). Tenemos por lo tanto
\begin{gather*}
\sigma^{f}(\idl{p}') = \idl{p}'\quad\text{y}\quad\sigma^{f}(x)\equiv x^{q^{f}}
\end{gather*}
(mod $\idl{p}'$) para todo $x\in A'$. Por definici\'on de $f$, $q^{f}$ es el cardinal del cuerpo residual
$(A'\cap F)/(\idl{p}'\cap F)$. Adem\'as, el grupo de descomposic\'on de $\idl{p}'$ sobre $F$
es evidentemnete un subgrupo del grupo de descomposic\'on $D$ de $\idl{p}'$ sobre $K$ y tiene orden
\begin{gather*}
[A'/\idl{p}':(A'\cap F)/(\idl{p}'\cap F)] = f^{-1}[A'/\idl{p}':A/\idl{p}] = f^{-1}\cdot\card(D)
\end{gather*}
por~\eqref{eq6.2.2} de \S\ref{sec6.2}. Como $D$ es c\'iclico y generado por $\sigma$, su \'unico subgrupo de orden $f^{-1}\cdot\card(D)$
est\'a generado por $\sigma^{f}$. Esto demuestra {\itshape a}).

Supongamos ahora que $F$ es galoisiana sobre $K$ y notemos $\sigma'$ la restricci\'on de $\sigma$ a $F$
(\S\ref{sec6.1}, teorema~\ref{teo6.1.1}, {\itshape b})). Como $\sigma'(\idl{p}') = \idl{p}'$, se tiene $\sigma'(\idl{p}'\cap F)=\idl{p}'\cap F$
y $\sigma'$ pertenece al grupo de descomposici\'on de $\idl{p}'\cap F$ sobre $K$. Adem\'as evidentemente
se tiene $\sigma'(x)\equiv x^{q}$ para todo $x\in A'\cap F$, con $q = \card(A/\idl{p})$. Esto demuestra {\itshape b}).

\section{Aplicaci\'on a los cuerpos ciclot\'omicos}\label{sec6.4}

Ahora utilizaremos lo anterior para demostrar un resultado que generaliza la irreducibilidad del polinomio ciclot\'omico
y para dar una tercera demostraci\'on (cf.~cap\'itulo~\ref{cap2}, \S\ref{sec2.9}, teorema~\ref{teo2.9.1}
y cap\'itulo~\ref{cap5}, \S\ref{sec5.2}, ex.) de este hecho.

\begin{theorem}\label{teo6.4.1}
Sea $z$ una ra\'iz primitiva $n$-\'esima de la unidad en $\CC$. Entonces
\begin{enumerate}
\item Ning\'un n\'umero primo $p$ que no divide a $n$ ramifica en $\QQ[z]$;
\item $\QQ[z]$ es una extensi\'on abeliana de $\QQ$ de grado $\varphi(n)$ y de grupo de Galois isomorfo
a $(\ZZ/n\ZZ)^{*}$.
\end{enumerate}
\end{theorem}

En efecto, sea $F(X)$ el polinomio minimal de $z$ sobre $\QQ$ y $d$ su grado
(tenemos $d = [\QQ[z]:\QQ]$). El polinomio $F(X)$ es un divisor de $X^{n}-1$, digamos
$X^{n}-1=F(X)G(X)$. Tenemos $D(1,z,\dots,z^{d-1}) = N(F'(z))$ (cap\'itulo~\ref{cap2}, \S\ref{sec2.7},~\eqref{eq2.7.6}).
De $nX^{n-1}=F'(X)G(X)+F(X)G'(X)$, deducimos que $nz^{n-1}=F'(z)G(z)$. Como $z$ es una unidad de
$\QQ[z]$ y por lo tanto tiene norma $\pm 1$, se deduce, tomando norma, que $N(F'(z))$ divide a $n^{d}$.
Por \'ultimo, como $z$ es un entero de $\QQ[z]$, el discriminante absoluto de $\QQ[z]$ divide a
$D(1,z,\dots,z^{d-1})$ y por lo tanto a $n^{d}$. As\'i, por el cap\'itulo~\ref{cap5}, \S\ref{sec5.3},
teorema~\ref{teo5.3.1},
ning\'un primo $p$ que no divida a $n$ ramifica en $\QQ[z]$. Esto demuestra {\itshape a}).

Para {\itshape b}) recordemos (\S\ref{sec6.1}, ejemplo~\ref{ej6.1.2}) que $\QQ[z]$ es una extensi\'on abeliana de $\QQ$ y
que se tiene un homomorfismo inyectivo $j$ del grupo de Galois $G$ de $\QQ[z]$ sobre $\QQ$ en
$(\ZZ/n\ZZ)^{*}$. M\'as precisamente el elemento $\sigma\in G$ eleva todas las ra\'ices $n$-\'esimas de la
unidad a la potencia $j(\sigma)$. Sea entonces $p$ un n\'umero primo que no divide a $n$. Por {\itshape a}),
el automorfismo de Frobenius $\left(\frac{\QQ[z]/\QQ}{p}\right)$ est\'a definido; not\'emoslo $\sigma_{p}$.
Escribiendo $A$ por el anillo de enteros de $\QQ[z]$ y $\idl{p}$ por un factor primo cualquiera de $Ap$,
tenemos por definici\'on que $\sigma_{p}(x)\equiv x^{p}\bmod\idl{p}$ para todo $x\in A$. En particular, poniendo
$j = j(\sigma_{p})$, tenemos $z^{j}\equiv z^{p}\bmod\idl{p}$. Ahora bien, tambi\'en tenemos
\begin{gather*}
\prod_{\substack{0\leq r\leq n-1\\ r\not\equiv p\pmod n}}(z^{p}-z^{r})=P'(z^{p}) = nz^{p(n-1)},
\end{gather*}
donde $P(X) = X^{n}-1 = \prod_{0\leq r\leq n-1}(X-z^{r})$. Como $n$ es coprimo con $p$,
$\idl{p}\cap\ZZ = p\ZZ$ y $z$ es inversible, deducimos que
\begin{gather*}
\prod_{\substack{0\leq r\leq n-1\\ r\not\equiv p\pmod n}}(z^{p}-z^{r})\notin\idl{p}.
\end{gather*}
La relaci\'on $z^{j}\equiv z^{p}\pmod{\idl{p}}$ implica por lo tanto que $j$ es la clase de $p$
mod $n$. As\'i $j(G)$ contiene las clases mod $n$ de todos los n\'umeros rimos $p$ que no dividen a
$n$ y por lo tanto, por multiplicatividad, las clases de todos los enteros coprimos con $n$. En otras
palabras $j(G) = (\ZZ/n\ZZ)^{*}$ y esto demuestra {\itshape b}).

\section{Otra demostraci\'on de la ley de reciprocidad cuadr\'atica}

Sea $q$ un n\'umero primo \emph{impar} y $K$ el cuerpo ciclot\'omico generado por una ra\'iz primitiva
$q$-\'esima de la unidad en $\CC$. El grupo de Galois $G$ de $K$ sobre $\QQ$ es isomorfo a $\FF_{q}^{*}$
(\S\ref{sec6.4}, teorema~\ref{teo6.4.1}, {\itshape b})) y por lo tanto es \emph{c\'iclico} de orden par $q-1$. Por lo tanto admite
un \'unico subgrupo $H$ de \'indice $2$, que corresponde al subgrupo de los cuadrados $(\FF_{q}^{*})^{2}$.
As\'i $K$ contiene un \'unico subcuerpo \emph{cuadr\'atico} $F$ (\S\ref{sec6.1}, teorema~\ref{teo6.1.2}, {\itshape b})). Ning\'un n\'umero
primo $p\neq q$ se ramifica en $F$ pues, sino, se ramificar\'ia en $K$, lo que contradice el teorema~\ref{teo6.4.1}, {\itshape a})
de \S\ref{sec6.4}. El c\'alculo del discriminante de un cuerpo cuadr\'atico (cap\'itulo~\ref{cap5}, \S\ref{sec5.3}, ex.) muestra que
se tiene necesariamente $F=\QQ[\sqrt{q}]$ si $q\equiv 1\bmod 4$ y $F = \QQ[\sqrt{-q}]$ si $q\equiv 3\bmod 4$.
Poniendo $q^{*} = (-1)^{\frac{q-1}{2}}q$, tenemos en todo caso que $F = \QQ[\sqrt{q^{*}}]$.

Sea $p$ un n\'umero primo distinto a $q$. Notemos $\sigma_{p}$ el automorfismo de Frobenius
$\left(\frac{K/\QQ}{p}\right)$ (cf.~\S\ref{sec6.4}). Su restricci\'on a $F$ es $\left(\frac{F/\QQ}{p}\right)$
(\S\ref{sec6.3}, proposici\'on~\ref{prop6.3.1}, {\itshape b})) y es la identidad si $\sigma_{p}\in H$, es decir si el exponente $j(\sigma_{p})$
$=$ a la clase de $p\bmod q$ (cf.~\S\ref{sec6.4}) es un \emph{cuadrado} en $\FF_{q}^{*}$. En el caso contrario el el
automorfismo distinto a la identidad. En otras palabras, identificando el grupo de Galois $G/H$ de $F$ sobre
$\QQ$ con $\{+1,-1\}$, tenemos
\begin{gather}\label{eq-6.5-1}
\left(\frac{F/\QQ}{p}\right) = \left(\frac{p}{q}\right)
\end{gather}
por definici\'on del s\'imbolo de Legendre $\left(\frac{p}{q}\right)$ (cap\'itulo~\ref{cap5}, \S\ref{sec5.5}).

Por otra parte los resultados sobre la descomposici\'on de un n\'umero primo $p$ en $F = \QQ[\sqrt{q^{*}}]$
(cap\'itulo~\ref{cap5}, \S\ref{sec5.4}) nos otorgan??? m\'as informaci\'on sobre $\left(\frac{F/\QQ}{p}\right)$.

Por definici\'on, es la identidad si $p$ se descompone totalmente en $F$ y el automorfismo no trivial si $p$
es inerte. Por la proposici\'on~\ref{prop5.4.1} de \S\ref{sec5.4}, cap\'itulo~\ref{cap5}, tenemos por lo tanto, si $p$ es \emph{impar}
\begin{gather}\label{eq-6.5-2}
\left(\frac{F/\QQ}{p}\right) = \left(\frac{q^{*}}{p}\right).
\end{gather}
Comparando~\eqref{eq-6.5-1} y~\eqref{eq-6.5-2} obtenemos $\left(\frac{p}{q}\right)=\left(\frac{q^{*}}{p}\right)
=\left(\frac{-1}{p}\right)^{\frac{q-1}{2}}\left(\frac{q}{p}\right)$. Como $\left(\frac{-1}{p}\right)=(-1)^{\frac{p-2}{2}}$
por el criterio elemental de Euler (cap\'itulo~\ref{cap5}, \S\ref{sec5.5}, proposici\'on~\ref{prop5.5.1}), se sigue que $\left(\frac{p}{q}\right)
=(-1)^{\frac{(p-1)(q-1)}{4}}\left(\frac{q}{p}\right)$ y volemos a obtener la ley de la reciprocidad cuadr\'atica
(cap\'itulo~\ref{cap5}, \S\ref{sec5.5}, teorema~\ref{teo5.5.1}).

Si $p=2$, recordemos que $2$ se descompone totalmente en $F$ si $q^{*}\equiv 1\pmod 8$ y que es inerte
si $q^{*}\equiv 5\pmod 8$ (cap\'itulo~\ref{cap5}, \S\ref{sec5.4}, proposici\'on~\ref{prop5.4.1}). Como
\begin{gather*}
(-1)^{\frac{q^{2}-1}{8}}=(-1)^{\frac{q^{*2}-1}{8}}
\end{gather*}
vale $1$ si $q^{*}\equiv 1\pmod 8$ y $-1$ si $q^{*}\equiv 5\pmod 8$, tenemos por lo tanto
\begin{gather}\label{eq-6.5-3}
\left(\frac{F/\QQ}{2}\right)=(-1)^{\frac{q^{2}-1}{8}}.
\end{gather}
Comparando~\eqref{eq-6.5-1} y~\eqref{eq-6.5-3} obtenemos $\left(\frac{2}{q}\right) = (-1)^{\frac{q^{2}-1}{8}}$,
que no es otra cosa que la ``f\'ormula comlementaria'' dif\'icil (cap\'itulo~\ref{cap5}, \S\ref{sec5.5},
proposici\'on~\ref{prop5.5.2}, {\itshape b})).

\chapter*{Complementos sin demostraci\'on}

\setcounter{equation}{0}

Aqu\'i damos, sin demostraci\'on, algunos complementos a lo se hizo en el texto. Se tratan de cuestiones
muy cercanas a aquellas tratadas en el texto y cuyo grado de profundida y dificultad es an\'alogo tambi\'en.
No fueron incluidas para poder mantener este libro de un tama\~no razonable y tambi\'en porque se las trata
en otras obras (ver p. ej. el cap\'itulo~V de~\cite{ZariskiSamuel}, que se puede leer directamente despu\'es de este
libro).

El autor di\'o marcha atr\'as a la idea de dar, sin demostraci\'on, una descripci\'on de los desarollos m\'as avanzados
de la teor\'ia de n\'umeros (ad\`eles, cuerpos de clases, funciones zeta y series $L$, aritm\'etica de \'algebras
simples, teor\'ia anal\'itica, formas cuadr\'aticas, etc.). Para eso, env\'ia al lector a los t\'itulos nombrados
en la primera mitad de la bibliograf\'ia (que aparecen sin n\'umero).

\section*{F\'ormulas de transitividad}

Dados $3$ cuerpos encajados $K\subset L\subset M$, cada uno extensi\'on de \emph{grado finito} del anterior, tenemos las
aplicaciones ``traza''
\begin{gather*}
\Tr_{L/K}:L\to K,\quad\Tr_{M/L}:M\to L,\quad\Tr_{M/K}:M\to K,
\end{gather*}
y las aplicaciones ``norma'' an\'alogas (cap\'itulo~\ref{cap2}, \S\ref{sec2.6}). Entonces, para $x\in M$, se tiene
\begin{gather}
\left\{\begin{aligned}
\Tr_{M/K}(x) &= \Tr_{L/K}(\Tr_{M/L}(x))\\
N_{M/K}(x) &= N_{L/K}(N_{M/L}(x))
\end{aligned}
\right.
\end{gather}

\section*{Norma relativa de un ideal}

Dados dos cuerpos de n\'umeros ecajados $K\subset K'$ y un ideal (entero o fraccionario) $\idl{a}'$ de $K'$,
el ideal de $K$ generado por los $N_{K'/K}(x)$ ($x\in\idl{a}$) se llama \emph{norma relativa} de $\idl{a}'$,
y se nota $N_{K'/K}(\idl{a}')$ (o $N(\idl{a}')$). Si $\idl{a}'$ es un ideal principal $(a')$ se tiene
\begin{gather}
N_{K'/K}((a')) = (N_{K'/K}(a)).
\end{gather}
Si $K=\QQ$ recuperamos la noci\'on expuesta en el cap\'itulo~\ref{cap3}, \S\ref{sec3.5}: si $\idl{a}'$ es un
ideal entero de $K'$, y si $A'$ es el anillo de enteros de $K'$, se tiene
\begin{gather}
N_{K'/\QQ}(\idl{a}') = \card(A'/\idl{a}')\ZZ.
\end{gather}
Volviendo al caso genera, si $\idl{a}$ es un ideal de $K$ y si $n=[K':K]$, tenemos
\begin{gather}
N_{K'/K}(A'\idl{a}) = \idl{a}^{n}.
\end{gather}
Si $\idl{a}'$ y $\idl{b}'$ son dos ideales de $K'$, tenemos una f\'ormula de multiplicatividad:
\begin{gather}
N_{K'/K}(\idl{a}'\idl{b}') = N_{K'/K}(\idl{a}')N_{K'/K}(\idl{b}')
\end{gather}
Por \'ultimo, si $\idl{p}'$ es un ideal primo de $K'$ y $\idl{p} = \idl{p}'\cap K$ y si $f$ es el grado
residual de $\idl{p}'$ sobre $K$, se tiene
\begin{gather}
N_{K'/K}(\idl{p}') = \idl{p}^{f}.
\end{gather}
Dados tres cuerpos encajados $K\subset K'\subset K''$, tenemos la siguiente f\'ormula de transitividad, donde
$\idl{a}''$ denota un ideal de $K''$:
\begin{gather}
N_{K''/K}(\idl{a}'') = N_{K'/K}(N_{K''/K'}(\idl{a}'')).
\end{gather}
En la misma situaci\'on, la noci\'on de norma relativa de un ideal permite dar una f\'ormula de transitivdad
para los discriminantes (donde $\disc_{K'/K}$ denota el discriminante de $K'$ sobre $K$;
cf.~cap\'itulo~\ref{cap5}, \S\ref{sec5.3}, definici\'on~\ref{defV.3.1})
\begin{gather}
\disc_{K''/K}=N_{K'/K}(\disc_{K''/K'})\cdot\disc_{K'/K}^{[K'':K']}.
\end{gather}
Todo lo anterior se generaliza a un anillo de Dedekind $A$ y a su clausura \'integra $A'$ en una extensi\'on
de grado finito del cuerpo de fracciones de $A$.

\section*{El diferente}

Lo que sigue es v\'alido para un anillo de Dedekind $A$ y a la clausura \'integra de $A$ en una extensi\'on
de grado finito de su cuerpo de fracciones. Para simplificar, supondremos que estamos en el caso de un cuerpo
de n\'umeros.

Sean $K\subset K'$ dos cuerpos encajados, $A$ y $A'$ sus anillos de enteros. Decimos que un ideal maximal
$\idl{p}'$ de $A'$ \emph{es ramificado} sobre $A$ (o sobre $K$) si su \'indice de ramificaci\'on sobre $A$ es $>1$.
Entonces el ideal maximal $\idl{p} = \idl{p}'\cap A$ de $A$ \emph{ramifica} en $A'$ (cap\'itulo~\ref{cap5}, \S\ref{sec5.3}).
Resulta f\'acilmente del cap\'itulo~\ref{cap5}, \S\ref{sec5.3}, teorema~\ref{teo5.3.1} que s\'olo hay \emph{un n\'umero finito} de
ideales maximales de $A'$ que son ramificados sobre $A$. Nosotros caracterizaremos un ideal $\diff_{K'/K}$
de $A'$, el \emph{``diferente''} de $K'$ sobre $K$, tal que esos ideales son exactamente aquellos que contienen
a $\diff_{K'/K}$ (observar la analog\'ia con el cap\'itulo~\ref{cap5}, \S\ref{sec5.3}, teorema~\ref{teo5.3.1}).

Se demuestra primero que nada que el conjunto de los $x\in K'$ tales que
\begin{gather}
\Tr_{K'/K}(xA') \subset A
\end{gather}
es un ideal fraccionario $\mathfrak C$ de $A'$, que lo llamamos el \emph{codiferente} de $K'$ sobre $K$.
Por definici\'on el \emph{diferente} $\diff_{K'/K}$ es el ideal inverso $\mathfrak C^{-1}$. Es un ideal
\emph{entero} no nulo de $A'$. Se demuestra que est\'a \emph{generado por los $F'(x)$,} donde $x$ recorre $A'$ y
$F$ denota el polinomio minimal de $x$ sobre $K$. En particular, si $A'$ es de la forma $A[y]$ (que no
es el caso siempre) y si $G$ es el polinomio minimal de $y$ sobre $K$, entonces el diferente $\diff_{K'/K}$
es el ideal principal de $A'$ generado por $G'(y)$.

Los ideales primos no nulos de $A'$ que son ramificados sobre $A$ son aquellos que contienen $\diff_{K'/K}$.
M\'as precisamente, sea
\begin{gather}
\diff_{K'/K}=\prod_{i}{\idl{p}'_{i}}^{m_{i}}\quad(m_{i}>0)
\end{gather}
la descomposici\'on del diferente en ideales primos y sea $e_{i}$ el \'indice de ramificaci\'on de $\idl{p}'_{i}$
sobre $A$. Entonces los ideales primos no nulos de $A'$ que son ramificados sobre $A$ son los $\idl{p}'_{i}$
y se tiene $m_{i}\geq e_{i}-1$ para todo $i$. Adem\'as, se tiene $m_{i}=e_{i}-1$ si y s\'olo si $e_{i}$ es
coprimo con la caracter\'istica del cuerpo residual $A'/\idl{p}'_{i}$.

El \index{diferente} $\diff_{K'/K}$ (ideal de $A'$) y el discriminante $\disc_{K'/K}$ (ideal de $A$) est\'an relacionados
de la siguiente manera:
\begin{gather}
\disc_{K'/K}=N_{K'/K}(\diff_{K'/K})
\end{gather}
(cf.~cap\'itulo~\ref{cap2}, \S\ref{sec2.7}, f\'ormula~\eqref{eq2.7.6}). As\'i la informaci\'on del diferente es m\'as
precisa que aquella del discriminante.

Por \'ultimo, dados tres cuerpos encajados $K\subset K'\subset K''$, se tiene la siguiente f\'ormula de transitividad
para los diferentes:
\begin{gather}
\diff_{K''/K} = \diff_{K''/K'}\cdot\diff_{K'/K}.
\end{gather}

\chapter*{Ejercicios}

Los ejercicios marcados $A$ son ejercicios ``de reflexi\'on inmediata'', destinados al
control directo del conocicimiento ?. Los ejercicios marcados B son m\'as elaborados. Los
``problemas de revisi\'on'' (al final) son problemas de examen, donde a veces hay menos gu\'ias
que en algunos de los ejercicios B.

\section*{Cap\'itulo I}

\begin{exer*}{1 B}
Sea $p$ un n\'umero primo y $r\in\NN$. Mostrar que el grupo multiplicativo
$(\ZZ/p^r\ZZ)^\times$ es c\'iclico, salvo si $p=2$ y $r\geq 3$ (para $p$ impar,
mostrar que la clase de $1+p$ tiene orden $p^{r-1}$; para $p=2$ estudiar el orden
de la clase de $5$; utilizar despu\'es el Corolario~\ref{cor1.5.4} del Teorema~\ref{teo1.5.1},
\S\ref{sec1.5}).
\end{exer*}

\begin{exer*}{2 B}
\end{exer*}

\begin{exer*}{3 B}
Rehacer la demostraci\'on cla\'sica del hecho que hay una infinidad de n\'umeros primos. Inspirandose
en esa, mostrar que hay una infinidad de n\'umeros primos de la forma $4k-1$ ($k\in\NN$).
\end{exer*}

\begin{exer*}{4B}
Para que $n\in\NN$ sea primo, es necesario y suficiente que $n$ divida a
$(n-1)!+1$. (Si $n$ es primo, calcular el producto de los elementos de $\FF_n^\times$;
examinar a continuaci\'on el caso cuando $n$ no es primo).
\end{exer*}

\begin{exer*}{5B}
Mostrar que, en un cuerpo finito $K$, todo elemento es suma de dos cuadrados (tratar
primero el caso donde $q = \card(K)$ es par; si $q$ es impar, calcular el numero de valores
que toma la funci\'on $x\mapsto x^2$ y $y\mapsto a-y^2$, $x, y\in K$, $a\in K$ dados).
\end{exer*}

\begin{exer*}{6B}
Descomponer el polinomio $X^3-X+1$ sobre $\FF_{23}$ y el polynomio $X^3+X+1$ sobre
$\FF_{31}$ (cada uno tiene una ra\'iz doble y una ra\'iz simple).
\end{exer*}

\begin{exer*}{7A}
Dar un ejemplo de dos ideales $\idl{a}, \idl{b}$ de un anillo $A$ tal que
$\idl{a}\cap\idl{b}\neq\idl{a}\idl{b}$. Mostrar que siempre se tiene
$\idl{a}\idl{b}\subset\idl{a}\cap\idl{b}$.
\end{exer*}

\begin{exer*}{8B}
Sea $A$ un dominio \'integro, $a, b\in A$, y $B = A[X]/(aX+b)$.
Mostrar que, si $Aa\cap Ab=Aab$, entonces $B$ es \'integro (considerar
el elemento $-b/a$ del cuerpo de fracciones $K$ de $A$, y mostrar que el
morfismo $\varphi : A[X]\to K$ definido por $\varphi(X) = -b/a$ y $\varphi(y) = y$
para $y\in A$ tiene n\'ucleo exactamente $(aX+b)$.
\end{exer*}

\begin{exer*}{9B}
Sea $A$ un dominio \'integro y $i, j$ dos enteros $\geq 1$ coprimos. Mostrar que el
ideal $(X^i-Y^j)$ del anillo de polinomios $A[X,Y]$ es primo (definir un morfismo
$A[X,Y]\to A[X]$ que tenga este ideal como n\'ucleo.
\end{exer*}

\begin{exer*}{10B}
Sea $A$ un dominio \'integro, $K$ su cuerpo de fracciones y $b$ un elemento no nulo
de $A$. Mostrar
\end{exer*}

\section*{Cap\'itulo II}

\begin{exer*}{1A}
El teorema~\ref{teo2.7.1} de \S\ref{sec2.7} tambi\'en vale si, en vez de suponer que $K$ es
de caracter\'istica cero, suponemos que $K$ es finito. Explicar porqu\'e. Es interesante
el caso de $K$ es finito?
\end{exer*}

\backmatter

\begin{thebibliography}{21}

\bibitem{} {\scshape E. Artin.} {\itshape Theory of algebraic numbers}
(G. Striker, Schildweg 12, G\"ottingen, Allemagne-1957)
(Explica el pasaje a la teor\'ia de las valuaciones; muy
eleganate; muchos ejemplos).

\bibitem{} {\scshape H. Hasse.} {\itshape Zahlentheorie}
(Akademie Verlag, Berlin, 1949) (masivo y muy completo).

\bibitem{} {\scshape H. Hasse.} {\itshape Vorlesungen \"uber
Zahlen theorie} (Springer, 1964) (describe muchos aspectos
de la teor\'ia de n\'umeros).

\bibitem{} {\scshape G.H. Hardy} y {\scshape E.M. Wright.} {\itshape An introduction to the theory
of numbers} (Clarendon Press-Oxford, 1965) (profundo y atractivo; un notable sentido est\'etico en
la elecci\'on de los temas).

\bibitem{} {\scshape E. Hecke,} {\itshape Vorlesungen \"uber die Theorie der algebraischen Zahlen}
(Chelsea, New York, 1948) (un cl\'asico, muy eficaz y completo).

\bibitem{} {\scshape S. Lang,} {\itshape Algebraic Numbers,} (Addison-Wesley, 1964) (un
libro peque\~no, muy denso y concentrado).

\bibitem{} {\scshape S. Lang,} {\itshape Diophantine Geometry} (Interscience Tract n\textsuperscript{$\circ$}
11, J. Wiley, New York, 1962) (orientado hacie las ecuaciones diof\'anticas; presenta muy claramente su
conexi\'on con la Goemetr\'ia algebrica).

\bibitem{} {\scshape O'Meara,} {\itshape Introduction to quadratic forms} (Springer, 1963) (una exposici\'on
muy eficaz de la teor\'ia de n\'umeros algebraicos, seguida de una de sus m\'as bellas aplicaciones).

\bibitem{} {\scshape J.P. Serre,} {\itshape Corps locaux} (Hermann, Paris, 1962) (el acento se posa aqu\'i
sobre los cuerpos $p$-\'adicos; una presentaci\'on muy clara y l\'ucida de los m\'etodos algebraicos m\'as
recientes de la Teor\'ia de n\'umeros; muy rico y autocontenido; muchos ejemplos).

\bibitem{} {\scshape E. Artin} and {\scshape J. Tate.} {\itshape Class-field theory}
(Math. Dept. Harvard University) (la exposici\'on m\'as moderna de la famosa teor\'ia de llos ``cuerpos de clases,''
es decir las extensiones abelianas de los cuerpos de n\'umeros).

\bibitem{} {\scshape A. Weil,} {\itshape Basic number theory} (Springer, 1967) (utiliza los m\'etodos
de los ad\`eles, y trata al mismo tiempo el caso de los cuerpos de n\'umeros y los cuerpos de funciones).

\bibitem{} {\scshape Z.I. Borovic} et {\scshape I.R. Safarevic.} {\itshape Th\'eorie de nombres}
(Gauthiers Villars, 1966)
(muy completo; excelentes cap\'itulos sobre los m\'etodos anal\'iticos, complejos y p-\'adicos; numerosas
tabl\'as n\'umericas).

%%%%% SHOULD START NUMBERING HERE

\bibitem{Bourbaki1} {\scshape N. Bourbaki.} {\itshape Alg\`ebre} (Paris, Hermann).
Sobre todo el cap\'itulo V para lo que concierne a los cuerpos,
cap\'itulo VI para lo que concierne la divisibilidad y el
cap\'itulo VII para lo que concierne a los m\'odulos sobre
dominios de ideales principales.
\bibitem{Bourbaki2} {\scshape N. Bourbaki.} {\itshape Alg\`ebre commutative} (\textit{ibid}). Sobre todo el
cap\'itulo V para lo que concierne los elementos enteros, y el cap\'itulo VII para lo concerniente a los
anillos de Dedekind y de factorizaci\'on \'unica. En el cap\'itulo II se puede encontrar una teor\'ia muy
completa y general de los anillos de fracciones. Una buena exposici\'on de la teor\'ia de valuaciones en
el cap\'itulo VI.

\bibitem{Cartan} {\scshape H. Cartan.} {\itshape Th\'eorie \'el\'ementaire des fonctions analytiques...}
(Paris, Hermann, 1962).

\bibitem{Choquet} {\scshape G. Choquet.} {\itshape Cours d'analyse} (Paris, Masson, 1963).

\bibitem{Lang} {\scshape S. Lang.} {\itshape On quasi-algebraic closure} (Ann. of math., 55 (1962), 373--390).

\bibitem{Samuel1} {\scshape P. Samuel.} {\itshape A propos du th\'eor\`eme des unit\'es}
(Bull. Sci. math., 90, (1966) 89--96).

\bibitem{Samuel2} {\scshape P. Samuel.} {\itshape Anneaux factoriels} (Publ. Soc. mat. S\~ao Paulo, 1964).

\bibitem{Terjanian} {\scshape G. Terjanian,} {\itshape Sur une conjecture de M. Artin} (C.R. Acad. Sci. Paris,
(1966)).

\bibitem{ZariskiSamuel} {\scshape O. Zariski} and {\scshape P. Samuel,} {\itshape Commutative algebra,} Vol. I
(Van Nostrand, Princeton, 1958). Cap\'itulo II para los cuerpos, cap\'itulo IV para los anillos noetherianos,
cap\'itulo V para los elementos enteros y los anillos de Dedekind.
\end{thebibliography}

%\printindex

\end{document}
